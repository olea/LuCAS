% Linux Installation and Getting Started    -*- TeX -*-
% problems.tex
% Copyright (c) 1992-1994 by Matt Welsh <mdw@sunsite.unc.edu>
%
% This file is freely redistributable, but you must preserve this copyright 
% notice on all copies, and it must be distributed only as part of "Linux 
% Installation and Getting Started". This file's use is covered by the 
% copyright for the entire document, in the file "copyright.tex".
%
% Copyright (c) 1998 by Specialized Systems Consultants Inc. 
% <ligs@ssc.com>
%Revisi�n 1 15/07/2002 por Francisco Javier Fern�ndez <serrador@arrakis.es>
%Para publicar

\section{Cuando se tienen problemas}
\markboth{Obtenci�n e instalaci�n de {\linux}}{Cuando se tienen problemas}
\namedlabel{sec-install-probs}{Problemas de instalaci�n}
\index{instalaci�n!problemas|(}

Casi todos hemos tenido alguna clase de problema o cuelgue cuando se intenta
instalar {\linux} la primera vez. La mayor�a de las veces, el problema es causado
por un simple malentendido. Sin embargo, a veces el problema puede ser
algo m�s serio, como un descuido de uno de los desarrolladores o un bug.

%Almost everyone runs into some kind of snag or hangup when attempting
%to install Linux the first time. Most of the time, the problem is caused
%by a simple misunderstanding. Sometimes, however, it can be something
%more serious, like an oversight by one of the developers, or a bug.

%Esta seccion describir� algunos de los problemas de instalaci�n m�s comunes
%y c�mo resolverlos (N del T Comentado en el original)
Si su instalaci�n parece ser exitosa pero se recibieron mensajes de error 
inesperados, �stos se describen aqu�.

%% This section will describe some of the
%% most common installation problems, and how to solve them. 
%If your installation
%appears to be successful, but you received unexpected error messages,
%these are described here as well.

%Si su instalaci�n aparenta haber sido exitosa pero se recibieron mensajes inesperados
%de error, �stos se describen aqu�.
% Linux Installation and Getting Started    -*- TeX -*-
% booting.tex
% Copyright (c) 1992-1994 by Matt Welsh <mdw@sunsite.unc.edu>
%
% This file is freely redistributable, but you must preserve this copyright 
% notice on all copies, and it must be distributed only as part of "Linux 
% Installation and Getting Started". This file's use is covered by the 
% copyright for the entire document, in the file "copyright.tex".
%
% Copyright (c) 1998 by Specialized Systems Consultants Inc. 
% <ligs@ssc.com>

%Traducido al espa�ol por Sebasti�n Gurin, Cancerbero <anon@adinet.com.uy>
%Revisi�n 1 realizada el 9 de julio de 2002 por Francisco Javier Fern�ndez <serrador@arrakis.es>
% Revisi�n 2 realizada el 15 de julio de 2002 por Fco. javier fernandez 

\subsection{Problemas arrancando desde el disquete de instalaci�n}
%Problems with booting the installation media}
\namedlabel{sec-install-probs-booting}{problemas en el inicio}

\index{instalaci�n!problemas iniciando}
\index{inicio!problemas}

Cuando se intenta iniciar el disquete de instalaci�n por primera vez, puede encontrarse con algunos problemas, los cuales se enumeran abajo. Por favor, tenga en cuenta que estos inconvenientes
 {\em no est�n\/} relacionados con el inicio de su sistema GNU/Linux nuevo y ya instalado. Consulte la p�gina~\pageref{sec-install-probs-postinstall} para m�s informaci�n sobre estos problemas. 

\begin{itemize}
\item {\bf Error en el disquete o en otro dispositivo al intentar arrancar el sistema}

La causa m�s frecuente de esta clase de problema es que el disquete est� corrupto. Si el disquete se encuentra f�sicamente da�ado, deber�a construirlo de nuevo usando un disquete en buen estado. 
Si los datos del disquete son los que se encuentran defectuosos, deber�a verificar que ha descargado y transferido los datos al disquete correctamente. Generalmente, para solucionar este problema, bastar�
simplemente con volver a crear el disquete de arranque. Repita todos los pasos, e intente nuevamente. 

Si ha recibido su disquete de arranque por correo o alg�n otro distribuidor en lugar de descargarlo y crearlo por usted mismo, comun�quese con el distribuidor para pedirle uno nuevo --pero s�lo despu�s
de verificar que �ste es, efectivamente el problema--.

\item {\bf El sistema se cuelga durante el arranque, o despu�s de arrancar} 

Despu�s de que el disquete arranca, ver� un n�mero de mensajes del n�cleo, indicando cu�les dispositivos fueron detectados y configurados. Despu�s de �sto, normalmente, se le presentar� un indicador
de ingreso\NT{``login indicador de �rdenes'' en el original}, permiti�ndole proceder con la instalaci�n (en lugar de esto, algunas distribuciones le lanzar�n justo al programa de instalaci�n). El 
sistema  puede parecer como si estuviera bloqueado, durante muchos de estos pasos. Tenga paciencia: cargar software desde un disquete es un proceso lento. En muchos casos el sistema puede no haberse 
colgado de ninguna manera, simplemente necesita algo de tiempo. Verifique que no haya ninguna actividad, ya sea en el disco o en el sistema, por lo menos por unos cuantos minutos, antes de suponer que 
el sistema est� bloqueado. 

\begin{enumerate}
\index{LILO!problemas iniciando}

\item Despu�s de iniciar el sistema desde el indicador {\tt LILO}, se debe cargar la imagen del n�cleo del disquete. Esto puede tomar unos cuantos segundos; usted podr� asegurarse de que todo marcha bien
si la luz de la disquetera est� encendida. 

\item Cuando arranca el n�cleo, los dispositivos SCSI deben ser detectados. Si no posee ning�n dispositivo SCSI, el sistema har� un alto de 15 segundos, mientras contin�a la detecci�n de los posibles 
SCSI; esto ocurre, generalmente, despu�s de que la l�nea
\begin{tscreen}
lp\_init: lp1 exists (0), using polling driver
\end{tscreen}
aparece en pantalla. 

\item Una vez que  el n�cleo halla terminado de cargarse, el control pasa a los ficheros de arranque que hay en el disquete. Al cabo de esto, usted podr� ver un indicador de entrada \NT{ login}, o bien, 
se iniciar� un programa de instalaci�n. Si se le presenta un indicador de ingreso como
\begin{tscreen}
Linux login:
\end{tscreen}
entonces deber�a entrar, (por lo normal es  como {\tt root} o {\tt install} --- esto var�a seg�n cada distribuci�n). Antes de ingresar el nombre de usuario, el sistema puede detenerse por 20 segundos o
m�s, mientras el programa de instalaci�n o el int�rprete de �rdenes se  carga desde el disquete. Nuevamente, si esto sucediera, la luz de la disquetera deber� estar encendida. No suponga que el sistema 
est� colgado. 

\end{enumerate}

Cualquiera de las causas comentadas arriba, puede ser el origen de su problema. De todos modos, es posible que el sistema realmente se cuelgue mientras se inicia, lo cual puede deberse a muchas causas.
En primer lugar, puede ser que usted no posea suficiente memoria RAM para iniciar el disquete de instalaci�n. (Vea el siguiente punto al respecto para m�s informaci�n acerca de c�mo desactivar el disco
RAM (ramdisk) para liberar memoria.)

La causa del cuelgue de muchos sistemas, es la incompatibilidad del hardware. El �ltimo cap�tulo muestra un vistazo general del hardware soportado en GNU/Linux. Incluso si sus dispositivos son soportados,
 usted puede meterse en problemas si tiene configuraciones incompatibles de hardware, las cu�les pueden estar causando que el sistema se cuelgue. Vea la p�gina~\pageref{sec-install-probs-hardware}, 
adelante, para una discusi�n sobre incompatibilidades de hardware. 


\item {\bf El sistema informa de errores a causa de falta de memoria, mientras intenta arrancar o instalar el software} 

Este punto trata sobre la cantidad de RAM con la que se dispone. En sistemas con 4 Mbytes o menos de RAM, se  pueden tener problemas al arrancar el disquete, o al instalar el software. Esto se produce 
ya que muchas distribuciones usan un disco RAM (ramdisk), que se trata de un sistema de ficheros cargado directamente en la memoria RAM, para las operaciones que se ejecutan mientras el disquete de 
instalaci�n est� siendo utilizado. La imagen entera del disquete de instalaci�n, por ejemplo, puede  cargarse en un disco RAM (ramdisk), lo que puede requerir m�s de un Mbyte de memoria RAM. 

% La soluci�n para este problema, es desactivar la opci�n de disco RAM 
% cuando se inicia el disquete de instalci�n. El procedimiento para efectuar 
% esto depende de la versi�n de Linux. Por ejemplo, en SLS, usted 
% deber� teclear ``{\tt floppy}'' en el indicador de �rdenes LILO, cuando se inicia el disco 
% {\tt a1}. 

Puede suceder que, cuando se intenta iniciar o instalar el software, en lugar de un mensaje de error por falta de memoria, su sistema se bloquee inesperadamente durante el arranque. 
Si su sistema se cuelga, y ninguna de las explicaciones de la secci�n anterior parezcan ser la causa, trate de desactivar el disco RAM. Vea la documentaci�n de su distribuci�n para m�s detalles. 

Acu�rdese de que GNU/Linux necesita para s� mismo, por lo menos 2 Mbytes de memoria RAM para ejecutarse; distribuciones m�s modernas requieren 4 Mbytes o m�s. 


\item {\bf El sistema informa de errores como ``{\tt permission denied}\NT{ permiso denegado}'', o ``{\tt file not found}\NT{ fichero no encontrado}'', durante el arranque}

Esto es un indicio de que su disquete de instalaci�n est� da�ado o contiene datos corruptos. Si usted est� tratando de iniciar el sistema desde el disquete (y est� seguro de que est� haciendo todo
correctamente), no deber�a estar viendo errores como �ste. Contacte con  su distribuidor de software GNU/Linux e indague sobre el problema. Quiz� pueda obtener otra copia del disquete de arranque si
 es necesario. Si usted ha descargado el disco de inicio, trate de construirlo nuevamente en un disquete sano. Posiblemente esto resuelva su problema. 


\item {\bf El sistema informa del error ``{\tt VFS: Unable to mount root}\NT{ no se puede montar ra�z}'' cuando se inicia}

Este mensaje de error indica que el sistema de ficheros ra�z (que se encuentra en el disquete de arranque), no puede ser localizado. Esto puede significar que su disquete de arranque est� corrupto de
 alguna manera, o que el sistema no se est� inicializando correctamente. 

Por ejemplo, muchas distribuciones de CD-ROM requieren que se encuentre el CD-ROM en la lectora, cuando se inicia la instalaci�n. Aseg�rese de que la lectora de CD-ROM est� encendida y  que funcione 
correctamente. Tambi�n es posible que el sistema no pueda localizar su unidad de CD-ROM cuando se inicia; para m�s informaci�n consulte la p�gina~\pageref{sec-install-probs-hardware}. 

% Si usted est� seguro de estar iniciando el sistema correctamente, 
% entonces su disquete de instalaci�n puede estar de veras da�ado. Este 
% es un problema bastante extra�o, y es por esto que deber�a tratar con 
% otras soluciones antes de intentar usar otro disquete o cinta magn�tica 
% de arranque. 


\end{itemize}

\index{instalaci�n!problemas en el inicio}
\index{iniciando!problemas}


% Traducci�n terminada por Sebasti�n Gurin, Cancerbero <anon@adinet.com.uy> 
% Linux Installation and Getting Started    -*- TeX -*-
% hardware.tex
% Copyright (c) 1992-1994 by Matt Welsh <mdw@sunsite.unc.edu>
%
% This file is freely redistributable, but you must preserve this copyright 
% notice on all copies, and it must be distributed only as part of "Linux 
% Installation and Getting Started". This file's use is covered by the 
% copyright for the entire document, in the file "copyright.tex".
%
% Copyright (c) 1998 by Specialized Systems Consultants Inc. 
% <ligs@ssc.com>
% Traducido por Francisco Javier Frenandez <serrador@arrakis.es>
% Revision 1 6 de julio por Fco. Javier Fern�ndez <serrador@arrakis.es>
%Revisi�n 2 15 julio 2002
%gold
\subsection{Problemas con el hardware}
\label{sec-install-probs-hardware}

\index{hardware!problemas}
\index{instalaci�n!problemas con el hardware}

%The most common form of problem when attempting to install or use Linux
%is an incompatibility with hardware. Even if all of your hardware is supported
%by Linux, a misconfiguration or hardware conflict can sometimes cause
%strange results---your devices may not be detected at boot time, or
%the system may hang. 
El problema m�s com�n cuando se intenta instalar o usar GNU/Linux es una incompatibilidad con el hardware. 
Incluso si todo su hardware est� soportado por GNU/Linux, una configuraci�n err�nea o un conflicto con otro
dispositivo puede algunas veces ocasionar resultados extra�os---los dispositivos pueden no ser detectados al arrancar,
o el sistema se puede colgar.

%It is important to isolate these hardware problems if you suspect 
%that they may be the source of your trouble. 
Es importante aislar estos problemas con el hardware si se sospecha que �stos pueden ser la fuente de sus problemas.
%Comentado en la versi�n original
% In the following sections 
% we will describe some common hardware problems and how to resolve them.

% Linux Installation and Getting Started    -*- TeX -*-
% conflicts.tex
% Copyright (c) 1992-1994 by Matt Welsh <mdw@sunsite.unc.edu>
%
% This file is freely redistributable, but you must preserve this copyright 
% notice on all copies, and it must be distributed only as part of "Linux 
% Installation and Getting Started". This file's use is covered by the 
% copyright for the entire document, in the file "copyright.tex".
%
% Copyright (c) 1998 by Specialized Systems Consultants Inc. 
% <ligs@ssc.com>

%Tradu por Fco. Javier Fern�ndez <serrador@arrakis.es>
%Revisi�n 1 16/7/2002  por Francisco Javier Fernandez

\subparagraph*{Aislando problemas de hardware}
\namedlabel{sec-install-probs-hardware-conflicts}{Conflictos con el hardware}
\index{hardware!conflictos}

Si experimentas alg�n problema que creas que est� relacionado con el hardware, 
lo primero que deber�as hacer es intentar aislar el problema.  
Esto significa  eliminar  todas las posibles variables y (usualmente) 
desmontar el sistema, pieza a pieza, hasta que el componente es aislado.

%If you experience a problem that you believe to be hardware-related, 
%the first thing that you should to do is attempt to isolate the problem.
%This means eliminating all possible variables and (usually) taking the
%system apart, piece-by-piece, until the offending piece of hardware is
%isolated.

Esto no es tan terrible como suena. B�sicamente, se deber� retirar todo
el hardware prescindible del equipo, y entonces determinar cu�l de los
dispositivos est� causando el problema, posiblemente reconectando cada
dispositivo, uno cada vez. Esto quiere decir que se deber� retirar todo el
hardware excepto la unidad de disquettes y la tarjeta de v�deo y por supuesto
el teclado. Incluso los dispositivos aparentemente inocentes como los ratones
pueden causar insospechados problemas a no ser que se les considere 
no esenciales.

%This is not as frightening as it may sound. Basically, you should
%remove all nonessential hardware from your system, and then determine
%which device is causing the trouble---possibly by reinserting each
%device, one at a time. This means that you should remove all hardware
%other than the floppy and video controllers, and of course the
%keyboard. Even innocent-looking devices such as mouse controllers can
%wreak unknown havoc on your peace of mind unless you consider them
%nonessential.

Por ejemplo, digamos que el sistema se cuelga durante la secuencia de 
detecci�n de la placa Ethernet en el arranque. Quiz� pueda hipotetizar
que hay un conflicto con la tarjeta Ethernet en su computadora. La manera
r�pida y f�cil de encontrarlo es extraer la tarjeta Ethernet e intentar
arrancar otra vez. Si todo va bien, entonces  sabe que o (a) la tarjeta 
Ethernet no tiene soporte en Linux (ver p�gina~\pageref{sec-intro-hardware}),
o (b) hay un conflicto de direcci�n o IRQ con la tarjeta.

%For example, let's say that the system hangs during the Ethernet board
%detection sequence at boot time. You might hypothesize that there is a
%conflict or problem with the Ethernet board in your machine. The quick
%and easy way to find out is to pull the Ethernet board, and try
%booting again. If everything goes well, then you know that either (a)
%the Ethernet board is not supported by Linux (see
%P�gina~\pageref{sec-intro-hardware}), or (b) there is an address or IRQ
%conflict with the board.

\index{IRQ}
``�Conflicto de direcci�n o IRQ?'' �Qu� diablos significa esto?
Todos los dispositivos en un computador usan una {\bf IRQ}, o 
{\em Interrupt ReQuest line}, \NT{l�nea de petici�n de interrupci�n}
para decirle al sistema que necesitan
algo hecho. Puedes pensar en la IRQ como un cordel del que el dispositivo tira
cuando necesita que el sistema se haga cargo de  alguna petici�n pendiente.
Si m�s de un dispositivo est� tirando del mismo cordel, el n�cleo no es capaz
de determinar cu�l de los dispositivos necesita su atenci�n. Desastre al instante.


%\index{IRQ}
%``Address or IRQ conflict?'' What on earth does that mean? 
%All devices in your machine use an {\bf IRQ}, or 
%{\em interrupt request line}, to tell the system that they need something
%done on their behalf. You can think of the IRQ as a cord that the device
%tugs when it needs the system to take care of some pending request.
%If more than one
%device is tugging on the same cord, the kernel won't be able to detemine
%which device it needs to service. Instant mayhem.

Entonces, hay que asegurarse de que todos los dispositivos instalados usan
una �nica IRQ. En general la IRQ de un dispositivo puede establecerse mediante
jumpers en la tarjeta; mira la documentaci�n para detalles espec�ficos 
del dispositivo.
Algunos dispositivos no requieren el uso de una IRQ, pero se sugiere
que si hay alguna disponible, se ponga. (Las controladoras SCSI Seagate
ST01 y ST02 son buenos ejemplos).

%Therefore, be sure that all of your installed devices use unique IRQ
%lines. In general, the IRQ for a device can be set by jumpers on the
%card; see the documentation for the particular device for details.
%Some devices do not require the use of an IRQ at all, but it is
%suggested that you configure them to use one if possible. (The Seagate
%ST01 and ST02 SCSI controllers are good examples).

En algunos casos, el n�cleo proporcionado por tu medio de instalaci�n est�
configurado para usar ciertas IRQs para ciertos dispositivos. Por ejemplo, la 
controladora SCSI TMC-950, la controladora de CD-ROM Mitsumi y el driver del bus del rat�n.
Si se quiere usar dos o m�s de estos dispositivos, habr� que instalar primero
{\linux} con s�lo uno de estos dispositivos activados, despu�s recompilar
el n�cleo para cambiar la IRQ  predeterminada de uno de ellos.
(Ver cap�tulo~\ref{chap-sysadm-num} para informaci�n acerca de recompilar el n�cleo.)

%In some cases, the kernel provided on your installation media is configured
%to use certain IRQs for certain devices. For example, on some distributions
%of Linux, the kernel is preconfigured to use IRQ 5 for the TMC-950 SCSI 
%controller, the Mitsumi CD-ROM controller, and the bus mouse driver. 
%If you want to use two or more of these devices, you'll need to first
%install Linux with only one of these devices enabled, then recompile the
%kernel in order to change the default IRQ for one of them.
%(See Chapter~\ref{chap-sysadm-num} for information
%on recompiling the kernel.) 


Otro �rea donde pueden aparecer conflictos de hardware es con los canales DMA
(Direct Memory Access)\NT{acceso directo a memoria}, direcciones de E/S y direcciones de
memoria compartida. Todos estos t�rminos describen mecanismos a trav�s de los cuales el sistema
se comunica con los dispositivos f�sicos. Algunas tarjetas Ethernet, por ejemplo,
usan una direcci�n compartida de memoria as� como una IRQ para comunicarse con el sistema.
Si cualquiera de �stas est� en conflicto con otros dispositivos, entonces el sistema puede comportarse
de manera err�tica.
Deber�as ser capaz de cambiar el canal DMA, las direcciones de E/S o memoria compartida para varios
dispositivos mediante los jumpers \NT{ unas clavijas en la placa} de las tarjetas. (Desafortunadamente, algunos
dispositivos no permiten cambiar estos ajustes.)




%Another area where hardware conflicts can arise is with DMA (direct
%memory access) channels, I/O addresses, and shared memory addresses. 
%All of these terms describe mechanisms through which the system interfaces 
%with hardware devices. Some Ethernet boards, for example, use a shared memory 
%address as well as an IRQ to interface with the system. If any of these
%are in conflict with other devices, then the system may behave unexpectedly.
%You should be able to change the DMA channel, I/O or shared
%memory addresses for your various devices with jumper settings. (Unfortunately,
%some devices don't allow you to change these settings.)

La documentaci�n para varios dispositivos hardware deber�a especificar la IRQ,
el canal DMA, direcci�n E/S o direcci�n de memoria compartida que el dispositivo
usa, y c�mo configurarlo. Otra vez, la manera m�s simple de evitar estos problemas
es deshabilitar temporalmente los dispositivos en conflicto hasta que se tenga
tiempo de determinar la causa del problema.

%The documentation for various hardware devices should specify the IRQ,
%DMA channel, I/O address, or shared memory address that the devices
%use, and how to configure them. Again, the simple way to get around
%these problems is to temporarily disable the conflicting devices until
%you have time to determine the cause of the problem.

En el cuadro se puede ver una lista de  las IRQ y canales DMA utilizados
por varios dispositivos ``est�ndar'' en la mayor�a de sistemas. Casi
todos los sistemas tienen alguno de estos dispositivos, as� que se puede
evitar el poner la IRQ o el DMA de otro dispositivo en conflicto con estos valores.

%The table below is a list of IRQ and DMA channels used by various
%``standard'' devices on most systems. Almost all systems have some of
%these devices, so you should avoid setting the IRQ or DMA of other
%devices in conflict with these values.

\begin{table}\begin{center}
\small\begin{tabular}{|l|l|l|l|}
\hline
Device                     &   I/O address  & IRQ & DMA \\
\hline
{\tt ttyS0} ({\tt COM1})   &   3f8          &  4  &  n/a \\
{\tt ttyS1} ({\tt COM2})   &   2f8          &  3  &  n/a \\
{\tt ttyS2} ({\tt COM3})   &   3e8          &  4  &  n/a \\
{\tt ttyS3} ({\tt COM4})   &   2e8          &  3  &  n/a \\

{\tt lp0} ({\tt LPT1})     &   378 - 37f    &  7  &  n/a \\
{\tt lp1} ({\tt LPT2})     &   278 - 27f    &  5  &  n/a \\

{\tt fd0}, {\tt fd1} (disqueteras 1 y 2) & 3f0 - 3f7 & 6 & 2 \\
{\tt fd2}, {\tt fd3} (disqueteras 3 y 4) & 370 - 377 & 10 & 3 \\
\hline
\end{tabular}\normalsize\rm
\caption{Preajustes por omisi�n de dispositivos est�ndar.}
\label{table-dev-settings}
\end{center}\end{table}

\index{hardware!conflictos|)}

% Linux Installation and Getting Started    -*- TeX -*-
% hd.tex
% Copyright (c) 1992-1994 by Matt Welsh <mdw@sunsite.unc.edu>
%
% This file is freely redistributable, but you must preserve this copyright 
% notice on all copies, and it must be distributed only as part of "Linux 
% Installation and Getting Started". This file's use is covered by the 
% copyright for the entire document, in the file "copyright.tex".
%
% Copyright (c) 1998 by Specialized Systems Consultants Inc. 
% <ligs@ssc.com>
% Traducci�n realizada por Francisco javier Fern�ndez <serrador@arrakis.es>
%Revisi�n 1 por FJFS 
%Gold
\subparagraph*{Problemas reconociendo la controladora de disco}%Problems recognizing hard drive or controller.}
\index{hardware!problemas con el disco duro}

%When Linux boots, you should see a series of messages on your screen such
%as: 
Cuando {\linux} arranca, se deber�a ver una serie de mensajes en la pantalla como:
\begin{tscreen}
Console: colour EGA+ 80x25, 8 virtual consoles \\
Serial driver version 3.96 with no serial options enabled \\
tty00 at 0x03f8 (irq = 4) is a 16450 \\
tty03 at 0x02e8 (irq = 3) is a 16550A \\
lp\_init: lp1 exists (0), using polling driver \\
\ldots
\end{tscreen}
%Here, the kernel is detecting the various hardware devices present on your
%system. At some point, you should see the line
Aqu�, el n�cleo est� detectando los distintos dispositivos hardware presentes en el sistema. En alg�n punto se deber�a ver la l�nea:
\begin{tscreen}
Partition check:
\end{tscreen}
%followed by a list of recognized partitions, for example:
seguida por una lista de las particiones reconocidas, por ejemplo:
\begin{tscreen}
Partition check: \\
\ \ hda: hda1 hda2 \\
\ \ hdb: hdb1 hdb2 hdb3
\end{tscreen}
%If, for some reason, your drives or partitions are not recognized, then
%you will not be able to access them in any way. 
Si por alguna raz�n, las unidades de disco o las particiones no se reconocen, entonces no se podr� acceder
a ellas de ninguna manera.
%There are several things that can cause this to happen:
Hay varias cosas que pueden causar que esto pase:
\begin{itemize}
\item {\bf La controladora del disco duro no est� soportada.}%Hard drive or controller not supported.} If you have a
%hard drive controller (IDE, SCSI, or otherwise) that is not supported
%by Linux, the kernel will not recognize your partitions at boot time.

Si se tiene una controladora de disco (IDE,SCSI, o lo que sea) que no tenga soporte en Linux, el
n�cleo no reconocer� las particiones al arrancar.
\index{disco duro!problemas}

\item {\bf Unidad o controladora configurada incorrectamente.}%Drive or controller improperly configured.}
%Even if your controller is supported by Linux, it may not be
%configured correctly. (This is particularly a problem for SCSI
%controllers. Most non-SCSI controllers should work fine without any
%additional configuration).

Incluso si la controladora est� soportada por Linux, quiz� no se haya configurado apropiadamente. (Este es un
problema particular para las controladoras SCSI. La mayor�a de las controladoras no SCSI deber�an funcionar bien sin
ninguna configuraci�n adicional).
%Refer to the documentation for your hard drive and/or controller. In
%particular, many hard drives need to have a jumper set to be used as a
%slave drive (the second device on either the primary or secondary IDE
%bus). The acid test of this kind of condition is to boot MS-DOS or
%some other operating system that is known to work with your drive and
%controller. If you can access the drive and controller from another
%operating system, then it is not a problem with your hardware
%configuration.
Echa un vistazo a la documentaci�n del disco duro o la controladora. En particular, 
muchos discos duros necesitan tener un jumper puesto para ser usado como unidad esclava 
(el segundo dispositivo en cualquiera del bus IDE primario o secundario)
Una prueba para esta clase de condici�n es arrancar MS-DOS o alg�n otro
sistema operativo que se sepa que funciona con la controladora y la unidad de disco.
Si se puede acceder al disco duro y la controladora desde otro sistema operativo,
entonces no es un problema de la configuraci�n de hardware.
%See P�gina~\pageref{sec-install-probs-hardware-conflicts}, above, for
%information on resolving possible device conflicts, and
%P�gina~\pageref{sec-install-probs-hardware-scsi}, below, for information
%on configuring SCSI devices.

Consulta la p�gina~\pageref{sec-install-probs-hardware-conflicts}, arriba, para
informarte acerca de la posible resoluci�n de conflictos de dispositivos, y la p�gina~\pageref{sec-install-probs-hardware-scsi} m�s abajo, para
m�s informaci�n acerca de la configuraci�n de dispositivos SCSI.

\item {\bf La controladora est� configurada apropiadamente, pero no es detectada.}%Controller properly configured, but not detected.}
%Some BIOS-less SCSI controllers require the user to specify
%information about the controller at boot time.  A description of how
%to force hardware detection for these controllers begins on
%P�gina~\pageref{sec-install-probs-hardware-scsi}.

Algunas BIOS de las controladoras SCSI requieren que el usuario especifique informaci�n
acerca de la controladora al inicio. Hay una descripci�n de c�mo
forzar la detecci�n de hardware para estas controladoras en la
p�gina~\pageref{sec-install-probs-hardware-scsi}.

\item {\bf No se reconoce la geometr�a del disco.} %Hard drive geometry not recognized.} Some systems, like
%the IBM PS/ValuePoint, do not store hard drive geometry information in
%the CMOS memory, where Linux expects to find it. Also, certain SCSI
%controllers need to be told where to find drive geometry in order for
%Linux to recognize the layout of your drive.

Algunos sistemas como los IBM PS/Valuepoint, no guardan la informaci�n de la geometr�a del disco duro en la memoria CMOS,
donde Linux espera encontrarla. Tambi�n ciertas controladoras SCSI necesitan que se las diga expl�citamente d�nde
encontrar la geometr�a de la unidad para que Linux reconozca la disposici�n del disco.

%Most distributions provide a bootup option to specify the 
%drive geometry. In general, when booting the installation
%media, you can specify the drive geometry at the LILO boot indicador de �rdenes with
%a command such as:
Muchas distribuciones proporcionan una opci�n de arranque para especificar la geometr�a del disco.
En general, cuando se arranca el medio de instalaci�n, se puede especificar la geometr�a de la unidad en
el indicador de LILO con una orden como:
\begin{tscreen}
boot: {\em linux hd=\cparam{cilindros},\cparam{cabezas},\cparam{sectores}}
\end{tscreen}
%where \cparam{cylinders}, \cparam{heads}, and \cparam{sectors} correspond
%to the number of cylinders, heads, and sectors per track for your hard
%drive. 
donde \cparam{cilindros}, \cparam{cabezas}, y \cparam{sectores} corresponden
al n�mero de cilindros, cabezas y sectores por pista del disco duro.

%After installing Linux, you will be able to install LILO, allowing you
%to boot from the hard drive. At that time, you can specify the drive
%geometry to LILO, making it unnecessary to enter the drive geometry
%each time you boot. See Chapter~\ref{chap-sysadm-num} for more
%information about LILO.

Tras instalar {\linux}, deber� instalar LILO, permiti�ndole arrancar
desde el disco duro. En este momento, se puede especificar la geometr�a de la unidad a LILO,
haciendo innecesario introducir la geometr�a del disco cada vez que arranca. Consulta el
Cap�tulo~\ref{chap-sysadm-num} para m�s informaci�n acerca de LILO.
\end{itemize}

\index{hardware!problemas con el disco duro}

% Linux Installation and Getting Started    -*- TeX -*-
% scsi.tex
% Copyright (c) 1992-1994 by Matt Welsh <mdw@sunsite.unc.edu>
%
% This file is freely redistributable, but you must preserve this copyright 
% notice on all copies, and it must be distributed only as part of "Linux 
% Installation and Getting Started". This file's use is covered by the 
% copyright for the entire document, in the file "copyright.tex".
%
% Copyright (c) 1998 by Specialized Systems Consultants Inc. 
% <ligs@ssc.com>
%Traducido por Francisco Javier Fernandez <serrador@arrakis.es>
%Revisado el 6 de julio de 2002 por Francisco Javier Fern�ndez
% Revisi�n 2 16 de julio 2002 por Francisco Javier Fernandez
%gold

\subparagraph*{Problemas con las controladoras y los dispositivos SCSI} %Problems with SCSI controllers and devices.}
\namedlabel{sec-install-probs-hardware-scsi}{}
\index{hardware!problemas con SCSI}
\index{SCSI!problemas}
%Presented here are some of the most common problems with SCSI
%controllers and devices like CD-ROMs, hard drives, and tape drives. If
%you have problems getting Linux to recognize your drive or controller,
%read on.

Aqu� se presentan algunos de los problemas m�s comunes con las controladoras SCSI
y los dispositivos como CD-ROMs, discos duros, y unidades de cinta. Si
se tiene alg�n problema con {\linux} reconociendo un disco o controladora, siga leyendo.

%The Linux SCSI HOWTO (see App�ndice~\ref{app-sources-num}) contains much useful
%information on SCSI devices in addition to that listed here. SCSI can be
%particularly tricky to configure at times.

El COMO de Linux SCSI (ver Ap�ndice~\ref{app-sources-num}) contiene mucha informaci�n
�til acerca de dispositivos SCSI en adici�n de  lo que se muestra aqu�. SCSI puede ser dif�cil de 
configurar a veces.


\begin{itemize}

\item {\bf Un dispositivo SCSI se detecta en todos los posibles IDs.}
%A SCSI device is detected at all possible IDs.} This is caused
%by strapping the device to the same address as the controller. You need to
%change the jumper settings so that the drive uses a different address than
%the controller.

Esto es causado al poner el dispositivo con el mismo identificador que la controladora. Es necesario cambiar el
ajuste del jumper para que el dispositivo use una direcci�n diferente que la controladora.


\item {\bf Linux informa de errores, incluso si se sabe que los dispositivos est�n libres de errores.} 
%Linux reports sense errors, even if the devices are known to be
%error-free.} This can be caused by bad cables or bad termination. If
%your SCSI bus is not terminated at both ends, you may have errors
%accessing SCSI devices. When in doubt, always check your cables.

Esto puede ser causado por cables defectuosos o de baja calidad o una mala terminaci�n de la cadena SCSI.
Si el bus SCSI no est� terminado a ambos extremos, se pueden producir errores accediendo a los dispositivos SCSI.
Si se tiene alguna duda, {\em siempre revise los cables}.

\item {\bf Los dispositivos SCSI informan de errores ``timeout''.}
%SCSI devices report timeout errors.} This is usually caused by 

Los errores de tiempo de espera agotado, normalmente son producidos por un conflicto con una IRQ, DMA o direcci�n de dispositivo. 
Revisa las interrupciones de la controladora, a ver si est�n en su sitio.
%a conflict with IRQ, DMA, or device addresses. Also check that interrupts
%are enabled correctly on your controller.

\item {\bf Las controladoras SCSI que usan BIOS no son detectadas.}
%SCSI controllers that use BIOS are not detected.} Detection of
%controllers that use BIOS will fail if the BIOS is disabled, or if
%your controller's signature is not recognized by the kernel. See the
%Linux SCSI HOWTO, available from the sources in
%App�ndice~\ref{app-sources-num}, for more information about this.

La detecci�n de las controladoras que usan BIOS fallar� si la BIOS est� deshabilitada, o si
la firma del controlador no la reconoce el n�cleo. Consulta el C�MO Linux SCSI, disponible
desde las fuentes de informaci�n disponibles en el Ap�ndice~\ref{app-sources-num}, para m�s informaci�n acerca de esto.

\item {\bf Las controladoras que usan  memoria de E/S mapeada no funcionan.} 
%Controllers using memory mapped I/O do not work.} This is caused
%when the memory-mapped I/O ports are incorrectly cached. Either mark the
%board's address space as uncacheable in the XCMOS settings, or disable
%cache altogether.

Esto ocurre cuando los puertos de E/S mapeados a memoria se cachean incorrectamente.
Hay dos soluciones: una es marcar el espacio de direccionamiento de la tarjeta como 
no cacheable en los ajustes de la CMOS. La segunda soluci�n es deshabilitar toda la cach�.

\item {\bf Mientras se particiona, se obtiene una advertencia tipo ``cylinders $>$ 1024'', o no se puede
iniciar desde una partici�n usando cilindros numerados por encima del 1023.}
%When partitioning, you get a warning that ``cylinders $>$ 1024'', or
%you are unable to boot from a partition using cylinders numbered above 1023.}

%BIOS limits the number of cylinders to 1024, and any partition using
%cylinders numbered above this won't be accessible from the BIOS. As far as
%Linux is concerned, this affects only booting; once the system has booted
%you should be able to access the partition. Your options are to either 
%boot Linux from a boot floppy, or boot from a partition using 
%cylinders numbered below 1024. 

La BIOS limita el numero de cilindros a 1024, y cualquier partici�n que use cilindros numerados por encima
de esto no ser� accesible por la BIOS. Esto s�lo afecta a Linux al arranque; una vez que el sistema ha arrancado
se podr� acceder a la partici�n. Las opciones son o arrancar Linux desde un disquete, o arrancar desde
una partici�n usando cilindros por debajo del 1024.

\item {\bf Al arrancar no se reconocen unidades de CD-ROM u otros dispositivos extra�bles.} %CD-ROM drive or other removeable media devices are not recognized
%at boot time.} Try booting with a CD-ROM (or disk) in the drive. This is 
%necessary for some devices. 

Intenta arrancando con un CD-ROM (o disco) en la unidad. Esto es necesario en algunos dispositivos.
\end{itemize}

%If your SCSI controller is not recognized, you may need to 
%force hardware detection at boot time. This is particularly important for
%BIOS-less SCSI controllers. Most distributions allow you to
%specify the controller IRQ and shared memory address when booting the
%installation media. For example, if you are using a TMC-8xx controller,
%you may be able to enter

Si no se reconoce su controladora SCSI, quiz� se necesite forzar la detecci�n de hardware al arrancar. Esto es particularmente importante
para las controladoras SCSI que carecen de BIOS. Muchas distribuciones permiten especificar la IRQ de la controladora y
la direcci�n de memoria compartida cuando se arranca el medio de instalaci�n. Por ejemplo, si se usa una controladora TMC-8xx,
se podr� introducir:

\begin{tscreen}
boot: linux tmx8xx=\cparam{interrupci�n},\cparam{direcci�n}
\end {tscreen}
%at the LILO boot indicador de �rdenes, where \textsl{interrupt} is the IRQ of
%controller, and \textsl{memory-address} is the shared memory
%address. Whether or not this is possible depends on the distribution
%of Linux; consult your documentation for details.
en el indicador de inicio de LILO, donde \textsl{interrupci�n} es la IRQ de la controladora, y 
\textsl{direcci�n} es la direcci�n de memoria compartida. Esto es o no posible dependiendo de la 
distribuci�n de {\linux}; consulta la documentaci�n para m�s detalles.

\index{hardware!problemas con SCSI}
\index{SCSI!problemas}


\index{hardware!problemas}
\index{instalaci�n!problemas con el hardware}

% Linux Installation and Getting Started    -*- TeX -*-
% install.tex
% Copyright (c) 1992-1994 by Matt Welsh <mdw@sunsite.unc.edu>
%
% This file is freely redistributable, but you must preserve this copyright 
% notice on all copies, and it must be distributed only as part of "Linux 
% Installation and Getting Started". This file's use is covered by the 
% copyright for the entire document, in the file "copyright.tex".
%
% Copyright (c) 1998 by Specialized Systems Consultants Inc. 
% <ligs@ssc.com>
%9 de julio de 2002 Traducci�n realizada por Francisco Javier Fern�ndez <serrador@arrakis.es>
%12 de julio de 2001 Revisi�n 1 por Francisco Javier Mart�nez <serrador@arrakis.es>
% 15 de julio revision 2 realizada por pakojavi2000
\subsection{Problemas instalando el software} %Problems installing the software.}
\namedlabel{sec-install-probs-install}

%Actually installing the Linux software should be quite trouble-free,
%if you're lucky. The only problems that you might experience would be
%related to corrupt installation media or lack of space on your Linux
%filesystems. Here is a list of these common problems.
Actualmente instalar el software {\linux} deber�a estar libre de problemas
si se tiene suerte. Los �nicos problemas que quiz� se puedan experimentar deber�an ser
los relacionados con medios de instalaci�n defectuosos o la falta de espacio en los sistemas de 
ficheros de su {\linux}. aqu� hay una lista de estos problemas comunes.

\begin{itemize}
\index{instalaci�n!errores en los medios}
%\item {\bf System reports ``{\tt read error}'' , ``{\tt file not found}'',
%or other errors while attempting to install the software.} This
%indicates a problem with the installation media. If you install from
%floppy, keep in mind that floppies are quite succeptible to media
%errors of this type. Be sure to use brand-new, newly formatted
%floppies. If you have an MS-DOS partition on your drive, many Linux
%distributions allow you to install the software from the hard
%drive. This may be faster and more reliable than using floppies.
\item  {\bf El sistema informa de  ``{\tt read error}\NT{ error de lectura}'', ``{\tt file not found}\NT{ fichero no encontrado}'', u otros errores mientras se intenta instalar el software}


Esto indica un problema con el medio de instalaci�n. Si se instala desde disquete, hay que tener
en cuenta que los disquetes son muy susceptibles de tener errores de este tipo.
Es conveniente asegurarse de utilizar un disquete nuevo de marca, reci�n formateado.
Si se tiene una particion de MS-DOS en el disco, muchas distribuciones de {\linux}
permiten la instalaci�n de software desde disco duro. Esto puede ser m�s r�pido y
seguro que usar disquetes.

%If you use a CD-ROM, be sure to check the disc for scratches, dust, or
%other problems that may cause media errors.

Si se utiliza un CD-ROM, hay que asegurarse de verificar el disco buscando
ara�azos, polvo o otros problemas que puedan causar errores en el medio.

%The cause of the problem may be that the media is in the incorrect
%format. Many Linux distributions require that the floppies be
%formatted in high-density MS-DOS format. (The boot floppy is the
%exception; it is not in MS-DOS format in most cases.) If all else
%fails, either obtain a new set of floppies, or recreate the floppies
%(using new diskettes) if you downloaded the software yourself.
La causa del problema puede ser que el medio est� en el formato incorrecto.
Muchas distribuciones de {\linux} requieren que los disquetes est�n formateados
en el formato MS-DOS de alta densidad.  El disquete de arranque es la
excepci�n; no est� en formato MS-DOS en la mayor�a de los casos. Si todo lo
dem�s falla, mejor obtener un nuevo conjunto de disquetes, o rehacer los
disquetes (utilizando disquetes {\em nuevos}) si se descarg� el software usted mismo.

% System reports errors such as ``{\tt tar: read error}'' or 
%``{\tt gzip: not in gzip format}''.}  This problem is usually caused
%by corrupt files on the installation media. In other words, your
%floppy may be error-free, but the data on the floppy is in some way
%corrupted. If you downloaded the Linux software using text mode,
%rather than binary mode, then your files will be corrupt, and
%unreadable by the installation software.
\item {\bf El sistema informa de errores como ``{\tt tar: read error}\NT{ error de lectura}'' o ``{\tt gzip: not in gzip format}\NT{ no en formato gzip}''} 


Este problema est� causado usualmente por ficheros corrompidos en el medio de instalaci�n. En otras palabras, 
los disquetes pueden estar libres de errores, pero los datos en los disquetes
est�n de alg�n modo corrompidos. Si se descarg� el software de {\linux} usando
modo texto en vez de modo binario, entonces los ficheros estar�n corrompidos, y
el programa de instalaci�n no podr� leerlos.

\item {\bf El sistema informa de errores como ``{\tt device full}\NT{ dispositivo lleno}''  mientras se instala}
%This is a clear-cut sign that you have run out of space
%when installing the software. Not all Linux distributions can pick up
%the mess cleanly; you shouldn't be able to abort the installation and expect
%the system to work.


Este es un s�ntoma claro de que se ha acabado el espacio cuando se instala el software.
No todas las distribuciones de {\linux} pueden solucionar el 
desastre limpiamente; no se podr� abortar la instalaci�n y esperar que el sistema funcione.
%The solution is usually to re-create your file systems (with {\tt
%mke2fs}) which deletes the partially installed software. You can
%attempt to re-install the software, this time selecting a smaller
%amount of software to install. In other cases, you may need to start
%completely from scratch, and rethink your partition and filesystem
%sizes.


La soluci�n pasa por volver a crear el sistema de ficheros (con {\tt
mke2fs}), que borra el software parcialmente instalado). Se puede intentar
una reinstalaci�n, seleccionando esta vez una menor cantidad de paquetes a
instalar. En otros casos, quiz� se necesite empezar completamente desde el principio,
y replantearse el particionado y los tama�os de los sistemas de ficheros.

%\item {\bf System reports errors such as ``{\tt read\_intr: 0x10}'' while
%accessing the hard drive.}  This usually indicates bad blocks on your
%drive. However, if you receive these errors while using {\tt mkswap}
%or {\tt mke2fs}, the system may be having trouble accessing your
%drive. This can either be a hardware problem (see
%P�gina~\pageref{sec-install-probs-hardware}), or it might be a case of
%poorly specified geometry. If you used the

\item {\bf El sistema informa de errores como ``{\tt read\_intr: 0x10}\NT{ lectura interrumpida}''  mientras se accede al disco duro}


Esto usualmente indica que hay bloques defectuosos en la unidad. Sin embargo,
si se reciben estos errores mientras se usa {\tt mkswap} o {\tt mke2fs}, el
sistema puede estar teniendo problemas accediendo a la unidad. Esto puede
ser o un problema de hardware (ver p�gina~\pageref{sec-install-probs-hardware}), o
puede quiz� ser un caso de una geometr�a mal especificada. Si se utiliz� la opci�n
\begin{tscreen}
hd=\cparam{cylindros},\cparam{cabezas},\cparam{sectores}
\end{tscreen}
%option at boot time to force detection of your drive geometry, and 
%incorrectly specified the geometry, you could be prone to this problem.
%This can also happen if your drive geometry is incorrectly specified in
%the system CMOS. 
en tiempo de arranque, para forzar la detecci�n de la geometr�a de la unidad,
y  se especific� incorrectamente la geometr�a, se puede ser candidato a este problema.
Esto tambi�n puede ocurrir cuando la geometr�a de la unidad est� especificada
incorrectamente en la CMOS del sistema.

%\item {\bf System reports errors like ``{\tt file not found}'' or 
%``{\tt permission denied}''.} This problem can occur if not all of the 
%necessary files are present on the installation media (see the next 
%paragraph) or if there is a permissions problem with the installation
%software. For example, some distributions of Linux have been known to
%have bugs in the installation software itself. These are usually fixed
%very rapidly, and are quite infrequent.
%If you suspect that the distribution software contains bugs, and
%you're sure that you have not done anything wrong, contact the maintainer
%of the distribution to report the bug. 
\item {\bf El sistema informa de errores como ``{\tt file not found}\NT{ fichero no encontrado}'' o ``{\tt permission denied}\NT{ permiso denegado}''} 


Este problema puede ocurrir cuando no todos los
ficheros necesarios se encuentran presentes en el medio de instalaci�n (ver el 
siguiente p�rrafo) o si hay un problema de permisos con el software de instalaci�n.
Por ejemplo, se sabe que algunas distribuciones de {\linux} tienen errores en el
programa de instalaci�n mismo. �stas normalmente se solucionan r�pidamente, y son 
muy infrecuentes.
Si se sospecha que el software de la distribuci�n contiene errores, y se est�
seguro de que no se ha hecho nada equivocadamente, contacte con el mantenedor
de la distribuci�n para informar del error.
\end{itemize}

%If you have other strange errors when installing Linux (especially if you 
%downloaded the software yourself), be sure that you actually obtained all
%of the necessary files when downloading. For example, some people use the
%FTP command 
Si se tienen otros errores extra�os cuando se instala {\linux} (especialmente si 
descarg� el software usted mismo), aseg�rese de que se obtuvo todos los ficheros necesarios
en la descarga. Por ejemplo, alguna gente utiliza la orden FTP

\begin{tscreen}
mget *.*
\end{tscreen}

%when downloading the Linux software via FTP. This will download only those
%files that contain a ``{\tt .}'' in their filenames; if there are any files
%without the ``{\tt .}'', you will miss them. The correct command to use
%in this case is
cuando descarga el software de {\linux} por FTP. Esto descargar� �nicamente aquellos
ficheros que contengan un punto ``{\tt .}'' en sus nombres de fichero; si
alg�n fichero no tiene el punto ``{\tt .}'', no se descargar�. La orden correcta
para descargarlo todo en este caso es: 

\begin{tscreen}
mget *
\end{tscreen}

%The best advice is to retrace your steps when something goes
%wrong. You may think that you have done everything correctly, when in
%fact you forgot a small but important step somewhere along the way. In
%many cases, re-downloading and re-installing the software can solve
%the problem. Don't beat your head against the wall any longer than you
%have to!
La mejor advertencia es repasar los pasos que se han dado cuando algo va mal.
Qu�z� se piense que se ha hecho todo correctamente, cuando de hecho
se olvid� un peque�o pero importante paso en alg�n lugar a lo largo del proceso.
En muchas ocasiones, volver a descargar y reinstalar el software puede 
resolver el problema. �No se d� coscorrones frente a un muro m�s de lo necesario!

%Also, if Linux unexpectedly hangs during installation, there may be a
%hardware problem of some kind. See page ~lala for hints
Adem�s, si {\linux} se cuelga sin esperarlo durante la instalaci�n, quiz� haya un
problema de hardware de alguna clase. Consulte la p�gina~\pageref{sec-install-probs-hardware}
para m�s datos.

% Linux Installation and Getting Started    -*- TeX -*-
% postinstall.tex
% Copyright (c) 1992-1994 by Matt Welsh <mdw@sunsite.unc.edu>
%
% This file is freely redistributable, but you must preserve this copyright 
% notice on all copies, and it must be distributed only as part of "Linux 
% Installation and Getting Started". This file's use is covered by the 
% copyright for the entire document, in the file "copyright.tex".
%
% Copyright (c) 1998 by Specialized Systems Consultants Inc. 
% <ligs@ssc.com>
% Traducido al espa�ol por Sebasti�n Gurin, Cancerbero <anon@adinet.com.uy>
% Revisi�n 6-7-2002 por Francisco javier Fern�ndez <serrador@arrakis.es> 
% Revisi�n 15-7-2002 por Francisco Javier fernandes <serrador@arrakis.es>
%Gold
\subsection{Problemas despu�s de instalar {\linux}}
\label{sec-install-probs-postinstall}

\index{instalaci�n!problemas en la post-instalaci�n}
Se ha pasado toda una tarde instalando {\linux}. Con el fin de dejar espacio para �ste, ha tenido que reducir sus particiones de MS-DOS y de OS/2, y con l�grimas en sus
ojos, borrar sus copias de SimCity y Wing Commander. Reinici� el sistema pero no pasa nada; o peor a�n, {\em algo\/} pasa, pero no es lo que deber�a pasar. �Qu� se debe hacer 
ahora?

En la Secci�n~\pageref{sec-install-probs-booting}, se han cubierto algunos de los problemas m�s comunes que pueden ocurrir al iniciar un sistema {\linux} nuevo 
desde los disquetes de instalaci�n --de los cu�les algunos son puestos en pr�ctica aqu�--. Adem�s de �sos, se puede ser una v�ctima de alguna de las siguientes dificultades. 


\subparagraph*{Problemas al iniciar {\linux} desde el disquete}
\index{instalaci�n!problemas en el inicio}
\index{inicio!problemas}

Si se usa un disquete para iniciar Linux, cuando el sistema arranca tal vez necesite especificar la ubicaci�n de la partici�n ra�z de {\linux}. 
Esto ser� realmente necesario, si se est� usando el disquete original de la instalaci�n, y no el disquete com�n, creado durante la misma. 

Mientras se inicia el disquete apriete las teclas \key{Shift} o \key{Ctrl}. Esto deber�a mostrarle un men� de inicio. Presione la tecla \key{Tab} para ver una lista 
con las opciones disponibles. Por ejemplo, muchas distribuciones le permitir�n escribir
\begin{tscreen}
boot: linux hd=\cparam{partici�n}
\end{tscreen}

en dicho men� de inicio, en donde \textsl{partici�n} es el nombre de la partici�n ra�z de {\linux}, por ejemplo, {\tt /dev/hda2}. Para m�s informaci�n, 
consulte la documentaci�n de su distribuci�n. 

\subparagraph*{Problemas al iniciar {\linux} desde el disco duro}
\index{instalaci�n!problemas en el inicio}
\index{inicio!problemas}

Si ha optado por instalar LILO, en lugar de crear un disquete de arranque, deber�a ser capaz de iniciar Linux desde el disco duro. De todos modos, el procedimiento de 
instalaci�n autom�tico de LILO que poseen muchas distribuciones, no siempre es perfecto. La instalaci�n pudo haber usado informaci�n incorrecta de su esquema de particiones, 
haciendo que LILO se configure de forma incorrecta. Si es as�, entonces se necesitar� reinstalar LILO para conseguir que todo funcione bien. La instalaci�n de LILO se
analiza en el Cap�tulo~\ref{chap-sysadm-num}.

\begin{itemize}
\item {\bf El Sistema informa ``{\tt Drive not bootable---Please insert system disk.}\NT{ Disco no iniciable---por favor inserte disco de sistema.}''}  

Esto es indicio de que el registro maestro de arranque del disco duro (MBR), est� mal por alguna causa. En la mayor�a de los casos, esto no es un problema da�ino y por lo
general todos los dem�s datos de su disco deber�an estar intactos. Hay varias causas para que esto haya sucedido:

\begin{enumerate}
\item Mientras estuvo reparticionando su disco usando {\tt fdisk}, se pudo haber borrado la partici�n que estaba marcada como ``activa''. MS-DOS y otros sistemas operativos 
intentan siempre arrancar la partici�n ``activa'' cuando se inicia el sistema, (a Linux le da igual si la partici�n es ``activa'' o no lo es). Una de las maneras para 
solucionar este problema, puede ser iniciar MS-DOS desde un disquete y ejecutar {\tt FDISK.EXE}, para establecer como activa, la partici�n de MS-DOS. 

Tambi�n podr� tratar con la siguiente orden (en versiones de MS-DOS 5.0 o posteriores) :
\begin{tscreen}
FDISK /MBR
\end{tscreen}
Esta orden intentar� reconstruir el registro maestro de arranque del disco duro para que se inicie MS-DOS, borrando a LILO. Si usted no posee MS-DOS, entonces necesitar� 
arrancar {\linux} desde un disquete e intentar instalar LILO nuevamente. 

\item Es muy probable que la causa de este error tenga que ver con el hecho de haber creado una partici�n MS-DOS usando la versi�n {\tt fdisk} de {\linux}, o viceversa.
Se deber�an crear particiones MS-DOS solamente utilizando la versi�n de {\tt FDISK.EXE} de MS-DOS. (Esto se aplica tambi�n para otros sistemas operativos distintos de MS-DOS).
La mejor soluci�n es, empezar desde cero y reparticionar el disco correctamente, o simplemente borrar y crear nuevamente las particiones corruptas, siempre con la versi�n 
correcta de  {\tt fdisk}.

\index{installaci�n!LILO}
\index{LILO!problemas en la instalaci�n}
\item Pudo haber sucedido que el proceso de instalaci�n de LILO haya fallado. Si fuera as�, usted deber�a iniciar {\linux} usando el disquete de arranque (si es que posee 
uno), o desde el medio de instalaci�n original. En cualquiera de los dos casos, usted deber�a proporcionar opciones para especificar la partici�n ra�z que {\linux} debe usar
para arrancar. Presione las teclas \key{Shift} or \key{Ctrl} al arrancar el sistema, y en el men� presione \key{Tab} para ver las opciones disponibles. 

% Si todo lo dem�s falla, puede, sencillamente iniciar {\linux} desde el disquete 
% hasta que tenga el sistema configurado de tal manera que sea posible intentar reinstalar LILO. 

\end{enumerate}

\item {\bf Cuando el sistema arranca desde el disco duro, se inicia MS-DOS (u otro sistema operativo) en vez de {\linux}} 

Antes que nada, aseg�rese de que realmente ha 
instalado LILO durante el proceso de instalaci�n de {\linux}. Si no es as�, entonces el sistema seguir� arrancando MS-DOS (o el sistema operativo que usted posea), 
cuando se intente cargar desde el disco duro. Para que {\linux} pueda iniciarse desde el disco duro, usted necesitar� instalar LILO (see Chapter~\ref{chap-sysadm-num}).

Si por otro lado, {\em ha} instalado LILO, pero arranca otro sistema operativo en lugar de {\linux}, puede ser que se tenga configurado LILO para que arranque predeterminadamente ese sistema
operativo. Para que LILO se presente en el arranque, apriete las teclas \key{Shift} o \key{Ctrl}, y en el indicador apriete \key{Tab}. Esto deber�a mostrarle una
lista con los sistemas operativos disponibles. Elija la opci�n apropiada (usualmente ``{\tt linux}'') para iniciar {\linux}. 

Si desea tener a {\linux} como el sistema operativo de arranque por omisi�n, tendr� que reinstalar LILO. Vea el cap�tulo~\ref{chap-sysadm-num}.

Asimismo, puede ser posible que usted haya intentado instalar LILO, pero que el procedimiento de instalaci�n haya fallado de alguna manera. Vea el asunto anterior. 

\end{itemize}


\subparagraph*{Problemas al registrarse en  {\linux}}
\index{instalaci�n!problemas present�ndose}
\index{presentacion!problemas}


Despu�s de iniciar Linux, ser� presentado ante un indicador de �rdenes de registro, {\tt login}, similar a 
\begin{tscreen}
linux login:
\end{tscreen}
En este punto, tanto la documentaci�n de su distribuci�n como el sistema, le dir�n qu� debe hacerse. En muchas distribuciones,  simplemente tendr� que registrarse como {\tt root},
sin ninguna contrase�a. Otros nombres de usuario posibles son {\tt guest} o {\tt test}.

Generalmente, un sistema {\linux} reci�n instalado, no nos deber�a pedir una contrase�a en el indicador de �rdenes de {\tt login} inicial. De cualquier modo, si el sistema le pide una, 
puede que haya un problema con su nuevo sistema. Antes que nada, trate de usar una contrase�a equivalente al nombre de usuario; por ejemplo, si usted ha utilizado {\tt root}
para ingresar, use ``{\tt root}'' como contrase�a. 

Si sencillamente usted no puede entrar, es muy posible que exista un problema. Consulte la documentaci�n de su distribuci�n: el nombre de usuario y la contrase�a para entrar
pueden estar perdidos por ah�. Tambi�n pudo haber sucedido, que durante el proceso de instalaci�n, el sistema le haya mostrado el nombre de usuario y la contrase�a para 
ingresar por primera vez, o que ambos datos est�n justo delante suyo, en el indicador de {\tt login}. 

Otra de las causas de esta dificultad puede ser un problema con la instalaci�n del programa {\tt login} de {\linux} y sus ficheros de inicio. Si es as�, necesitar�
reinstalar (por lo menos algunas partes de) {\linux}, o arrancar el programa de instalaci�n e intentar resolver el problema a mano---vea el Cap�tulor~\ref{chap-sysadm} para 
algunas sugerencias. 


\subparagraph*{Problemas al usar el sistema}
Si se ha logrado ingresar exitosamente al sistema, podr� ver un indicador del int�rprete de �rdenes (shell), (como por ejemplo ``{\tt \#}'' o ``{\tt \$}''), y podr� 
felizmente, deambular por su nuevo sistema. De todos modos, hay algunos problemas iniciales, que a veces pueden complicarnos la vida. 

El m�s com�n de estos inconvenientes es el permiso incorrecto que se tiene sobre algunos ficheros o directorios. Esto puede ser la causa del mensaje de error
\begin{tscreen}
Shell-init: permission denied
\end{tscreen}
que se nos presente despu�s de ingresar al sistema, (de hecho, cuando veamos cualquier mensaje que diga algo como ``{\tt permission denied}''\NT{permiso denegado}, podremos estar seguros de que se
trata de un problema con los permisos de los ficheros)

\index{permisos!problemas}
\index{instalaci�n!permisos de ficheros}
En casi todos los casos, se puede resolver este problema simplemente usando {\tt chmod}, para ajustar los permisos de los ficheros o directorios adecuados. Por ejemplo, en 
algunas distribuciones de {\linux} se ha usado (incorrectamente) el modo de fichero \NT{ file mode} 0644 para el directorio ra�z ({\tt /}). El arreglo para esto fue introducir 
la orden


\begin{tscreen}
\# chmod 755 /
\end{tscreen}


como {\tt root}. Sin embargo, para poder ejecutar dicha orden, se necesitar� arrancar el sistema desde el disquete de instalaci�n, y montar el sistema de ficheros ra�z de 
{\linux} a mano ---lo que puede ser una tarea algo dura para los m�s principiantes.

Conforme utilice el sistema, podr� encontrar lugares de su sistema, en donde los permisos para los ficheros y directorios sean incorrectos, o programas que no se ejecuten como
se los configur�. �Bienvenido al mundo de {\linux}!  Mientras que la mayor�a de las distribuciones pr�cticamente no dan problemas, muy pocas de ellas son perfectas. 
No queremos cubrir aqu� todos estos problemas. En su lugar, le ayudaremos a resolver la mayor�a de estos obst�culos de configuraci�n a lo largo del libro, ense��ndole c�mo
encontrarlos y repararlos por usted mismo. En el Cap�tulo~\ref{chap-intro-num}, se ha comentado parte de esta filosof�a. En el Cap�tulo~\ref{chap-sysadm-num}, se dar�n algunas
pistas para arreglar muchos de estos frecuentes problemas de configuraci�n. 

\index{instalaci�n!problemas en la postinstalaci�n}



\index{instalaci�n!problemas|)}

