%%%%%%%%%%%%%%%%%%%%%%%%%%%%%%%%%%%%%%%%%%%%%%%%%%%%%%%%%%%%%%%%%%
%\begin{document}% Linux Installation and Getting Started    -*- TeX -*-
% chap-intro.tex
% Copyright (c) 1992-1994 by Matt Welsh <mdw@sunsite.unc.edu>
%
% This file is freely redistributable, but you must preserve this copyright 
% notice on all copies, and it must be distributed only as part of "Linux 
% Installation and Getting Started". This file's use is covered by the 
% copyright for the entire document, in the file "copyright.tex".
%
% Copyright (c) 1998 by Specialized Systems Consultants Inc. 
% <ligs@ssc.com>


\chapter{Introducci�n a {\linux}}
\markboth{Introducci�n a {\linux}}{}
\pagenumbering{arabic}
\namedlabel{chap-intro}{Introducci�n a {\linux}}
\label{chap-intro-num}
\index{{\linux}|(} % A little joke

Muy posiblemente {\linux} sea el logro m�s importante del Software Libre 
desde el Space War original, o m�s recientemente, GNU/Emacs. {\linux} se ha convertido
en un sistema operativo para las empresas, la educaci�n y la productividad
personal. {\linux} ya ha dejado de ser s�lo para aquellos expertos de Unix 
que se sientan durante horas ante una consola parpadeante, si bien 
podemos asegurarle de buena tinta que muchos usuarios entran dentro de tal
 categor�a. Este libro le ayudar� a sacarle el mayor partido a su {\linux}.

{\linux} (pronunciado con una {\em i} breve como en L�nux) es un sistema operativo, cl�nico de Unix,
que se ejecuta en varias plataformas, principalmente en PCs (computadoras personales)
con un procesador Intel 80386 o superior. Soporta una amplia variedad de programas,
desde el sistema de procesado de documentos \TeX hasta el sistema X-Window, 
pasando por gcc (compilador GNU de C/C++) y TCP/IP\@. Es una implementaci�n de Unix
vers�til y {\em bona fide}, libremente redistribuible en los t�rminos
de la Licencia P�blica General GPL (v�ase Ap�ndice~\ref{app-gpl-num})

{\linux} puede convertir un computador personal 80386 o  superior 
en una estaci�n de trabajo que pone al alcance de su mano
toda la potencia de Unix. Las empresas instalan {\linux} en redes enteras
de m�quinas, y utilizan este sistema operativo para gestionar registros 
financieros y hospitalarios, entornos inform�ticos distribuidos y
telecomunicaciones. Las Universidades del mundo entero usan {\linux} para 
impartir cursos de programaci�n y dise�o de sistemas operativos. Los aficionados
a la inform�tica de todo el mundo usan {\linux} en su casa para 
programar, para productividad personal y para el hackeo sano en general.

El que sea una implementaci�n libre de Unix es lo que hace a {\linux} tan 
diferente. Se desarroll� y sigue desarroll�ndose de forma cooperativa, 
principalmente a trav�s de Internet, por parte de un grupo de voluntarios que
intercambian c�digo fuente, informan de los errores y solucionan los 
problemas en un entorno abierto. Cualquiera es bienvenido a sumarse al
esfuerzo de desarrollar {\linux}. Todo lo que se necesita es inter�s en hackear
un cl�nico libre de Unix, y ciertos conocimientos de programaci�n. El libro
que tiene en las manos es una gu�a para ese viaje.

% {\linux} Installation and Getting Started    -*- TeX -*-
% about.tex
% Copyright (c) 1992-1994 by Matt Welsh <mdw@sunsite.unc.edu>
%
% This file is freely redistributable, but you must preserve this copyright 
% notice on all copies, and it must be distributed only as part of "{\linux} 
% Installation and Getting Started". This file's use is covered by the 
% copyright for the entire document, in the file "copyright.tex".
%
% Copyright (c) 1998 by Specialized Systems Consultants Inc. 
% <ligs@ssc.com>
%Revisi�n 1 por Francisco Javier Fernande <serrador@arrakis.es>
%Gold


\section{Acerca de este libro}
\markboth{Introducci�n a {\linux}}{Acerca de este libro}

Este libro es una gu�a para la instalaci�n y el manejo b�sico de {\linux}.
Su finalidad es conseguir que los nuevos usuarios se pongan en marcha 
reuniendo la mayor cantidad posible de contenido relevante en un s�lo libro. En 
lugar de cubrir f�tiles detalles t�cnicos que tienden a cambiar debido al 
r�pido desarrollo de {\linux}, le damos unos conocimientos de base claros y 
simples, que le permitan seguir avanzando por s� mismo.

Instalar y utilizar {\linux} no es dif�cil. Sin embargo, al igual que con
cualquier otra implementaci�n de Unix, para ponerlo todo a funcionar a 
menudo hace falta algo de mano izquierda. Esperamos que este libro le suba al 
tren de {\linux} y le muestre lo grande que puede llegar a ser un sistema 
operativo.

En este libro cubrimos los siguientes asuntos:
\begin{itemize}
\item �Qu� es {\linux}? El dise�o y filosof�a de este sistema operativo �nico,
      y lo que {\linux} puede hacer por usted.

\item Detalles sobre c�mo ejecutar {\linux}, adem�s de sugerencias sobre la 
      configuraci�n de hardware recomendada.


%\item How to obtain and install {\linux}. There are many distributions of the 
%{\linux} software. We present a general discussion of {\linux} software
%distributions, how to obtain them, and general instructions for installing
%the software (which should apply to any distribution).

%\ifodd\igsslack
\item Instrucciones espec�ficas para instalar diversas distribuciones de {\linux},
      incluyendo Debian, Red Hat y Slackware.
%\fi

\item Un breve tutorial de introducci�n a Unix para usuarios sin 
      experiencia previa en Unix. Este tutorial proporciona suficiente material
      a los novatos como para que puedan moverse solos por el sistema.


\item Una introducci�n a la administraci�n del sistema bajo {\linux}. Se cubren
      las tareas m�s importantes que los admistradores de  {\linux} necesitan
      llevar a cabo como la creaci�n de cuentas de usuario o el manejo de 
      los ficheros del sistema.

\item Informaci�n sobre la configuraci�n de las caracter�sticas avanzadas de 
      {\linux}, como el sistema X-Window, la red TCP/IP, las noticias
      o el correo electr�nico.

\end{itemize}

Este libro est� dirigido al usuario de ordenadores personales que desee
empezar con {\linux}. No damos por sentado que se tenga experiencia previa con 
Unix pero s� esperamos que, durante su aprendizaje, el principiante se 
remita a otras fuentes de informaci�n. Proporcionamos una lista de 
referencias �tiles en el Ap�ndice~\ref{app-sources-num} para quienes no 
est�n familiarizados con Unix.
En general, se supone que este libro ha de leerse junto con otro libro de 
conceptos b�sicos de Unix.


% Linux Installation and Getting Started    -*- TeX -*-
% history.tex
% Copyright (c) 1992-1994 by Matt Welsh <mdw@sunsite.unc.edu>
%
% This file is freely redistributable, but you must preserve this copyright 
% notice on all copies, and it must be distributed only as part of "Linux 
% Installation and Getting Started". This file's use is covered by the 
% copyright for the entire document, in the file "copyright.tex".
%
% Copyright (c) 1998 by Specialized Systems Consultants Inc. 
% <ligs@ssc.com>


\section{Breve historia de {\linux}}
\index{{\linux}!historia}
\index{origen}
\markboth{Introducci�n a {\linux}}{Breve historia de {\linux}}
UNIX es uno de los sistemas operativos mundialmente m�s famosos a causa de su
amplia distribuici�n y base soportada. Se desarroll� originalmente en la
AT\&T como sistema multitarea para miniordenadores y mainframes en la d�cada de
los 70, pero desde entonces ha crecido hasta convertirse en uno de los sistemas m�s
ampliamente usados por doquier, a pesar de su interfaz, a veces confuso, 
y de su falta de estandarizaci�n por parte de una entidad centralizadora.

Muchos hackers sienten que Unix es Lo Que Vale la Pena, el �nico sistema operativo
de verdad. De ah� proviene el desarrollo de {\linux} por parte de un grupo, siempre
en aumento, de hackers del Unix que quieren ``curr�rselo'' con un sistema que puedan 
llamar propio.

Existen versiones de Unix para muchos sistemas, desde ordenadores personales
hasta superordenadores como el Cray Y-MP\@. La mayor parte de las versiones de Unix
para ordenadores personales son caras y dif�ciles. Al escribir esto, una
versi�n de UNIX System V para una sola m�quina 386 cuesta unos 1500 \$ USA.

{\linux} es una versi�n libre de Unix desarrollada originalmente por Linus
Torvalds en la Universidad de Helsinki en Finlandia, con la ayuda, a trav�s de 
Internet, de numerosos programadores y expertos en Unix.
Cualquiera que tenga el instinto y los conocimientos suficientes
puede desarrollar y modificar el sistema. El n�cleo de {\linux} no 
utiliza c�digo de AT\&T o de cualquier otra fuente propietaria,
y gran parte del software disponible para {\linux} ha sido desarrollado
por el proyecto GNU de la Free Software Foundation en Cambridge, 
Massachusetts (EEUU) No obstante, tambi�n programadores de todo el mundo
contribuyen a que aumente cada vez m�s el software disponible para {\linux}.

Linux, el n�cleo o kernel de GNU/Linux, se desarroll� originalmente como un 
proyecto que Linus Torvalds emprendi� en su tiempo libre. 
Se inspir� en Minix, un sistema Unix b�sico desarrollado por 
Andy Tanenbaum. Las primeras discusiones acerca del n�cleo Linux tuvieron 
lugar en el grupo de noticias de Usenet {\tt comp.os.minix}.
Estas discusiones se centraban sobre todo en el desarrollo de un sistema 
peque�o y acad�mico de Unix para usuarios de Minix que quer�an algo m�s.


El primitivo desarrollo del n�cleo Linux se centr� en las caracter�sticas multitarea
del interfaz en modo protegido del 80386, escrito en c�digo ensamblador.
Linus escribe:
\begin{quotation}
``Despu�s de todo, ha sido una navegaci�n tranquila; el c�digo era tremendo,
pero ten�a algunos dispositivos, y la depuraci�n fue m�s f�cil. En esta etapa 
comenc� a usar C, y ciertamente acelera el desarrollo. Tambi�n fue entonces
cuando me empec� a poner serio en mi megaloman�aca idea de hacer `un Minix
mejor que Minix'. Esperaba poder recompilar {\tt gcc} bajo el n�cleo Linux alg�n d�a\ldots''.

``Dos meses para la configuraci�n b�sica, pero luego s�lo un poco m�s 
hasta que tuve un controlador de disco (gravemente plagado de errores,
pero result� que funcionaba en mi ordenador) y un peque�o sistema de ficheros.
Fue por aquel entonces cuando dej� disponible la versi�n 0.01
(m�s o menos a finales de agosto de 1991): no era bonito, no ten�a controlador
de disquetera, y no pod�a hacer mucho en ning�n sentido. No creo siquiera que nadie
compilara jam�s esa versi�n. Pero para entonces ya estaba enganchado, y no quer�a 
parar hasta conseguir dejar fuera a Minix''.
\end{quotation}

Nunca se hizo un anuncio de la versi�n 0.01. Las fuentes del 0.01 ni
siquiera eran ejecutables. Conten�an s�lo los rudimentos b�sicos de
las fuentes del n�cleo y daban por supuesto que se ten�a acceso a una 
m�quina con Minix para compilarlas y experimentar con ellas.

El 5 de octubre de 1991 Linus anunci� la primera versi�n ``oficial''
del n�cleo Linux, la versi�n 0.02. En este punto, Linus pod�a ejecutar {\tt bash}
(el GNU Bourne Again Shell) y {\tt gcc} (el compilador C GNU) pero no mucho
m�s. De nuevo, estaba pensado como un sistema para hackers. La orientaci�n
principal fue el desarrollo del n�cleo; el soporte de usuarios, la documentaci�n
y la distribuci�n todav�a no hab�an sido atendidos. A�n hoy, la comunidad
{\linux} parece que a�n trata estas cosas como secundarias frente a la
``programaci�n de verdad'' (el desarrollo del n�cleo).

Seg�n escribi� Linus en {\tt comp.os.minix},
\begin{quotation}
``�Suspiras por los fabulosos d�as de Minix-1.1, cuando los hombres eran 
hombres y escrib�an sus propios controladores de dispositivo? �Te encuentras 
sin un buen proyecto y te mueres por hincar los dientes a un sistema operativo 
que puedas intentar modificar para tus necesidades? Encuentras frustrante que 
todo en Minix funcione? �Se acabaron las amanecidas para conseguir que funcione 
ese programa ca�ero? Entonces este mensaje puede que sea para ti.

``Tal y como mencion� hace un mes, estoy trabajando en una versi�n libre
de una especie de Minix para ordenadores AT-386.  Por fin ha alcanzado el
estado en el que incluso se puede usar (aunque a lo mejor no se puede, depende
de para qu� lo quieras), y deseo dejar el c�digo fuente libre para que alcance
mayor distribuci�n. S�lo es la versi�n 0.02\ldots, pero ya he ejecutado con �xito
{\tt bash}, {\tt gcc}, {\tt gnu-make}, {\tt gnu-sed}, {\tt compress}, 
etc�tera, bajo este sistema''.
\end{quotation}

Despu�s de la versi�n 0.03 Linus dio el salto a la versi�n 0.10, seg�n
empez� a trabajar m�s gente en el sistema. Despu�s de varias revisiones
posteriores, Linus increment� el n�mero de versi�n a la 0.95 en marzo
de 1992 para reflejar su impresi�n de que el sistema estaba preparado para
un inminente lanzamiento ``oficial''. (Generalmente a un programas no se le
numera con la versi�n 1.0 hasta que no est� en teor�a completo, o libre de 
errores). Casi a�o y medio despu�s, a finales de diciembre de 1993, el n�cleo
de {\linux} estaba todav�a en la versi�n 0.99.p114, acerc�ndose asint�ticamente
a la versi�n 1.0. En el momento de escribir esto, la versi�n estable actual es la
2.2.10 y est� en desarrollo la versi�n 2.3.

Casi todos los paquetes de programas UNIX importantes libremente redistribuibles
han sido portados a {\linux}, y tambi�n hay abundante software comercial. La lista
de hardware soportado es mayor que la del n�cleo original. Mucha gente ha 
ejecutado benchmarks (pruebas de rendimiento) en sistemas Linux 80486 y han 
encontrado que es comparable a estaciones de trabajo de Sun Microsystems y Digital
Equipment Corporation. �Qui�n hubiera adivinado que este ``peque�o'' cl�nico de
UNIX iba a crecer tanto como para dominar el mundo de la computaci�n personal en
su totalidad?

%%% Local Variables: 
%%% mode: latex
%%% TeX-master: t
%%% End: 

% Linux Installation and Getting Started    -*- TeX -*-
% features.tex
% Copyright (c) 1992-1994 by Matt Welsh <mdw@sunsite.unc.edu>
%
% This file is freely redistributable, but you must preserve this copyright 
% notice on all copies, and it must be distributed only as part of ``Linux 
% Installation and Getting Started''. This file's use is covered by the 
% copyright for the entire document, in the file ``copyright.tex''.
%
% Copyright (c) 1998 by Specialized Systems Consultants Inc. 
% <ligs@ssc.com>
%Revisi�n 1 por Francisco javiermartine <serrador@arrakis.es>
%Revisi�n 2 por FJFS <serrador@arrakis.es>
\section{Caracter�sticas del sistema}
\markboth{Introducci�n a {\linux}}{Caracter�sticas del sistema}

{\linux} soporta caracter�sticas que tambi�n se encuentran en otras implementaciones
de UNIX, y otras muchas que no se encuentran en ninguna otra. En esta
secci�n, daremos un r�pido paseo por las caracter�sticas del n�cleo
de {\linux}.

{\linux} es un sistema operativo de multitarea real y multiusuario, como
lo son todas las otras versiones de UNIX. Esto significa que muchos usuarios
pueden autentificarse en el sistema y ejecutar programas, y hacerlo de 
forma simult�nea.

El sistema {\linux} es en su mayor�a compatible con varios est�ndares de UNIX
(hasta donde pueda tener est�ndares el UNIX) en lo que respecta al c�digo
fuente de los programas, entre ellos los est�ndares POSIX.1, UNIX System V, 
and Berkely System Distribution UNIX.  {\linux} se ha desarrollado con la idea
de que el c�digo fuente sea portable de un sistema a otro, y as� es
f�cil encontrar caracter�sticas de uso general que son compartidas por m�s de
una plataforma. Gran parte del software para UNIX disponible en Internet
y en otros lugares compila para {\linux} sin hacer modificaciones.
Adem�s, es libremente redistribuible todo el c�digo fuente del sistema {\linux}, 
a saber, el n�cleo, los controladores de dispositivo, las bibliotecas, 
los programas de usuario y las herramientas de desarrollo.

Otros rasgos internos espec�ficos de {\linux} incluyen control de tareas
POSIX (que utilizan int�rpretes de �rdenes como {\tt chs} y {\tt bash},
pseudoterminales (dispositivos tty), y soporte para teclados nacionales o
personalizados que se cargan din�micamente. {\linux} soporta {\tt consolas
virtuales} que le permiten cambiar entre sesiones de login en una �nica
consola del sistema. Los usuarios del programa {\tt screen} encontrar�n
familiar la implementaci�n de la consola virtual de {\linux}.

El kernel puede emular instrucciones del coprocesador 387; los sistemas
sin un coprocesador matem�tico pueden ejecutar programas que requieren
capacidades matem�ticas de coma flotante.

El sistema operativo  soporta varios sistemas de ficheros para almacenar 
los datos, como el sistema de ficheros ext2, dise�ado espec�ficamente para {\linux}.
Hay soporte para los sistemas de ficheros de Xenix y UNIX System V, as� 
como los sistemas de ficheros de MS-DOS y el VFAT de Windows 98, en disco
duro y en disquete.  El sistema de ficheros de CD-ROM ISO 9660 tambi�n 
est� soportado. Hablaremos m�s acerca de los sistemas de ficheros en los 
cap�tulos~\ref{chap-install-num} y~\ref{chap-sysadm-num}.

{\linux} proporciona una implementaci�n completa del software de redes
TCP/IP. Incluidos controladores de dispositivo para muchas tarjetas
Ethernet habituales, y tambi�n SLIP (Serial Line Internet Protocol) 
y PPP (Point-to-Point Protocol), que proporcionan acceso a una red TCP/IP
a trav�s de una conexi�n serie, PLIP (Parallel Line Internet Protocol), y NFS
(Network File System - Sistema de Ficheros de Red).
Tambi�n est� soportada toda la gama de clientes y servicios TCP/IP, lo que
incluye FTP, {\tt telnet}, NNTP y SMTP. Hablaremos m�s acerca del trabajo
en red en el cap�tulo~\ref{chap-networking}.

El n�cleo de {\linux} se ha desarrollado para utilizar las caracter�sticas
del modo protegido del procesador 80386 o superior. En particular, {\linux}
usa el paradigma de manejo de la memoria basado en descriptores y en 
modo protegido. Cualquiera que est� familiarizado con el modo protegido
del 386 sabe que fue dise�ado para sistemas multitarea como el Unix. {\linux}
explota esta funcionalidad.


El n�cleo soporta ejecutables con paginaci�n por demanda: s�lo aquellos
segmentos de un programa que realmente se utilizan se pasan a la memoria
desde el disco. Igualmente, se comparten las p�ginas de memoria de los
ejecutables mediante la t�cnica {\it copy-on-write}. Si varias copias de un
programa se est�n ejecutando a la vez, comparten la memoria f�sica, lo cual
reduce su uso global.


Para conseguir aumentar la cantidad total de memoria disponible, {\linux} implementa
tambi�n la paginaci�n de disco. Puede reservarse en el disco hasta un Gigabyte 
de {\bf espacio de intercambio}\footnote{El espacio de intercambio no tendr�a que
llamarse as�; no se mandan al espacio de intercambio procesos enteros, sino m�s bien
determinadas p�ginas. Claro est� que en la mayor�a de los casos ir�n al fichero de
intercambio procesos enteros, pero esto no siempre es cierto} en hasta 8 particiones
de 128 megas cada una). Cuando el sistema requiere m�s memoria f�sica,
manda al fichero de intercambio las aplicaciones inactivas, permiti�ndole ejecutar
aplicaciones m�s grandes y dar servicio a otros usuarios. Aun as�, 
el intercambio de p�ginas al disco no sustituye a la memoria RAM, que 
es mucho m�s r�pida.

El n�cleo de {\linux} implementa tambi�n una unificaci�n de la memoria f�sica
y de la memoria de intercambio en el disco. Toda la memoria que quede libre
es usada para intercambio, y se reduce al ejecutar programas grandes.

Los ejecutables usan bibliotecas compartidas; esto significa 
que los ejecutables comparten el c�digo com�n de las
bibliotecas en un �nico fichero, como sucede en SunOS.
Los ficheros ejecutables ocupan menos espacio en disco, especialmente
cuando usan funciones de muchas bibliotecas distintas. 
Tambi�n existen bibliotecas enlazadas est�ticamente  para el depurado
de objetos y para mantener ficheros ejecutables ``completos'' cuando
las bibliotecas din�micas no est�n instaladas.
Las bibliotecas se enlazan din�micamente en tiempo de ejecuci�n, y el 
programador puede usar sus propias rutinas en lugar de las rutinas de
la biblioteca est�ndar.

Para facilitar la depuraci�n de programas, el n�cleo genera volcados
de memoria {\tt core dump} para el an�lisis post-mortem cuando una aplicaci�n
falla. Mediante los {\tt core dump} y un ejecutable enlazado con soporte de
depuraci�n, los programadores pueden determinar la causa de que el programa
haya fallado.


%%% Local Variables: 
%%% mode: latex
%%% TeX-master: t
%%% End: 

% Linux Installation and Getting Started    -*- TeX -*-
% software.tex
% Copyright (c) 1992-1994 by Matt Welsh <mdw@sunsite.unc.edu>
%
% This file is freely redistributable, but you must preserve this copyright 
% notice on all copies, and it must be distributed only as part of "Linux 
% Installation and Getting Started". This file's use is covered by the 
% copyright for the entire document, in the file "copyright.tex".
%
% Copyright (c) 1998 by Specialized Systems Consultants Inc. 
% <ligs@ssc.com>
% Traducido por Ivan Juanes <ivanjuanes@geocities.com>

\section{Programas}
\markboth{Introducci�n a {\linux}}{Programas}


% general.tex
% Copyright (c) 1992-1994 by Matt Welsh <mdw@sunsite.unc.edu>
%
% This file is freely redistributable, but you must preserve this copyright 
% notice on all copies, and it must be distributed only as part of "{\linux} 
% Installation and Getting Started". This file's use is covered by the 
% copyright for the entire document, in the file "copyright.tex".
%
% Copyright (c) 1998 by Specialized Systems Consultants Inc. 
% <ligs@ssc.com>
% Revisi�n 1 para LIPP2 por Francisco Javier Fern�ndez

Pr�cticamente ha sido portada a {\linux} cualquier utilidad que pudiera 
encontrarse en un sistema UNIX est�ndar, entre ellas las �rdenes b�sicas
como {\tt ls}, {\tt awk}, {\tt tr}, {\tt sed}, {\tt bc} y {\tt more}.
El entorno de trabajo familiar en otros sistemas UNIX se ha replicado
en {\linux}. Se incluyen todas las �rdenes y utilidades (Los nuevos
usuarios de UNIX o {\linux} deben ver el Cap�tulo~\ref{chap-tutorial} para una
introducci�n a las �rdenes b�sicas de UNIX)

Est�n disponibles muchos editores de texto, entre ellos {\tt vi}, {\tt ex}, 
{\tt pico}, {\tt jove} y  {\tt GNU Emacs}, y variantes como Lucid {\tt
emacs},  que incorpora extensiones para el sistema X-Window, y {\tt joe}.  
Es muy posible que el editor de texto al que este acostumbrado haya sido
portado a {\linux}.

Es interesante la cuesti�n de elegir un editor de texto. Muchos usuarios
de UNIX prefieren editores ``sencillos'' como {\tt vi}; el autor original
escribi� este libro con {\tt vi}. Pero {\tt vi} tiene muchas limitaciones
debido a su antig�edad, y los modernos editores como {\tt emacs}
han ganado popularidad. {\tt GNU Emacs} soporta un completo lenguaje e int�rprete
de macros basado en Lisp, una potente sintaxis de �rdenes y otras 
extensiones. Hay paquetes de macros para {\tt emacs} que te permiten leer
correo electr�nico y noticias, editar el contenido de directorios, e
incluso atreverse con sesiones de psicoterapia de inteligencia artificial.
(indispensables para hackers del {\linux} muy estresados).

Muchas de las utilidades b�sicas de {\linux} son software GNU. Las utilidades
GNU soportan caracter�sticas avanzadas que no se encuentran en las versiones
est�ndar de los programas de BSD y System V. Por ejemplo el clon GNU de
{\tt vi}, llamado {\tt elvis}, incluye un lenguaje estructurado de macros
que difiere de la interpretaci�n original. Sin embargo, se pretende que las
utilidades GNU permanezcan compatibles con sus hom�logos de BSD y System V.
Mucha gente considera que las versiones GNU son superiores a los originales.

Una {\bf shell} o int�rprete de �rdenes es un programa que lee y ejecuta �rdenes
del usuario. Adem�s muchas shells proporcionan caracter�sticas como el 
{\bf control de tareas}, manejo de varias tareas a la vez, redirecci�n de las
entradas y salidas, y un lenguaje de �rdenes para escribir {\bf shell scripts},
un gui�n de �rdenes.
Un gui�n de �rdenes es un programa escrito en el lenguaje de �rdenes de la shell,
an�logo a un fichero .bat del DOS. 

Est�n disponibles para {\linux} muchos tipos de shells. La diferencia m�s
importante entre las shells es el lenguaje de �rdenes. Por ejemplo, el
C SHell ({\tt csh}) utiliza un lenguaje de �rdenes similar al lenguaje
de programaci�n C. El cl�sico Bourne SHell {\tt sh} usa un lenguaje de 
�rdenes diferente. La elecci�n de una shell se basa a menudo en el lenguaje
de �rdenes que proporciona, y determina en gran medida la calidad de tu
entorno de trabajo en {\linux}.

La Bourne Again Shell GNU ({\tt bash}) es una variante de la Bourne
Shell que incluye muchas caracter�sticas avanzadas como el control de 
tareas,  el historial de �rdenes, conclusi�n de �rdenes y nombres de ficheros, 
un interface tipo  {\tt emacs} para editar l�neas de �rdenes y otras poderosas
extensiones al lenguaje est�ndar de la shell Bourne est�ndard. Otra shell popular
es {\tt tcsh}, una versi�n de la C Shell con funciones avanzadas similares a las
que encontramos en {\tt bash}. Otras shells son {\tt zsh}, una shell peque�a similar
a la Bourne shell; Korn Shel; la {\tt ash} del BSD y {\tt rc} Shell de Plan 9.

Si es la �nica persona que va a usar el sistema y planea usar s�lo {\tt vi}
y {\tt bash} como editor y shell, respectivamente, no hay raz�n para instalar
otros editores o shells. Esta actitud de ``h�galo usted mismo'' es la t�nica general
entre los usuarios y hackers de {\linux}.


%%% Local Variables: 
%%% mode: plain-tex
%%% TeX-master: t
%%% End: 

% {\linux} Installation and Getting Started    -*- TeX -*-
% text.tex
% Copyright (c) 1992-1994 by Matt Welsh <mdw@sunsite.unc.edu>
%
% This file is freely redistributable, but you must preserve this copyright 
% notice on all copies, and it must be distributed only as part of ``{\linux} 
% Installation and Getting Started''. This file's use is covered by the 
% copyright for the entire document, in the file ``copyright.tex''.
%
% Copyright (c) 1998 by Specialized Systems Consultants Inc. 
% <ligs@ssc.com>

\subsection{Formateado y procesado de textos}

Pr�cticamente todos los usuarios de ordenadores necesitan de un sistema
para preparar documentos. En el mundo de los ordenadores personales, el
{\bf procesamiento de textos} es lo m�s habitual: 
Se trata de editar y manipular textos en un entorno de tipo 
``lo que ves es lo que obtienes'', obteniendo copias impresas del 
documento completo con sus gr�ficos, tablas y adornos.


Existen procesadores de textos comerciales para el mundo UNIX producidos
por Corel, Star Division y Applix, pero es m�s com�n el {\bf formateo de textos},
que es conceptualmente diferente. En los sistemas de formateo de documentos,
el texto es introducido en un {\bf lenguaje de descripci�n de p�ginas}
que describe la forma en que ha de darse formato al texto. En lugar de 
introducir el texto en un entorno de procesador de textos, se puede modificar
el texto con cualquier editor, como {\tt vi} o {\tt GNU Emacs}. Una vez
hemos acabado con la introducci�n del texto fuente (en el lenguaje de formateo
de texto) un programa aparte convierte el texto fuente en un formato adecuado
para su impresi�n. Es un proceso an�logo al de programar en un lenguaje, como 
el C, y ``compilar'' el documento en un formato imprimible.

En {\linux} hay disponibles muchos sistemas de formateo de documentos. Uno de
ellos es {\tt groff}, la versi�n GNU del formateador de documentos {\tt troff},
un cl�sico desarrollado originalmente en los laboratorios Bell y que a�n se usa
en muchos sistemas UNIX de todo el mundo. Otro sistema moderno es {\TeX},
desarrollado por Donald Knuth, famoso en la ciencia de la computaci�n. Tambi�n
pueden obtenerse dialectos de {\TeX}, como {\LaTeX}.

Los formateadores como {\tt groff} y \TeX\ se diferencian sobre todo en la 
sintaxis de su lenguaje de formateo. Elegir un sistema concreto en vez de otro
es una decisi�n que se toma bas�ndose en las utilidades disponibles que satisfagan
nuestras necesidades, y en el gusto personal.

Muchos consideran que el lenguaje de formateo de {\tt groff} es un poco oscuro
y encuentran que {\TeX} es m�s legible. Sin embargo, {\tt groff}, produce
salida ASCII, que puede leerse en un terminal, mientras que {\TeX} est� pensado 
sobre todo para que la salida sea por un dispositivo de impresi�n.
Hay que disponer de algunos programas adicionales para conseguir salida
ASCII a partir de textos formateados en {\TeX} y para convertir un texto 
introducido en {\TeX} al formato de {\tt groff}.

Otro programa es {\tt texinfo}, una extensi�n de {\TeX} utilizada para
la documentaci�n de los programas llevada a cabo por la Free Software
Foundation. {\tt Texinfo} puede producir salida de impresora o un
hypertexto navegable en l�nea de tipo ``Info'', y todo ello a partir
de un s�lo fichero de texto fuente. Los ficheros ``info'' son la
principal fuente de documentaci�n de los programas GNU, como por
ejemplo {\tt GNU Emacs}.

En la comunidad inform�tica se usan extensamente los formateadores de
documentos para producir comunicaciones, tesis, art�culos para
revistas y libros. (Este libro ha sido generado con {\LaTeX}). La
capacidad de manejar como un fichero de texto el documento fuente
escrito en lenguaje de formateo abre la puerta a muchas posibilidades
de extensi�n del sistema mismo de formateado. Gracias a que el
documento fuente no se almacena en un oscuro formato que s�lo puede
leer un procesador de textos determinado, los programadores pueden
escribir analizadores y traductores para el lenguaje de formateo,
ampliando as� el sistema.

�A qu� se parece un sistema de formateo? En general el fichero de
texto fuente formateado est� compuesto en su mayor parte por el texto en s�,
adem�s de unos {\bf c�digos de control} para producir efectos como cambios
de tipo de letra y m�rgenes, creaci�n de listas, etc�tera.

Examina el siguiente texto:

\begin{quote}
Sr. Torvalds: 

Andamos bastante mosqueados con sus planes de 
implementar {\em sugesti�n post-hipn�tica\/} 
en el c�digo del controlador del terminal de 
{\bf {\linux}}.
Nos sentimos as� por tres razones:

\begin{enumerate}
\item Mostrar mensajes subliminales en el controlador del terminal 
      no s�lo es inmoral, sino adem�s una p�rdida de tiempo.
\item Se ha demostrado que las ``sugestiones post-hipn�ticas'' 
      carecen de efecto cuando se usan contra incautos hackers de 
      UNIX;
\item Ya hemos implementado descargas el�ctricas de alto voltaje 
      en el c�digo del {\tt login} de {\linux}, como medida de 
     seguridad.
\end{enumerate}
Esperamos que lo reconsidere.
\end{quote}

Este texto deberia aparecer en el lenguaje de formateo {\LaTeX} como lo siguiente:
\begin{verbatim}
\begin{quote}
Sr. Torvalds: 

Andamos bastante mosqueados con sus planes de 
implementar {\em sugesti�n post-hipn�tica\/}
en el c�digo del controlador del terminal de 
{\bf {\linux}}.
Nos sentimos as� por tres razones:
\begin{enumerate}
\item Mostrar mensajes subliminales en el controlador del terminal 
      no s�lo es inmoral, sino adem�s una p�rdida de tiempo.
\item Se ha demostrado que las ``sugestiones post-hipn�ticas'' 
      carecen de efecto cuando se usan contra incautos hackers de 
      UNIX;
\item Ya hemos implementado descargas el�ctricas de alto voltaje 
      en el c�digo del {\tt login} de {\linux}, como medida de 
      seguridad.
\end{enumerate}
Esperamos que lo reconsidere.
\end{quote}
\end{verbatim}

El autor introduce el texto utilizando un editor de texto cualquiera y
genera una salida formateada tras procesar el texto fuente con {\LaTeX}.
A primera vista, el sistema de formateo puede parecer arcano, pero en
realidad es bastante sencillo de entender. Utilizar un sistema de
formateo de documentos obliga a usar est�ndares tipogr�ficos al
escribir. Todas las listas numeradas dentro del documento tendr�n el
mismo aspecto, a menos que su autor modifique la definici�n de lista
numerada. El objetivo es permitir al autor concentrarse en el texto, no
en las convenciones tipogr�ficas.

Cuando se escribe con un editor de textos, generalmente no se piensa
en el aspecto que tendr� el texto impreso. El autor se acostumbra a 
imaginar la apariencia que tendr� el texto gracias a las �rdenes de 
formateo en el fichero fuente.

Los procesadores de textos tipo ``Lo que ves es lo que obtienes'' 
(en ingl�s WYSIWYG) son atractivos por muchos conceptos. Proporcionan
un interfaz visual f�cil de usar para la edici�n de documentos.
Pero este interfaz est� limitado a los aspectos de la disposici�n del
texto que son accesibles para el usuario. Por ejemplo, muchos procesadores
de textos proporcionan todav�a un lenguaje especial de formateo para 
producir expresiones complicadas, como las f�rmulas matem�ticas. Esto �ltimo
constituye formateo de documentos, aunque en una escala mucho m�s
modesta.

Un beneficio no tan sutil del formateo de documentos es que se puede
especificar exactamente el formato que se necesita. En muchos casos, el
sistema de procesado de documentos requiere de una especificaci�n de formato.
Los sistemas de formateo de documentos permiten adem�s editar el texto
con cualquier editor de texto, en lugar de basarse en c�digos de formato
que queden escondidos detr�s del opaco interfaz de usuario de un procesador
de textos. El precio a pagar por esta potencia y flexibilidad es la falta
de un interfaz de tipo ``lo que ves es lo que obtienes''.

Existes programas que te dejan visualizar el documento formateado en
un dispositivo de pantalla gr�fica antes de imprimirlo. El programa
{\tt xdvi} muestra bajo el entorno X-Window un fichero ``independiente
del dispositivo'', generado por el sistema \TeX. Aplicaciones del tipo
de {\tt xfig} y {\tt gimp} proporcionan interfaces gr�ficos para dibujar
figuras y diagramas, que se convierten posteriormente al sistema de formateo
de texto para poder incluirlos en tus documentos.

Los formateadores de texto como {\tt troff} ya estaban ah� mucho antes de
que hubiera a disposici�n del p�blico ning�n procesador de textos. Muchos
prefieren a�n hoy su versatilildad y su independiencia del entorno gr�fico.

% Sorry about this.
%\font\manual=manfnt
%\def\MF{{\manual META}\-{\manual FONT}}

Existen muchas utilidades relacionadas con el formateo de texto. El
poderoso sistema METAFONT, que se usa para dise�ar fuentes de \TeX,
viene incluido con la versi�n de {\TeX} que se ha portado a {\linux}.
Otros programas son {\tt ispell}, un corrector ortogr�fico; {\tt
  makeindex} genera �ndices para los documentos en {\LaTeX}; hay muchos
otros paquetes basados en {\tt groff} y {\TeX} que son capaces de
formatear diversos tipos de textos t�cnicos y matem�ticos. Tambi�n
existen programas que transforman el texto en {\TeX} o {\tt groff}
a una infinidad de formatos, y viceversa.

Un lenguaje nuevo en la escena es YODL, escrito por Karel Kubat.
Se trata de un lenguaje f�cil de aprender, y que incorpora filtros para 
producir salidas en diversos formatos, como {\LaTeX}, SGML y HTML.


%%% Local Variables: 
%%% mode: latex
%%% TeX-master: t
%%% End: 

% {\linux} Installation and Getting Started    -*- TeX -*-
% programming.tex
% Copyright (c) 1992-1994 by Matt Welsh <mdw@sunsite.unc.edu>
%
% This file is freely redistributable, but you must preserve this copyright 
% notice on all copies, and it must be distributed only as part of "{\linux} 
% Installation and Getting Started". This file's use is covered by the 
% copyright for the entire document, in the file "copyright.tex".
%
% Copyright (c) 1998 by Specialized Systems Consultants Inc. 
% <ligs@ssc.com>
%revisi�n 1 Francisco javier Fernandez <serrador@arrakis.es>
\subsection{Lenguajes y utilidades de programaci�n}

{\linux} proporciona un completo entorno de programaci�n UNIX que incluye
todas las bibliotecas est�ndar, herramientas de programaci�n, compiladores
y depuradores que podr�an esperarse en otro sistema UNIX.

Est�n soportados los est�ndares, como POSIX.1, lo que permite que los programas
escritos en {\linux} puedan portarse f�cilmente a otros sistemas. Los 
programadores UNIX profesionales y los administradores de sistemas usan 
{\linux} para desarrollar programas en casa, luego trasladan los programas
a sus sistemas UNIX en el trabajo. Ello no s�lo les ahorra gran cantidad
de tiempo y dinero, sino que adem�s les permite trabajar en la comodidad de su
propia casa. (Uno de los autores de este libro usa su sistema para desarrollar
y probar aplicaciones para el sistema X-Window en casa, y que pueden compilarse
directamente en estaciones de trabajo en otro lugar). Los estudiantes de
ingenier�a inform�tica aprenden programaci�n UNIX y exploran otros aspectos
del sistema, como la arquitectura del n�cleo.

Con {\linux} tienes acceso a un completo juego de bibliotecas y utilidades
de programaci�n, adem�s del n�cleo completo y el c�digo fuente de las
bibliotecas.

Dentro del mundo de los programas UNIX, los sistemas operativos y las 
aplicaciones normalmente est�n programados en C o en C++. El compilador
est�ndar de C y C++ para {\linux} es el compilador GNU {\tt gcc}; se trata
de un compilador avanzado y moderno que soporta C++ con las caracter�sticas
de AT\&T 3.0, adem�s de Objective C, otro dialecto de C orientado a objetos.

Adem�s de C y C++, se han portado a {\linux} otros lenguajes de programaci�n 
interpretados o compilados, por ejemplo Smalltalk, FORTRAN, Java, Pascal, 
LISP, Scheme y Ada (si es tan masoquista como para programar en Ada, pues
adelante, no le detendremos). Tambi�n existen varios ensambladores para 
escribir c�digo en el modo protegido del 80386, as� como los favoritos para
el hacking del UNIX como Perl (el lenguaje de guiones que acabar�a con 
todos los de su especie), el Tcl/Tk (un procesador de �rdenes al estilo
 del int�rprete que incluye soporte para desarrollar peque�as aplicaciones
bajo el sistema X-Window).

El depurador avanzado {\tt gdb} permite examinar un programa de l�nea
de c�digo en l�nea de c�digo, o bien examinar la causa del ``cuelgue'' de
una aplicaci�n examinando un volcado del sistema (core dump).
La utilidad {\tt gprof} para perfilado permite obtener estad�siticas
sobre las prestaciones de su programa, indic�ndole en qu� puntos el programa
pasa la mayor parte de su tiempo de ejecuci�n.
Como ya se ha mencionado antes, el editor de texto {\tt GNU Emacs} 
proporciona edici�n interactiva y entornos de compilaci�n para varios
lenguajes de programaci�n. Otras herramientas son el GNU {\tt make} e
{\tt imake} que controlan la compilaci�n de grandes aplicaciones, y RCS
un sistema para el bloqueo de c�digo fuente y el control de revisiones.

Finalmente, {\linux} soporta bibliotecas compartidas enlazadas din�micamente
(DLLs, bibliotecas de enlace din�mico). El c�digo com�n a muchas subrutinas
se enlaza en tiempo de ejecuci�n. Estas bibliotecas (DLLs) le permiten
sobreescribir las funciones predeterminadas con su propio c�digo. Pongamos
que quiere escribir su propia versi�n de la rutina {\tt malloc()}, entonces
el enlazador usar� su nueva rutina en lugar de la que hay en las 
bibliotecas.


%%% Local Variables: 
%%% mode: plain-tex
%%% TeX-master: t
%%% End: 

% Linux Installation and Getting Started    -*- TeX -*-
% soft-xwindows.tex
% Copyright (c) 1992-1994 by Matt Welsh <mdw@sunsite.unc.edu>
%
% This file is freely redistributable, but you must preserve this copyright 
% notice on all copies, and it must be distributed only as part of "{\linux} 
% Installation and Getting Started". This file's use is covered by the 
% copyright for the entire document, in the file "copyright.tex".
%
% Copyright (c) 1998 by Specialized Systems Consultants Inc. 
% <ligs@ssc.com>
% Traduccion Ivan Juanes <ivanjuanes@geocities.com> 

\subsection{Introducci�n al sistema X-Window}
\index{X-Window!introducci�n|(}

El sistema X-Window (atenci�n al singular: X-Window) o simplemente ``las X'',
es un interfaz gr�fico de usuario est�ndar en las m�quinas UNIX; se trata
de un potente entorno que soporta todo tipo de aplicaciones. Dentro del 
sistema de ventanas X-Window, puede tener a la vez varias terminales en la 
pantalla, cada una con su sesi�n propia. A menudo se usa un dispositivo 
se�alador, como un rat�n, aunque en teor�a se puede trabajar sin �l.

Se han escrito espec�ficamente para las X muchas aplicaciones, entre
ellas juegos, utilidades para gr�ficos y para programaci�n. {\linux}
y las X hacen de su sistema una estaci�n de trabajo de toda confianza.
Con una red TCP/IP, su m�quina {\linux} puede visualizar aplicaciones que
se est�n ejecutando en otra m�quina.

El sistema de ventanas X-Window se desarroll� originariamente en el
Massachussets Institute of Technology (MIT) y es de libre distribuci�n. Muchas
empresas, por su parte,  han distribuido ampliaciones propietarias
al sistema X-Window. La versi�n de las X para {\linux} es XFree86, una versi�n
libremente distribuible de X11R6. XFree86 soporta una amplia variedad
de tarjetas de v�deo VGA, S�per VGA, y tarjetas aceleradoras y 3D.
XFree86 es una distribuci�n completa del software X-Window, y
contiene lo que en terminolog�a de las X se llama el servidor, adem�s de
muchas aplicaciones, utilidades, bibliotecas para el programador y
documentaci�n.

Entre las aplicaciones X est�ndar se incluye {\tt xterm}, un emulador
de terminal que se usa para la mayor parte de aplicaciones basadas
en texto que se ejecutan dentro de una ventana, {\tt xdm}, que
gestiona las sesiones (los 'logins'), {\tt xclock}, que muestra un reloj
sencillo, {\tt xman}, un lector de p�ginas de manual basado en las X,
y {\tt xless}.
Las aplicaciones {\linux} para X-Window son muchas; tantas, de hecho, que
son demasiadas para nombrarlas aqu�, pero entre ellas se cuentan hojas
de c�lculo, procesadores de texto, aplicaciones para gr�ficos y 
navegadores de internet como el Netscape Navigator. Muchas otras aplicaciones
se pueden conseguir aparte. En teor�a, cualquier aplicaci�n escrita
para las X compila sin problemas bajo {\linux}.

El interfaz del sistema X-Window est� controlado en gran medida por el
{\bf gestor de ventanas.} Este amigable programa se encarga de situar
las ventanas y el interfaz de usuario que las mueve y las cambia de
tama�o, minimiza las ventanas en iconos, y controla la apariencia de los
marcos de las ventanas, entre otras tareas. XFree86 incluye {\tt twm},
el cl�sico gestor de ventanas del MIT, pero tambi�n gestores avanzados
como el Open Look Virtual Window Manager ({\tt olvwm}). Es popular
entre los usuarios de UNIX el {\tt fvwm}, un gestor de ventanas compacto
que requiere menos de la mitad de memoria que el {\tt twm}. Proporciona
apariencia tridimensional para las ventanas y un escritorio virtual.
El usuario mueve el rat�n hasta el borde de la ventana, y el escritorio
se desplaza como si fuera mucho m�s grande de lo que en realidad es.
{\tt Fvwm} es muy configurable y permite acceso a sus funciones tanto
desde el teclado como desde el rat�n. Muchas distribuciones de {\linux} 
han usado {\tt fvwm} como su gestor de ventanas est�ndar. Una versi�n
de {\tt fvwm} llamada  {\tt fvwm95-2} ofrece el aspecto del sistema de 
ventanas Microsoft Windows.

La distribuci�n de XFree86 incluye bibliotecas de programaci�n
para h�biles programadores que deseen desarrollar aplicaciones X. Los
conjuntos de controles Athena, Open Look y Xaw3D est�n soportados.
Se incluyen todas las fuentes est�ndar, los mapas de bits y las p�ginas
de manual. Est� soportado tambi�n PEX (un interfaz de programaci�n para 
gr�ficos tridimensionales).

Muchos programadores en X usan el conjunto de controles Motif para 
sus desarrollos. Varias empresas venden licencias mono o multiusuario
para las versiones compiladas de Motif. Puesto que Motif como tal es
relativamente caro, no hay demasiados usuarios de {\linux} que lo tengan.
No obstante, pueden redistribuirse librementes los binarios con Motif
vinculado est�ticamente. Si usas Motif para escribir un programa, puedes
producir un binario para que los usuarios que no tengan Motif 
puedan usarlo.

Una advertendia importante para usar el sistema X-Window es la cuesti�n
del hardware necesario. Un 386 con 4 megabytes de RAM puede ejecutar las
X, pero se necesitan 16 megas o m�s de memoria RAM para usarlas con 
comodidad. Es ideal tener tambi�n un procesador m�s r�pido, pero es mucho
m�s importante disponer de memoria f�sica suficiente. Adem�s de esto
si quiere adquirir un rendimiento de v�deo c�modo, recomendamos 
utilizar una tarjeta de v�deo aceleradora, en una ranura AGP o en una PCI.
En los tests de rendimiento se han conseguido resultados que exceden de 
los 300.000 xstones\NT{Ver http://www.rarcoa.com/~thebard/X11-performance.html}.
Con el hardware adecuado, hallar� que su m�quina {\linux} ejecutando las X
es tan r�pida o m�s que ejecutar las X en otras estaciones de trabajo
UNIX.

En el Cap�tulo~\ref{chap-xwindow} trataremos de la instalaci�n y uso de 
las X en su sistema.


%%% Local Variables: 
%%% mode: plain-tex
%%% TeX-master: t
%%% End: 

% {\linux} Installation and Getting Started    -*- TeX -*-
% soft-networking.tex
% Copyright (c) 1992-1994 by Matt Welsh <mdw@sunsite.unc.edu>
%
% This file is freely redistributable, but you must preserve this copyright 
% notice on all copies, and it must be distributed only as part of "{\linux} 
% Installation and Getting Started". This file's use is covered by the 
% copyright for the entire document, in the file "copyright.tex".
%
% Copyright (c) 1998 by Specialized Systems Consultants Inc. 
% <ligs@ssc.com>
% Traducido por Ivan Juanes <ivanjuanes@geocities.com>
% Revisi�n 1 por FRJMS
\subsection{Introducci�n a las redes}

�Desea comunicarse con el mundo? {\linux} soporta dos protocolos 
b�sicos en UNIX: TCP/IP y UUCP. El protocolo TCP/IP 
(Transmission Control Protocol/Internet Protocol) es el paradigma de
redes que permite a los sistemas de todo el mundo intercomunicarse
mediante una sola red, {\bf Internet.} Con {\linux}, TCP/IP y una conexi�n
a Internet \NT{Que en Espa�a han bajado mucho de precio o son gratuitas},
puedes comunicarte con usuarios y ordenadores a trav�s del correo
electr�nico, noticias, y transferencia de ficheros por FTP.

La mayor parte de las redes TCP/IP utilizan Ethernet como transporte f�sico
de la red. {\linux} soporta las tarjetas de red Ethernet m�s usuales y los
interfazs para ordenadores personales, como los adaptadores pocket Ethernet 
y PCMCIA.

Sin embargo, y debido a que no todo el mundo tiene una conexi�n Ethernet en
casa, {\linux} soporta tambi�n {\bf SLIP} (Serial Line Internet Protocol, protocolo
de Internet por l�nea serie) y {\bf PPP} (Point-to-Point Protocol, protocolo de
punto a punto), que proporcionan el acceso a Internet a trav�s de un m�dem.
Muchos negocios y universidades disponen de servidores PPP a los que conectarse.
De hecho, si su sistema {\linux} dispone de una conexi�n Ethernet y un m�dem,
puede convertirse en un servidor SLIP o PPP para otros ordenadores.

NFS (Network File System, sistema de ficheros de red) le permite compartir ficheros
con otras m�quinas de la red de forma transparente. FTP (File Transfer
Protocol, protocolo de transferencia de ficheros) le permite transferir ficheros
desde y hacia otro ordenador. {\tt Sendmail} env�a y recibe correo electr�nico
a trav�s del protocolo SMTP. C-News e INN son sistemas de noticias basados en
el protocolo NNTP; y {\tt telnet}, {\tt rlogin}, {\tt rsh} y {\tt ssh} le permiten
autentificarse y ejecutar �rdenes en otras m�quinas de la red. Con {\tt finger}
 podra obtener informaci�n sobre otros usuarios de Internet.

{\linux} tambi�n soporta la conectividad con Microsoft Windows gracias a
Samba\footnote{ V�ase {\em Samba: Integrating UNIX and Windows}, Copyright
1998 Specializad System Consultants.} y conectividad con Macintosh
a trav�s de AppleTalk y LocalTalk.  Tambi�n se incluye el soporte
para el protocolo IPX de Novell.

{\linux} tiene disponible una bater�a completa de lectores de correo y noticias,
entre ellos {\tt elm}, {\tt pine}, {\tt mutt}, {\tt rn}, {\tt nn}, y {\tt
tin}. Cualquiera que sea su preferencia, puede configurar el sistema {\linux}
para que env�e y reciba noticias y correo electr�nico desde todo el mundo.

El sistema proporciona un interfaz est�ndar de programaci�n de sockets 
UNIX. Se puede migrar a {\linux} pr�cticamente cualquier programa que use
TCP/IP. El servidor de las X para {\linux} tambi�n soporta TCP/IP, y las
aplicaciones que se ejecutan en otros sistemas pueden usar la pantalla
de su sistema local para visualizarse.

En el Cap�tulo~\ref{chap-networking} trataremos de la instalaci�n del
software TCP/IP, y con �l SLIP y PPP.

UUCP (UNIX-to-UNIX Copy, copia de UNIX a UNIX) es un mecanismo ya bastante
antiguo para transferir correo electr�nico y noticias entre m�quinas UNIX.
Hist�ricamente, las m�quinas UUCP estaban conectadas por l�neas telef�nicas
a trav�s de un m�dem, pero UUCP es capaz igualmente de transferir datos 
a trav�s de una red TCP/IP. Si no dispone de acceso a una red TCP/IP
(acceso f�cil de obtener en Espa�a con el acceso gratuito a Internet),
puede configurar su sistema para enviar ficheros y correo electr�nico 
utilizando UUCP. Para m�s informaci�n, v�ase el cap�tulo~\ref{chap-networking}.


% Linux Installation and Getting Started    -*- TeX -*-
% telecomm.tex
% Copyright (c) 1992-1994 by Matt Welsh <mdw@sunsite.unc.edu>
%
% This file is freely redistributable, but you must preserve this copyright 
% notice on all copies, and it must be distributed only as part of "Linux 
% Installation and Getting Started". This file's use is covered by the 
% copyright for the entire document, in the file "copyright.tex".
%
% Copyright (c) 1998 by Specialized Systems Consultants Inc. 
% <ligs@ssc.com>

\subsection{Telecomunicaciones y software para BBS}

Si dispone de un M�dem, podr� comunicarse con otras m�quinas gracias a los
paquetes de telecomunicaciones que proporciona Linux. Mucha gente usa
sus programas de telecomunicaciones para acceder a sistemas de BBS (Bulletin
Board System, Sistema de tabl�n de anuncios electr�nico), y a proveedores de
servicios en l�nea como Prodigy, Compuserve America Online. La gente utiliza
el m�dem para conectarse a los sistemas UNIX del trabajo o el centro educativo.
Con el m�dem se pueden enviar y recibir faxes.

Un conocido paquete de comunicaciones para Linux es {\tt seyon,} que 
nos proporciona un interfaz de usuario c�modo y configurable bajo X~Window, y 
que lleva incluido el soporte para los protocolos de transferencia de ficheros
Kermit y Z-Modem. Otros programas de telecomunicaciones son C-Kermit, {\tt pcomm}
y {\tt minicom}. Son parecidos a los programas de telecomunicaciones disponibles
para otros sistemas operativos, y resultan bastante f�ciles de utilizar.

Si no tiene acceso a un servidor SLIP o PPP (v�ase la secci�n anterior)
puede utilizar {\tt term} para multiplexar su l�nea serie. El programa
{\tt term} le hace posible abrir m�s de una sesi�n de login sobre una conexi�n 
por m�dem. Le permite redirigir conexiones de un cliente X a su servidor X local
a trav�s de una l�nea serie. Otro paquete de software, KA9Q, implementa un interfaz
parecido, estilo SLIP.

Ser un SySop de una BBS fue en tiempos una afici�n predilecta y una forma
de obtener ingresos para mucha gente. {\linux} soporta una amplia gama de 
software para BBS, que en general es mucho m�s potente que el disponible
para otros sistemas operativos. Con una l�nea telef�nica, un m�dem y
{\linux}, puedes transformar tu sistema en una BSS y proporcionar acceso
a los usuarios de todo el mundo. Entre los programas de BBS para linux  
est�n XBBS y UniBoard BBS.

La mayor�a de programas de BBS constri�en al usuario a un sistema de men�s
en el que s�lo est�n disponibles determinadas funciones y aplicaciones.
Una alternativa al acceso por BBS es el acceso UNIX completo, que permite
al usuario llamar a tu sistema y autentificarse normalmente. Esto �ltimo 
requiere de una nada despreciable tarea de administraci�n por parte del 
administrador, pero no es dif�cil proporcionar acceso p�blico a UNIX. 
Adem�s de la red TCP/IP, puede hacer posible el acceso al correo y las 
noticias en su sistema.

Si no dispone de acceso a una red TCP/IP o de una pasarela UUCP,
{\linux} le permite todav�a comunicarse con redes de BBS como Fidonet, que le
permiten intercambiar correo y noticias a trav�s de una l�nea telef�nica.
Para m�s informaci�n sobre telecomunicaciones y software de BBS bajo {\linux}, 
v�ase el Cap�tulo~\ref{chap-networking}.



\chapter{Lugares �tiles en Web y FTP}
\label{www-appendix}
Los siguientes lugares de la Red WWW y de FTP le ser�n de utilidad:
\begin{description}
\item [http://www.azstarnet.com/\ $\tilde{}$axplinux] Este es el lugar
  en la web del \Linux\ para \axp\ de David
  Mosberger-Tang, y es el lugar donde debe ir a buscar todos los
  HOWTOs sobre \axp.  Tambi�n tiene una gran cantidad de punteros
  hacia informaci�n sobre \Linux\ y espec�fica sobre \axp, como por
  ejemplo las hojas de datos de la CPU.
  
\item [http://www.redhat.com/] El lugar de Red Ha en la web.  Hay aqu�
  un mont�n de punteros �tiles.
  
\item [ftp://sunsite.unc.edu] Este es el principal lugar para un
  mont�n de software libre.  El software espec�fico sobre \Linux\ se
  encuentra en \emph{pub/Linux}.
  
\item [http://www.intel.com] El lugar de Intel en la web, y un buen
  lugar donde encontrar informaci�n sobre chips Intel.
  
\item [http://www.ssc.com/lj/index.html] "<Linux Journal"> es una muy
  buena revista sobre \Linux\ y bien vale la pena el precio de la
  suscripci�n anual para leer sus excelentes art�culos.
  
\item [http://www.blackdown.org/java-linux.html] Este es el lugar
  principal con respecto a la informaci�n sobre Java para \Linux.
  
\item [ftp://tsx-11.mit.edu/\ $\tilde{}$ftp/pub/linux] El lugar FTP
  sobre \Linux\ del MIT.
  
\item [ftp://ftp.cs.helsinki.fi/pub/Software/Linux/Kernel] Fuentes del
  n�cleo de Linus.
  
\item [http://www.linux.org.uk] El "<UK Linux User Group">.
  
\item [http://sunsite.unc.edu/mdw/linux.html] P�gina principal para el
  "<Linux Documentation Project"> (Proyecto de documentaci�n para
  \Linux), al cual est� afiliado LuCAS ({\tt http://lucas.ctv.es/})
  que se dedica a traducir al castellano ---como en este caso--- dicha
  documentaci�n.
  
\item [http://www.digital.com] El lugar en la web de "<Digital
  Equipment Corporation">
  
\item [http://altavista.digital.com] La m�quina de b�squeda Altavista
  es un producto de la empresa Digital, y un muy buen lugar para
  buscar informaci�n dentrl de la web y los grupos de discusi�n.
  
\item [http://www.linuxhq.com] El lugar web "<Linux HQ"> contiene los
  actualizados parches tanto oficiales como no-oficiales, consejos, y
  punteros a la web que le ayudar�n a conseguir el mejor conjunto de
  fuentes posibles para su sistema.

\item [http://www.amd.com] El lugar de AMD en la web.
  
\item [http://www.cyrix.com] El lugar de Cyrix en la web.

\end{description}


% Linux Installation and Getting Started    -*- TeX -*-
% soft-msdos.tex
% Copyright (c) 1992-1994 by Matt Welsh <mdw@sunsite.unc.edu>
%
% This file is freely redistributable, but you must preserve this copyright 
% notice on all copies, and it must be distributed only as part of "Linux 
% Installation and Getting Started". This file's use is covered by the 
% copyright for the entire document, in the file "copyright.tex".
%
% Copyright (c) 1998 by Specialized Systems Consultants Inc. 
% <ligs@ssc.com>
%Revisi�n 1 sin fallos FRJMS
%\subsection{Interfacing and MS-DOS.}
\subsection{Interacci�n con MS-DOS}

%Various utilities exist to interface with MS-DOS. The most well-known
%application is the Linux MS-DOS Emulator, which lets you run MS-DOS
%applications directly from Linux.  Although Linux and MS-DOS are
%completely different operating systems, the 80386 protected-mode
%environment allows MS-DOS applications to behave as if they were
%running in their native 8086 environment.

Diferentes utilidades existen como una interfaz con MS-DOS. La
m�s conocida es el emulador de MS-DOS de {\linux}, que permite ejecutar
aplicaciones de MS-DOS directamente desde {\linux}. Aunque {\linux} y MS-DOS
son sistemas operativos totalmente diferentes, el entorno en modo
protegido 80386 le permite a las aplicaciones MS-DOS comportarse como
si estuviesen en su entorno original 8086.

%The MS-DOS emulator is still under development, but many popular
%applications run under it. Understandably, MS-DOS applications that
%use bizarre or esoteric features of the system may never be supported,
%because of the limitations inherent in any emulator.  For example, you
%shouldn't expect to run programs that use 80386 protected-mode
%features, like Microsoft Windows (in 386 enhanced mode, that is).

El emulador de MS-DOS est� desarroll�ndose, pero muchas
aplicaciones se ejecutan ya bajo �l. Evidentemente, las aplicaciones
de MS-DOS que usen extra�as o esot�ricas caracter�sticas del sistema,
nunca podr�n soportarse, debido a las limitaciones inherentes de
cualquier emulador. Por ejemplo, no espere ejecutar programas que usen
el modo protegido 80386, como Microsoft Windows (es decir, en modo 386
mejorado).

%Standard MS-DOS commands and utilities like {\tt PKZIP.EXE} work under
%the emulators, as do 4DOS, a {\tt COMMAND.COM} replacement, FoxPro
%2.0, Harvard Graphics, MathCad, Stacker 3.1, Turbo Assembler, Turbo
%C/C++, Turbo Pascal, Microsoft Windows 3.0 (in real mode), and
%WordPerfect 5.1.

Ordenes est�ndar de MS-DOS y utlidades como {\tt PKZIP.EXE} funcionan
con los emuladores, al igual que 4DOS, un sustituto de {\tt
COMMAND.COM}, FoxPro 2.0, Harvard Graphics, MathCad, Stacker 3.1, Turbo
Assembler, Turbo C/C++, Turbo Pascal, Microsoft Windows 3.0 (en modo
real) y WordPerfect 5.1.

%The MS-DOS Emulator is meant mostly as an ad-hoc solution for those
%who need MS-DOS for only a few applications and use {\linux} for
%everything else. It's not meant to be a complete implementation of
%MS-DOS.  Of course, if the Emulator doesn't satisfy your needs, you
%can always run MS-DOS as well as {\linux} on the same system. Using the
%LILO boot loader, you can specify at boot time which operating system
%to start. {\linux} can also coexist with other operating systems, like
%OS/2.

El emulador de MS-DOS viene a ser como una soluci�n {\em ad-hoc} para
aquellos que necesitan MS-DOS s�lo para algunas aplicaciones y usan
{\linux} para todo lo dem�s. Esto no quiere decir que sea una completa
implementaci�n de MS-DOS. Por supuesto que si el emulador de MS-DOS no
satisface sus necesidades, siempre puede utilizar alternativamente
MS-DOS y {\linux} en el mismo sistema. Utilizando el ``LILO boot loader'',
puede especificar en el inicio qu� sistema operativo debe arrancar. {\linux}
adem�s puede coexistir con otros sistemas operativos, como OS/2.

%{\linux} provides a seamless interface to transfer files between {\linux}
%and MS-DOS. You can mount a MS-DOS partition or floppy under {\linux},
%and directly access MS-DOS files as you would any file.

{\linux} proporciona una interfaz sin fisuras para transferir ficheros
entre {\linux} y MS-DOS. Puede montar una partici�n MS-DOS o un floppy
bajo {\linux} y acceder directamente a los ficheros de MS-DOS que desee.

%Currently under development is {\bf WINE}---a Microsoft Windows
%emulator for the X~Window System under {\linux}.  Once WINE is complete,
%users will be able to run MS-Windows applications directly from
%{\linux}. This is similar to the commercial WABI Windows emulator from
%Sun Microsystems, which is also available for {\linux}.

Actualmente se est� desarrollando {\bf WINE}---un emulador de
Microsoft Windows para el sistema X-Window bajo {\linux}. Una vez que
WINE est� completado, los usuarios podr�n ejecutar aplicaciones para
MS-Windows directamente desde {\linux}. Esto es parecido a la aplicaci�n
comercial ``WABI Windows emulator'' de Sun Microsystems, tambi�n
disponible para {\linux}.

%In Chapter~\ref{chap-advanced}, we talk about the MS-DOS tools
%available for {\linux}.

En el Cap�tulo~\ref{chap-tutorial} y en el Cap�tulo~\ref{chap-networking}, hablaremos acerca de las
utilidades de MS-DOS disponibles para {\linux}. 



% {\linux} Installation and Getting Started    -*- TeX -*-
% misc.tex
% Copyright (c) 1992-1994 by Matt Welsh <mdw@sunsite.unc.edu>
%
% This file is freely redistributable, but you must preserve this copyright 
% notice on all copies, and it must be distributed only as part of "{\linux} 
% Installation and Getting Started". This file's use is covered by the 
% copyright for the entire document, in the file "copyright.tex".
%
% Copyright (c) 1998 by Specialized Systems Consultants Inc. 
% <ligs@ssc.com>
% Revisi�n 1 por Francisco javier Fernandez --sin fallos--
%Gold
\subsection{Otras aplicaciones}

Existen para {\linux} multitud de aplicaciones y utilidades de todo tipo, 
como cabe esperar de un sistema tan variado. El principal objetivo de
{\linux} es la inform�tica personal con UNIX, pero no es �ste el �nico campo
en donde sobresale. El cat�logo de programas cient�ficos y para empresas,
sigue creciendo y los desarrolladores de software comercial hace tiempo que 
han comenzado a contribuir al creciente fondo de aplicaciones para {\linux}.

Hay disponibles en {\linux} varias bases de datos relacionales, por ejemplo
Postgres, Ingres, Mbase, Oracle, IBM DB2, Interbase, Sybase y ADABAS.
Se trata de aplicaciones de bases datos profesionales, con todo tipo 
de caracter�sticas avanzadas, y de arquitectura cliente/servidor, 
semejantes a las que se encuentran en otras plataformas UNIX. 
Existen igualmente otros sistemas comerciales de bases de datos para {\linux}.

Entre las aplicaciones cient�ficas se cuentan FELT (finite element
analysis, an�lisis de elementos finitos); {\tt gnuplot} (representaci�n 
y an�lisis de datos); Octave (un paquete de matem�tica simb�lica similar
a MATLAB); {\tt xspread} (calculadora y hoja de c�lculo); {\tt xfractint}
(una adaptaci�n a X~Window del conocido generador de fractales Fractint)
y {\tt xlispstat} (para estad�sticas). Otras aplicaciones: SPICE 
(dise�o y an�lisis de circuitos) y Khoros (dise�o y visualizaci�n de im�genes
y se�ales digitales). Tambi�n existen aplicaciones comerciales como Maple y 
Matlab.

Se han adaptado a {\linux} muchas m�s aplicaciones, y �ltimamente el n�mero
crece vertiginosamente. Si de ninguna manera encuentra lo que busca,
siempre puede intentar migrar usted mismo la aplicaci�n desde otra
plataforma. Migrar aplicaciones UNIX, del campo que sea, a {\linux} no suele
presentar problemas. El completo entorno de programaci�n UNIX del que dispone
{\linux} sirve de base para cualquier aplicaci�n cient�fica.

{\linux} cuenta tambi�n con un creciente n�mero de juegos. Existen los cl�sicos
juegos de dragones y mazmorras basados en texto, como Nethack y Moria;
luego est�n los {\bf MUDs} (multi-user dungeons, dragones y mazmorras multiusuario) 
que permiten que muchos usuarios interact�en en una aventura basada en texto, como
DikuMUD y TinyMUD; y una pl�yade de juegos para las X~Window, como {\tt xtetris}, 
{\tt netrek}, y {\tt xboard}, la versi�n X11 de {\tt gnuchess}. El popular 
juego de disparar a todo lo que se mueva, Doom, y los arcades que lo continuaron,
Quake, Quake II y Quake III, han sido portados a {\linux}. �ste �ltimo ha salido para 
{\linux} antes que para algunas plataformas mayoritarias, 

Para los mel�manos, {\linux} soporta gran variedad de tarjetas de sonido
y programas asociados, como CDplayer, que convierte su unidad de CD-Rom
en un reproductor de CD's, secuenciadores y editores MIDI, que permiten
componer m�sica para su reproducci�n en un sintetizador u otro instrumento
controlado por MIDI, editores de sonido para sonidos digitalizados, y 
codificadores y reproductores de ficheros en formato MP3\NT{Durante la traducci�n
de este documento, el formato libre Ogg Vorbis ha alcanzado la versi�n 1.0}.

�No encuentra la aplicaci�n que busca? El Mapa de Software {\linux}, que
se describe en el Ap�ndice~\ref{app-sources-num}, enumera los paquetes 
de software que se han escrito o migrado a {\linux}. Otra manera de encontrar
aplicaciones para {\linux} es buscar en los ficheros {\tt INDEX} que se 
encuentran en los sitios FTP con programas para {\linux}, en el caso de que tenga
acceso a Internet.

La mayor parte del software libremente redistribuible disponible para UNIX
compila sin problemas en {\linux}, o al menos con poca dificultad.
Pero si todo lo dem�s fallara, siempre puede programarse usted mismo la
aplicaci�n. Si anda buscado una aplicaci�n comercial, puede existir un 
cl�nico libre. Incluso puede considerar la posibilidad de animar a su
compa��a proveedora de software a que lance una versi�n de su programa
para {\linux}. Muchos individuos y organizaciones han contactado ya con 
compa��as de software y les han pedido que porten sus aplicaciones a {\linux},
con diferentes grados de �xito.


%%% Local Variables: 
%%% mode: plain-tex
%%% TeX-master: t
%%% End: 




%%% Local Variables: 
%%% mode: latex
%%% TeX-master: t
%%% End: 

% Linux Installation and Getting Started    -*- TeX -*-
% gpl.tex
% Copyright (c) 1992-1994 by Matt Welsh <mdw@sunsite.unc.edu>
%
% This file is freely redistributable, but you must preserve this copyright 
% notice on all copies, and it must be distributed only as part of "Linux 
% Installation and Getting Started". This file's use is covered by the 
% copyright for the entire document, in the file "copyright.tex".
%
% Copyright (c) 1998 by Specialized Systems Consultants Inc. 
% <ligs@ssc.com>
%Revisado por Francisco javier fernandes <serrador@arrakis.es> el 16 de julio de 2002

\section{Acerca del Copyright}
\markboth{Introducci�n a {\linux}}{Acerca del Copyright}
\namedlabel{sec-intro-gpl}{Acerca del Copyright}

Linux est� regido por lo que se conoce como la {\em Licencia P�blica
General} de GNU, o {\em GPL, General Public License}. La GPL fue
desarrollada para el proyecto GNU por la {\em Free Software
Foundation}, que podemos traducir como ``Fundaci�n por el Software
Libre''. La licencia hace una serie de previsiones sobre la
distribuci�n y modificaci�n del ``software libre''. ``Free'' en
este sentido se refiere a libertad, y no necesariamente al coste. La GPL puede ser
interpretada de distintas formas, y esperamos que este resumen le
ayude a entenderla y c�mo afecta a Linux. Se incluye una copia
completa de la Licencia al final del libro, en el Ap�ndice~\ref{app-gpl-num}.
\index{free software}
\index{software libre}

Originalmente, Linus Torvalds lanz� Linux bajo una licencia m�s
restrictiva que la GPL, que permit�a que el software fuera libremente
distribuido y modificado, pero prohib�a su uso para ganar dinero. Sin
embargo, la GPL autoriza que la gente venda su software, aunque no le
permite restringir el derecho que su comprador tiene a copiarlo y
venderlo a su vez.

En primer lugar, hay que aclarar que el ``software libre'' de la GPL
{\em no es} software de dominio p�blico. El software de dominio
p�blico carece de {\it copyright} y pertenece literalmente al
p�blico. El software regido por la GPL s� tiene el copyright de su
autor o autores. Esto significa que est� protegido por las leyes
internacionales del copyright y que el autor del software est�
declarado legalmente. No solo porque un programa sea de libre
distribuci�n puede consider�rsele del dominio p�blico.

El software regido por la GPL tampoco es ``shareware''. Por lo
general, el ``shareware'' es propiedad del autor, y exige a los
usuarios que le paguen cierta cantidad por utilizarlo despu�s de la
distribuci�n. Sin embargo, el software que se rige por la GPL puede
ser distribuido y usado sin pagar a nadie.

La GPL permite a los usuarios modificar el software y
redistribuirlo. Sin embargo, cualquier trabajo derivado de un programa
GPL se regir� tambi�n por la GPL. En otras palabras, una compa��a
nunca puede tomar \linux, modificarlo y venderlo bajo una licencia
restringida. Si un software se deriva de \linux, �ste deber� regirse
por la GPL tambi�n.

La GPL permite distribuir y usar el software sin cargo alguno. Sin
embargo, tambi�n permite que una persona u organizaci�n gane dinero
distribuyendo el software. Sin embargo, cuando se venden programas
GPL, el distribuidor no puede poner ninguna restricci�n a la
redistribuci�n. Esto es, si usted compra un programa GPL, puede a su
vez redistribuirlo gratis o cobrando una cantidad.

Esto puede parecer contradictorio. ?`Por qu� vender software cuando la
GPL especifica que puede obtenerse gratis? Por ejemplo, supongamos que
una empresa decide reunir una gran cantidad de software GPL en un
CD-ROM y venderlo. La empresa necesitar� cobrar por el hecho de haber
producido el CD, y as�mismo querr� ganar dinero. Esto est� permitido
por la GPL.

Las organizaciones que vendan el software regido por la GPL deben
tener en cuenta algunas restricciones. En primer lugar, no pueden
restringir ning�n derecho al comprador del programa. Esto significa
que si usted compra un CD-ROM con software GPL, podr� copiar ese CD y
revenderlo sin ninguna restricci�n. En segundo lugar, los
distribuidores deben hacer saber que el software se rige por la
GPL. En tercer lugar, el vendedor debe proporcionar, sin coste
adicional, el c�digo fuente del software a distribuir. Esto permite a
cualquiera comprar el software y modificarlo a placer.

Permitir a una empresa distribuir y vender programas que son gratis es
bueno. No todo el mundo tiene acceso a Internet para conseguir los
programas, como \linux, gratis. La GPL permite a las empresas vender y
distribuir programas a esas personas que no pueden acceder al software
con un coste bajo. Por ejemplo, muchas empresas venden \linux en
disquetes o CD-ROM por correo, y hacen negocio de esas ventas. Los
desarrolladores de \linux pueden no tener constancia de estos
negocios. Por ejemplo, Linus sabe que ciertas compa��as venden \linux,
y �l no va a cobrar nada por esas ventas.

En el mundo del software libre, lo importante no es el dinero. El
objetivo es permitir desarrollar y distribuir software fant�stico
asequible a cualquiera. En la siguiente secci�n, hablaremos de c�mo
esto se aplica al desarrollo de \linux.




% {\linux} Installation and Getting Started    -*- TeX -*-
% design.tex
% Copyright (c) 1992-1994 by Matt Welsh <mdw@sunsite.unc.edu>
%
% This file is freely redistributable, but you must preserve this copyright 
% notice on all copies, and it must be distributed only as part of "{\linux} 
% Installation and Getting Started". This file's use is covered by the 
% copyright for the entire document, in the file "copyright.tex".

% Traduccion realizada por Juan Jose Amor. Envie sus comentarios a:
%            Juan Jose Amor, 2:341/12.19 (FidoNet)
%            jjamor@gedeon.ls.fi.upm.es (InterNet)
%
% $Log: design.tex,v $
% Revision 1.7  2002/07/19 23:51:35  pakojavi2000
% Beta 1
%
% Revision 1.6  2002/07/17 22:36:32  pakojavi2000
% Alpha2
%
% Revision 1.5  2002/07/13 13:43:37  pakojavi2000
% Gold
%
% Revision 1.4  2002/07/13 13:31:09  pakojavi2000
% Gold
%
% Revision 1.3  2002/03/20 00:40:17  elzo
% Diversas correciones
%
% Revision 1.2  2000/11/13 19:13:15  amolina
% *** empty log message ***
%
% Revision 0.5.0.1  1996/02/10 23:45:10  rcamus
% Primera beta publica
%
%

\section{Dise�o y filosof�a de {\linux}}
\namedlabel{sec-intro-design}{Dise�o y filosof�a de {\linux}}

\index{{\linux}!filosof�a|(}
En ocasiones, los nuevos usuarios de {\linux} se crean falsas
expectativas acerca de �ste. {\linux} es un sistema operativo �nico, y es
importante entender su filosof�a y dise�o para usarlo
de una manera eficiente. Aunque usted sea un experimentado ``gur�'' de UNIX,
lo que viene a continuaci�n le interesar� con total seguridad.

\index{UNIX!comercial}
En las versiones comerciales de UNIX, el sistema se desarrolla siguiendo
una rigurosa pol�tica de mantenimiento de la calidad, con sistemas de
control de revisiones para las fuentes y documentaci�n, etc. Los
desarrolladores no pueden a�adir cosas nuevas por su cuenta: cualquier
cambio ser� en respuesta a un informe de un {\it bug} detectado, y se
registrar� cuidadosamente en el sistema de control de versiones, de
manera que podr� volverse atr�s sin problemas. Cada desarrollador
tiene asignada una o m�s partes del c�digo, y solo ese desarrollador
puede alterar esas secciones del c�digo.

Internamente, el departamento de calidad realiza rigurosas pruebas
en cada nueva versi�n del sistema operativo, e informa de los errores. Es
responsabilidad de los desarrolladores corregir esos errores. Se
utiliza un complicado sistema de an�lisis estad�stico para asegurarse
de que se corrige cierto porcentaje de errores antes de lanzar la
versi�n siguiente.

Como vemos, el proceso seguido por los desarrolladores de los UNIX comerciales
para mantenerlo y darle soporte es muy complicado, pero razonable. La
compa��a debe tener cierta seguridad de que la pr�xima revisi�n del
sistema est� lista para comercializarse, a trav�s de las pruebas
que hemos comentado. Esto supone un gran trabajo que involucra a
cientos (si no miles) de programadores, {\it betatesters}, redactores
de documentaci�n y personal administrativo. Por supuesto, no todos los
fabricantes de UNIX trabajan as�, pero esto nos da una idea de la
panor�mica habitual.

Con {\linux}, hay que olvidarse del concepto de desarrollo organizado,
sistemas de control de versiones, informaci�n de errores estructurada
o an�lisis estad�sticos. {\linux} es un sistema operativo hecho por
``{\it hackers}''\footnote{Por ``hacker'' queremos referirnos a
gente que programa con verdadera pasi�n, por {\it hobby}, a explotar
sus ordenadores al m�ximo, con resultados �tiles para otras
personas. Este concepto es contrario al habitualmente aceptado, seg�n
el cual un ``hacker'' es un pirata inform�tico.}

Linux ha sido desarrollado principalmente por un grupo de
programadores de todo el mundo unidos por Internet. A trav�s de
Internet, cualquiera tiene la oportunidad de unirse al grupo y ayudar
al desarrollo y depuraci�n del n�cleo, portar nuevo software, escribir
documentaci�n o ayudar a los nuevos usuarios. La comunidad de {\linux} se
comunica principalmente mediante diversas listas de correo y grupos de
USENET. Existen algunos acuerdos en el desarrollo, como que cualquiera
que desee que su c�digo sea incluido en la versi�n ``oficial'' del
n�cleo deber� ponerse en contacto con Linus Torvalds,
\index{Torvalds, Linus}
quien comprobar� el c�digo y lo incluir� en el n�cleo. Por lo general,
estar� encantado en hacerlo, siempre que no estropee otras cosas.

\index{{\linux}!desarrollo}

El sistema se ha dise�ado siguiendo una filosof�a abierta y de
crecimiento. Por regla general hay una nueva versi�n del n�cleo
cada pocas semanas.
Esto depende del n�mero de errores a corregir, la cantidad de
informaci�n recibida de los usuarios, y lo que haya dormido Linus esta semana.

\index{errores}
\index{{\linux}!errores}
Por lo tanto, resulta dif�cil decir que todos los errores vayan a poder
corregirse para cierta fecha. Pero conforme el sistema va apareciendo
libre de errores cr�ticos o manifiestos, se considera ``estable'' y se
lanzan nuevas revisiones. Hay que recordar que no se pretende realizar
un sistema perfecto, sin errores. Se trata sencillamente de desarrollar
una implementaci�n libre de UNIX. {\linux} est� hecho {\em para\/}
desarrolladores, m�s que para otro tipo de personas. 

Si alguien desarrolla una aplicaci�n o nueva caracter�stica para el
n�cleo, se a�ade inicialmente en una fase ``alfa'',
\index{alpha}
\index{desarrollo!alpha}
es decir, pensada para que la puedan probar los usuarios m�s atrevidos
aficionados a enredar con los problemas que surgen en el c�digo cuando
se encuentra en fases tempranas de desarrollo. Dado que la comunidad de 
{\linux} se basa sobre todo en
Internet, el software ``alfa'' se env�a, normalmente, a servidores
FTP dedicados a {\linux} (vea el ap�ndice~\ref{app-ftplist-num}) y
se anuncia mediante un mensaje puesto en un grupo USENET dedicado a
{\linux}. Los usuarios que descargan y prueban el software pueden entonces
enviar sugerencias, correcciones de errores o preguntas al autor por
correo electr�nico.

Una vez que se corrigen los problemas iniciales, el c�digo pasa a una
fase ``beta'',
\index{beta}
\index{desarrollo!beta}
en la que se considera estable pero incompleto (o sea, funciona, pero
no incluye todas las funcionalidades previstas). Tambi�n se puede
pasar a una etapa ``final'' en la que el software se considera
terminado. 

Recuerden que lo anterior son s�lo convenciones, y no leyes.
\index{desarrollo!convenciones}
Algunos programas pueden no necesitar fases ``alpha''. Es el
desarrollador quien tomar� las decisiones al respecto en todo caso.

Estar� sorprendido de que un grupo de programadores y aficionados
voluntarios, relativamente desorganizados, hayan podido hacer
algo. Sin embargo, este grupo constituye uno de los m�s eficientes y
motivados. Todo el n�cleo de Linux se ha escrito {\em desde cero},
sin emplear c�digo alguno de fuentes propietarias. Todo el software,
bibliotecas, sistemas de ficheros y controladores se han desarrollado
o se ha portado desde otros sistemas; y se han programado controladores para
los dispositivos m�s populares.

Normalmente, {\linux} se distribuye junto con otro software en lo que se
conoce como una {\em distribuci�n},
\index{distribuciones}
\index{{\linux}!distribuciones}
un paquete de software que permite poner a punto un sistema
completo. Dado que crear un sistema UNIX partiendo del n�cleo y
programas de diversas fuentes puede resultar dif�cil para los usuarios 
m�s noveles, se crearon las distribuciones con el fin de facilitar esta tarea: 
con una
distribuci�n, usted s�lo tiene que tomar el CD-ROM o los disquetes y
proceder a su instalaci�n para disponer de un sistema completo con programas 
de aplicaci�n incluidos. Como es de esperar, no hay ninguna distribuci�n
``est�ndar''. Hay muchas, cada una con sus ventajas e inconvenientes. 
Hablaremos m�s sobre distribuciones en la Secci�n~\ref{sec-install-distributions}.

% Linux Installation and Getting Started    -*- TeX -*-
% differences.tex
% Copyright (c) 1992-1994 by Matt Welsh <mdw@sunsite.unc.edu>
%
% This file is freely redistributable, but you must preserve this copyright 
% notice on all copies, and it must be distributed only as part of "Linux 
% Installation and Getting Started". This file's use is covered by the 
% copyright for the entire document, in the file "copyright.tex".

% Traduccion realizada por Alfonso Belloso. Envie sus comentarios a:
%            Alfonso Belloso, 2:344/17.2 (FidoNet)
%            alfon@bipv02.bi.ehu.es.es (InterNet)
% Versi�n revisada LIPP 2.0 por Alberto Molina. Comentarios a:
%            alberto@nucle.us.es
%
% Revisi�n 2 16/7/2002 por Francisco Javier Fernandez <serrador@arrakis.es

\section{Diferencias entre {\linux} y otros sistemas operativos}
\label{sec-intro-differences}

Es importante entender las diferencias entre {\linux} y otros sistemas
operativos, tales como MS-DOS, OS/2, y otras implementaciones de UNIX para
ordenador personal. Antes de nada, conviene aclarar que {\linux} puede convivir
felizmente con otros sistemas operativos en la misma m�quina: es decir, Ud.
puede ejecutar MS-DOS y OS/2 en compa��a de {\linux} sobre el mismo sistema sin
problemas. Hay incluso formas de interactuar entre los diversos sistemas
operativos como veremos.

\subsection{�Por qu� usar {\linux}?}

�Por qu� usar {\linux} en lugar de un sistema operativo comercial conocido, bien
probado, y bien documentado? Podr�amos darle miles de razones. Una de las
m�s importantes es, sin embargo, que {\linux} es una excelente elecci�n para
trabajar con UNIX a nivel personal. Si Ud. es un desarrollador de software
UNIX, �por qu� usar MS-DOS en casa? {\linux} le permitir� desarrollar y probar
el software UNIX en su PC, incluyendo aplicaciones de bases de datos y
X~Window. Si es Ud. estudiante, la oportunidad est� en que los sistemas de
su universidad ejecutar�n UNIX. Con {\linux}, podr� ejecutar su propio sistema UNIX
y adaptarlo a sus necesidades. La instalaci�n y uso de {\linux} es tambi�n una
excelente manera de aprender UNIX si no tiene acceso a otras m�quinas UNIX.

Pero no perdamos la vista. {\linux} no es s�lo para los usuarios personales de
UNIX. Es robusto y suficientemente completo para manejar grandes tareas, asi
como necesidades de c�mputo distribuidas. Muchos negocios---especialmente
los peque�os---se est�n cambiando a {\linux} en lugar de otros entornos de
estaci�n de trabajo basados en UNIX. Las universidades encuentran a {\linux}
perfecto para dar cursos de dise�o de sistemas operativos. Grandes
vendedores de software comercial se est�n dando cuenta de las oportunidades
que puede brindar un sistema operativo gratuito.

\subsection{{\linux} frente a MS-DOS}

\index{MS-DOS|(} 
No es raro tener ambos, {\linux} y MS-DOS, en el mismo sistema. Muchos usuarios
de {\linux} conf�an en MS-DOS para aplicaciones tales como procesadores de
texto. Aunque {\linux} proporciona sus propios an�logos para estas
aplicaciones (por ejemplo, \TeX), existen varias razones por las que un
usuario concreto desear�a ejecutar tanto MS-DOS como {\linux}. Si toda su
tesis est� escrita en WordPerfect \NT{El ejemplo del autor se ha quedado
obsoleto: Ya existe un WordPerfect 6.1 nativo para {\linux}} para MS-DOS, puede
no ser capaz de convertirla f�cilmente a \TeX o alg�n otro formato. Hay
muchas aplicaciones comerciales para MS-DOS que no est�n disponibles para
{\linux}, y no hay ninguna raz�n por la que no pueda usar ambos.

Como puede saber, MS-DOS no utiliza completamente la funcionalidad de los
procesadores 80386 y 80486. Por otro lado, {\linux} corre completamente en el
modo protegido del procesador y explota todas las caracter�sticas de
�ste. Puede acceder directamente a toda su memoria disponible (e incluso
m�s all� de la disponible, usando RAM virtual). {\linux} proporciona un
interfaz UNIX completo no disponible bajo MS-DOS---el desarrollo y
adaptaci�n de aplicaciones UNIX bajo {\linux} es cosa f�cil, mientras que, bajo
MS-DOS, Ud. est� limitado a un peque�o subgrupo de la funcionalidad de
programaci�n UNIX. Al ser {\linux} un verdadero sistema UNIX, Ud. no tendr�
estas limitaciones.

Podr�amos debatir los pros y contras de MS-DOS y {\linux} durante p�ginas y
p�ginas. Sin embargo, baste decir que {\linux} y MS-DOS son entidades
completamente diferentes. MS-DOS no es caro (comparado con otros sistemas
operativos comerciales), y tiene un fuerte asentamiento en el mundo de los
PC's. Ning�n otro sistema operativo para PC ha conseguido el nivel de
popularidad de MS-DOS---b�sicamente porque el coste de esos otros sistemas
operativos es inaccesible para la mayor�a de los usuarios de PC's. Muy pocos
usuarios de PC pueden imaginar gastarse 200000 ptas. o m�s solamente en el
sistema operativo. {\linux}, sin embargo, es gratis, y por fin tiene la
oportunidad de decidirse.

Le permitiremos emitir sus propios juicios de {\linux} y MS-DOS basados en sus
expectativas y necesidades. {\linux} no est� destinado a todo el mundo. Si
siempre ha querido tener un sistema UNIX completo en casa, si es alto el coste
de otras implementaciones UNIX para PC, {\linux} puede ser lo que estaba
buscando.

\index{MS-DOS|)}

\subsection{{\linux} frente a otros sistemas operativos}
Est�n surgiendo un gran n�mero de sistemas operativos avanzados en el mundo
del PC. Concretamente, OS/2 de IBM y Windows NT de Microsoft comienzan a
tener popularidad a medida que los usuarios de MS-DOS migran a ellos.

\index{OS/2|(}
\index{Windows NT|(}
Ambos, OS/2 y Windows NT son sistemas operativos completamente multitarea,
muy parecidos a {\linux}. T�cnicamente, OS/2, Windows NT y {\linux} son bastante
similares: Soportan aproximadamente las mismas caracter�sticas en
t�rminos de interfaz de usuario, redes, seguridad, y dem�s. Sin embargo,
la diferencia real entre {\linux} y los otros es el hecho de que {\linux} es una
versi�n de UNIX, y por ello se beneficia de las contribuciones de la
comunidad UNIX en pleno.

�Qu� hace a UNIX tan importante? No s�lo es el sistema operativo m�s
popular para m�quinas multiusuario, tambi�n es la base de la
mayor�a del mundo del software de libre distribuci�n. Si tiene acceso
a Internet, casi todo el software de libre distribuci�n disponible est�
espec�ficamente escrito para sistemas UNIX. (Internet en s� est�
profundamente basada en UNIX.)

Hay muchas implementaciones de UNIX, de muchos vendedores, y ni una sola
organizaci�n es responsable de su distribuci�n.
Hay un gran pulso en la comunidad UNIX por la estandarizaci�n en forma de
sistemas abiertos, pero ninguna corporaci�n controla este dise�o. Por eso,
ning�n vendedor (o, como parece, ning�n hacker\glossary{hacker})
puede adoptar estos est�ndares en una implementaci�n de UNIX.

Por otro lado, OS/2 y Windows NT son sistemas propietarios. El interfaz y
dise�o est�n controlados por una sola corporaci�n, y s�lo esa corporaci�n
puede implementar ese dise�o. (No espere encontrar una versi�n gratis de
OS/2 en un futuro cercano.) De alguna forma, este tipo de organizaci�n es
beneficiosa: establece un est�ndar estricto para la programaci�n y el
interfaz de usuario distinto al encontrado incluso en la comunidad de
sistemas abiertos. 

Varias organizaciones est�n intentando estandarizar el modelo de
programaci�n, pero la tarea es muy dif�cil. {\linux}, en particular, es en su
mayor�a compatible con el estandar POSIX.1 para el interfaz de programaci�n
UNIX. A medida que pase el tiempo, se espera que el sistema se adhiera a
otros estandares, pero la estandarizaci�n no es la etapa primaria en la
comunidad de desarrollo de {\linux}.

\index{OS/2|)}
\index{Windows NT|)}

\subsection{{\linux} frente a otras implementaciones de UNIX}
Hay otras implementaciones de UNIX para el 80386 y 80486. La
arquitectura 80386 se presta al dise�o de UNIX y buen un n�mero de vendedores
han sacado ventaja de este factor.

\index{UNIX!comercial|(}
\index{UNIX!para PCs|(}
Hablando de caracter�sticas, otras implementaciones de UNIX para PC son
bastante similares a {\linux}. Podr� ver que casi todas las versiones
comerciales de UNIX soportan b�sicamente el mismo software, entorno de
programaci�n, y caracter�sticas de red. Sin embargo, hay algunas fuertes
diferencias entre {\linux} y las versiones comerciales de UNIX.

En primer lugar, {\linux} soporta un rango de hardware diferente de las
implementaciones comerciales. En general, {\linux} soporta la mayor�a de
dispositivos hardware conocidos, pero el soporte est� a�n limitado a ese
hardware al que los desarrolladores tengan acceso actualmente. Sin embargo,
los vendedores de UNIX comercial por lo general tienen una base de soporte
m�s amplia, y tienden a soportar m�s hardware, aunque {\linux} no esta tan
lejos de ellos. Cubriremos los requerimientos hardware de {\linux} en la
Secci�n~\ref{sec-intro-hardware}.

\index{{\linux}!estabilidad}
\index{estabilidad}
En lo que concierne a estabilidad y robustez, muchos usuarios han comentado
que {\linux} es al menos tan estable como los sistemas UNIX comerciales. {\linux}
est� a�n en desarrollo, el hecho de trabajar en dos frentes, produce
versiones estables a la vez sin parar el desarrollo.

\index{dinero}
\index{{\linux}!y el coste}
El factor m�s importante a considerar por muchos usuarios es el precio. El
software de {\linux} es gratis, si tiene acceso a Internet (o a otra red de
ordenadores) y puede descarg�rselo. Si no tiene acceso a tales redes, tiene la
opci�n de comprarlo pidi�ndolo por correo en disquetes, cinta o CD-ROM (vea
el Ap�ndice~\ref{app-vendor-num}). 
\index{{\linux}!copiar}
\index{copiar {\linux}}
Por supuesto, Ud. puede copiarse {\linux} de un amigo que puede tener ya el
software, o compartir el coste de comprarlo con alguien m�s. Si planea
instalar {\linux} en un gran n�mero de m�quinas, s�lo necesita comprar una
copia del software---{\linux} no se distribuye con licencia para ``una sola
m�quina''.

El valor de las implementaciones comerciales de UNIX no deber�a ser
rebajado: conjuntamente con el precio del software en s�, uno paga
generalmente por la documentaci�n, el soporte, y una etiqueta de calidad. Estos
factores son muy importantes para grandes instituciones, pero los usuarios
de ordenadores personales pueden no necesitar esos beneficios. En cualquier
caso, muchos negocios y universidades encuentran que ejecutar {\linux} en un
laboratorio con ordenadores baratos es preferible a ejecutar una versi�n
comercial de UNIX en un laboratorio de estaciones de trabajo. {\linux} es capaz
de proporcionar la funcionalidad de una estaci�n de trabajo sobre hardware
de PC a una fracci�n de su coste.

Como un ejemplo del ``mundo real'' sobre el uso de {\linux} dentro de la
comunidad inform�tica, los sistemas {\linux} han viajado hasta los grandes
mares del Pac�fico Norte, encarg�ndose de las telecomunicaciones y an�lisis
de datos para en un buque de investigaci�n oceanogr�fica. Los sistemas {\linux}
se est�n usando en estaciones de investigaci�n en la Ant�rtida. Como ejemplo
m�s mundano, quiz�, varios hospitales est�n usando {\linux} para mantener
registros de pacientes. 

\index{UNIX!implementaciones gratuitas}
\index{UNIX!para PCs}
\index{386BSD}
\index{NetBSD}
Hay otras implementaciones gratuitas o baratas de UNIX para el 386 y 486.
Una de las m�s conocidas es 386BSD, una implementaci�n y adaptaci�n del UNIX
BSD para el 386. 386BSD es comparable a {\linux} en muchos aspectos, pero cual
de ellos es ``mejor'' depende de las necesidades y espectativas personales.
La �nica distinci�n fuerte que se puede hacer es que {\linux} se desarrolla
abiertamente (donde cualquier voluntario puede colaborar en el proceso de
desarrollo), mientras 386BSD se desarrolla dentro de un equipo cerrado de
programadores que mantienen el sistema. Debido a esto, existen diferencias
filos�ficas y de dise�o serias entre los dos proyectos. Los objetivos de los
dos proyectos son completamente distintos: el objetivo de {\linux} es
desarrollar un sistema UNIX completo desde el desconocimiento (y divertirse
mucho en el proceso), y el objetivo de 386BSD es en parte modificar el
c�digo de BSD existente para usarlo en el 386.

NetBSD es otra adaptaci�n de la distribuci�n NET/2 de BSD a un n�mero de
m�quinas, incluyendo el 386. NetBSD tiene una estructura de desarrollo
ligeramente m�s abierta, y es comparable al 386BSD en muchos aspectos.

\index{HURD}
Otro proyecto conocido es HURD, un esfuerzo de la Free Software Foundation
(Fundaci�n de Software Libre) para desarrollar y distribuir una versi�n
gratis de UNIX para muchas plataformas. Contacte con la Free Software
Foundation (la direcci�n se da en el Ap�ndice~\ref{app-gpl-num}) para
obtener m�s informaci�n sobre este proyecto. Al tiempo de escribir este
documento, HURD a�n est� en los primeros pasos de su desarrollo.

\index{Coherent}
\index{Minix}
Tambi�n existen otras versiones baratas de UNIX, como Minix (un
cl�nico de UNIX acad�mico, pero
�til, en el que se basaron los primeros pasos del desarrollo de {\linux}).
Algunas de estas implementaciones son de inter�s en mayor parte acad�mico,
mientras otras son sistemas ya maduros para productividad real.

\index{UNIX!comercial|)}
\index{UNIX!para PCs|)}



% Linux Installation and Getting Started    -*- TeX -*-
% hardware.tex
% Copyright (c) 1992-1994 by Matt Welsh <mdw@sunsite.unc.edu>
%
% This file is freely redistributable, but you must preserve this copyright 
% notice on all copies, and it must be distributed only as part of "Linux 
% Installation and Getting Started". This file's use is covered by the 
% copyright for the entire document, in the file "copyright.tex".
%
% Copyright (c) 1998 by Specialized Systems Consultants Inc. 
% <ligs@ssc.com>
% Traducido por Francisco Javier Frenandez <serrador@arrakis.es>
% Revision 1 6 de julio por Fco. Javier Fern�ndez <serrador@arrakis.es>
%Revisi�n 2 15 julio 2002
%gold
\subsection{Problemas con el hardware}
\label{sec-install-probs-hardware}

\index{hardware!problemas}
\index{instalaci�n!problemas con el hardware}

%The most common form of problem when attempting to install or use Linux
%is an incompatibility with hardware. Even if all of your hardware is supported
%by Linux, a misconfiguration or hardware conflict can sometimes cause
%strange results---your devices may not be detected at boot time, or
%the system may hang. 
El problema m�s com�n cuando se intenta instalar o usar GNU/Linux es una incompatibilidad con el hardware. 
Incluso si todo su hardware est� soportado por GNU/Linux, una configuraci�n err�nea o un conflicto con otro
dispositivo puede algunas veces ocasionar resultados extra�os---los dispositivos pueden no ser detectados al arrancar,
o el sistema se puede colgar.

%It is important to isolate these hardware problems if you suspect 
%that they may be the source of your trouble. 
Es importante aislar estos problemas con el hardware si se sospecha que �stos pueden ser la fuente de sus problemas.
%Comentado en la versi�n original
% In the following sections 
% we will describe some common hardware problems and how to resolve them.

% Linux Installation and Getting Started    -*- TeX -*-
% conflicts.tex
% Copyright (c) 1992-1994 by Matt Welsh <mdw@sunsite.unc.edu>
%
% This file is freely redistributable, but you must preserve this copyright 
% notice on all copies, and it must be distributed only as part of "Linux 
% Installation and Getting Started". This file's use is covered by the 
% copyright for the entire document, in the file "copyright.tex".
%
% Copyright (c) 1998 by Specialized Systems Consultants Inc. 
% <ligs@ssc.com>

%Tradu por Fco. Javier Fern�ndez <serrador@arrakis.es>
%Revisi�n 1 16/7/2002  por Francisco Javier Fernandez

\subparagraph*{Aislando problemas de hardware}
\namedlabel{sec-install-probs-hardware-conflicts}{Conflictos con el hardware}
\index{hardware!conflictos}

Si experimentas alg�n problema que creas que est� relacionado con el hardware, 
lo primero que deber�as hacer es intentar aislar el problema.  
Esto significa  eliminar  todas las posibles variables y (usualmente) 
desmontar el sistema, pieza a pieza, hasta que el componente es aislado.

%If you experience a problem that you believe to be hardware-related, 
%the first thing that you should to do is attempt to isolate the problem.
%This means eliminating all possible variables and (usually) taking the
%system apart, piece-by-piece, until the offending piece of hardware is
%isolated.

Esto no es tan terrible como suena. B�sicamente, se deber� retirar todo
el hardware prescindible del equipo, y entonces determinar cu�l de los
dispositivos est� causando el problema, posiblemente reconectando cada
dispositivo, uno cada vez. Esto quiere decir que se deber� retirar todo el
hardware excepto la unidad de disquettes y la tarjeta de v�deo y por supuesto
el teclado. Incluso los dispositivos aparentemente inocentes como los ratones
pueden causar insospechados problemas a no ser que se les considere 
no esenciales.

%This is not as frightening as it may sound. Basically, you should
%remove all nonessential hardware from your system, and then determine
%which device is causing the trouble---possibly by reinserting each
%device, one at a time. This means that you should remove all hardware
%other than the floppy and video controllers, and of course the
%keyboard. Even innocent-looking devices such as mouse controllers can
%wreak unknown havoc on your peace of mind unless you consider them
%nonessential.

Por ejemplo, digamos que el sistema se cuelga durante la secuencia de 
detecci�n de la placa Ethernet en el arranque. Quiz� pueda hipotetizar
que hay un conflicto con la tarjeta Ethernet en su computadora. La manera
r�pida y f�cil de encontrarlo es extraer la tarjeta Ethernet e intentar
arrancar otra vez. Si todo va bien, entonces  sabe que o (a) la tarjeta 
Ethernet no tiene soporte en Linux (ver p�gina~\pageref{sec-intro-hardware}),
o (b) hay un conflicto de direcci�n o IRQ con la tarjeta.

%For example, let's say that the system hangs during the Ethernet board
%detection sequence at boot time. You might hypothesize that there is a
%conflict or problem with the Ethernet board in your machine. The quick
%and easy way to find out is to pull the Ethernet board, and try
%booting again. If everything goes well, then you know that either (a)
%the Ethernet board is not supported by Linux (see
%P�gina~\pageref{sec-intro-hardware}), or (b) there is an address or IRQ
%conflict with the board.

\index{IRQ}
``�Conflicto de direcci�n o IRQ?'' �Qu� diablos significa esto?
Todos los dispositivos en un computador usan una {\bf IRQ}, o 
{\em Interrupt ReQuest line}, \NT{l�nea de petici�n de interrupci�n}
para decirle al sistema que necesitan
algo hecho. Puedes pensar en la IRQ como un cordel del que el dispositivo tira
cuando necesita que el sistema se haga cargo de  alguna petici�n pendiente.
Si m�s de un dispositivo est� tirando del mismo cordel, el n�cleo no es capaz
de determinar cu�l de los dispositivos necesita su atenci�n. Desastre al instante.


%\index{IRQ}
%``Address or IRQ conflict?'' What on earth does that mean? 
%All devices in your machine use an {\bf IRQ}, or 
%{\em interrupt request line}, to tell the system that they need something
%done on their behalf. You can think of the IRQ as a cord that the device
%tugs when it needs the system to take care of some pending request.
%If more than one
%device is tugging on the same cord, the kernel won't be able to detemine
%which device it needs to service. Instant mayhem.

Entonces, hay que asegurarse de que todos los dispositivos instalados usan
una �nica IRQ. En general la IRQ de un dispositivo puede establecerse mediante
jumpers en la tarjeta; mira la documentaci�n para detalles espec�ficos 
del dispositivo.
Algunos dispositivos no requieren el uso de una IRQ, pero se sugiere
que si hay alguna disponible, se ponga. (Las controladoras SCSI Seagate
ST01 y ST02 son buenos ejemplos).

%Therefore, be sure that all of your installed devices use unique IRQ
%lines. In general, the IRQ for a device can be set by jumpers on the
%card; see the documentation for the particular device for details.
%Some devices do not require the use of an IRQ at all, but it is
%suggested that you configure them to use one if possible. (The Seagate
%ST01 and ST02 SCSI controllers are good examples).

En algunos casos, el n�cleo proporcionado por tu medio de instalaci�n est�
configurado para usar ciertas IRQs para ciertos dispositivos. Por ejemplo, la 
controladora SCSI TMC-950, la controladora de CD-ROM Mitsumi y el driver del bus del rat�n.
Si se quiere usar dos o m�s de estos dispositivos, habr� que instalar primero
{\linux} con s�lo uno de estos dispositivos activados, despu�s recompilar
el n�cleo para cambiar la IRQ  predeterminada de uno de ellos.
(Ver cap�tulo~\ref{chap-sysadm-num} para informaci�n acerca de recompilar el n�cleo.)

%In some cases, the kernel provided on your installation media is configured
%to use certain IRQs for certain devices. For example, on some distributions
%of Linux, the kernel is preconfigured to use IRQ 5 for the TMC-950 SCSI 
%controller, the Mitsumi CD-ROM controller, and the bus mouse driver. 
%If you want to use two or more of these devices, you'll need to first
%install Linux with only one of these devices enabled, then recompile the
%kernel in order to change the default IRQ for one of them.
%(See Chapter~\ref{chap-sysadm-num} for information
%on recompiling the kernel.) 


Otro �rea donde pueden aparecer conflictos de hardware es con los canales DMA
(Direct Memory Access)\NT{acceso directo a memoria}, direcciones de E/S y direcciones de
memoria compartida. Todos estos t�rminos describen mecanismos a trav�s de los cuales el sistema
se comunica con los dispositivos f�sicos. Algunas tarjetas Ethernet, por ejemplo,
usan una direcci�n compartida de memoria as� como una IRQ para comunicarse con el sistema.
Si cualquiera de �stas est� en conflicto con otros dispositivos, entonces el sistema puede comportarse
de manera err�tica.
Deber�as ser capaz de cambiar el canal DMA, las direcciones de E/S o memoria compartida para varios
dispositivos mediante los jumpers \NT{ unas clavijas en la placa} de las tarjetas. (Desafortunadamente, algunos
dispositivos no permiten cambiar estos ajustes.)




%Another area where hardware conflicts can arise is with DMA (direct
%memory access) channels, I/O addresses, and shared memory addresses. 
%All of these terms describe mechanisms through which the system interfaces 
%with hardware devices. Some Ethernet boards, for example, use a shared memory 
%address as well as an IRQ to interface with the system. If any of these
%are in conflict with other devices, then the system may behave unexpectedly.
%You should be able to change the DMA channel, I/O or shared
%memory addresses for your various devices with jumper settings. (Unfortunately,
%some devices don't allow you to change these settings.)

La documentaci�n para varios dispositivos hardware deber�a especificar la IRQ,
el canal DMA, direcci�n E/S o direcci�n de memoria compartida que el dispositivo
usa, y c�mo configurarlo. Otra vez, la manera m�s simple de evitar estos problemas
es deshabilitar temporalmente los dispositivos en conflicto hasta que se tenga
tiempo de determinar la causa del problema.

%The documentation for various hardware devices should specify the IRQ,
%DMA channel, I/O address, or shared memory address that the devices
%use, and how to configure them. Again, the simple way to get around
%these problems is to temporarily disable the conflicting devices until
%you have time to determine the cause of the problem.

En el cuadro se puede ver una lista de  las IRQ y canales DMA utilizados
por varios dispositivos ``est�ndar'' en la mayor�a de sistemas. Casi
todos los sistemas tienen alguno de estos dispositivos, as� que se puede
evitar el poner la IRQ o el DMA de otro dispositivo en conflicto con estos valores.

%The table below is a list of IRQ and DMA channels used by various
%``standard'' devices on most systems. Almost all systems have some of
%these devices, so you should avoid setting the IRQ or DMA of other
%devices in conflict with these values.

\begin{table}\begin{center}
\small\begin{tabular}{|l|l|l|l|}
\hline
Device                     &   I/O address  & IRQ & DMA \\
\hline
{\tt ttyS0} ({\tt COM1})   &   3f8          &  4  &  n/a \\
{\tt ttyS1} ({\tt COM2})   &   2f8          &  3  &  n/a \\
{\tt ttyS2} ({\tt COM3})   &   3e8          &  4  &  n/a \\
{\tt ttyS3} ({\tt COM4})   &   2e8          &  3  &  n/a \\

{\tt lp0} ({\tt LPT1})     &   378 - 37f    &  7  &  n/a \\
{\tt lp1} ({\tt LPT2})     &   278 - 27f    &  5  &  n/a \\

{\tt fd0}, {\tt fd1} (disqueteras 1 y 2) & 3f0 - 3f7 & 6 & 2 \\
{\tt fd2}, {\tt fd3} (disqueteras 3 y 4) & 370 - 377 & 10 & 3 \\
\hline
\end{tabular}\normalsize\rm
\caption{Preajustes por omisi�n de dispositivos est�ndar.}
\label{table-dev-settings}
\end{center}\end{table}

\index{hardware!conflictos|)}

% Linux Installation and Getting Started    -*- TeX -*-
% hd.tex
% Copyright (c) 1992-1994 by Matt Welsh <mdw@sunsite.unc.edu>
%
% This file is freely redistributable, but you must preserve this copyright 
% notice on all copies, and it must be distributed only as part of "Linux 
% Installation and Getting Started". This file's use is covered by the 
% copyright for the entire document, in the file "copyright.tex".
%
% Copyright (c) 1998 by Specialized Systems Consultants Inc. 
% <ligs@ssc.com>
% Traducci�n realizada por Francisco javier Fern�ndez <serrador@arrakis.es>
%Revisi�n 1 por FJFS 
%Gold
\subparagraph*{Problemas reconociendo la controladora de disco}%Problems recognizing hard drive or controller.}
\index{hardware!problemas con el disco duro}

%When Linux boots, you should see a series of messages on your screen such
%as: 
Cuando {\linux} arranca, se deber�a ver una serie de mensajes en la pantalla como:
\begin{tscreen}
Console: colour EGA+ 80x25, 8 virtual consoles \\
Serial driver version 3.96 with no serial options enabled \\
tty00 at 0x03f8 (irq = 4) is a 16450 \\
tty03 at 0x02e8 (irq = 3) is a 16550A \\
lp\_init: lp1 exists (0), using polling driver \\
\ldots
\end{tscreen}
%Here, the kernel is detecting the various hardware devices present on your
%system. At some point, you should see the line
Aqu�, el n�cleo est� detectando los distintos dispositivos hardware presentes en el sistema. En alg�n punto se deber�a ver la l�nea:
\begin{tscreen}
Partition check:
\end{tscreen}
%followed by a list of recognized partitions, for example:
seguida por una lista de las particiones reconocidas, por ejemplo:
\begin{tscreen}
Partition check: \\
\ \ hda: hda1 hda2 \\
\ \ hdb: hdb1 hdb2 hdb3
\end{tscreen}
%If, for some reason, your drives or partitions are not recognized, then
%you will not be able to access them in any way. 
Si por alguna raz�n, las unidades de disco o las particiones no se reconocen, entonces no se podr� acceder
a ellas de ninguna manera.
%There are several things that can cause this to happen:
Hay varias cosas que pueden causar que esto pase:
\begin{itemize}
\item {\bf La controladora del disco duro no est� soportada.}%Hard drive or controller not supported.} If you have a
%hard drive controller (IDE, SCSI, or otherwise) that is not supported
%by Linux, the kernel will not recognize your partitions at boot time.

Si se tiene una controladora de disco (IDE,SCSI, o lo que sea) que no tenga soporte en Linux, el
n�cleo no reconocer� las particiones al arrancar.
\index{disco duro!problemas}

\item {\bf Unidad o controladora configurada incorrectamente.}%Drive or controller improperly configured.}
%Even if your controller is supported by Linux, it may not be
%configured correctly. (This is particularly a problem for SCSI
%controllers. Most non-SCSI controllers should work fine without any
%additional configuration).

Incluso si la controladora est� soportada por Linux, quiz� no se haya configurado apropiadamente. (Este es un
problema particular para las controladoras SCSI. La mayor�a de las controladoras no SCSI deber�an funcionar bien sin
ninguna configuraci�n adicional).
%Refer to the documentation for your hard drive and/or controller. In
%particular, many hard drives need to have a jumper set to be used as a
%slave drive (the second device on either the primary or secondary IDE
%bus). The acid test of this kind of condition is to boot MS-DOS or
%some other operating system that is known to work with your drive and
%controller. If you can access the drive and controller from another
%operating system, then it is not a problem with your hardware
%configuration.
Echa un vistazo a la documentaci�n del disco duro o la controladora. En particular, 
muchos discos duros necesitan tener un jumper puesto para ser usado como unidad esclava 
(el segundo dispositivo en cualquiera del bus IDE primario o secundario)
Una prueba para esta clase de condici�n es arrancar MS-DOS o alg�n otro
sistema operativo que se sepa que funciona con la controladora y la unidad de disco.
Si se puede acceder al disco duro y la controladora desde otro sistema operativo,
entonces no es un problema de la configuraci�n de hardware.
%See P�gina~\pageref{sec-install-probs-hardware-conflicts}, above, for
%information on resolving possible device conflicts, and
%P�gina~\pageref{sec-install-probs-hardware-scsi}, below, for information
%on configuring SCSI devices.

Consulta la p�gina~\pageref{sec-install-probs-hardware-conflicts}, arriba, para
informarte acerca de la posible resoluci�n de conflictos de dispositivos, y la p�gina~\pageref{sec-install-probs-hardware-scsi} m�s abajo, para
m�s informaci�n acerca de la configuraci�n de dispositivos SCSI.

\item {\bf La controladora est� configurada apropiadamente, pero no es detectada.}%Controller properly configured, but not detected.}
%Some BIOS-less SCSI controllers require the user to specify
%information about the controller at boot time.  A description of how
%to force hardware detection for these controllers begins on
%P�gina~\pageref{sec-install-probs-hardware-scsi}.

Algunas BIOS de las controladoras SCSI requieren que el usuario especifique informaci�n
acerca de la controladora al inicio. Hay una descripci�n de c�mo
forzar la detecci�n de hardware para estas controladoras en la
p�gina~\pageref{sec-install-probs-hardware-scsi}.

\item {\bf No se reconoce la geometr�a del disco.} %Hard drive geometry not recognized.} Some systems, like
%the IBM PS/ValuePoint, do not store hard drive geometry information in
%the CMOS memory, where Linux expects to find it. Also, certain SCSI
%controllers need to be told where to find drive geometry in order for
%Linux to recognize the layout of your drive.

Algunos sistemas como los IBM PS/Valuepoint, no guardan la informaci�n de la geometr�a del disco duro en la memoria CMOS,
donde Linux espera encontrarla. Tambi�n ciertas controladoras SCSI necesitan que se las diga expl�citamente d�nde
encontrar la geometr�a de la unidad para que Linux reconozca la disposici�n del disco.

%Most distributions provide a bootup option to specify the 
%drive geometry. In general, when booting the installation
%media, you can specify the drive geometry at the LILO boot indicador de �rdenes with
%a command such as:
Muchas distribuciones proporcionan una opci�n de arranque para especificar la geometr�a del disco.
En general, cuando se arranca el medio de instalaci�n, se puede especificar la geometr�a de la unidad en
el indicador de LILO con una orden como:
\begin{tscreen}
boot: {\em linux hd=\cparam{cilindros},\cparam{cabezas},\cparam{sectores}}
\end{tscreen}
%where \cparam{cylinders}, \cparam{heads}, and \cparam{sectors} correspond
%to the number of cylinders, heads, and sectors per track for your hard
%drive. 
donde \cparam{cilindros}, \cparam{cabezas}, y \cparam{sectores} corresponden
al n�mero de cilindros, cabezas y sectores por pista del disco duro.

%After installing Linux, you will be able to install LILO, allowing you
%to boot from the hard drive. At that time, you can specify the drive
%geometry to LILO, making it unnecessary to enter the drive geometry
%each time you boot. See Chapter~\ref{chap-sysadm-num} for more
%information about LILO.

Tras instalar {\linux}, deber� instalar LILO, permiti�ndole arrancar
desde el disco duro. En este momento, se puede especificar la geometr�a de la unidad a LILO,
haciendo innecesario introducir la geometr�a del disco cada vez que arranca. Consulta el
Cap�tulo~\ref{chap-sysadm-num} para m�s informaci�n acerca de LILO.
\end{itemize}

\index{hardware!problemas con el disco duro}

% Linux Installation and Getting Started    -*- TeX -*-
% scsi.tex
% Copyright (c) 1992-1994 by Matt Welsh <mdw@sunsite.unc.edu>
%
% This file is freely redistributable, but you must preserve this copyright 
% notice on all copies, and it must be distributed only as part of "Linux 
% Installation and Getting Started". This file's use is covered by the 
% copyright for the entire document, in the file "copyright.tex".
%
% Copyright (c) 1998 by Specialized Systems Consultants Inc. 
% <ligs@ssc.com>
%Traducido por Francisco Javier Fernandez <serrador@arrakis.es>
%Revisado el 6 de julio de 2002 por Francisco Javier Fern�ndez
% Revisi�n 2 16 de julio 2002 por Francisco Javier Fernandez
%gold

\subparagraph*{Problemas con las controladoras y los dispositivos SCSI} %Problems with SCSI controllers and devices.}
\namedlabel{sec-install-probs-hardware-scsi}{}
\index{hardware!problemas con SCSI}
\index{SCSI!problemas}
%Presented here are some of the most common problems with SCSI
%controllers and devices like CD-ROMs, hard drives, and tape drives. If
%you have problems getting Linux to recognize your drive or controller,
%read on.

Aqu� se presentan algunos de los problemas m�s comunes con las controladoras SCSI
y los dispositivos como CD-ROMs, discos duros, y unidades de cinta. Si
se tiene alg�n problema con {\linux} reconociendo un disco o controladora, siga leyendo.

%The Linux SCSI HOWTO (see App�ndice~\ref{app-sources-num}) contains much useful
%information on SCSI devices in addition to that listed here. SCSI can be
%particularly tricky to configure at times.

El COMO de Linux SCSI (ver Ap�ndice~\ref{app-sources-num}) contiene mucha informaci�n
�til acerca de dispositivos SCSI en adici�n de  lo que se muestra aqu�. SCSI puede ser dif�cil de 
configurar a veces.


\begin{itemize}

\item {\bf Un dispositivo SCSI se detecta en todos los posibles IDs.}
%A SCSI device is detected at all possible IDs.} This is caused
%by strapping the device to the same address as the controller. You need to
%change the jumper settings so that the drive uses a different address than
%the controller.

Esto es causado al poner el dispositivo con el mismo identificador que la controladora. Es necesario cambiar el
ajuste del jumper para que el dispositivo use una direcci�n diferente que la controladora.


\item {\bf Linux informa de errores, incluso si se sabe que los dispositivos est�n libres de errores.} 
%Linux reports sense errors, even if the devices are known to be
%error-free.} This can be caused by bad cables or bad termination. If
%your SCSI bus is not terminated at both ends, you may have errors
%accessing SCSI devices. When in doubt, always check your cables.

Esto puede ser causado por cables defectuosos o de baja calidad o una mala terminaci�n de la cadena SCSI.
Si el bus SCSI no est� terminado a ambos extremos, se pueden producir errores accediendo a los dispositivos SCSI.
Si se tiene alguna duda, {\em siempre revise los cables}.

\item {\bf Los dispositivos SCSI informan de errores ``timeout''.}
%SCSI devices report timeout errors.} This is usually caused by 

Los errores de tiempo de espera agotado, normalmente son producidos por un conflicto con una IRQ, DMA o direcci�n de dispositivo. 
Revisa las interrupciones de la controladora, a ver si est�n en su sitio.
%a conflict with IRQ, DMA, or device addresses. Also check that interrupts
%are enabled correctly on your controller.

\item {\bf Las controladoras SCSI que usan BIOS no son detectadas.}
%SCSI controllers that use BIOS are not detected.} Detection of
%controllers that use BIOS will fail if the BIOS is disabled, or if
%your controller's signature is not recognized by the kernel. See the
%Linux SCSI HOWTO, available from the sources in
%App�ndice~\ref{app-sources-num}, for more information about this.

La detecci�n de las controladoras que usan BIOS fallar� si la BIOS est� deshabilitada, o si
la firma del controlador no la reconoce el n�cleo. Consulta el C�MO Linux SCSI, disponible
desde las fuentes de informaci�n disponibles en el Ap�ndice~\ref{app-sources-num}, para m�s informaci�n acerca de esto.

\item {\bf Las controladoras que usan  memoria de E/S mapeada no funcionan.} 
%Controllers using memory mapped I/O do not work.} This is caused
%when the memory-mapped I/O ports are incorrectly cached. Either mark the
%board's address space as uncacheable in the XCMOS settings, or disable
%cache altogether.

Esto ocurre cuando los puertos de E/S mapeados a memoria se cachean incorrectamente.
Hay dos soluciones: una es marcar el espacio de direccionamiento de la tarjeta como 
no cacheable en los ajustes de la CMOS. La segunda soluci�n es deshabilitar toda la cach�.

\item {\bf Mientras se particiona, se obtiene una advertencia tipo ``cylinders $>$ 1024'', o no se puede
iniciar desde una partici�n usando cilindros numerados por encima del 1023.}
%When partitioning, you get a warning that ``cylinders $>$ 1024'', or
%you are unable to boot from a partition using cylinders numbered above 1023.}

%BIOS limits the number of cylinders to 1024, and any partition using
%cylinders numbered above this won't be accessible from the BIOS. As far as
%Linux is concerned, this affects only booting; once the system has booted
%you should be able to access the partition. Your options are to either 
%boot Linux from a boot floppy, or boot from a partition using 
%cylinders numbered below 1024. 

La BIOS limita el numero de cilindros a 1024, y cualquier partici�n que use cilindros numerados por encima
de esto no ser� accesible por la BIOS. Esto s�lo afecta a Linux al arranque; una vez que el sistema ha arrancado
se podr� acceder a la partici�n. Las opciones son o arrancar Linux desde un disquete, o arrancar desde
una partici�n usando cilindros por debajo del 1024.

\item {\bf Al arrancar no se reconocen unidades de CD-ROM u otros dispositivos extra�bles.} %CD-ROM drive or other removeable media devices are not recognized
%at boot time.} Try booting with a CD-ROM (or disk) in the drive. This is 
%necessary for some devices. 

Intenta arrancando con un CD-ROM (o disco) en la unidad. Esto es necesario en algunos dispositivos.
\end{itemize}

%If your SCSI controller is not recognized, you may need to 
%force hardware detection at boot time. This is particularly important for
%BIOS-less SCSI controllers. Most distributions allow you to
%specify the controller IRQ and shared memory address when booting the
%installation media. For example, if you are using a TMC-8xx controller,
%you may be able to enter

Si no se reconoce su controladora SCSI, quiz� se necesite forzar la detecci�n de hardware al arrancar. Esto es particularmente importante
para las controladoras SCSI que carecen de BIOS. Muchas distribuciones permiten especificar la IRQ de la controladora y
la direcci�n de memoria compartida cuando se arranca el medio de instalaci�n. Por ejemplo, si se usa una controladora TMC-8xx,
se podr� introducir:

\begin{tscreen}
boot: linux tmx8xx=\cparam{interrupci�n},\cparam{direcci�n}
\end {tscreen}
%at the LILO boot indicador de �rdenes, where \textsl{interrupt} is the IRQ of
%controller, and \textsl{memory-address} is the shared memory
%address. Whether or not this is possible depends on the distribution
%of Linux; consult your documentation for details.
en el indicador de inicio de LILO, donde \textsl{interrupci�n} es la IRQ de la controladora, y 
\textsl{direcci�n} es la direcci�n de memoria compartida. Esto es o no posible dependiendo de la 
distribuci�n de {\linux}; consulta la documentaci�n para m�s detalles.

\index{hardware!problemas con SCSI}
\index{SCSI!problemas}


\index{hardware!problemas}
\index{instalaci�n!problemas con el hardware}

% Linux Installation and Getting Started    -*- TeX -*-
% sources.tex
% Copyright (c) 1992-1994 by Matt Welsh <mdw@sunsite.unc.edu>
%
% This file is freely redistributable, but you must preserve this copyright 
% notice on all copies, and it must be distributed only as part of "Linux 
% Installation and Getting Started". This file's use is covered by the 
% copyright for the entire document, in the file "copyright.tex".
%
% Copyright (c) 1998 by Specialized Systems Consultants Inc. 
% <ligs@ssc.com>
%
% Traduccion realizada por Juan Jose Amor. Envie sus comentarios a:
%            Juan Jose Amor, 2:341/12.19 (FidoNet)
%            jjamor@gedeon.ls.fi.upm.es (InterNet)
%
% Versi�n revisada LIPP 2.0 por Alberto Molina. Comentarios a:
%            alberto@nucle.us.es
% Revisi�n 2 LIPP 2.0 por francisco Javier fern�ndez. Comentarios a serrador@arrakis.es


\section{Fuentes de informaci�n sobre {\linux}}
\namedlabel{sec-intro-sources}{Fuentes de informaci�n sobre Linux}

\index{Linux!fuentes de informaci�n}
\index{ayuda, consiguiendo}
Como podr� imaginar, adem�s de este libro, hay muchas otras fuentes de
informaci�n sobre Linux. Concretamente, hay numerosos libros sobre UNIX en
general, que recomendamos a aquellos lectores que no tengan experiencia
previa con UNIX. Si somos nuevos en UNIX, lo m�s indicado es leer uno de
estos libros antes de meternos en la ``peligrosa selva'' de {\linux}. Un buen
comienzo puede ser el libro {\em Learning the UNIX Operating System}, de
Grace Todino y John Strang.

Casi todas las fuentes de informaci�n sobre {\linux} est�n disponibles
principalmente de forma electr�nica. Esto es, deber� tener acceso a
una red, como Internet, USENET o Fidonet, con el fin de obtener la
documentaci�n. Un buen sitio para empezar es {\tt
  www.linuxresources.com} (ver Ap�ndice~\ref{app-info}). Si no tiene
acceso a ninguna red, siempre puede
encontrar la forma de obtener copias impresas en disquetes o CDROM de
los libros.

\subsection{Documentaci�n {\it en-l�nea}}
\index{documentaci�n!en l�nea}

Si tiene acceso a Internet, encontrar� variada documentaci�n en muchos
servidores de FTP del mundo. Si no tiene acceso directo a Internet, a�n
puede obtener los documentos: muchos distribuidores de {\linux} en CDROM
incluyen toda o casi toda la documentaci�n existente en la red. Adem�s,
se suelen distribuir por redes diferentes como Fidonet o Compuserve. Y si
tiene acceso �nicamente al correo en Internet, puede obtener ficheros de
servidores FTP sin m�s que usar un servidor de {\tt ftpmail}. Vea el
ap�ndice~\ref{app-ftplist-num} para m�s informaci�n.

Hay gran cantidad de servidores FTP que distribuyen software y
documentaci�n de {\linux}. En el ap�ndice~\ref{app-ftplist-num}
encontrar� una lista con servidores conocidos. Con el fin de reducir el
tr�fico de red, deber�a utilizar el servidor que le quede m�s
cercano\NT{Vea el ap�ndice ~\ref{app-tldpes} para localizar una
lista de ftps espa�oles} geogr�ficamente.

El Ap�ndice~\ref{app-sources-num} incluye una lista de algunos de los
documentos sobre {\linux} que se encuentran disponibles por FTP
an�nimo. Los nombres de los ficheros pueden no ser los mismos en todos
los servidores, pero suelen estar en el directorio {\tt docs} dentro
del directorio que dediquen a {\linux}. Por ejemplo, en {\tt
sunsite.unc.edu} los ficheros de {\linux} est�n en {\tt /pub/linux} y la
documentaci�n en {\tt /pub/linux/docs}.

\index{documentaci�n!en linea!HOWTO, documentos}
\index{documentaci�n!en linea!FAQ}
\index{HOWTO, documentos}
\index{FAQ}
Algunos documentos que puede encontrar son las {\em {\linux} FAQ}, una
colecci�n de FAQ \glossary{FAQ}
sobre {\linux}; los documentos {\em HOWTO\/}\glossary{HOWTO}, dedicados a
aspectos espec�ficos del sistema, como la configuraci�n de
impresoras y spoolers ({\em Printing HOWTO\/}), tarjetas Ethernet
({\em Ethernet HOWTO}) y las {\em {\linux} META-FAQ}, que es una lista de
las fuentes de informaci�n disponibles en Internet.

Algunos documentos se env�an regularmente a uno o m�s grupos USENET
sobre {\linux}. No deje de leer la Secci�n~\ref{sec-intro-usenet} sobre el
tema de las noticias.

\subsection{{\linux} en el WWW}

La p�gina inicial de la documentaci�n de {\linux} en el Web se encuentra
en la direcci�n URL
\begin{tscreen}
http://sunsite.unc.edu/mdw/linux.html
\end{tscreen}
Desde esta p�gina puede accederse a los {\it HOWTO}s y otros
documentos en formato HTML. Tambi�n se encuentran enlaces a otros
servidores de inter�s como {\tt ssc.com}, p�gina inicial de la
revista  {\em Linux Journal}, de periodicidad mensual. Se puede
encontrar en {\tt http://www.ssc.com/}.

\subsection{Libros y otras publicaciones}
\index{documentaci�n!libros}

En este momento, hay algunos trabajos publicados sobre {\linux}.
Principalmente, los libros del Proyecto de Documentaci�n de Linux (LDP),
que se lleva a cabo mediante Internet para escribir y distribuir una
colecci�n de manuales para {\linux}. Estos manuales son an�logos a los que
se publican junto con versiones comerciales de UNIX: tratan la instalaci�n
y puesta en marcha, programaci�n, trabajo en red, asuntos del n�cleo y
muchas cosas m�s.

\index{documentaci�n!Linux Documentation Project}
\index{Linux Documentation Project}
Los manuales del LDP\glossary{LDP} se
encuentran disponibles mediante FTP an�nimo en Internet, as� como
por correo a trav�s de algunos comercios. En el
ap�ndice~\ref{app-sources-num} se enumeran los manuales disponibles y
c�mo conseguirlos.

Gran cantidad de editoriales como MIT:Press, Digital Press, O'Reilly
\& Associates y SAMS han dado el salto a {\linux}. Se pueden encontrar en
bibliotecas especializadas o en la p�gina web de SSC en {\tt
  http://www.ssc.com}, incluso tambien el la p�gina web de la revista
{\em Linux Journal}, {\tt http://www.linuxjournal.com}.

No hay muchos m�s libros que traten el tema particular de {\linux}. Sin
embargo, s� que hay numerosos libros sobre UNIX en general que
normalmente son aplicables a {\linux}, como aquellos sobre c�mo utilizar
o programar sobre el sistema UNIX, ya que {\linux} no difiere mucho en su
interfaz con el usuario o programador. En resumen, lo que quiera saber
sobre el uso y programaci�n de {\linux} lo encontrar� en los libros sobre
UNIX. De hecho, este libro es un suplemento a los libros
disponibles de UNIX. Aqu�, se presentan los detalles espec�ficos de
{\linux} m�s importantes y esperamos que busque en otros lugares para
informaci�n m�s detallada.

Armado con buenos libros sobre UNIX y este libro, deber�a ser capaz
de enfrentarse a cualquier problema. Encontrar� los nombres de
algunos libros recomendados sobre UNIX en el
ap�ndice~\ref{app-sources-num}.

Tambi�n existe un {\it magazine} mensual sobre {\linux}, el {\em {\linux}
Journal}. Se distribuye por todo el mundo y es una excelente manera de
mantenerse al d�a en este tema, sobre todo si no se tiene acceso a
USENET. En el ap�ndice~\ref{app-sources-num} encontrar� informaci�n
sobre c�mo suscribirse a esta publicaci�n.

\subsection{Grupos de Noticias USENET} 
\namedlabel{sec-intro-usenet}{Grupos de NEWS USENET}
\index{USENET!grupos de noticias relacionadas con {\linux}}

``USENET'' es un foro mundial de art�culos electr�nicos organizado en
``grupos'', o sea, �reas de discusiones relacionadas con cada tema
concreto. Buena parte del desarrollo de {\linux} ha sido a trav�s de Internet
y USENET, con lo que no es extra�o que existan bastantes grupos que traten
el tema.

Inicialmente, el grupo sobre {\linux} era {\tt alt.os.linux}, y se cre� para
tratar aqu� las cuestiones que sobre {\linux} abundaban ya en {\tt
comp.os.minix} y varias listas de correo. El tr�fico en el grupo de {\linux}
fue creciendo lo suficiente como para permitirse el paso a la jerarqu�a
{\tt comp}, en Febrero de 1992.

{\tt comp.os.linux} se ha convertido en un grupo de News muy conocido,
m�s que cualquier otro de {\tt comp.os}. En Diciembre del 92 se vot�
la creaci�n del grupo {\tt comp.os.linux.announce} para reducir el
tr�fico de {\tt comp.os.linux}. En Julio de 1993 se parti� este grupo
de forma definitiva en la jerarqu�a que hoy existe.

Si no tiene acceso a USENET, pero s� puede usar el correo
electr�nico, existen pasarelas de correo a noticias disponibles para cada uno
de los grupos siguientes.

\begin{dispitems}
\ditem {{\tt comp.os.linux.advocacy }}
Un grupo para la discusi�n sobre los beneficios de {\linux} respecto a
otros sistemas operativos.

\ditem {{\tt comp.os.linux.alpha }}
Se debe usar para todas las discusiones relacionadas con la compra,
instalaci�n, ejecuci�n, mantenimiento y desarrollo de los sistemas
{\linux} basados el el procesador Alpha de Digital.

\ditem {{\tt comp.os.linux.announce}}
{\tt comp.os.linux.announce} es un grupo moderado, pensado para anuncios
importantes respecto a {\linux} (como informes sobre errores detectados,
lanzamiento de parches, etc). Si quiere leer grupos de {\linux}, empiece por
�ste. Los art�culos que aqu� se publican no son reenviados a
ning�n otro grupo normalmente. En �l se pueden encontrar adem�s muchos
art�culos que se env�an peri�dicamente, incluyendo documentos ya
mencionados como los {\it HOWTO}s.

Los env�os al grupo deben ser aceptados por los moderadores, Matt
Welsh y Lars Wirzenius. Si quiere enviar algo, normalmente basta con
que lo ordene a su software de noticias. Este software se ocupar� de
enviar el art�culo a los moderadores para que lo acepten. Sin embargo,
si su sistema no est� correctamente configurado, puede enviarlo
directamente a la direcci�n de correo {\tt linux-announce@tc.cornell.edu}.

\ditem {{\tt comp.os.linux.answers }}
Para enviar FAQs, {\em HOWTO}s, READMEs y otros documentos que
respondan dudas sobre {\linux}. Esto ayudar� a disminuir el tr�fico de
otros grupos c.o.l.* y dejar� libre comp.os.linux.announce para
verdaderos anuncios.

\ditem {{\tt comp.os.linux.development.apps}}
Un grupo de noticias sin moderar, para preguntas y discusiones acerca
de aplicaciones para {\linux} y sobre la migraci�n de aplicaciones a {\linux}.

\ditem {{\tt comp.os.linux.development.system}}
Un grupo de noticias sin moderar, para discusiones sobre el desarrollo
de todo lo relacionado con el n�cleo, controladores de dispositivos y
m�dulos cargables para el sistema {\linux}.

\ditem {{\tt comp.os.linux.hardware }}
Este grupo de noticias es para preguntas y discusiones espec�ficas
sobre alg�n componenete hardware, por ejemplo: ``�Puede este sistema
ejecutar {\linux}?'',�C�mo puedo utilizar esta unidad de disco con
{\linux}?'', etc.

\ditem {{\tt comp.os.linux.m68k}}
De inter�s y para el desarrollo de la migraci�n de {\linux} de la
arquitectura Motorola 680x0.

\ditem {{\tt comp.os.linux.misc }}
Todas las discusiones que no se ajustan a otros grupos de noticias de
{\linux} tienen aqu� su sitio, adem�s de otros ``discursos'' y
discusiones no t�cnicas.

\ditem {{\tt comp.os.linux.networking}}
Discusiones relacionadas con redes y comunicaciones, incluyendo
tarjetas Ethernet, SLIP y PPP.

\ditem {{\tt comp.os.linux.setup}}
Preguntas y discusiones acerca de la instalaci�n de {\linux} y su
administraci�n. 

\ditem {{\tt comp.os.linux.x}}
Discusiones acerca de las caracter�sticas �nicas de X Window para
{\linux}, incluyendo servidores, clientes, fuentes y bibliotecas.
\end{dispitems}

Esta lista no est� completa. Nuevos grupos se crean cuando se detecta
la necesidad de una divisi�n o subdivisi�n, adem�s de los grupos linux
que hay en otras jerarqu�as \NT{Por ejemplo en {\tt es.comp.os.linux}}.

\subsection{Listas de correo en Internet}
\index{Internet!listas de correo}
\index{listas de correo}
\index{documentaci�n!en linea!listas de correo}
Si tiene acceso al correo electr�nico de Internet, puede aun participar en
las listas de correo aunque no tenga acceso a USENET. A estas listas de
correo puede apuntarse incluso sin tener acceso alguno a Internet, gracias a
las pasarelas que ofrecen otros servicios, como UUCP, FidoNET o CompuServe.

Para m�s informaci�n sobre las listas de correo de {\linux}, env�e un
mensaje de correo a:
\begin{verbatim}
majordomo@vger.rutgers.edu
\end{verbatim}
\label{sec-source-mailing-list}
Con una l�nea en el cuerpo de mensaje con la palabra {\tt help}, se le
responder� con un mensaje que describe c�mo subscribirse y darse de
baja en varias listas de correo. La palabra {\tt lists} en una l�nea
hace que se nos responda con las listas de correo accesibles a trav�s
del servidor {\tt majordomo.vger.rutgers.edu}.

Hay adem�s, varias listas de correo de {\linux} espec�ficas. El mejor camino
para encontrarlas es leer los anuncios aparecidos en USENET, y el
listado de ``listas de correo'' disponible peri�dicamente en el grupo {\tt
news.answers.}

% Linux Installation and Getting Started    -*- TeX -*-
% help.tex
% Copyright (c) 1992-1994 by Matt Welsh <mdw@sunsite.unc.edu>
%
% This file is freely redistributable, but you must preserve this copyright 
% notice on all copies, and it must be distributed only as part of "Linux 
% Installation and Getting Started". This file's use is covered by the 
% copyright for the entire document, in the file "copyright.tex".

% Traduccion realizada por Juan Jose Amor. Envie sus comentarios a:
%            Juan Jose Amor, 2:341/12.19 (FidoNet)
%            jjamor@gedeon.ls.fi.upm.es (InterNet)
%
% Versi�n revisada LIPP 2.0 por Alberto Molina. Comentarios a:
%            alberto@nucle.us.es
%
%
%Revisi�n 1 por Francisco javier fernandez serrador <serrador2arrakis.es>
%gold
\section{C�mo obtener ayuda}
\namedlabel{sec-intro-help}{Conseguir ayuda}

Indudablemente, necesitar� cierta ayuda durante sus primeras aventuras
en el mundo de {\linux}. Aqu� veremos algunas indicaciones sobre c�mo
obtener esa ayuda.

\index{ayuda!consiguiendo|(}
La forma m�s inmediata de buscar ayuda es mediante las listas de
correo y grupos de USENET que se mantienen en Internet, tal como
explicamos en la Secci�n~\ref{sec-intro-sources}. Si no tiene acceso a
estas fuentes, puede encontrar ayuda en otros servicios on-line como los
BBS's o Compuserve. Tambi�n encontrar� ayuda dentro de {\em Linux
  Journal}, en la p�gina {\tt http://www.linuxjournal.com/techsup.html}.

\index{servicio t�cnico!comercial}
\index{ayuda!comercial}
\index{linux!servicio t�cnico comercial}
Tambi�n hay ciertas empresas que dan servicio t�cnico comercial
de {\linux}. Esto le permitir� pedir ayuda a los t�cnicos a cambio del
pago de una cuota. El ap�ndice~\ref{app-vendor-num} contiene una lista
de comerciantes de {\linux}, alguno de los cuales ofrece soporte
t�cnico. Sin embargo, si tiene acceso a USENET y al correo de
Internet, ver� que puede obtener servicio t�cnico de calidad y gratuito.

Lo que sigue son sugerencias que le hacemos para mejorar su
experiencia con {\linux} y garantizarle mayor probabilidad de �xito para
encontrar soluciones a los problemas que se le presenten.

{\em Consultar primero toda la documentaci�n disponible.}

Lo primero que debe hacer cuando encuentre un problema es leerse la
documentaci�n que se lista en la Secci�n~\ref{sec-intro-sources} y el
Ap�ndice~\ref{app-sources-num}. Estos documentos fueron laboriosamente
escritos para gente como usted---gente que necesitaba ayuda sobre
{\linux}. Los libros sobre Unix tambi�n se pueden usar para resolver
dudas de {\linux}.

Frecuentemente, y por dif�cil que parezca, se encuentran respuestas a
muchos problemas.

Si se tiene acceso a las {\it noticias} de USENET o a listas de correo
relacionadas con {\linux}, se debe {\em leer} su contenido antes de
poner una pregunta sobre sus problemas. Muchas veces hay problemas
comunes que no se encuentran f�cilmente en los libros pero que tienen
frecuente respuesta en los grupos de USENET o similar. Enviar la
pregunta a los grupos sin leerlos antes puede ser una p�rdida de tiempo.

{\em Aprenda a apreciar las cosas hechas por uno mismo.}

En la mayor�a de los casos se recomienda investigar por cuenta de uno
mismo antes de pedir ayuda al exterior. Recuerde que {\linux} no es un
sistema comercial y puede hacer con �l lo que quiera (modificarlo,
etc). Si aprende a hacerse las cosas por s� mismo, su experiencia le
permitir� llegar a ser, tal vez, uno de los famosos ``gur�s'' de {\linux}.

{\em Mantenga la calma.\/} 

Hay que evitar la desesperaci�n a pesar de todos los problemas. No se ha
o�do a�n ninguna historia de alguien que, en un ataque de ira, borrase
todos sus discos de {\linux} con un fuerte im�n. Los mismos autores se han
desahogado a pu�etazos con almohadas o similares en arrebatos de este
tipo. Hay que esperar un poco a que las distribuciones de {\linux} se hagan
m�s f�ciles a�n de instalar, aunque ya lo son m�s que casi cualquier
otro Unix comercial.

{\em Evite poner preguntas en USENET demasiado pronto.\/}

\index{USENET!poniendo mensajes}
Mucha gente comete el error de pedir ayuda en USENET demasiado
pronto. Cuando encuentre alg�n problema, no se vaya de inmediato al
terminal m�s cercano (insistimos, {\em no lo haga}) para escribir su
duda en un grupo de noticias de {\linux}. Es mejor que intente antes
resolverlo pues muchas veces es debido al nerviosismo inicial y se
puede caer en el error de preguntar cosas demasiado triviales. Vamos,
que si su PC no se enciende, mire antes a ver si est� enchufado.

{\em Si escribe su pregunta en USENET, haga que merezca la pena.\/}

Finalmente, puede que se vea obligado a pedir ayuda a los foros
dedicados a {\linux}, bien mediante listas de correo o con USENET. Cuando
redacte su pregunta, recuerde que la gente que va a leerla no est� ah�
para ayudarle. As� que trate de ser correcto en el trato y lo m�s
descriptivo posible.

�C�mo puede conseguir esto? En primer lugar, debe incluir toda la
informaci�n acerca de su sistema y el problema que crea relevante. Una
escueta pregunta como ``no puedo conseguir que el correo electr�nico
funcione'' dif�cilmente ser� contestada, a menos que incluya
informaci�n acerca de su sistema, qu� software utiliza, qu� ha
intentado hacer para resolverlo y qu� resultados ha
obtenido. Adem�s, suele ser interesante a�adir informaci�n m�s
general, como qu� versi�n del sistema utiliza (del n�cleo y
distribuci�n), as� como un peque�o resumen de su hardware. Pero
tampoco exagere incluyendo su tipo de monitor, por ejemplo, cuando su
problema sea con el software de red.

\index{ayuda!consiguiendo|)}

%\end{document}
