% {\linux} Installation and Getting Started    -*- TeX -*-
% about.tex
% Copyright (c) 1992-1994 by Matt Welsh <mdw@sunsite.unc.edu>
%
% This file is freely redistributable, but you must preserve this copyright 
% notice on all copies, and it must be distributed only as part of "{\linux} 
% Installation and Getting Started". This file's use is covered by the 
% copyright for the entire document, in the file "copyright.tex".
%
% Copyright (c) 1998 by Specialized Systems Consultants Inc. 
% <ligs@ssc.com>
%Revisi�n 1 por Francisco Javier Fernande <serrador@arrakis.es>
%Gold


\section{Acerca de este libro}
\markboth{Introducci�n a {\linux}}{Acerca de este libro}

Este libro es una gu�a para la instalaci�n y el manejo b�sico de {\linux}.
Su finalidad es conseguir que los nuevos usuarios se pongan en marcha 
reuniendo la mayor cantidad posible de contenido relevante en un s�lo libro. En 
lugar de cubrir f�tiles detalles t�cnicos que tienden a cambiar debido al 
r�pido desarrollo de {\linux}, le damos unos conocimientos de base claros y 
simples, que le permitan seguir avanzando por s� mismo.

Instalar y utilizar {\linux} no es dif�cil. Sin embargo, al igual que con
cualquier otra implementaci�n de Unix, para ponerlo todo a funcionar a 
menudo hace falta algo de mano izquierda. Esperamos que este libro le suba al 
tren de {\linux} y le muestre lo grande que puede llegar a ser un sistema 
operativo.

En este libro cubrimos los siguientes asuntos:
\begin{itemize}
\item �Qu� es {\linux}? El dise�o y filosof�a de este sistema operativo �nico,
      y lo que {\linux} puede hacer por usted.

\item Detalles sobre c�mo ejecutar {\linux}, adem�s de sugerencias sobre la 
      configuraci�n de hardware recomendada.


%\item How to obtain and install {\linux}. There are many distributions of the 
%{\linux} software. We present a general discussion of {\linux} software
%distributions, how to obtain them, and general instructions for installing
%the software (which should apply to any distribution).

%\ifodd\igsslack
\item Instrucciones espec�ficas para instalar diversas distribuciones de {\linux},
      incluyendo Debian, Red Hat y Slackware.
%\fi

\item Un breve tutorial de introducci�n a Unix para usuarios sin 
      experiencia previa en Unix. Este tutorial proporciona suficiente material
      a los novatos como para que puedan moverse solos por el sistema.


\item Una introducci�n a la administraci�n del sistema bajo {\linux}. Se cubren
      las tareas m�s importantes que los admistradores de  {\linux} necesitan
      llevar a cabo como la creaci�n de cuentas de usuario o el manejo de 
      los ficheros del sistema.

\item Informaci�n sobre la configuraci�n de las caracter�sticas avanzadas de 
      {\linux}, como el sistema X-Window, la red TCP/IP, las noticias
      o el correo electr�nico.

\end{itemize}

Este libro est� dirigido al usuario de ordenadores personales que desee
empezar con {\linux}. No damos por sentado que se tenga experiencia previa con 
Unix pero s� esperamos que, durante su aprendizaje, el principiante se 
remita a otras fuentes de informaci�n. Proporcionamos una lista de 
referencias �tiles en el Ap�ndice~\ref{app-sources-num} para quienes no 
est�n familiarizados con Unix.
En general, se supone que este libro ha de leerse junto con otro libro de 
conceptos b�sicos de Unix.

