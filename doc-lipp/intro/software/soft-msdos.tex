% Linux Installation and Getting Started    -*- TeX -*-
% soft-msdos.tex
% Copyright (c) 1992-1994 by Matt Welsh <mdw@sunsite.unc.edu>
%
% This file is freely redistributable, but you must preserve this copyright 
% notice on all copies, and it must be distributed only as part of "Linux 
% Installation and Getting Started". This file's use is covered by the 
% copyright for the entire document, in the file "copyright.tex".
%
% Copyright (c) 1998 by Specialized Systems Consultants Inc. 
% <ligs@ssc.com>
%Revisi�n 1 sin fallos FRJMS
%\subsection{Interfacing and MS-DOS.}
\subsection{Interacci�n con MS-DOS}

%Various utilities exist to interface with MS-DOS. The most well-known
%application is the Linux MS-DOS Emulator, which lets you run MS-DOS
%applications directly from Linux.  Although Linux and MS-DOS are
%completely different operating systems, the 80386 protected-mode
%environment allows MS-DOS applications to behave as if they were
%running in their native 8086 environment.

Diferentes utilidades existen como una interfaz con MS-DOS. La
m�s conocida es el emulador de MS-DOS de {\linux}, que permite ejecutar
aplicaciones de MS-DOS directamente desde {\linux}. Aunque {\linux} y MS-DOS
son sistemas operativos totalmente diferentes, el entorno en modo
protegido 80386 le permite a las aplicaciones MS-DOS comportarse como
si estuviesen en su entorno original 8086.

%The MS-DOS emulator is still under development, but many popular
%applications run under it. Understandably, MS-DOS applications that
%use bizarre or esoteric features of the system may never be supported,
%because of the limitations inherent in any emulator.  For example, you
%shouldn't expect to run programs that use 80386 protected-mode
%features, like Microsoft Windows (in 386 enhanced mode, that is).

El emulador de MS-DOS est� desarroll�ndose, pero muchas
aplicaciones se ejecutan ya bajo �l. Evidentemente, las aplicaciones
de MS-DOS que usen extra�as o esot�ricas caracter�sticas del sistema,
nunca podr�n soportarse, debido a las limitaciones inherentes de
cualquier emulador. Por ejemplo, no espere ejecutar programas que usen
el modo protegido 80386, como Microsoft Windows (es decir, en modo 386
mejorado).

%Standard MS-DOS commands and utilities like {\tt PKZIP.EXE} work under
%the emulators, as do 4DOS, a {\tt COMMAND.COM} replacement, FoxPro
%2.0, Harvard Graphics, MathCad, Stacker 3.1, Turbo Assembler, Turbo
%C/C++, Turbo Pascal, Microsoft Windows 3.0 (in real mode), and
%WordPerfect 5.1.

Ordenes est�ndar de MS-DOS y utlidades como {\tt PKZIP.EXE} funcionan
con los emuladores, al igual que 4DOS, un sustituto de {\tt
COMMAND.COM}, FoxPro 2.0, Harvard Graphics, MathCad, Stacker 3.1, Turbo
Assembler, Turbo C/C++, Turbo Pascal, Microsoft Windows 3.0 (en modo
real) y WordPerfect 5.1.

%The MS-DOS Emulator is meant mostly as an ad-hoc solution for those
%who need MS-DOS for only a few applications and use {\linux} for
%everything else. It's not meant to be a complete implementation of
%MS-DOS.  Of course, if the Emulator doesn't satisfy your needs, you
%can always run MS-DOS as well as {\linux} on the same system. Using the
%LILO boot loader, you can specify at boot time which operating system
%to start. {\linux} can also coexist with other operating systems, like
%OS/2.

El emulador de MS-DOS viene a ser como una soluci�n {\em ad-hoc} para
aquellos que necesitan MS-DOS s�lo para algunas aplicaciones y usan
{\linux} para todo lo dem�s. Esto no quiere decir que sea una completa
implementaci�n de MS-DOS. Por supuesto que si el emulador de MS-DOS no
satisface sus necesidades, siempre puede utilizar alternativamente
MS-DOS y {\linux} en el mismo sistema. Utilizando el ``LILO boot loader'',
puede especificar en el inicio qu� sistema operativo debe arrancar. {\linux}
adem�s puede coexistir con otros sistemas operativos, como OS/2.

%{\linux} provides a seamless interface to transfer files between {\linux}
%and MS-DOS. You can mount a MS-DOS partition or floppy under {\linux},
%and directly access MS-DOS files as you would any file.

{\linux} proporciona una interfaz sin fisuras para transferir ficheros
entre {\linux} y MS-DOS. Puede montar una partici�n MS-DOS o un floppy
bajo {\linux} y acceder directamente a los ficheros de MS-DOS que desee.

%Currently under development is {\bf WINE}---a Microsoft Windows
%emulator for the X~Window System under {\linux}.  Once WINE is complete,
%users will be able to run MS-Windows applications directly from
%{\linux}. This is similar to the commercial WABI Windows emulator from
%Sun Microsystems, which is also available for {\linux}.

Actualmente se est� desarrollando {\bf WINE}---un emulador de
Microsoft Windows para el sistema X-Window bajo {\linux}. Una vez que
WINE est� completado, los usuarios podr�n ejecutar aplicaciones para
MS-Windows directamente desde {\linux}. Esto es parecido a la aplicaci�n
comercial ``WABI Windows emulator'' de Sun Microsystems, tambi�n
disponible para {\linux}.

%In Chapter~\ref{chap-advanced}, we talk about the MS-DOS tools
%available for {\linux}.

En el Cap�tulo~\ref{chap-tutorial} y en el Cap�tulo~\ref{chap-networking}, hablaremos acerca de las
utilidades de MS-DOS disponibles para {\linux}. 


