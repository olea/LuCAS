% Linux Installation and Getting Started    -*- TeX -*-
% audience.tex
% Copyright (c) 1992-1994 by Matt Welsh <mdw@sunsite.unc.edu>
%
% This file is freely redistributable, but you must preserve this copyright 
% notice on all copies, and it must be distributed only as part of "Linux 
% Installation and Getting Started". This file's use is covered by
% the copyright for the entire document, in the file "copyright.tex".
%
% Copyright (c) 1998 by Specialized Systems Consultants Inc. 
% <ligs@ssc.com>
% Corregido y comparado con el original por Antonio Rueda (24/11/00)
% montuno@openbank.es
%
% Corregido por Antonio Rueda (02/06/2002)
%Versi�n final
\section*{Destinatarios}
\addcontentsline{toc}{section}{Destinatarios}
Este libro esta destinado a usuarios de ordenadores personales que deseen
instalar {\linux}. Se da por supuesto que el usuario tiene un conocimiento
b�sico de ordenadores y sistemas operativos como MS-DOS, pero no
un conocimiento previo de {\linux} o UNIX.

A pesar de ello, insistimos en sugerir a los principiantes en UNIX que consigan 
un buen manual de UNIX, de buena calidad y traducidos al castellano, de los varios disponibles.
Esto es asi porque aun es necesario el conocimiento del UNIX para instalar 
y mantener en marcha un sistema completo. No hay distribuci�n de {\linux} que est�
completamente libre de errores. Quiz� tenga que corregir peque�os errores a 
mano. Hacer funcionar un sistema UNIX no es tarea sencilla, incluso trabajando
con versiones comerciales de UNIX. Si pretende tomarse {\linux} en serio, tenga en 
cuenta que mantener el sistema funcionando requiere un esfuerzo y atenci�n 
considerable. Esto as� para cualquier sistema UNIX. 
Debido a la variedad de la comunidad {\linux} y de las muchas
necesidades que el software intenta satisfacer, no se pueden tener resueltas todas las
necesidades, para todos, durante todo el tiempo.


