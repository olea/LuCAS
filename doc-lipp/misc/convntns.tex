% Linux Installation and Getting Started    -*- TeX -*-
% Documentation conventions (convntns.tex)
% Copyright 1992, 1993 Michael K. Johnson, to be distributed freely.
% Use as directed ;-)
%
% Copyright (c) 1998 by Specialized Systems Consultants Inc. 
% <ligs@ssc.com>
%
% Corregido por Antonio Rueda (02/06/2002)



\pagebreak
\section*{Notaci�n usada en este documento}
\addcontentsline{toc}{section}{Convenios en este documento}

Hemos intentado usar la siguiente notaci�n en esta gu�a:

\begin{dispitems}
\ditem {{\bf Negrita}} Usado para resaltar {\bf conceptos nuevos}, {\bf AVISOS},
y {\bf palabras clave} de un lenguaje.

\ditem {{\em It�lica}} Usada para {\em enfatizar} el texto, y
ocasionalmente para citas o presentaciones al comienzo de una  seccion.

\ditem {\cparam{Sesgado}} Usado para marcar {\bf meta variables} en el texto,
especialmente en l�neas de �rdenes.  Por ejemplo, en: 
\begin{quote}{\tt ls -l} \cparam{foo}\end{quote}
\cparam{foo} representa un nombre de fichero, como {\tt /bin/cp}.

\ditem {{\tt Escritura de m�quina}} Usado para representar interacci�n con la pantalla, como en:
\begin{tscreen}
\$ ls --l /bin/cp \\
\verb!-rwxr-xr-x  1 root    wheel    12104 Sep 25 15:53 /bin/cp !
\end{tscreen}

Tambi�n  usada para ejemplos de c�digo, ya sea codigo C, guiones del interprete 
de ordenes, o para mostrar ficheros, tales como ficheros de configuraci�n. Cuando sea 
necesario, y para una mejor claridad, estos ejemplos o figuras se incluiran en cajas.

\ditem {\key{Tecla}} Representa una tecla a pulsar, como en este ejemplo:

\begin{quote}Pulse \ret para continuar.\end{quote}
% note that this is in a san-serif font...

\ditem {\hfill$\diamond$} Un diamante en el margen, como un diamante negro en una 
pista de esqu�, se�ala ``peligro'' o ``precauci�n''. Lea detenidamente los 
p�rrafos marcados de esta forma.

\end{dispitems}




