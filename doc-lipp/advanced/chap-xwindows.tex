% Linux Installation and Getting Started    -*- TeX -*-
% chap-xwindows.tex.
% Copyright (c) 1992, 1993 by Matt Welsh <mdw@sunsite.unc.edu>
%
% This file is freely redistributable, but you must preserve this copyright 
% notice on all copies, and it must be distributed only as part of "Linux 
% Installation and Getting Started". This file's use is covered by the 
% copyright for the entire document, in the file "copyright.tex".
%
% Copyright (c) 1998 by Specialized Systems Consultants Inc. 
% <ligs@ssc.com>
%Revision 5 de julio 2002 por Francisco Javier Fern�ndez <serrador@arrakis.es>
%Gold

%\chapter{The X Window System}\label{chap-xwindows}
\chapter{El Sistema X-Window}
\label{chap-xwindow}
\markboth{El Sistema X-Window}{}

% derived from the monstrous "advanced features" chapter.
\index{X~Window~System|(}
\index{X~Window~System!definici�n}
\markboth{Caracter�sticas Avanzadas}{El Sistema X~Window}

%The X~Window System is a graphical user interface (GUI) that was
%originally developed at the Massachusetts Institute of
%Technology. Commercial vendors have since made X the industry standard
%GUI for UNIX platforms. Virtually every UNIX workstation in the world
%now runs some form of X.

El sistema X-Window es una interfaz gr�fica para usuario (``GUI''
en sus siglas en ingl�s), que se desarroll� originalmente en el Instituto de Tecnolog�a de Massachussetts (Massachusetts Institute of
 Technology, m�s conocido como MIT). X es la ``GUI'' est�ndar para plataformas UNIX
 comerciales. Pr�cticamente todas las estaciones de trabajo UNIX del
 mundo trabajan bajo alguna forma de X.


\index{X11R6}
\index{XFree86}
%A free port of the MIT X~Window System version 11, release 6 (X11R6)
%for 80386, 80486, and Pentium UNIX systems was developed by a team of
%programmers that was originally headed by David Wexelblat. This
%release, known as XFree86\footnote{XFree86 is a trademark of The
%XFree86 Project, Inc.}, is available for System V/386, 386BSD, and
%other Intel x86 UNIX implementations, including Linux. It provides all
%of the binaries, support files, libraries, and tools required for
%installation.

Un equipo  de programadores encabezados inicialmente por David Welxelblat 
desarrollaron un porte libre del sistema X-Window del MIT,
versi�n 11 y ``release'' 6 (X11R6) para sistemas UNIX con 80386, 80486
y Pentium. Esta ``release'', conocida como
XFree86\footnote{XFree86 es una marca registrada por XFree86 Project,
  Inc.}, est� disponible para sistemas V/386, 386BSD y otras
implementaciones UNIX de Intel x86, incluyendo {\linux}. Proporciona
todos los binarios, ficheros de soporte, bibliotecas y utilidades para
la instalaci�n. 

%Some features offered by this release are:
%\begin{itemize}
%\item complete inclusion of the X Consortium's X11R6.3 release;
%\item a new DPMS extension, donated by Digital Equipment Corporation;
%\item the Low Bandwidth X (LBX) extension in all X servers;
%\item Microsoft IntelliMouse support;
%\item support for {\tt gzip} font compression.
%\end{itemize}

Algunas de las caracter�sticas que ofrece esta versi�n son:
\begin{itemize}
\item Inclusi�n de la versi�n completa de X11R6.3 del {\em X Consortium}.
\item Una nueva extensi�n ``DPMS'', donada por Digital Equipment
  Corporation.
\item La extensi�n de {\em Low Bandwidth X}(LBX) en todos los
  servidores X.
\item Soporte {\em Microsoft Intellimouse}
\item Soporte para la compresi�n de fuentes {\tt gzip}
\end{itemize}

%To use the X Window System, you are encouraged to read {\em The
%X~Window System: A User's Guide\/} (see Appendix~\ref{app-info}).
%Here, we describe step-by-step an XFree86 installation under Linux.
%You still need to fill in some of the details by reading the XFree86
%documentation, which is discussed below.  The Linux {\em XFree86 HOW
%TO\/} is another good information source.

Para usar el sistema X-Window, se recomienda leer {\em The X-Window
  System: A User's Guide\/} (ver Ap�ndice). Aqu�
  describiremos paso a paso la instalaci�n de XFree86 bajo {\linux}. Debe
  fijarse en algunos detalles leyendo la documentaci�n de XFree86, que
  se discute m�s adelante. El {\em XFree86 C�MO\/} de {\linux} es otra fuente de
  informaci�n interesante.
%%% Input each section

% Linux Installation and Getting Started    -*- TeX -*-
% xwindows.tex
% Copyright (c) 1992, 1993 by Matt Welsh <mdw@sunsite.unc.edu>
%
% This file is freely redistributable, but you must preserve this copyright 
% notice on all copies, and it must be distributed only as part of "Linux 
% Installation and Getting Started". This file's use is covered by the 
% copyright for the entire document, in the file "copyright.tex".
%
% Copyright (c) 1998 by Specialized Systems Consultants Inc. 
% <ligs@ssc.com>
% Revisi�n 1 7 de julio 2002 por F Javier Fernandez <serrador@arrakis.es>
% Revisi�n 2 13 de julio de 2002 por F. javier Fernandez <serrador@arrakis.es>
% Revisi�n 3 13 de julio de 2002 12:26:00 por JFS <serrador@arrakis.es>
%Gold
\section{Requerimientos hardware de X-Window}
\label{sec-xwindows-reqs}
\subsection{Gr�ficos}
\index{XFree86!requerimientos hardware para}
\index{XFree86!chipsets de v�deo soportados por}
\index{hardware!tarjeta de v�deo}
%% As of XFree86 version 3.3, released in June, 1997, the following video
%% chip sets are supported. 
La documentaci�n para su adaptador de v�deo deber�a especificar el chip
gr�fico. Si est� dispuesto a comprar una nueva tarjeta, o va a comprar una
m�quina que viene con tarjeta de v�deo, pregunte a su distribuidor cu�l es
exactamente la marca, modelo y chip de la tarjeta que tiene. El distribuidor
puede que necesite llamar al departamento de apoyo t�cnico del fabricante.
Muchos distribuidores de hardware de ordenadores personales afirman que su
tarjeta de v�deo es ``Super VGA est�ndar,'' que ``deber�a funcionar,'' con su
ordenador. Explique que su software (mencione {\linux} y XFree86!) no permite
todos los chips gr�ficos y que deber�a tener informaci�n detallada.

Tambi�n puede determinar el chip de la tarjeta de v�deo ejecutando el 
programa {\tt SuperProbe} incluido junto con el paquete XFree86. Esto se 
detalla m�s abajo.

 Concretamente se permiten los siguientes chips de v�deo para la versi�n
de XFree86 3.3 liberada en Junio 1997:
 
 \begin{itemize}
 \item Ark Logic ARK1000PV, ARK1000VL, ARK2000PV, ARK2000MT
 \item Alliance AP6422, AT24
 \item ATI 18800, 18800-1, 28800-2, 28800-4, 28800-5, 28800-6, 68800-3,
             68800-6, 68800AX, 68800LX, 88800GX-C, 88800GX-D, 88800GX-E,
             88800GX-F, 88800CX, 264CT, 264ET, 264VT, 264VT2, 264GT (esta lista
             incluye los Mach8, Mach32, Mach64, 3D Rage y 3D Rage II)
 \item Avance Logic ALG2101, ALG2228, ALG2301, ALG2302, ALG2308, ALG2401
 \item Chips \& Technologies
             65520, 65530, 65540, 65545, 65520, 65530, 65540, 65545, 65546,
             65548, 65550, 65554
 \item Cirrus Logic
             CLGD5420, CLGD5422, CLGD5424, CLGD5426, CLGD5428, CLGD5429,
             CLGD5430, CLGD5434, CLGD5436, CLGD5440, CLGD5446, CLGD5462,
             CLGD5464, CLGD5465, CLGD5480, CLGD6205, CLGD6215, CLGD6225,
             CLGD6235, CLGD6410, CLGD6412, CLGD6420, CLGD6440, CLGD7541,
             CLGD7543, CLGD7548, CLGD7555
 \item Digital Equipment Corporation TGA
 \item Compaq AVGA
 \item Genoa GVGA
 \item IBM 8514/A (y clones aut�nticos), XGA-2
 \item IIT AGX-014, AGX-015, AGX-016
 \item Matrox MGA2064W (Millennium), MGA1064SG (Mystique)
 \item MX MX68000, MX680010
 \item NCR 77C22, 77C22E, 77C22E+
 \item Number Nine I128 (series I y II)
 \item NVidia/SGS Thomson NV1, STG2000
 \item OAK OTI067, OTI077, OTI087
 \item RealTek RTG3106
 \item S3 86C911, 86C924, 86C801, 86C805, 86C805i, 86C928, 86C864, 86C964,
             86C732, 86C764, 86C765, 86C775, 86C868, 86C968, 86C325, 86C375,
             86C385, 86C988, 86CM65
 \item SiS 86C201, 86C202, 86C205
 \item Tseng ET3000, ET4000AX, ET4000/W32, ET4000/W32i, ET4000/W32p, ET6000
 \item Trident
             TVGA8800CS, TVGA8900B, TVGA8900C, TVGA8900CL, TVGA9000, TVGA9000i,
             TVGA9100B, TVGA9200CXR, TVGA9320, TVGA9400CXi, TVGA9420,
             TGUI9420DGi, TGUI9430DGi, TGUI9440AGi, TGUI9660XGi, TGUI9680, 
             ProVidia 9682, ProVidia 9685, ProVidia 9692, Cyber 9382, Cyber 9385
 \item Video 7/Headland Technologies HT216-32
 \item Weitek P9000
 \item Western Digital/Paradise PVGA1
 \item Western Digital WD90C00, WD90C10, WD90C11, WD90C24, WD90C24A, WD90C30, 
              WD90C31, WD90C33
 \end{itemize}

Las tarjetas de v�deo con estos chips se permiten en todos los tipos de buses.
Todas las tarjetas permiten virtualmente los modos gr�ficos de 256 colores. 
Adem�s, algunas de ellas permiten modos de color como monocromo,  15-bit, 
16-bit, 24-bit y 32-bit. Para profundidades de color superiores a 
256 (8-bit), debe tener instalada la cantidad requerida de RAM din�mica de 
v�deo (DRAM). La configuraci�n t�pica es 16 bits por pixel (65536 colores).

El servidor monocromo permite asimismo tarjetas VGA gen�ricas, la tarjeta 
monocroma Hercules, las tarjetas monocromas Hyundai HGC1280, Sigma LaserView 
y Apollo.

Las anotaciones de la versi�n actual de XFree86 deber�an contener la
lista completa de chips permitidos. La distribuci�n XFree86 tiene 
ficheros README espec�ficos para cada chip que dan informaci�n detallada
sobre la posibilidad de utilizar ese chip.

Un problema al que se enfrentaron los desarrolladores del XFree86 es que
algunos fabricantes de tarjetas de v�deo utilizan mecanismos no est�ndar
para determinar las frecuencias del reloj utilizadas para trabajar con la
tarjeta. O no publican especificaciones que describan como programar la 
tarjeta o exigen a los programadores que firmen declaraciones de no 
revelaci�n para conseguir la informaci�n. Esta pr�ctica restringe la 
libre distribuci�n de XFree86 y el equipo de desarrollo de XFree86 no 
est� dispuesto a aceptarla. Esto ha sido un problema con las tarjetas de
v�deo m�s antiguas de Diamond, pero a partir de la versi�n 3.3, Diamond
apoya activamente el Proyecto XFree86.

%% Note however, that the Diamond SpeedStar 24 and possibly some SpeedStar+ board
%% are NOT supported, even though they use the ET4000 chipset. Also, there are
%% many of the newer chipsets are not supported yet due to lack of documentation
%% from the manufacturer and/or lack of programmer resources to dedicate to the
%% coding.
N�tese sin embargo que las tarjetas Diamond SpeedStar 24 y posiblemente algunas
tarjetas SpeeedStar+ NO son soportadas, incluso aunque usen el chipset ET4000.
Tambi�n, hay muchos de los chipsets m�s modernos que no se soportan debido
a la escasez de documentaci�n del fabricante y/o la escasez de programadores
dedicados a la codificaci�n.

Se recomienda usar una tarjeta aceleradora, como el chip S3. Deber�a 
revisar la documentaci�n del XFree86 y verificar que su propia tarjeta 
est� permitida antes de dar el paso decisivo y comprar hardware caro.
Se env�an comparaciones de rendimiento de tarjetas de v�deo de forma
rutinaria a los grupos de noticias de Usenet 
{\tt comp.windows.x.i386unix} y {\tt comp.os.linux.misc}.

Es importante fijarse en que la tarjeta aceleradora media es 
significativamente m�s r�pida que la tarjeta gr�fica est�ndar de la 
mayor�a de las estaciones de trabajo. Un sistema {\linux} 80486DX2 a 
66-MHz con 20 megabytes de RAM equipado con tarjeta S3-864 
VESA Local Bus (VLB) con 2 megabytes de DRAM podr� por consiguiente
ser unas 7 veces m�s r�pido que una estaci�n de trabajo Sun Sparc IPX
en pruebas de rendimiento con el servidor XFree86 versi�n 3.1. La 
versi�n 3.3 es a�n m�s r�pida. En general, un sistema {\linux} con una
SVGA acelerada dar� un rendimiento mucho mayor que las estaciones de
trabajo UNIX comerciales, que normalmente emplean buffers de cuadro 
�nico para gr�ficos.

\subsection{Memoria, CPU y espacio en disco}
La instalaci�n recomendada para XFree86 bajo GNU/Linux es un 80486 o 
mejor con al menos 16 megabytes de RAM.
Cuanta m�s RAM f�sica haya instalada, menos intercambio a disco habr�
que hacer cuando quede poca memoria libre. Como el intercambio es
inherentemente lento (los discos son muy lentos comparados con la
memoria) tener 16 megabytes de RAM o m�s es necesario para ejecutar
XFree86 c�modamente. Un sistema con 4 megabytes de RAM f�sica podr�a
funcionar de 10 a 100 veces m�s despacio que uno con 16 megabytes o
m�s.

Una instalaci�n est�ndar de XFree86 necesita de 60 a 80 megabytes de
espacio en disco, como m�nimo. Esto incluye espacio para los 
servidores X, fuentes, bibliotecas y utilidades est�ndar. Si tiene 
en mente a�adir aplicaciones, probablemente podr� ejecutar XFree86 
con comodidad con 200 megabytes de espacio en disco.

\section{Instalaci�n de XFree86}
\index{XFree86!instalaci�n}

La distribuci�n binaria de XFree86 para {\linux} se halla en todas
las distribuciones {\linux} en CD y tambi�n puede ser encontrada en 
un cierto n�mero de sitios FTP. En {\tt sunsite.unc.edu} se encuentra en 
el directorio {\tt /pub/X11/XFree86}. En el momento de la redacci�n
de este documento, la versi�n actual es la 3.3.1. Peri�dicamente, 
salen versiones m�s nuevas. Si obtiene XFree86 como parte de una
distribuci�n {\linux}, no es necesario descargar el software de forma
separada.

% When you download XFree86 directly, be sure that you have the correct
% version. Recent releases of many of Linux distributions use the GNU
% version of the C libraries. There are now two versions of the XFree86
% distribution: one linked against the GNU C libraries and another
% version linked against the older, non-GNU version. There is no
% signficant difference functionally between the two versions.

Estos ficheros est�n presentes en la distribuci�n XFree86-3.3.1

Uno de los siguientes servidores es necesario:

\begin{center}
\small\begin{tabular}{lll} 
\hline
Fichero                                 & Descripci�n     \\
\hline
{\tt X338514.tgz}                 & Servidor para tarjetas basadas en 8514. \\
{\tt X33AGX.tgz}                  & Servidor para tarjetas basadas en AGX. \\
{\tt X33I128.tgz}                 & Servidor para las tarjetas Imagine I128.\\
{\tt X33Ma64.tgz}                 & Servidor para tarjetas basadas en Mach64.\\
{\tt X33Ma32.tgz}                 & Servidor para tarjetas basadas en Mach32.\\
{\tt X33Ma8.tgz}                  & Servidor para tarjetas basadas en Mach8. \\
{\tt X33Mono.tgz}                 & Servidor para modos de v�deo monocromos. \\
{\tt X33P9K.tgz}                  & Servidor para tarjetas basadas en P9000. \\
{\tt X33S3.tgz}                   & Servidor para tarjetas basadas en S3. \\
{\tt X33S3V.tgz}                  & Servidor para tarjetas tipo S3/Virge. \\
{\tt X33SVGA.tgz}                 & Servidor para tarjetas Super VGA. \\
{\tt X33VGA16.tgz}                & Servidor para tarjetas VGA/EGA. \\
{\tt X33W32.tgz}                  & Servidor para tarjetas tipo ET4000/W32.\\

\hline
\end{tabular}\normalsize\rm
\end{center}

Todos los ficheros siguientes son necesarios:
\begin{center}
\small\begin{tabular}{lll} 
\hline
Fichero                                 & Descripci�n     \\
\hline    
{\tt preinst.sh }    & Script de preinstalaci�n \\
{\tt postinst.sh}    & Script de postinstalaci�n \\
{\tt X33bin.tgz }    & Clientes, bibliotecas de tiempo de ejecuci�n y ficheros de aplicaci�n predeterminados \\
{\tt X33doc.tgz }    & Documentaci�n \\
{\tt X33fnts.tgz}    & Fuentes 75dpi, misc y PEX \\
{\tt X33lib.tgz }    & Ficheros de datos necesarios en tiempo de ejecuci�n \\ 
{\tt X33man.tgz }    & P�ginas del manual \\
{\tt X33set.tgz }    & Utilidad XF86Setup \\
{\tt X33VG16.tgz}    & Servidor VGA de 16 colores (XF86Setup lo necesita) \\
\hline
\end{tabular}\normalsize\rm
\end{center}

Lo siguiente es necesario para nuevas instalaciones y opcionalmente 
para instalaciones existentes:
\begin{center}
\small\begin{tabular}{lll} 
\hline
Fichero                                 & Descripci�n     \\
\hline
{\tt X33cfg.tgz}     & ficheros ejemplo de configuraci�n para xinit y xdm \\
\hline
\end{tabular}\normalsize\rm
\end{center}


\blackdiamond 
No instalar {\tt X33cfg.tgz} sobre una instalaci�n XFree86 existente
sin crear una copia de seguridad de los ficheros de configuraci�n.
Desempaquetar {\tt X33cfg.tgz} sobreescribe �stos y otros ficheros. 
De todas formas, si usted s� que tiene ficheros de configuraci�n 
personalizados, no hay necesidad de instalar este paquete.

\blackdiamond 
Las fuentes de mapa de bits distribuidas con la versi�n 3.3.1 est�n
comprimidas con el programa {\tt gzip} en vez de con {\tt compress}.
Probablemente tendr� antes de  borrar las fuentes antiguas  hacer
copia de seguridad de ellas. Los servidores X y servidores de fuentes
de las versiones anteriores no pod�an leer fuentes comprimidas por
{\tt gzip}, as� que copie las fuentes antiguas si desea utilizar los
servidores m�s antiguos.

Los siguientes ficheros son opcionales:
\begin{center}
\small\begin{tabular}{lll} 
\hline
Fichero                                 & Descripci�n     \\
\hline
{\tt X33f100.tgz}   & Fuentes 100dpi \\
{\tt X33fcyr.tgz}   & Fuentes del alfabeto cir�lico \\
{\tt X33fnon.tgz}   & Otras fuentes (chino, japon�s, koreano, hebreo) \\
{\tt X33fscl.tgz}   & Fuentes escalables (Speedo y Type1) \\
{\tt X33fsrv.tgz}   & Servidor de fuentes y ficheros de configuraci�n \\
{\tt X33prog.tgz}   & Ficheros de cabecera de X, ficheros de configuraci�n y bibliotecas de tiempo de compilaci�n \\
{\tt X33nest.tgz}   & Servidor X anidado \\
{\tt X33vfb.tgz }   & Servidor X de framebuffer virtual \\
{\tt X33prt.tgz }   & Servidor de impresi�n X \\
{\tt X33ps.tgz  }   & Versi�n PostScript de la documentaci�n \\
{\tt X33html.tgz}   & Versi�n HTML de la documentaci�n \\
{\tt X33jdoc.tgz}   & Documentaci�n en japon�s (para la versi�n 3.2) \\
{\tt X33jhtm.tgz}   & Versi�n HTML de la documentaci�n en japon�s (3.2) \\
{\tt X33lkit.tgz}   & Kit de enlazado del servidor X \\
\hline
\end{tabular}\normalsize\rm
\end{center}

El directorio XFree86 deber�a contener ficheros {\tt README} y
apuntes de instalaci�n para la versi�n actual.

Despu�s, como root, cree el directorio {\tt /usr/X11R6} si no 
existe todav�a. Despu�s ejecute el script de preinstalaci�n, 
{\tt preinst.sh}. Deber�a copiar del directorio {\tt /var/tmp} 
este fichero y todos los ficheros comprimidos para su sistema 
antes de ejecutar {\tt preinst.sh}. {\tt /usr/X11R6} debe ser 
el directorio actual cuando se ejecute el script de 
preinstalaci�n y descomprima todos los ficheros.

\begin{tscreen}
\# cd /usr/X11R6 \\
\# sh /var/tmp/preinst.sh \\
\end{tscreen}

A continuaci�n debe descomprimir los ficheros desde 
{\tt /var/tmp} a {\tt /usr/X11R6} con una instrucci�n como:
\begin{tscreen}
\# gzip -d < /var/tmp/X33prog.tgz  $\mid$ tar vxf - \\
\end{tscreen}

\blackdiamond 
Estos archivos {\tt tar} est�n comprimidos con la ruta
relativa a {\tt /usr/X11R6}. Debe descomprimir los ficheros
ah�. En algunas distribuciones {\linux}, el directorio ra�z es
{\tt /var/X11R6}.

Despu�s de haber descomprimido los ficheros necesarios y 
todos los ficheros opcionales que haya seleccionado, ejecute
el script de postinstalaci�n {\tt postinst.sh}.
\begin{tscreen}
\# cd /usr/X11R6 \\
\# sh /var/tmp/postinst.sh \\
\end{tscreen}

Ahora enlace el fichero {\tt /usr/X11R6/bin/X} al servidor
que permite su tarjeta gr�fica. Por ejemplo el servidor de
color de SVGA, {\tt /usr/bin/X11/X} deber�a estar enlazado
con {\tt /usr/X11R6/bin/XF86\_SVGA}. Para utilizar el 
servidor monocromo enlace {\tt X} a {\tt XF86\_MONO} con 
la instrucci�n:

\begin{tscreen}
\# ln --sf /usr/X11R6/bin/XF86\_MONO\ \ /usr/X11R6/bin/X
\end{tscreen}
Lo mismo sirve para el resto de servidores.

Tambi�n tendr� que asegurarse de que el directorio 
{\tt /usr/X11R6/bin} est� en el path. Esto puede hacerse 
editando los valores por omisi�n del sistema 
{\tt /etc/profile} o {\tt /etc/csh.login} (basado en el
shell que usted u otros usuarios utilizan). Tambi�n puede
simplemente a�adir el directorio a su path personal 
modificando {\tt /etc/.bashrc} o {\tt /etc/.cshrc}, basado
en su shell.

Finalmente, aseg�rese de que {\tt /usr/X11R6/lib} puede ser
localizado por {\tt ld.so}, el enlazador en tiempo de 
ejecuci�n. Para esto, a�ada la l�nea
\begin{tscreen}
/usr/X11R6/lib
\end{tscreen}
al fichero {\tt /etc/ld.so.conf}, y ejecute 
{\tt /sbin/ldconfig}, como {\tt root}. 

\section{Examinando la configuraci�n hardware}

Si no est� seguro de qu� servidor usar o no sabe cu�l 
es el chip de la tarjeta gr�fica, el programa 
{\tt SuperProbe}, que est� en {\tt /usr/X11R6/bin} puede 
intentar determinarlo, as� como m�s informaci�n. Anote
estos datos para uso posterior.

Para ejecutar SuperProbe desde la l�nea de ordenes, 
escriba simplemente
\begin{tscreen}
\# SuperProbe
\end{tscreen}

\blackdiamond Es posible que {\tt SuperProbe} se confunda
con hardware que utilice direcciones de puertos de 
entrada/salida que puedan ser utilizadas por tarjetas de
v�deo. Para evitar que SuperProbe compruebe estas 
direcciones, utilice el argumento {\tt excl} seguido de la
lista de direcciones que {\tt SuperProbe} no va a examinar.
Por ejemplo:

\begin{tscreen}
\# SuperProbe -excl 0x200-0x230,0x240
\end{tscreen}
Las direcciones vienen dadas como n�meros hexadecimales 
que est�n precedidos por {\tt 0x}.

Para mostrar una lista de los dispositivos de v�deo que 
SuperProbe conoce, utilice la instrucci�n
\begin{tscreen}
\# SuperProbe -info
\end{tscreen}

{\tt SuperProbe} puede escribir gran cantidad de 
informaci�n si le a�ade el argumento {\tt -verbose}. 
Puede redireccionar la salida a un fichero:
\begin{tscreen}
\# SuperProbe -verbose >superprobe.out 
\end{tscreen}

\blackdiamond Ejecutar SuperProbe puede provocar que
el sistema se cuelgue. Aseg�rese de que no est�n 
ejecut�ndose aplicaciones esenciales, o al menos de
que tienen todos sus datos grabados en disco de manera
segura, y cerci�rese de que todos los usuarios est�n
desconectados. Similarmente, un sistema cargado (que
est� imprimiendo en segundo plano, por ejemplo) puede
tergiversar la salida de software como SuperProbe o de
un servidor X que est� intentando medir las 
especificaciones de tiempo de una tarjeta de v�deo.


\section{Generar de forma autom�tica el fichero {\tt XF86Config}}

Crear el fichero {\tt XF86Config} a mano es una tarea ardua, 
pero no imposible. Varias herramientas de la versi�n 3.3.1 de 
XFree86 podr�n ayudarle. Una de ellas, el programa 
{\tt XF86Setup} puede generar autom�ticamente un fichero
XF86Config con formato correcto. Debe conocer las 
especificaciones exactas de su tarjeta de v�deo as� como
los valores de refresco vertical y horizontal de su monitor.
La mayor parte de la informaci�n puede ser encontrada en los
propios manuales.

Hay otros tantos programas de configuraci�n, dependiendo de
la distribuci�n {\linux}. Los m�s comunes son 
{\tt Xconfigurator} y {\tt xf86config}. El �ltimo es una 
versi�n antigua de {\tt XF86Setup} y est� incluido en versiones
anteriores de XFree86. Deber�a usar siempre {\tt XF86Setup} en
caso de que tenga disponibles �ste y {\tt xf86config}.

\section{Configurar XFree86}
\label{chap-advanced-xconfiguration}
\index{XFree86!configuring}
% Setting up XFree86 is not difficult. However, if you happen to be
% using hardware for which drivers are under development, or wish to
% obtain the best performance or resolution from an accelerated graphics
% card, configuring XFree86 can be somewhat time-consuming.
Configurar XFree86 no es dif�cil. Sin embargo si ocurre que se est�
usando hardware para el cu�l los controladores est�n en desarrollo,
o se desea obtener el mejor rendimiento o resoluci�n de una tarjeta
aceleradora, configurar XFree86 puede tomar su tiempo.

En este apartado, se describe c�mo crear y editar el fichero
{\tt XF86Config}, que configura el servidor XFree86. En la mayor�a
de los casos lo mejor es empezar con una configuraci�n XFree86 que
utilice baja resoluci�n como 640x480, que sea permitida por la
pr�ctica totalidad de las tarjetas de v�deo y monitores. Una vez
que XFree86 trabaje a una resoluci�n est�ndar baja, podr� 
modificar la configuraci�n para aprovechar las capacidades de su
tarjeta de v�deo. Esto asegura que XFree86 funcione en su sistema y
que la instalaci�n es esencialmente correcta antes de que empiece
con la a menudo dif�cil tarea de configurar XFree86 para un alto
rendimiento.

Adem�s de la informaci�n listada aqu�, deber�a leer los siguientes
documentos:
\begin{itemize}
\item La documentaci�n de XFree86 en {\tt /usr/X11R6/lib/X11/doc} 
(del paquete {\tt XFree86-3.1-doc}). Deber�a revisar especialmente
el fichero {\tt README.Config}, que es un tutorial de configuraci�n
de XFree86.
\item Varios chips de v�deo tienen ficheros separados en el 
directorio arriba mencionado (como {\tt README.Cirrus} y 
{\tt README.S3}). Lea el fichero que concierne a su tarjeta de
v�deo. 
\item La p�gina del manual de {\tt XFree86}.
\item La p�gina del manual de {\tt XF86Config}.
\item La p�gina del manual del servidor que est� utilizando, como
{\tt XF86\_SVGA} o {\tt XF86\_S3}.
\end{itemize}

\index{XFree86!fichero de configuracio para}
\index{/usr/X11R6/lib/X11/XF86Config@{\tt /usr/X11R6/lib/X11/XF86Config}}
\index{XF86Config@{\tt XF86Config}}
El fichero principal de configuraci�n de XFree86 es 
{\tt /usr/X11R6/lib/X11/XF86Config}. Este fichero contiene 
informaci�n sobre su rat�n, par�metros de su tarjeta de v�deo, 
etc. El fichero {\tt XF86Config.eg} viene con la distribuci�n
XFree86 como ejemplo. Copie este fichero a {\tt XF86Config} y
ed�telo como punto de inicio.

La p�gina del manual de {\tt XF86Config} explica el formato
del fichero {\tt XF86Config}. Lea la p�gina del manual si a�n
no lo ha hecho. 

Se va a describir un ejemplo de {\tt XF86Config}, una secci�n 
de cada vez. Este fichero puede no resultar exactamente igual
al fichero de ejemplo incluido en la distribuci�n de XFree86, 
pero la estructura es la misma.

\blackdiamond Observe que el formato del fichero {\tt XF86Config}
puede cambiar con cada versi�n de XFree86. Lea las notas de su 
distribuci�n para erratas.

\blackdiamond {\bf No copie el fichero de configuraci�n mostrado aqu�
 a su sistema y trate de utilizarlo.} Un fichero de
configuraci�n que no se corresponda con su hardware puede
poner el monitor en frecuencias demasiado altas. Se han dado
casos de da�os del monitor, especialmente monitores de 
frecuencia fija, esto ha sido provocado por ficheros 
{\tt XF86Config} mal configurados. {\bf Cerci�rese 
completamente de que su fichero {\tt XF86Config}} se 
corresponde con el hardware antes de utilizarlo.

Todas las secciones del fichero {\tt XF86Config} est�n 
rodeadas por un par de l�neas con la sintaxis 
{\tt Section ``\cparam{section-name}''}\ldots{\tt EndSection}. 

La primera secci�n del fichero {\tt XF86Config} es {\tt Files},
que tiene este aspecto:
\begin{tscreen}\begin{verbatim}
Section "Files"
    RgbPath     "/usr/X11R6/lib/X11/rgb"
    FontPath    "/usr/X11R6/lib/X11/fonts/misc/"
    FontPath    "/usr/X11R6/lib/X11/fonts/75dpi/"
EndSection
\end{verbatim}\end{tscreen}
La l�nea {\tt RgbPath} establece la ruta a la base de datos
de colores RGB de X11R6, y cada l�nea {\tt FontPath} pone la
ruta a un directorio que contiene fuentes X11. No deber�a 
tener por qu� modificar estas l�neas. Simplemente aseg�rese 
de que hay una entrada {\tt FontPath} para cada tipo de fuente
que haya instalado; esto es, para cada directorio en 
{\tt /usr/X11R6/lib/X11/fonts}.

La siguiente secci�n es {\tt ServerFlags}, que especifica 
varios flags globales para el servidor. En general esta
secci�n est� vac�a.
\begin{tscreen}\begin{verbatim}
Section "ServerFlags"
# Elimine este comentario para provocar un volcado de memoria en el
# punto donde se reciba una se�al. Esto puede dejar la terminal en 
# estado inutilizable, pero puede ofrecer una mejor traza de la pila
# para ayudar a depurar en caso de volcado de memoria
#    NoTrapSignals
# Elimine este comentario para deshabilitar la secuencia 
# <Crtl><Alt><BS> de abortar el servidor
#    DontZap
EndSection
\end{verbatim}\end{tscreen}
En esta secci�n de {\tt ServerFlags} todas las l�neas est�n 
comentadas.

La siguiente secci�n es {\tt Keyboard}. Este ejemplo muestra una
configuraci�n b�sica que deber�a funcionar en la mayor�a de los
sistemas. El fichero {\tt XF86Config} describe como modificar la
configuraci�n.
\begin{tscreen}\begin{verbatim}
Section "Keyboard"
    Protocol    "Standard"
    AutoRepeat  500 5
    ServerNumLock
EndSection
\end{verbatim}\end{tscreen}

La siguiente secci�n es {\tt Pointer}, que especifica par�metros
para el dispositivo de rat�n:
\begin{tscreen}\begin{verbatim}
Section "Pointer"

    Protocol    "MouseSystems"
    Device      "/dev/mouse"

# Baudrate y SampleRate se aplican solo a algunos ratones Logitech
#    BaudRate   9600
#    SampleRate 150

# Emulate3Buttons es una opci�n para ratones Microsoft de 2 botones
#    Emulate3Buttons

# ChordMiddle es una opci�n para algunos ratones Logitech de 3 botones
#    ChordMiddle

EndSection
\end{verbatim}\end{tscreen}
De momento las �nicas opciones que le preocupan son {\tt Protocol} y 
{\tt Device}. {\tt Protocol} especifica el rat�n {\em protocol,\/} que
no es necesariamente el mismo que el del fabricante. XFree86 bajo {\linux}
reconoce los siguientes protocolos de rat�n:
\begin{itemize}
\item {\tt BusMouse} 
\item {\tt Logitech}
\item {\tt Microsoft}
\item {\tt MMSeries}
\item {\tt Mouseman}
\item {\tt MouseSystems}
\item {\tt PS/2}
\item {\tt MMHitTab}
\end{itemize}
{\tt BusMouse} deber�a ser usado con los ratones de bus Logitech. Los
ratones Logitech m�s antiguos utilizan {\tt Logitech}, y los nuevos 
ratones serie Logitech utilizan protocolos {\tt Microsoft} o 
{\tt Mouseman}. 

{\tt Device} especifica el fichero de dispositivo por el cual el rat�n
podr� ser accedido. En la mayor�a de los sistemas {\linux}, es 
{\tt /dev/mouse}, que generalmente es un enlace al puerto serie 
apropiado, como {\tt /dev/cua0} para ratones serie y el dispositivo 
de rat�n de bus apropiado para los ratones de bus. En cualquier caso
aseg�rese de que el fichero de dispositivo existe.

La secci�n siguiente es {\tt Monitor}, que especifica las 
caracter�sticas de su monitor. Como en otras secciones del fichero
{\tt XF86Config} puede haber m�s de una secci�n {\tt Monitor}. Esto es
�til si tiene varios monitores conectados al sistema o si utiliza el
mismo fichero {\tt XF86Config} para varias configuraciones de hardware.

\begin{tscreen}\begin{verbatim}
Section "Monitor"

    Identifier  "CTX 5468 NI"

    # !`Estos valores son para un CTX 5468NI s�lo! No intente usarlos 
    # con su monitor (a menos que posea este modelo)

    Bandwidth    60
    HorizSync    30-38,47-50
    VertRefresh  50-90

    # Modos: Nombre    dotclock  horiz                vert 

    ModeLine "640x480"  25       640 664 760 800      480 491 493 525
    ModeLine "800x600"  36       800 824 896 1024     600 601 603 625
    ModeLine "1024x768" 65       1024 1088 1200 1328  768 783 789 818

EndSection
\end{verbatim}\end{tscreen}
{\tt Identifier} es un nombre arbitrario para la entrada {\tt Monitor}.
Puede ser cualquier cadena y se utiliza para referirse m�s tarde a la 
entrada {\tt Monitor} en el fichero {\tt XF86Config}.

{\tt HorizSync} especifica las frecuencias v�lidas horizontales
para su monitor, en kHz. Los monitores multisync pueden tener
un rango de valores o varios rangos separados por comas. Los
monitores de frecuencia fija exigen una lista de valores discretos,
por ejemplo:
\begin{tscreen}\begin{verbatim}
    HorizSync    31.5, 35.2, 37.9, 35.5, 48.95
\end{verbatim}\end{tscreen}
El manual del monitor deber�a listar estos valores en la secci�n de
especificaciones t�cnicas. Si no es as�, p�ngase en contacto con el
fabricante o proveedor del monitor para obtenerlos.

{\tt VertRefresh} especifica los valores v�lidos de refresco vertical
(o frecuencias de sincronizaci�n vertical) para su monitor, en kHz.
Al igual que {\tt HorizSync}, puede ser un rango o una lista de valores
discretos. El manual de su monitor deber�a listarlos.

{\tt HorizSync} y {\tt VertRefresh} se utilizan solamente para
comprobar que las resoluciones del monitor est�n en rangos v�lidos. 
Esto reduce la posibilidad de que se da�e el monitor haci�ndolo trabajar
a una frecuencia para la que no est� preparado.

La directiva {\tt ModeLine} se utiliza para especificar los modos de 
resoluci�n para su monitor. El formato es
\begin{tscreen}
ModeLine \cparam{name} \cparam{clock} \cparam{horiz-values} \cparam{vert-values}
\end{tscreen}
\textsl{name} es una cadena arbitraria que se utiliza para referirse al
modo de resoluci�n m�s tarde en el fichero. \textsl{dot-clock} es la
frecuencia de reloj o ``dot clock'' asociada al modo de resoluci�n.
La dot clock se expresa normalmente en MHz. Esta es la proporci�n a
la que la tarjeta de v�deo env�a pixels al monitor a esta resoluci�n.
\textsl{horiz-values} y \textsl{vert-values} son cuatro n�meros cada
uno que especifican cu�ndo debe disparar la pistola de electrones del
monitor y cuando disparan los pulsos de sincronizaci�n horizontal y 
vertical en cada barrido.

El fichero {\tt VideoModes.doc}, incluido en la distribuci�n XFree86 
describe con detalle como determinar los valores de {\tt ModeLine}
para cada modo de resoluci�n que permita su monitor. \textsl{clock}
ha de corresponderse con uno de los valores de dot clock que permita
su tarjeta de v�deo. M�s tarde en el fichero {\tt XF86Config}, podr�
especificarlos. 

Dos ficheros, {\tt modeDB.txt} y {\tt Monitors}, pueden contener
los datos de {\tt ModeLine} para su monitor. Est�n en 
{\tt /usr/X11R6/lib/X11/doc}.

Empiece con valores de {\tt ModeLine} para los tiempos de monitores
VESA est�ndar porque la mayor�a de los monitores los permiten. 
{\tt ModeDB.txt} incluye los valores de tiempo para las resoluciones
VESA est�ndar. Por ejemplo la entrada
\begin{tscreen}\begin{verbatim}
# 640x480@60Hz Modo no entrelazado
# Sincronizaci�n horizontal = 31.5kHz
# Tiempos: H=(0.95us, 3.81us, 1.59us), V=(0.35ms, 0.064ms, 1.02ms)
#
# nombre       reloj   tiempo horizontal     tiempo vertical     flags
 "640x480"     25.175  640  664  760  800    480  491  493  525
\end{verbatim}\end{tscreen}
es el tiempo VESA est�ndar para un modo 640x480. Tiene un dot clock
de 25.175 que debe ser permitido por su tarjeta de v�deo como se 
describe m�s abajo. Para incluir esta entrada en el fichero 
{\tt XF86Config} utilice la l�nea 
\begin{tscreen}
ModeLine "640x480" 25.175\ \ 640 664 760 800\ \ 480 491 493 525
\end{tscreen}
El argumento \textsl{name} a {\tt ModeLine} ({\tt\verb+"640x480"+})
es una cadena arbitraria. Por convenci�n los modos se llaman por sus
resoluciones, pero \textsl{name} puede t�cnicamente ser una etiqueta
descriptiva.

Para cada {\tt ModeLine} el servidor comprueba las especificaciones
del modo y se asegura de que caen dentro del rango de valores 
especificado por {\tt Bandwidth}, {\tt HorizSync} y {\tt VertRefresh}. 
Si no es as�, el servidor se quejar� cuando intente iniciar X. Por 
alguna raz�n, el dot clock utilizado por el modo no deber�a ser mayor
que el valor utilizado para {\tt Bandwidth}. Sin embargo, en muchos
casos, es seguro utilizar un modo que tenga un ancho de banda 
ligeramente mayor que el que permite su monitor.

Si los tiempos de VESA est�ndar no funcionan (lo sabr� despu�s de que
intente utilizarlos) mire los ficheros {\tt modeDB.txt} y 
{\tt Monitors}, que incluyen valores de modo espec�ficos para muchos
tipos de monitor. Tambi�n puede crear las entradas {\tt ModeLine} a 
partir de estos valores. Aseg�rese de utilizar s�lo valores para su
monitor concreto. Muchos monitores de 14 y 15 pulgadas no permiten
modos de resoluci�n mayores y a menudo las resoluciones de 1024x768
son a valores de dot clock bajos. Si no puede encontrar modos de alta
resoluci�n para su monitor en estos ficheros, seguramente se debe a
que su monitor no los permite.

Si est� completamente perdido y no puede encontrar valores de 
{\tt ModeLine} para su monitor, siga las instrucciones del fichero
{\tt VideoModes.doc}, que est� incluido en la distribuci�n XFree86
y genere valores a partir de las especificaciones del manual de su
monitor. Su recorrido seguramente variar� cuando intente valores de
{\tt ModeLine} a mano. Pero este es un buen lugar donde mirar si no
puede encontrar los valores que necesita. {\tt VideoModes.doc} 
describe tambi�n el formato de la directiva {\tt ModeLine} y otros
aspectos del servidor XFree86 con detalles escabrosos.

Finalmente si logra obtener valores de {\tt Modeline} que son casi
pero no completamente correctos, podr�a conseguir modificarlos un
poco para alcanzar el resultado deseado. Por ejemplo, si la imagen
de XFree86 est� ligeramente desplazada o la imagen parece 
``movida,'' siga las instrucciones del fichero {\tt VideoModes.doc}
y corrija los valores. Aseg�rese de comprobar los controles del 
monitor. En muchos casos deber� cambiar el tama�o horizontal y 
vertical de la pantalla tras arrancar XFree86 para centrar y calibrar
la imagen.

\blackdiamond No utilice los valores de tiempo del monitor o los 
valores de {\tt ModeLine} para monitores distintos de su modelo. Si
intenta hacer funcionar el monitor a una frecuencia para la que no
fue dise�ado, puede da�arlo o incluso destruirlo.

La siguiente secci�n del fichero {\tt XF86Config} es {\tt Device},
que especifica par�metros para la tarjeta de v�deo. Este es un ejemplo.
\begin{tscreen}\begin{verbatim}
Section "Device" 
        Identifier "#9 GXE 64"

        # De momento nada, despu�s se completar�n estos valores

EndSection
\end{verbatim}\end{tscreen}

Esta secci�n define propiedades para una tarjeta de v�deo concreta. 
{\tt Identifier} es una cadena arbitraria descriptiva. Se utilizar�
para referirse a la tarjeta m�s tarde.

En principio no necesita incluir nada en la secci�n {\tt Device} 
aparte de {\tt Identifier}. Se usar� el propio servidor X para
comprobar las propiedades de su tarjeta de v�deo y ponerlas en la
secci�n {\tt Device} m�s tarde. El servidor XFree86 es capaz de 
comprobar el chip de v�deo, relojes, RAMDAC y la cantidad de RAM de
v�deo de la tarjeta. Esto se describe en la secci�n
\ref{sec-video-card-info}.

Antes de hacer esto, hay que acabar de escribir el fichero
{\tt XF86Config}. La siguiente secci�n es {\tt Screen}, que
especifica la combinaci�n monitor/tarjeta de v�deo a utilizar
con cada servidor concreto.
\begin{tscreen}\begin{verbatim}
 Section "Screen"
     Driver     "Accel"
     Device     "#9 GXE 64"
     Monitor    "CTX 5468 NI"
     Subsection "Display"
         Depth      16
         Modes      "1024x768" "800x600" "640x480"
         ViewPort   0 0
         Virtual    1024 768
     EndSubsection
 EndSection
\end{verbatim}\end{tscreen}

La l�nea {\tt Driver} especifica el servidor X que va a utilizar.
Valores v�lidos de {\tt Driver} son:
\begin{itemize}
\item {\tt Accel}: Para los servidores {\tt XF86\_S3}, {\tt XF86\_Mach32}, {\tt XF86\_Mach8}, {\tt XF86\_8514}, {\tt XF86\_P9000}, {\tt XF86\_AGX} y {\tt XF86\_W32}.
\item {\tt SVGA}: Para el servidor {\tt XF86\_SVGA}.
\item {\tt VGA16}: Para el servidor {\tt XF86\_VGA16}.
\item {\tt VGA2}: Para el servidor {\tt XF86\_Mono}.
\item {\tt Mono}: Para los drivers monocromos no VGA en servidores {\tt XF86\_Mono} y {\tt XF86\_VGA16}.
\end{itemize}
Aseg�rese de que {\tt /usr/X11R6/bin/X} es un enlace simb�lico a este servidor.

La l�nea {\tt Device} especifica el {\tt Identifier} de la secci�n 
{\tt Device} que se corresponde con la tarjeta de v�deo que va a 
utilizar con este servidor. Anteriormente se cre� una secci�n 
{\tt Device} con la l�nea
\begin{tscreen}\begin{verbatim}
Identifier "#9 GXE 64"
\end{verbatim}\end{tscreen}
Por ello se emplea {\tt\verb+"#9 GXE 64"+} en la l�nea {\tt Device} ahora.

De forma similar, la l�nea {\tt Monitor} especifica el nombre de la
secci�n {\tt Monitor} que se va a utilizar con este servidor. Aqu�
{\tt\verb+"CTX 5468 NI"+} es el {\tt Identifier} utilizado en la 
secci�n {\tt Monitor} arriba descrita.

{\tt\verb+Subsection "Display"+} define varias propiedades del
servidor XFree86 correspondientes a su combinaci�n monitor/tarjeta.
El fichero {\tt XF86Config} describe todas estas opciones en detalle,
pero la mayor�a de ellas no son necesarias para conseguir que el 
sistema funcione.

Las opciones que deber�a conocer son:
\begin{itemize}
\item {\tt Depth}. Define el n�mero de planos de color, es decir, el
n�mero de bits por pixel. Normalmente {\tt Depth} tiene el valor 16. 
Para el servidor {\tt VGA16} deber�a utilizar una profundidad de 4 y
para el servidor monocromo una profundidad de 1. Si utiliza una tarjeta
de v�deo acelerada con bastante memoria para permitir m�s bits por
pixel, puede poner {\tt Depth} a 24 o 32. Si tiene problemas con 
profundidades mayores de 16, d�jelo otra vez en 16 e intente resolver
el problema m�s tarde.

\item {\tt Modes}. Esta es la lista de nombres de modos que han sido 
definidos con las directivas {\tt ModeLine} en la secci�n anterior.
Se emplearon {\tt ModeLine}s llamadas {\tt\verb+"1024x768"+}, 
{\tt\verb+"800x600"+} y {\tt\verb+"640x48"0+}. Por tanto, se utiliz�
una l�nea {\tt Modes} con
\begin{tscreen}\begin{verbatim}
         Modes    "1024x768" "800x600" "640x480"
\end{verbatim}\end{tscreen}
El primer modo mostrado en esta l�nea es el empleado por omisi�n
cuando arranca XFree86. Una vez que XFree86 est� funcionando, puede
cambiar entre los modos mostrados aqu� con las teclas 
\key{Ctrl}-\key{Alt}-\key{Numeric +} y
\key{Ctrl}-\key{Alt}-\key{Numeric -}. 

Lo mejor ser�a que cuando configure XFree86 utilice modos de v�deo
de resoluci�n m�s bajos como 640x480, que tienden a funcionar en
la mayor�a de los sistemas. Una vez que tenga la configuraci�n 
b�sica funcionando, puede modificar {\tt XF86Config} para permitir
resoluciones m�s altas.

\item {\tt Virtual}. Pone el tama�o del escritorio virtual. XFree86
puede utilizar memoria adicional de su tarjeta de v�deo para extender
el tama�o del escritorio. Cuando mueva el puntero del rat�n al borde
de la pantalla, el escritorio se desplaza poniendo en primer plano
el espacio adicional. Incluso si ejecuta el servidor con una 
resoluci�n de v�deo como 800x600 puede poner {\tt Virtual} a la 
resoluci�n total permitida por la tarjeta de v�deo. Una tarjeta de
v�deo de 1 megabyte puede permitir 1024x768 con una profundidad de
8 bits por pixel, una tarjeta de 2 megabytes permite 1280x1024 con
una profundidad 8 o 1024x768 con profundidad 16. Por supuesto el
�rea completa no ser� visible a la vez, pero puede ser utilizada.

La caracter�stica {\tt Virtual} es bastante limitada. Si desea 
utilizar un aut�ntico escritorio virtual los gestores de ventanas 
como {\tt fvwm} y similares le permitir�n tener escritorios 
virtuales grandes tapando ventanas y utilizando otras t�cnicas, en
vez de almacenar el escritorio entero en memoria de v�deo. Vea las
p�ginas del manual de {\tt fvwm} para m�s detalles acerca de esto. 
Muchos sistemas {\linux} utilizan {\tt fvwm} por omisi�n.

\item {\tt ViewPort}. Si est� utilizando la opci�n {\tt Virtual}
descrita anteriormente, {\tt ViewPort} pone las coordenadas de la
esquina superior izquierda del escritorio virtual cuando se inicia
XFree86. {\tt Virtual 0 0} es un valor utilizado a menudo. Si no
est� especificado, el escritorio se centra en la pantalla del 
escritorio virtual, lo que puede no ser lo deseado.
\end{itemize}

Hay muchas otras opciones para esta secci�n, vea la p�gina del 
manual de {\tt XF86Config} para una descripci�n exhaustiva. En la
pr�ctica estas opciones no son necesarias para que funcione XFree86
inicialmente. 

\section{Rellenando la informaci�n de la tarjeta de v�deo}
\label{sec-video-card-info}

Ahora ya tiene el fichero {\tt XF86Config} listo a excepci�n de la
informaci�n completa de la tarjeta de v�deo. Se utilizar� el 
servidor X para comprobar estos datos y a�adirlos a {\tt XF86Config}.

En vez de comprobar esta informaci�n con el servidor X, los 
valores de {\tt XF86Config} para muchas tarjetas est�n listados en 
los ficheros {\tt modeDB.txt}, {\tt AccelCards} y {\tt Devices}. 
Estos ficheros se encuentran todos en {\tt /usr/X11R6/lib/X11/doc}. 
Adem�s hay varios ficheros {\tt README} para ciertos chips. Deber�a
mirar estos ficheros para informaci�n de su tarjeta de v�deo y 
utilizarla (clock values, tipo de chip y otras opciones) en el 
fichero {\tt XF86Config}. Si falta alg�n dato, puede comprobarlo.

En la mayor�a de estos ejemplos se prueban configuraciones de una
tarjeta \#9 GXE 64, que utiliza el chip {\tt XF86\_S3}. En primer
lugar determine el chip de v�deo de la tarjeta. Ejecutar 
{\tt SuperProbe} (en {\tt /usr/X11R6/bin}) le dir� estos datos,
pero debe conocer el nombre del chip tal como es conocido por el 
servidor X.

Para ello, ejecute la instrucci�n
\begin{tscreen}
X -showconfig
\end{tscreen}
Esto le dar� los nombres de chips conocidos por el servidor X. (La
p�gina del manual para cada servidor X los lista tambi�n.) Por ejemplo
con el servidor acelerado {\tt XF86\_S3} se obtiene:
\begin{tscreen}\begin{verbatim}
XFree86 Version 3.1 / X Window System
(protocol Version 11, revision 0, vendor release 6000)
Operating System: Linux 
Configured drivers:
  S3: accelerated server for S3 graphics adaptors (Patchlevel 0)
      mmio_928, s3_generic
\end{verbatim}\end{tscreen}

Los nombres v�lidos de los chips para este servidor son {\tt mmio\_928}
y {\tt s3\_generic}. La p�gina del manual de {\tt XF86\_S3} describe
estos chips y las tarjetas de v�deo que las usan. En el caso de la
tarjeta de v�deo \#9 GXE 64, es apropiado {\tt mmio\_928}.

Si no sabe qu� chip se est� utilizando, el servidor X puede comprobarlo
por usted. Para hace esto, ejecute la instrucci�n
\begin{tscreen}
\verb!X -probeonly > /tmp/x.out 2>&1 !
\end{tscreen}
si utiliza {\tt bash} como shell. 
Si utiliza {\tt csh}, pruebe:
\begin{tscreen}
\verb!X -probeonly &> /tmp/x.out !
\end{tscreen}

Deber�a ejecutar esta instrucci�n mientras el sistema no ha sido
cargado, es decir, mientras ninguna otra actividad ocurre en el 
sistema. Esta instrucci�n tambi�n comprueba dot clocks para su 
tarjeta de v�deo (como se ver� m�s abajo) y la carga del sistema
puede sesgar este c�lculo.

La salida de arriba, en {\tt /tmp/x.out} deber�a contener l�neas como:
\begin{tscreen}
\verb!XFree86 Version 3.1 / X Window System !\\
\verb!(protocol Version 11, revision 0, vendor release 6000) !\\
\verb!Operating System: Linux  !\\
\verb!Configured drivers: !\\
\verb!  S3: accelerated server for S3 graphics adaptors (Patchlevel 0) !\\
\verb!      mmio_928, s3_generic !\\
{\em Several lines deleted\ldots} \\
\verb!(--) S3: card type: 386/486 localbus !\\
\verb!(--) S3: chipset:   864 rev. 0 !\\
\verb!(--) S3: chipset driver: mmio_928 !
\end{tscreen}
Aqu� se ve que los dos chips v�lidos para este servidor (en este caso
{\tt XF86\_S3}) son {\tt mmio\_928} y {\tt s3\_generic}. El servidor
comprob� y encontr� una tarjeta de v�deo que tiene el chip 
{\tt mmio\_928}.

En la secci�n {\tt Device} del fichero {\tt XF86Config} a�ada una l�nea
{\tt Chipset} que tenga el nombre del chip determinado como arriba. Por
ejemplo,
\begin{tscreen}\begin{verbatim}
Section "Device" 
        # Ya tenemos Identifier aqu� ...
        Identifier "#9 GXE 64"  
        # A�ada esta l�nea:
        Chipset "mmio_928"
EndSection
\end{verbatim}\end{tscreen}

Ahora hay que determinar las frecuencias de funcionamiento del reloj
utilizadas por la tarjeta de v�deo. Una frecuencia de funcionamiento
del reloj o dot clock, es simplemente la frecuencia a la que la tarjeta
de v�deo puede enviar pixels al monitor. Como se describi� m�s arriba,
cada resoluci�n de monitor tiene un dot clock asociado a ella. Se
requiere determinar qu� dot clocks se ofrecen con la tarjeta de v�deo.

En primer lugar, deber�a mirar la documentaci�n arriba mencionada y ver
si los dot clocks de la tarjeta est�n listados all�. Los dot clocks son
generalmente una lista de 8 o 16 valores, todos en MHz. Por ejemplo, si
mira en {\tt modeDB.txt} ver� una entrada para la tarjeta Cardinal ET4000
que tiene este aspecto:
\begin{tscreen}\begin{verbatim}
# chip    ram   virtual   clocks                           default-mode  flags
 ET4000   1024  1024 768   25  28  38  36  40  45  32   0  "1024x768"    
\end{verbatim}\end{tscreen}
Los dot clocks para esta tarjeta son 25, 28, 38, 36, 40, 45, 32 y 0 MHz.

En la secci�n {\tt Devices} del archivo {\tt XF86Config} a�ada una l�nea
{\tt Clocks} que contenga la lista de los dot clocks para su tarjeta. Por
ejemplo, para los dot clocks de arriba, a�ada la l�nea
\begin{tscreen}\begin{verbatim}
        Clocks 25 28 38 36 40 45 32 0
\end{verbatim}\end{tscreen}
a la secci�n {\tt Devices} del fichero, despu�s de {\tt Chipset}.

\blackdiamond {\bf El orden de los dot clocks es importante.} No reordene
la lista o elimine duplicados.

Si no puede encontrar los dot clocks asociados con su tarjeta, el 
servidor X puede comprobarlos tambi�n. Utilice {\tt X -probeonly} como
se describi� arriba. La salida deber�a contener l�neas similares a la
siguiente:
\begin{tscreen}\begin{verbatim}
(--) S3: clocks:  25.18  28.32  38.02  36.15  40.33  45.32  32.00  00.00
\end{verbatim}\end{tscreen}
Se puede a�adir una l�nea {\tt Clocks} que contenga todos estos valores.
Puede utilizar m�s de una l�nea {\tt Clocks} en {\tt XF86Config} si 
todos los valores (a veces hay m�s de 8 valores de clock) no caben en
una �nica l�nea. De nuevo aseg�rese de poner la lista de clocks en el
orden en que aparecen.

\blackdiamond
Aseg�rese de que no hay l�neas de {\tt Clocks} (o de que est�n comentadas)
en la secci�n {\tt Devices} del fichero cuando utilice {\tt X -probeonly}.
Si hay una l�nea {\tt Clocks} presente, el servidor no comprueba los
clocks, emplea los valores de {\tt XF86Config}.

Algunas tarjetas de v�deo utilizan un chip de reloj programable. Mire
la p�gina del manual de su servidor X o el fichero {\tt README} de 
XFree86 que describe su tarjeta de v�deo. Esencialmente, el chip 
permite que el servidor X le diga a la tarjeta los dot clocks a usar.
Para tarjetas de v�deo que tienen clock chips, puede ser que no 
encuentre una lista de los dot clocks para la tarjeta en ninguno de 
los ficheros arriba mencionados o que la lista de dot clocks impresa 
cuando utilice {\tt X -probeonly} contenga uno o dos valores discretos
de clock, siendo los dem�s duplicados o cero. O tambi�n puede que el
servidor X s�lo facilite un aviso expl�cito de que la tarjeta de v�deo
tiene un chip de reloj programable como:
\begin{tscreen}\begin{verbatim}
(--) SVGA: cldg5434: Specifying a Clocks line makes no sense for this driver
\end{verbatim}\end{tscreen}
Este ejemplo est� sacado de un servidor {\tt XF86\_SVGA} con tarjeta Cirrus
Logic PCI.

Para tarjetas que emplean chips de reloj programables, utilice una 
l�nea {\tt ClockChip} en vez de una l�nea de {\tt Clocks} en el fichero
{\tt XF86Config}. {\tt ClockChip} es el nombre del chip de reloj tal
como es usado por la tarjeta de v�deo. Las p�ginas del manual de cada
servidor los describen. Por ejemplo en el fichero {\tt README.S3}, se
puede observar que varias tarjetas de v�deo S3-864 utilican un chip de 
reloj ``ICD2061A'' y que se debe utilizar la l�nea
\begin{tscreen}
\verb!        ClockChip "icd2061a" !
\end{tscreen}
en vez de {\tt Clocks} en el fichero {\tt XF86Config}. Esta l�nea como
la de {\tt Clocks}, va dentro de la secci�n {\tt Devices}, tras 
{\tt Chipset}.

De modo similar, algunas tarjetas de v�deo exigen que especifique el
tipo de chip RAMDAC en el fichero {\tt XF86Config}. Esto puede hacerse
con una l�nea {\tt Ramdac}. La p�gina del manual {\tt XF86\_Accel} 
describe esta opci�n. A menudo el servidor X ser� capaz de calcular el
RAMDAC correctamente.

Algunas tarjetas de v�deo exigen que especifique varias opciones en la
secci�n {\tt Devices} de {\tt XF86Config}. Estas opciones se describen
en la p�gina del manual del servidor, as� como en varios ficheros como
{\tt README.cirrus} y {\tt README.S3}. Estas opciones se habilitan con
una l�nea {\tt Option}. Por ejemplo la tarjeta \#9 GXE 64 necesita dos 
opciones: 
\begin{tscreen}
\verb!        Option "number_nine" !\\
\verb!        Option "dac_8_bit" !
\end{tscreen}
Un servidor X puede funcionar sin las l�neas {\tt Option}, pero son 
necesarias para lograr el mejor rendimiento de la tarjeta. Hay tambi�n
muchas otras opciones que listar aqu�, son diferentes para cada tarjeta.
Si debe usar una, las p�ginas del manual del servidor X y varios ficheros
en /usr/X11R6/lib/X11/doc le dir�n cu�les son.

Cuando termine, deber�a tener una secci�n {\tt Devices} que se parezca 
a esta:
\begin{tscreen}
\verb!Section "Device"  !\\
\verb!        # !` Secci�n Device s�lo para la \#9 GXE 64 !\\
\verb!        Identifier "#9 GXE 64" !\\
\verb!        Chipset "mmio_928" !\\
\verb!        ClockChip "icd2061a" !\\
\verb!        Option "number_nine" !\\
\verb!        Option "dac_8_bit" !\\
\verb!EndSection !
\end{tscreen}
Hay otras opciones que puede incluir en la entrada {\tt Devices}. Las 
p�ginas del manual del servidor X ofrecen los detalles exactos.

\section{Ejecutar XFree86.}

Con el fichero {\tt XF86Config} ya configurado, puede arrancar el 
servidor X y darse una vuelta. De nuevo, aseg�rese de que el directorio
{\tt /usr/X11R6/bin} est� en el path.

La orden para arrancar XFree86 es 
\begin{tscreen}
startx
\end{tscreen}
No es m�s que una fachada de {\tt xinit}. Inicia el servidor X y 
ejecuta las �rdenes del fichero {\tt .xinitrc} de su directorio 
home. {\tt .xinitrc} es un script del shell que contiene las l�neas de 
�rdenes de los clientes X para ejecutarse cuando arranque el servidor X.
Si este fichero no existe se utiliza por omisi�n el 
{\tt /usr/X11R6/lib/X11/xinit/xinitrc} del sistema.

Un fichero {\tt .xinitrc} sencillo tiene este aspecto: 
\begin{tscreen}\begin{verbatim}
#!/bin/sh

xterm -fn 7x13bold -geometry 80x32+10+50 &
xterm -fn 9x15bold -geometry 80x34+30-10 &
oclock -geometry 70x70-7+7 &
xsetroot -solid midnightblue &

exec twm 
\end{verbatim}\end{tscreen}
Este script inicia dos clientes {\tt xterm} y un {\tt oclock} y
pone el color de la ventana principal (background) a {\tt midnightblue}.
Inicia {\tt twm}, el gestor de ventanas. {\tt twm} se ejecuta con la
instrucci�n del shell {\tt exec}. Esto provoca que el proceso {\tt init}
sea reemplazado por {\tt twm}. Despu�s de que el proceso {\tt twm} 
finalice, el servidor X se desconecta. Puede provocar la finalizaci�n
de {\tt twm} utilizando el men� principal. Pulse el bot�n 1 del rat�n
en el fondo del escritorio. Se desplegar� un men� que le permitir� 
elegir {\tt Exit Twm}.

Aseg�rese de que la �ltima instrucci�n de {\tt .xinitrc} empieza con
{\tt exec} y de que no se lanza en segundo plano (no debe haber ampersand
al final de la l�nea). Si no, el servidor X se desconectar� en cuanto
inicie los programas cliente del fichero {\tt .xinitrc}.

Alternativamente, puede salir de X pulsando a la vez
\key{Ctrl}-\key{Alt}-\key{Backspace}. Esto mata al proceso del 
servidor X directamente, saliendo del sistema de ventanas.

Arriba se muestra una configuraci�n simple del escritorio. 
% Many wonderful
% programs and configurations are available with a bit of work on your
% {\tt .xinitrc} file. For example, the {\tt fvwm} window manager will
% provide a virtual desktop, and you can customize colors, fonts, 
% window sizes and positions, and so forth to your heart's content. 
% Although the X~Window System might appear to be simplistic at first,
% it is extremely powerful once you customize it for yourself.
De nuevo se sugiere que lea un libro como {\em The X~Window System:
A User's Guide\/} (vea Ap�ndice \ref{app-info}). Las posibles 
variaciones en el uso de X y su configuraci�n son demasiadas para
describirlas aqu�. Las p�ginas de manual de {\tt xterm}, {\tt oclock}
y {\tt twm} le ofrecen pistas acerca de c�mo empezar.

\section{Si se l�a demasiado el asunto}

A menudo tendr� alg�n problema con algo que no funcione a la 
perfecci�n la primera vez que inicie el servidor X. Esto est�
provocado casi siempre por algo en su fichero {\tt XF86Config}. 
Normalmente los valores de tiempo del monitor o los dot clocks de la
tarjeta gr�fica contienen valores incorrectos. Si la pantalla parece
movida o los bordes son borrosos es un indicativo de que los valores
de tiempo del monitor o de los dot clocks son err�neos. Aseg�rese 
tambi�n de que especific� correctamente el chip de la tarjeta de v�deo
y dem�s opciones en la secci�n {\tt Device} de {\tt XF86Config}. 
Cerci�rese absolutamente de que est� usando el servidor X correcto y
de que {\tt /usr/X11R6/bin/X} es un enlace simb�lico a �l.

Si algo m�s falla, intente iniciar X ``desnudo'', es decir con una
orden como:
\begin{tscreen}
\verb!X > /tmp/x.out 2>&1 !
\end{tscreen}
Entonces podr� matar el servidor X (con 
\key{Ctrl}-\key{Alt}-\key{Backspace}) y examinar los contenidos de
{\tt /tmp/x.out}. El servidor X informa de cualquier aviso o error,
por ejemplo, si su tarjeta de v�deo no tiene un dot clock 
correspondiente al modo permitido por su monitor.

El fichero {\tt VideoModes.doc} incluido en la distribuci�n XFree86
contiene muchas ayudas para ajustar los valores de su fichero 
{\tt XF86Config}.

Recuerde que puede utilizar \key{Ctrl}-\key{Alt}-\key{Numeric +} 
y \key{Ctrl}-\key{Alt}-\key{Numeric -} para cambiar entre los modos
de v�deo listados en la l�nea {\tt Modes} de la secci�n {\tt Screen}
de {\tt XF86Config}. Si el modo de resoluci�n m�s alto no tiene 
buen aspecto, intente cambiar a una resoluci�n m�s baja. Al menos
esto le deja averiguar qu� partes de su configuraci�n de X est�n
funcionando correctamente.

Ajuste tambi�n los botones de tama�o vertical y horizontal de su
monitor. En muchos casos, es necesario ajustarlos cuando inicie X.
Por ejemplo si la pantalla parece estar ligeramente desplazada a un
lado, muchas veces puede corregir esto utilizando los controles del
monitor. 

De nuevo, el grupo de noticias de USENET {\tt comp.windows.x.i386unix}
est� dedicado a discusiones acerca de XFree86. Ser�a buena idea leer los
grupos de noticias relacionados con configuraci�n de v�deo. Puede ser
que se encuentre a alguien con el mismo problema.

Tambi�n hay ficheros {\tt XF86Config} de ejemplo creados por usuarios.
Algunos de ellos est�n disponibles en el dep�sito de 
{\tt sunsite.unc.edu} en el directorio {\tt /pub/Linux/X11} y en otros
lugares. Tambi�n podr�a encontrar un fichero de configuraci�n que 
alguien haya escrito ya para su hardware.

\index{X~Window~System|)}








