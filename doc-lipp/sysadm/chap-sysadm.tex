% Linux Installation and Getting Started    -*- TeX -*-
% chap-sysadm.tex
% Copyright (c) 1993 by Matt Welsh and Lars Wirzenius
%
% This file is freely redistributable, but you must preserve this copyright 
% notice on all copies, and it must be distributed only as part of "Linux 
% Installation and Getting Started". This file's use is covered by
% the copyright for the entire document, in the file "copyright.tex".
%
% Copyright (c) 1998 by Specialized Systems Consultants Inc. 
% <ligs@ssc.com>

% Traducci�n realizada por Alberto Molina. Comentarios a:
%            alberto@nucle.us.es
% Revisi�n 1 por Fco. javier Fernandez <serrador@arrakis.es>
%Gold

\def\fn#1{{\tt #1}}
\def\cmd#1{{#1}}

\chapter{Administraci�n del Sistema}\label{chap-sysadm}
\markboth{Administraci�n del Sistema}{}
\label{chap-sysadm-num}

Este cap�tulo cubre las cosas m�s importantes que se necesitan saber
acerca de la administraci�n del sistema para comenzar a usarlo sin problemas. Para que el cap�tulo tenga un tama�o razonable,
cubre s�lo lo b�sico y omite muchos detalles importantes. El libro
{\em Linux System Administrator's Guide}, de Lars Wirzenius (ver
Ap�ndice~\ref{app-info}) proporciona m�s detalles sobre la
administraci�n del sistema. Adem�s ayudar� a entender mejor c�mo
trabajan y cuelgan las cosas entre s�.

% Linux Installation and Getting Started    -*- TeX -*-
% hats.tex
% Copyright (c) 1993 by Matt Welsh and Lars Wirzenius
%
% This file is freely redistributable, but you must preserve this copyright 
% notice on all copies, and it must be distributed only as part of "Linux 
% Installation and Getting Started". This file's use is covered by
% the copyright for the entire document, in the file "copyright.tex".
%
% Copyright (c) 1998 by Specialized Systems Consultants Inc. 
% <ligs@ssc.com>

% Traducci�n realizada por Alberto Molina. Comentarios a:
%            alberto@nucle.us.es
% Revisi�n 1 7/7/2002 <serrador@arrakis.es>
% Revisi�n 2 30/7/2002 <serrador@arrakis.es>

\section{La cuenta {\tt root}} 
\markboth{Administraci�n del Sistema}{La cuenta {\tt root}}

GNU/Linux diferencia entre varios usuarios. Lo que puede hacer cada uno
con respecto a los dem�s est� regulado. Los permisos de ficheros est�n
regulados de manera que los usuarios normales no puedan borrar o
modificar ficheros de directorios como {\tt /bin} y {\tt
  /usr/bin}. Muchos usuarios protegen sus ficheros con los permisos
apropiados, para que otros usuarios no tengan acceso a ellos (uno no
querr�a que nadie leyese sus cartas de amor). Cada usuario tiene una
{\bf cuenta} que incluye su nombre de usuario y su directorio
``home''. Adem�s, hay cuentas especiales definidas por el sistema que
tienen privilegios especiales. La m�s importante de todas es la {\bf
  cuenta root}, que usa el administrador del sistema. Por convenio, el
administrador del sistema es el usuario {\tt root}.

No hay restricciones para {\tt root}. �l o ella puede leer, modificar
o borrar cualquier fichero del sistema, cambiar los permisos y la
propiedad de los ficheros y ejecutar programas especiales como los que
particionan un disco duro o crean sistemas de ficheros. La idea
fundamental es que es una persona que vigila los registros del sistema
y que realiza tareas que no pueden ejecutar los usuarios
normales. Puesto que {\tt root} puede hacer cualquier cosa, es f�cil
cometer errores con consecuencias catastr�ficas.

Si un usuario normal tratase inadvertidamente de borrar todos los
ficheros de {\tt /etc}, el sistema no se lo permitir�a. Sin embargo,
si lo intentase {\tt root} el sistema no se lo impedir�a. Es muy f�cil
destrozar un sistema GNU/Linux usando {\tt root}. La mejor manera de
prevenir accidentes es:

\begin{itemize}
\item Pens�rselo dos veces antes de pulsar \key{Enter} para una orden no
  reversible. Si se va a borrar un directorio, revisar la orden completa
  para estar seguro de que es correcta.

\item Usar un prompt diferente para la cuenta {\tt root}. En los
  ficheros {\tt .bashrc} o {\tt .login} de la cuenta {\tt root}
  deber�a especificarse el prompt con algo diferente al del resto de
  usuarios. Mucha gente reserva el car�cter ``{\tt \#}'' para el
  prompt de {\tt root} y usa ``{\tt \$}'' para el del resto de usuarios.

\item Entrar como {\tt root} s�lo cuando sea estrictamente
  necesario. Cuando se hayan finalizado las tareas como administrador del
  sistema, salir de dicha cuenta. Cuanto menos se utilice dicha cuenta,
  menos da�o podr� provocarle al sistema. 
\end{itemize}

Uno se puede imaginar la cuenta {\tt root} como un sombrero m�gico que le da
inmensos poderes y con el que se puede, simplemente moviendo las manos,
destruir ciudades enteras. Es una buena imagen para ser cuidadoso y
saber lo que se tiene entre manos. Puesto que es tan f�cil destruir
cosas con sus manos, no es una buena idea ponerse el sombrero cuando
no hace falta, a pesar de la magn�fica sensaci�n.

Comentaremos con m�s detalle las responsabilidades del administrador
del sistema a partir de la p�gina~\pageref{sec-manage-users}.




  %% rak
% Linux Installation and Getting Started    -*- TeX -*-
% booting.tex
% Copyright (c) 1992-1994 by Matt Welsh <mdw@sunsite.unc.edu>
%
% This file is freely redistributable, but you must preserve this copyright 
% notice on all copies, and it must be distributed only as part of "Linux 
% Installation and Getting Started". This file's use is covered by the 
% copyright for the entire document, in the file "copyright.tex".
%
% Copyright (c) 1998 by Specialized Systems Consultants Inc. 
% <ligs@ssc.com>

%Traducido al espa�ol por Sebasti�n Gurin, Cancerbero <anon@adinet.com.uy>
%Revisi�n 1 realizada el 9 de julio de 2002 por Francisco Javier Fern�ndez <serrador@arrakis.es>
% Revisi�n 2 realizada el 15 de julio de 2002 por Fco. javier fernandez 

\subsection{Problemas arrancando desde el disquete de instalaci�n}
%Problems with booting the installation media}
\namedlabel{sec-install-probs-booting}{problemas en el inicio}

\index{instalaci�n!problemas iniciando}
\index{inicio!problemas}

Cuando se intenta iniciar el disquete de instalaci�n por primera vez, puede encontrarse con algunos problemas, los cuales se enumeran abajo. Por favor, tenga en cuenta que estos inconvenientes
 {\em no est�n\/} relacionados con el inicio de su sistema GNU/Linux nuevo y ya instalado. Consulte la p�gina~\pageref{sec-install-probs-postinstall} para m�s informaci�n sobre estos problemas. 

\begin{itemize}
\item {\bf Error en el disquete o en otro dispositivo al intentar arrancar el sistema}

La causa m�s frecuente de esta clase de problema es que el disquete est� corrupto. Si el disquete se encuentra f�sicamente da�ado, deber�a construirlo de nuevo usando un disquete en buen estado. 
Si los datos del disquete son los que se encuentran defectuosos, deber�a verificar que ha descargado y transferido los datos al disquete correctamente. Generalmente, para solucionar este problema, bastar�
simplemente con volver a crear el disquete de arranque. Repita todos los pasos, e intente nuevamente. 

Si ha recibido su disquete de arranque por correo o alg�n otro distribuidor en lugar de descargarlo y crearlo por usted mismo, comun�quese con el distribuidor para pedirle uno nuevo --pero s�lo despu�s
de verificar que �ste es, efectivamente el problema--.

\item {\bf El sistema se cuelga durante el arranque, o despu�s de arrancar} 

Despu�s de que el disquete arranca, ver� un n�mero de mensajes del n�cleo, indicando cu�les dispositivos fueron detectados y configurados. Despu�s de �sto, normalmente, se le presentar� un indicador
de ingreso\NT{``login indicador de �rdenes'' en el original}, permiti�ndole proceder con la instalaci�n (en lugar de esto, algunas distribuciones le lanzar�n justo al programa de instalaci�n). El 
sistema  puede parecer como si estuviera bloqueado, durante muchos de estos pasos. Tenga paciencia: cargar software desde un disquete es un proceso lento. En muchos casos el sistema puede no haberse 
colgado de ninguna manera, simplemente necesita algo de tiempo. Verifique que no haya ninguna actividad, ya sea en el disco o en el sistema, por lo menos por unos cuantos minutos, antes de suponer que 
el sistema est� bloqueado. 

\begin{enumerate}
\index{LILO!problemas iniciando}

\item Despu�s de iniciar el sistema desde el indicador {\tt LILO}, se debe cargar la imagen del n�cleo del disquete. Esto puede tomar unos cuantos segundos; usted podr� asegurarse de que todo marcha bien
si la luz de la disquetera est� encendida. 

\item Cuando arranca el n�cleo, los dispositivos SCSI deben ser detectados. Si no posee ning�n dispositivo SCSI, el sistema har� un alto de 15 segundos, mientras contin�a la detecci�n de los posibles 
SCSI; esto ocurre, generalmente, despu�s de que la l�nea
\begin{tscreen}
lp\_init: lp1 exists (0), using polling driver
\end{tscreen}
aparece en pantalla. 

\item Una vez que  el n�cleo halla terminado de cargarse, el control pasa a los ficheros de arranque que hay en el disquete. Al cabo de esto, usted podr� ver un indicador de entrada \NT{ login}, o bien, 
se iniciar� un programa de instalaci�n. Si se le presenta un indicador de ingreso como
\begin{tscreen}
Linux login:
\end{tscreen}
entonces deber�a entrar, (por lo normal es  como {\tt root} o {\tt install} --- esto var�a seg�n cada distribuci�n). Antes de ingresar el nombre de usuario, el sistema puede detenerse por 20 segundos o
m�s, mientras el programa de instalaci�n o el int�rprete de �rdenes se  carga desde el disquete. Nuevamente, si esto sucediera, la luz de la disquetera deber� estar encendida. No suponga que el sistema 
est� colgado. 

\end{enumerate}

Cualquiera de las causas comentadas arriba, puede ser el origen de su problema. De todos modos, es posible que el sistema realmente se cuelgue mientras se inicia, lo cual puede deberse a muchas causas.
En primer lugar, puede ser que usted no posea suficiente memoria RAM para iniciar el disquete de instalaci�n. (Vea el siguiente punto al respecto para m�s informaci�n acerca de c�mo desactivar el disco
RAM (ramdisk) para liberar memoria.)

La causa del cuelgue de muchos sistemas, es la incompatibilidad del hardware. El �ltimo cap�tulo muestra un vistazo general del hardware soportado en GNU/Linux. Incluso si sus dispositivos son soportados,
 usted puede meterse en problemas si tiene configuraciones incompatibles de hardware, las cu�les pueden estar causando que el sistema se cuelgue. Vea la p�gina~\pageref{sec-install-probs-hardware}, 
adelante, para una discusi�n sobre incompatibilidades de hardware. 


\item {\bf El sistema informa de errores a causa de falta de memoria, mientras intenta arrancar o instalar el software} 

Este punto trata sobre la cantidad de RAM con la que se dispone. En sistemas con 4 Mbytes o menos de RAM, se  pueden tener problemas al arrancar el disquete, o al instalar el software. Esto se produce 
ya que muchas distribuciones usan un disco RAM (ramdisk), que se trata de un sistema de ficheros cargado directamente en la memoria RAM, para las operaciones que se ejecutan mientras el disquete de 
instalaci�n est� siendo utilizado. La imagen entera del disquete de instalaci�n, por ejemplo, puede  cargarse en un disco RAM (ramdisk), lo que puede requerir m�s de un Mbyte de memoria RAM. 

% La soluci�n para este problema, es desactivar la opci�n de disco RAM 
% cuando se inicia el disquete de instalci�n. El procedimiento para efectuar 
% esto depende de la versi�n de Linux. Por ejemplo, en SLS, usted 
% deber� teclear ``{\tt floppy}'' en el indicador de �rdenes LILO, cuando se inicia el disco 
% {\tt a1}. 

Puede suceder que, cuando se intenta iniciar o instalar el software, en lugar de un mensaje de error por falta de memoria, su sistema se bloquee inesperadamente durante el arranque. 
Si su sistema se cuelga, y ninguna de las explicaciones de la secci�n anterior parezcan ser la causa, trate de desactivar el disco RAM. Vea la documentaci�n de su distribuci�n para m�s detalles. 

Acu�rdese de que GNU/Linux necesita para s� mismo, por lo menos 2 Mbytes de memoria RAM para ejecutarse; distribuciones m�s modernas requieren 4 Mbytes o m�s. 


\item {\bf El sistema informa de errores como ``{\tt permission denied}\NT{ permiso denegado}'', o ``{\tt file not found}\NT{ fichero no encontrado}'', durante el arranque}

Esto es un indicio de que su disquete de instalaci�n est� da�ado o contiene datos corruptos. Si usted est� tratando de iniciar el sistema desde el disquete (y est� seguro de que est� haciendo todo
correctamente), no deber�a estar viendo errores como �ste. Contacte con  su distribuidor de software GNU/Linux e indague sobre el problema. Quiz� pueda obtener otra copia del disquete de arranque si
 es necesario. Si usted ha descargado el disco de inicio, trate de construirlo nuevamente en un disquete sano. Posiblemente esto resuelva su problema. 


\item {\bf El sistema informa del error ``{\tt VFS: Unable to mount root}\NT{ no se puede montar ra�z}'' cuando se inicia}

Este mensaje de error indica que el sistema de ficheros ra�z (que se encuentra en el disquete de arranque), no puede ser localizado. Esto puede significar que su disquete de arranque est� corrupto de
 alguna manera, o que el sistema no se est� inicializando correctamente. 

Por ejemplo, muchas distribuciones de CD-ROM requieren que se encuentre el CD-ROM en la lectora, cuando se inicia la instalaci�n. Aseg�rese de que la lectora de CD-ROM est� encendida y  que funcione 
correctamente. Tambi�n es posible que el sistema no pueda localizar su unidad de CD-ROM cuando se inicia; para m�s informaci�n consulte la p�gina~\pageref{sec-install-probs-hardware}. 

% Si usted est� seguro de estar iniciando el sistema correctamente, 
% entonces su disquete de instalaci�n puede estar de veras da�ado. Este 
% es un problema bastante extra�o, y es por esto que deber�a tratar con 
% otras soluciones antes de intentar usar otro disquete o cinta magn�tica 
% de arranque. 


\end{itemize}

\index{instalaci�n!problemas en el inicio}
\index{iniciando!problemas}


% Traducci�n terminada por Sebasti�n Gurin, Cancerbero <anon@adinet.com.uy> 
% Linux Installation and Getting Started    -*- TeX -*-
% shutdown.tex
% Copyright (c) 1993 by Matt Welsh and Lars Wirzenius
%
% This file is freely redistributable, but you must preserve this copyright 
% notice on all copies, and it must be distributed only as part of "Linux 
% Installation and Getting Started". This file's use is covered by
% the copyright for the entire document, in the file "copyright.tex".
%
% Copyright (c) 1998 by Specialized Systems Consultants Inc. 
% <ligs@ssc.com>
%
% Este fichero es de distribuci�n libre, pero debe mantenerse esta 
% informaci�n de Copyright en todas las copias, y debe distribuirse solo como
% parte de "Instalaci�n y Primeros Pasos en Linux". El uso de este fichero esta
% cubierto por el Copyright del documento completo, en el fichero "copyright.tex"
% Copyright (c) 1995 por Gerardo Izquierdo para la versi�n al Castellano
% $Log: shutdown.tex,v $
% Revision 1.9  2003/07/19 20:28:24  pakojavi2000
% Arreglando un peque�o destrozo de macros
%
% Revision 1.8  2003/07/19 06:20:33  joseluis.ranz
% Correcciones varias.
%
% Revision 1.7  2002/07/30 16:23:05  pakojavi2000
% Beta 2.2 Formatos de p�rrafo
%
% Revision 1.6  2002/07/20 22:24:29  pakojavi2000
% Beta2
%
% Revision 1.5  2002/07/20 17:41:16  pakojavi2000
% beta2
%
% Revision 1.4  2002/07/13 12:50:07  pakojavi2000
% Gold
%
% Revision 1.3  2001/04/18 16:29:10  amolina
% Segunda revisi�n de los ficheros
%
% Revision 1.2  2000/12/20 16:51:28  amolina
%
% Primera versi�n traducida de sysadm/shutdown.tex
%
% Revision 0.5.0.1  1996/02/10 23:45:13  rcamus
% Primera beta publica
%

% Versi�n para lipp 2.0 por Alberto Molina
%         comentarios a alberto@nucle.us.es
%
%Revisi�n 1 13 de julio 2002 por JFS <serrador@arrakis.es
%Gold
\section{Parada del sistema}
\markboth{Administraci�n del Sistema}{Parada del Sistema}
\label{sec-sysadm-shutdown}

\index{administraci�n del sistema!cierre del sistema|(}
\index{cierre del sistema|(}
Cerrar un sistema {\linux} tiene algo de truco. Hay que recordar que nunca se debe
cortar la corriente o pulsar el bot�n de apagado mientras el sistema
est� ejecut�ndose. El n�cleo sigue la pista de la entrada/salida a disco
en ``buffers'' de memoria. Si se reinicializa el sistema sin darle al n�cleo
la oportunidad de escribir sus ``buffers'' a disco, puede corromper sus sistemas
de ficheros.

En tiempo de cierre se toman tambi�n otras precauciones. Todos los procesos
reciben una se�al que les permite morir airosamente (escribiendo y cerrando 
todos los ficheros y ese tipo de cosas). Los sistemas de ficheros se 
desmontan por seguridad. Si se desea, el sistema tambi�n puede alertar a los
usuarios de que se est� cerrando y darles la posibilidad de desconectarse.

\index{orden shutdown@comando {\tt shutdown}}
La forma m�s simple de cerrar el sistema es con la orden {\tt 
shutdown}. El formato es
\begin{tscreen}
shutdown \cparam{tiempo} \cparam{mensaje-de-aviso}
\end{tscreen}
El argumento \cparam{tiempo} es el momento de cierre del sistema (en el
formato {\em hh:mm:ss}), y \cparam{mensaje-de-aviso} es un mensaje mostrado
en todos los terminales de usuario antes de cerrar. Alternativamente, se
puede especificar el par�metro \cparam{tiempo} como ``{\tt now}'', para
cerrar inmediatamente. Se le puede suministrar la opci�n {\tt -r} a 
{\tt shutdown} para reinicializar el sistema tras el cierre.

Por ejemplo, para cerrar el sistema a las 8:00pm, se puede utilizar la
siguiente orden
\begin{tscreen}
\# {\em shutdown -r 20:00}
\end{tscreen}

\index{halt@{\tt halt}}
La orden {\tt halt} puede utilizarse para forzar un cierre inmediato, sin
ning�n mensaje de aviso ni periodo de gracia. {\tt halt} se utiliza si se
es el �nico usuario del sistema y se quiere cerrar el sistema y apagarlo.

\blackdiamond No apagar o reinicializar el sistema hasta que se vea el mensaje:
\begin{tscreen}
The system is halted
\end{tscreen}
Es muy importante que cierre el sistema ``limpiamente'' utilizando la
orden {\tt shutdown} o el {\tt halt}. En algunos sistemas, se reconocer� 
el pulsar \key{ctrl-alt-del}, que causar� un {\tt shutdown}; en otros 
sistemas, sin embargo, el utilizar el ``Apret�n de Cuello de Vulcano''
reinicializar� el sistema inmediatamente y puede causar un desastre.

\index{administraci�n del sistema!cierre del sistema|)}
\index{cierre del sistema|)}



% inittabl.tex. this is new to the SSC edition, 12/8/97 -- rak
%
% Copyright (c) 1998 by Specialized Systems Consultants Inc. 
% <ligs@ssc.com> 

% Traducci�n realizada por Alberto Molina. Comentarios a:
%            alberto@nucle.us.es
% 


\subsection{El fichero {\tt /etc/inittab}} \label{sec-inittab}
\markboth{Administraci�n del Sistema}{El fichero {\tt /etc/inittab}}

Despu�s de que {\linux} arranque y el n�cleo monte el sistema de
ficheros de root, el primer programa que ejecuta el sistema es {\tt
  init}. Este programa es el encargado de lanzar los guiones de
inicializaci�n del sistema y de modificar el sistema operativo de su
estado inicial de arranque al estado est�ndar multiusuario. Tambi�n
define los int�rpretes de �rdenes {\tt login:} de todos los dispositivos tty del
sistema y especifica otras caracter�sticas del arranque y apagado.

Tras el arranque, {\tt init} permanece latente en segundo plano,
``monitoreando'' y si fuera necesario alterando la ejecuci�n del
sistema. Hay muchos detalles que deben comentarse del programa {\tt
  init}. Todas las tareas que realiza se definen en el fichero {\tt
  /etc/inittab}. Un ejemplo de dicho fichero se muestra a continuaci�n.

\blackdiamond Modificar el fichero {\tt /etc/inittab} de forma
incorrecta, puede impedirle registrarse en el sistema. Por ello, cuando se
modifique dicho fichero, hay que guardar una copia del fichero
original, adem�s de tener a mano el disquete de inicio, para el caso
en que se cometiera alg�n error.

\begin{tscreen}\begin{verbatim}
#
# inittab       Este fichero describe como el proceso INIT debe 
#               ajustar el sistema en ciertos niveles de ejecuci�n.
#
# Version:      @(#)inittab             2.04    17/05/93        MvS
#                                       2.10    02/10/95        PV
#
# Author:       Miquel van Smoorenburg, <miquels@drinkel.nl.mugnet.org>
# Modified by:  Patrick J. Volkerding, <volkerdi@ftp.cdrom.com>
# Minor modifications by:
#               Robert Kiesling, <kiesling@terracom.net>
#
# Nivel de ejecuci�n asumido.
id:3:initdefault:

# Iniciaci�n del sistema (se ejecuta al arrancar el sistema).
si:S:sysinit:/etc/rc.d/rc.S

# Script para ejecutarse cuando el sistema vaya a un usuario 
# (nivel de ejecuci�n 1).  
su:1S:wait:/etc/rc.d/rc.K

# Script para ejecutarse cuando el sistema vaya a multiusuario.
rc:23456:wait:/etc/rc.d/rc.M

# Qu� hacer cuando se pulse Ctrl-Alt-Del
ca::ctrlaltdel:/sbin/shutdown -t5 -rfn now

# El nivel de ejecuci�n 0 para el sistema.
l0:0:wait:/etc/rc.d/rc.0

# El nivel de ejecuci�n 6 reinicia el sistema.
l6:6:wait:/etc/rc.d/rc.6

# Qu� hacer cuando se va el suministro el�ctrico (bajar al nivel de
# ejecuci�n de un usuario).
pf::powerfail:/sbin/shutdown -f +5 "EL SUMINISTRO EL�CTRICO SE EST� CORTANDO"

# Si el suministro vuelve antes de bajar, cancelar el proceso.
pg:0123456:powerokwait:/sbin/shutdown -c "El SUMINISTRO EL�CTRICO EST�
VOLVIENDO"

# Si vuelve el suministro cuando se est� en modo de un usuario, volver
# al modo multiusuario.
ps:S:powerokwait:/sbin/init 5

# Los ``gettys'' en el modo multiusuario y las l�neas serie.
#
# NOTA NOTA NOTA �ajuste esto a su ``getty'' o no ser� capaz de ingresar!
#
# Nota: Debe especificar la velocidad de l�nea para ``agetty''.
# para ``getty_ps'' se usa una l�nea, se especifica la velocidad de
# l�nea y tambi�n se utiliza ``gettydefs''
c1:1235:respawn:/sbin/agetty 38400 tty1 linux
c2:1235:respawn:/sbin/agetty 38400 tty2 linux
c3:1235:respawn:/sbin/agetty 38400 tty3 linux
c4:1235:respawn:/sbin/agetty 38400 tty4 linux
c5:1235:respawn:/sbin/agetty 38400 tty5 linux
c6:12345:respawn:/sbin/agetty 38400 tty6 linux

# L�neas serie
# s1:12345:respawn:/sbin/agetty -L 9600 ttyS0 vt100
s2:12345:respawn:/sbin/agetty -L 9600 ttyS1 vt100

# L�neas de marcado telef�nico
d1:12345:respawn:/sbin/agetty -mt60 38400,19200,9600,2400,1200 ttyS0 vt100
#d2:12345:respawn:/sbin/agetty -mt60 38400,19200,9600,2400,1200 ttyS1 vt100

# El nivel de ejecuci�n 4 deber�a usarse para un sistema con X-window
# �nicamente, hasta que nos dimos cuenta de
# que lanzaba init en un bucle que manten�a la carga al menos a 1 todo  
# el tiempo. As� que, ahora hay un getty abierto en tty6. Esperemos que
# nadie se de cuenta. ;^)
# Quiz� no sea malo tener una consola de texto por ah�, en caso de que 
# le ocurriera algo a X.
x1:4:wait:/etc/rc.d/rc.4

# Fin de /etc/inittab

\end{verbatim}\end{tscreen}

Al iniciar, este {\tt /etc/inittab} lanza seis consolas virtuales, un
prompt de ingreso para el m�dem en {\tt /dev/ttys0} y un prompt de
ingreso para una terminal de caracteres conectada a trav�s de la l�nea
serie RS-232 a {\tt /dev/ttyS1}.

Brevemente podr�amos decir que el programa {\tt init} pasa a trav�s de
una serie de {\bf niveles de ejecuci�n}, que corresponden a varios
estados del sistema. Al nivel de ejecuci�n 1 se entra inmediatamente
despu�s de iniciar el sistema, los niveles de ejecuci�n 2 y 3 son los
modos de operaci�n del sistema normal y multiusuario respectivamente,
el nivel de ejecuci�n 4 lanza el sistema X Window a trav�s del X
display manager {\tt xdm} y el nivel de ejecuci�n 6 reinicia el
sistema. Los niveles de ejecuci�n asociados a cada orden, son el
segundo t�rmino de cada l�nea del fichero {\tt /etc/inittab}.

Por ejemplo, la l�nea:
\begin{tscreen}
s2:12345:respawn:/sbin/agetty -L 9600 ttyS1 vt100
\end{tscreen}
mantendr� un prompt de ingreso en una terminal serie para los niveles
de ejecuci�n 1--5. El ``{\tt s2}'' antes de los primeros dos puntos es
un identificador simb�lico que usa internamente {\tt init}. {\tt
  respawn} es una clave de {\tt init} que se usa a veces junto con las
terminales serie. Si tras un cierto per�odo de tiempo, el programa
{\tt agetty}, que genera los prompt de ingreso en las terminales, no
recibe se�al alguna en la terminal, termina su ejecuci�n. ``{\tt
respawn}'' hace que {\tt init} vuelva a ejecutar {\tt agetty},
asegurando que haya siempre un prompt de ingreso en la terminal,
independientemente de que haya alg�n otro ingreso. El resto de
par�metros se pasan directamente a {\tt agetty} y le especifican c�mo
debe generar la shell de ingreso, la capacidad de transferencia de
datos de la l�nea, el dispositivo serie y el tipo de terminal, como se
define en {\tt /etc/termcap} o {\tt /etc/terminfo}.

%The {\tt /sbin/agetty} program handles many details related to
%terminal I/O on the system. There are several different versions that
%are commonly in use on Linux systems. They include {\tt mgetty}, {\tt
%psgetty}, or simply, {\tt getty}.
El programa {\tt /sbin/agetty} maneja muchos detalles acerca de la E/S por
terminal en el sistema. Hay varias versiones diferentes que se unen regularmente en sistemas GNU/Linux.
Se incluyen {\tt mgetty}, {\tt psgetty} y {\tt getty}.

%In the case of the {\tt /etc/inittab} line 
En el caso de la l�nea de {\tt /etc/inittab}

\begin{tscreen}
d1:12345:respawn:/sbin/agetty -mt60 38400,19200,9600,2400,1200 ttyS0 vt100
\end{tscreen}
%which allows users to log in via a modem connected to serial line {\tt
%/dev/ttyS0}, the {\tt /sbin/agetty} parameters ``{\tt -mt60}'' allow
%the system to step through all of the modem speeds that a caller
%dialing into the system might use, and to shut down {\tt /sbin/agetty}
%if there is no connection after 60 seconds. This is called {\bf
%negotiating} a connection. The supported modem speeds are enumerated
%on the command line also, as well as the serial line to use, and the
%terminal type. Of course, both of the modems must support the data
%rate which is finally negotiated by both machines.

que permite a los usuarios ingresar usando un m�dem conectado a una l�nea serie
{\tt /dev/ttyS0}, los par�metros de {\tt /sbin/agetty} ``{\tt -m60}'' permiten
al sistema ir paso a paso por todas las velocidades del m�dem que un usuario
llamando al sistema puede usar, y apagar {\tt /sbin/getty}
si no hay ninguna conexi�n en 60 segundos. Esto se llama {\tt negociar} una
conexi�n. Las velocidades de modem soportadas se enumeran en la l�nea de �rdenes
tambi�n, as� como la l�nea serie a usar y el tipo de terminal. Desde luego, ambos m�dems
deben soportar el flujo de datos que se negocie finalmente por ambas m�quinas.


Se han pasado por alto muchos detalles importantes en esta
secci�n. Las tareas de {\tt /etc/inittab} ocupar�an un libro
completo. Para m�s informaci�n, pueden consultarse las p�ginas del
manual de {\tt init} y {\tt agetty} y los ``HOWTO'' del Proyecto de
Documentaci�n de Linux, disponibles en los lugares que se presentan en
el ap�ndice~\ref{app-sources-num}.



















\chapter{El sistema de archivos}
\label{filesystem-chapter}

\ChapterDescription{Este capitulo describe c�mo el kernel de Linux
  gestiona los ficheros en los sistemas de ficheros soportados por
  �ste.  Describe el Sistema de Ficheros Virtual (VFS) y explica c�mo
  los sistemas de ficheros reales del kernel de Linux son soportados.}

\index{File system} Una de los rasgos m�s importantes de Linux es su
soporte para diferentes sistemas de ficheros.  �sto lo hace muy
flexible y bien capacitado para coexistir con muchos otros sistemas
operativos.  En el momento de escribir �sto, Linux soporta 15 sistemas
de ficheros; \texttt{ext}, \texttt{ext2}, \texttt{xia},
\texttt{minix}, \texttt{umsdos}, \texttt{msdos}, \texttt{vfat},
\texttt{proc}, \texttt{smb}, \texttt{ncp}, \texttt{iso9660},
\texttt{sysv}, \texttt{hpfs}, \texttt{affs} and \texttt{ufs}, y sin
duda, con el tiempo se a�adir�n m�s.

En Linux, como en Unix\tm, a los distintos sistemas de ficheros que el
sistema puede usar no se accede por identificadores de dispositivo
(como un n�mero o nombre de unidad) pero, en cambio se combinan en una
simple structura jer�rquica de �rbol que representa el sistema de
ficheros como una entidad �nica y sencilla.  Linux a�ade cada sistema
de ficheros nuevo en este simple �rbol de sistemas de ficheros cuando
se monta.  Todos los sistemas de ficheros, de cualquier tipo, se
montan sobre un directorio y los ficheros del sistema de ficheros son
el contenido de ese directorio.  Este directorio se conoce como
directorio de montaje o punto de montaje.  Cuando el sistema de
ficheros se desmonta, los ficheros propios del directorio de montaje
son visibles de nuevo.

Cuando se inicializan los discos (usando \eg{fdisk}, por ejemplo)
tienen una estructura de partici�n inpuesta que divide el disco f�sico
en un n�mero de particiones l�gicas.  Cada partici�n puede mantener un
sistema de ficheros, por ejemplo un sistema de ficheros \texttt{EXT2}.
Los sistemas de ficheros organizan los ficheros en structuras
jer�rquicas l�gicas con directorios, enlaces flexibles y m�s
contenidos en los bloques de los dispositivos f�sicos.  Los
dispositivos que pueden contener sistemas de ficheros se conocen con
el nombre de dispositivos de bloque.  La partici�n de disco IDE
\fn{/dev/hda1}, la primera partici�n de la primera unidad de disco en
el sistema, es un dispositivo de bloque.  Los sistemas de ficheros de
Linux contemplan estos dispositivos de bloque como simples colecciones
lineales de bloques, ellos no saben o tienen en cuenta la geometr�a
del disco f�sico que hay debajo.  Es la tarea de cada controlador de
dispositivo de bloque asignar una petici�n de leer un bloque
particular de su dispositivo en t�rminos comprensibles para su
dispositivo; la pista en cuesti�n, sector y cilindro de su disco duro
donde se guarda el bloque.  Un sistema de ficheros tiene que mirar,
sentir y operar de la misma forma sin importarle con que dispositivo
est� tratando.  Por otra parte, al usar los sistemas de ficheros de
Linux, no importa (al menos para el usuario del sistema) que estos
distintos sistemas de ficheros est�n en diferentes soportes
controlados por diferentes controladores de hardware.  El sistema de
ficheros puede incluso no estar en el sistema local, puede ser
perfectamente un disco remoto montado sobre un enlace de red.
Considerese el siguiente ejemplo donde un sistema Linux tiene su
sistema de ficheros ra�z en un disco SCSI:
\begin{verbatim}
A         E         boot      etc       lib       opt       tmp       usr
C         F         cdrom     fd        proc      root      var       sbin
D         bin       dev       home      mnt       lost+found
\end{verbatim}
Ni los usuarios ni los programas que operan con los ficheros necesitan
saber que \fn{/C} de hecho es un sistema de ficheros VFAT montado que
est� en el primer disco IDE del sistema.  En el ejemplo (que es mi
sistema Linux en casa), \fn{/E} es el disco IDE primario en la segunda
controladora IDE.  No importa que la primera controladora IDE sea una
controladora PCI y que la segunda sea una controladora ISA que tambi�n
controla el IDE CDROM.  Puedo conectarme a la red donde trabajo usando
un modem y el protocolo de red PPP y en este caso puedo remotamente
montar mis sistemas de ficheros Linux \axp\ sobre \fn{/mnt/remote}.

Los ficheros en un sistema de ficheros son grupos de datos; el fichero
que contiene las fuentes de este cap�tulo es un fichero ASCII llamado
\fn{filesystems.tex}.  Un sistema de ficheros no s�lo posee los datos
contenidos dentro de los ficheros del sistema de ficheros, adem�s
mantiene la estructura del sistema de ficheros.  Mantiene toda la
informaci�n que los usuarios de Linux y procesos ven como ficheros,
directorios, enlaces flexibles, informaci�n de protecci�n de ficheros
y as�.  Por otro lado debe mantener esa informaci�n de forma eficiente
y segura, la integridad b�sica del sistema operativo depende de su
sistema de ficheros.  Nadie usaria un sistema operativo que perdiera
datos y ficheros de forma aleatoria\footnote{Bueno, no con
  conocimiento, sin embargo me he topado con sistemas operativos con
  m�s abogados que programadores tiene Linux}.

\texttt{Minix}, el primer sistema de ficheros que Linux tuvo es
bastante restrictivo y no era muy r�pido.  \texttt{Minix}, the first
file system that Linux had is rather restrictive and lacking in
performance.\index{Minix} Sus nombres de ficheros no pueden tener m�s
de 14 caracteres (que es mejor que nombres de ficheros 8.3) y el
tama�o m�ximo de ficheros es 64 MBytes.  64 MBytes puede a primera
vista ser suficiente pero se necesitan tama�os de ficheros m�s grandes
para soportar incluso modestas bases de datos.  El primer sistema de
ficheros dise�ado especificamente para Linux, el sistema de Ficheros
Extendido, o \texttt{EXT}, fue introducido en Abril de 1992 y solvent�
muchos problemas pero era aun falto de rapidez.  \index{EXT}
\index{Extended File system} As�, en 1993, el Segundo sistema de
Ficheros Extendido, o \texttt{EXT2}, fue a�adido.  \index{EXT2}
\index{Second Extended File system} Este es el sistema de ficheros que
se describe en detalle m�s tarde en este cap�tulo.

Un importante desarrollo tuvo lugar cuando se a�adi� en sistema de
ficheros EXT en Linux.  El sistema de ficheros real se separ� del
sistema operativo y servicios del sistema a favor de un interfaz
conocido como el sistema de Ficheros Virtual, o VFS.  \index{VFS}
\index{Virtual File system} VFS permite a Linux soportar muchos,
incluso muy diferentes, sistemas de ficheros, cada uno presentando un
interfaz software com�n al VFS.  Todos los detalles del sistema de
ficheros de Linux son traducidos mediante software de forma que todo
el sistema de ficheros parece id�ntico al resto del kernel de Linux y
a los programas que se ejecutan en el sistema.  La capa del sistema de
Ficheros Virtual de Linux permite al usuario montar de forma
transparente diferentes sistemas de ficheros al mismo tiempo.

El sistema de Ficheros Virtual est� implementado de forma que el
acceso a los ficheros es r�pida y tan eficiente como es posible.
Tambi�n debe asegurar que los ficheros y los datos que contiene son
correctos.  Estos dos requisitos pueden ser incompatibles uno con el
otro.  El VFS de Linux mantiene una antememoria con informaci�n de
cada sistema de ficheros montado y en uso.  Se debe tener mucho
cuidado al actualizar correctamente el sistema de ficheros ya que los
datos contenidos en las antememorias se modifican cuando cuando se
crean, escriben y borran ficheros y directorios.  Si se pudieran ver
las estructuras de datos del sistema de ficheros dentro del kernel en
ejecuci�n, se podria ver los bloques de datos que se leen y escriben
por el sistema de ficheros.  Las estructuras de datos, que describen
los ficheros y directorios que son accedidos serian creadas y
destruidas y todo el tiempo los controladores de los dispositivo
estarian trabajando, buascando y guardando datos.  La antememoria o
cach� m�s importantes es el Buffer Cache, que est� integrado entre
cada sistema de ficheros y su dispositivo de bloque.  Tal y como se
accede a los bloques se ponen en el Buffer Cache y se almacenan en
varias colas dependiendo de sus estados.  El Buffer Cache no s�lo
mantiene buffers de datos, tambien ayuda a administrar el interfaz
as�ncrono con los controladores de dispositivos de bloque.

\section{The Second Extended File system (EXT2)}
\index{EXT2}
\begin{figure}
\begin{center}
{\centering \includegraphics{fs/ext2.eps} \par}
\end{center}
\caption{Physical Layout of the EXT2 File system}
\label{ext2fs-figure}
\end{figure}
El Segundo sistema de ficheros Extendido fue pensado (por R�my Card)
como un sistema de ficheros extensible y poderoso para Linux.  Tambi�n
es el sistema de ficheros m�s �xito tiene en la comunidad Linux y es
b�sico para todas las distribuciones actuales de Linux.
\SeeModule{fs/�ext2/�*} El sistema de ficheros EXT2, como muchos
sistemas de ficheros, se construye con la premisa de que los datos
contenidos en los ficheros se guarden en bloques de datos.  Estos
bloques de datos son todos de la misma longitud y, si bien esa
longitud puede variar entre diferentes sistemas de ficheros EXT2 el
tama�o de los bloques de un sistema de ficheros EXT2 en particular se
decide cuando se crea (usando \eg{mke2fs}).  El tama�o de cada fichero
se redondea hasta un numero entero de bloques.  Si el tama�o de bloque
es 1024 bytes, entonces un fichero de 1025 bytes ocupar� dos bloques
de 1024 bytes.  Desafortunadamente esto significa que en promedio se
desperdicia un bloque por fichero.
Normalmente en ordenadores se cambia la carga de CPU por m�s espacio de memoria o de disco utilizado. En este caso Linux, como muchos sistemas operativos, cambia una ineficiencia relativa en el uso del disco a cambio de reducir la carga de la CPU.

No todos los bloques del sistema de ficheros contienen datos, algunos
deben usarse para mantener la informaci�n que describe la estructura
del sistema de ficheros.  EXT2 define la topologia del sistema de
ficheros describiendo cada fichero del sistema con una estructura de
datos inodo.  Un inodo describe que bloques ocupan los datos de un
fichero y tambi�n los permisos de acceso del fichero, las horas de
modificaci�n del fichero y el tipo del fichero.  Cada fichero en el
sistema de ficheros EXT2 se describe por un �nico inodo y cada inodo
tiene un �nico n�mero que lo identifica.  Los inodos del sistema de
ficheros se almacenan juntos en tablas de inodos.  Los directorios
EXT2 son simplemente ficheros especiales (ellos mismos descritos por
inodos) que contienen punteros a los inodos de sus entradas de
directorio.

La figura�\ref{ext2fs-figure} muestra la disposici�n del sistema de
ficheros EXT2 ocupando una serie de bloques en un dispositivo
estructurado bloque.  Por la parte que le toca a cada sistema de
ficheros, los dispositivos de bloque son s�lo una serie de bloques que
se pueden leer y escribir.  Un sistema de ficheros no se debe
preocupar donde se debe poner un bloque en el medio f�sico, eso es
trabajo del controlador del dispositivo.  Siempre que un sistema de
ficheros necesita leer informaci�n o datos del dispositivo de bloque
que los contiene, pide que su controlador de dispositivo lea un n�mero
entero de bloques.  El sistema de ficheros EXT2 divide las particiones
l�gicas que ocupa en Grupos de Bloque (Block Groups).  \index{EXT2
  Block Groups} Cada grupo duplica informaci�n cr�tica para la
integridad del sistema de ficheros ya sea valiendose de ficheros y
directorios como de bloques de informaci�n y datos.  Esta duplicaci�n
es necesaria por si ocurriera un desastre y el sistema de ficheros
necesitara recuperarse.  Los subapartados describen con m�s detalle
los contenidos de cada Grupo de Bloque.

\subsection{El inodo de EXT2}
\begin{figure}
\begin{center}
{\centering \includegraphics{fs/ext2_inode.eps} \par}
\end{center}
\caption{El inodo de EXT2}
\label{ext2fs-inode-figure}
\end{figure}
\index{El inodo de EXT2} \index{Estructuras de datos, el inodo EXT2} En el sistema de ficheros \texttt{EXT2}, el inodo es el bloque de construcci�n
b�sico; cada fichero y directorio del sistema de ficheros es descrito
por un y s�lo un inodo.  Los inodos EXT2 para cada Grupo de Bloque se
almacenan juntos en la table de inodos con un mapa de bits que permite
al sistema seguir la pista de inodos reservados y libres.  La
figura�\ref{ext2fs-inode-figure} muestra el formato de un inodo EXT2,
entre otra informaci�n, contiene los siguientes campos:
\SeeModule{include/\-linux/\-ext2\_fs\_i.h}
\begin{description}
\item [mode] Esto mantiene dos partes de informaci�n; qu� inodo
  describe y los permisos que tienen los usuarios.  Para EXT2, un
  inodo puede describir un ficheros, directorio, enlace simb�lico,
  dispositivo de bloque, dispositivo de caracter o FIFO.
\item [Owner Information] Los identificadores de usuario y grupo de
  los due�os de este fichero o directorio.  Esto permite al sistema de
  ficheros aplicar correctamente el tipo de acceso,
\item [Size] El tama�o en del fichero en bytes,
\item [Timestamps] La hora en la que el inodo fue creado y la �ltima
  hora en que se modific�,
\item [Datablocks] Punteros a los bloques que contienen los datos que
  este inodo describe.  Los doce primeros son punteros a los bloques
  f�sicos que contienen los datos descritos por este inodo y los tres
  �ltimos punteros contienen m�s y m�s niveles de indirecci�n.  Por
  ejemplo, el puntero de doble indirecci�n apunta a un bloque de
  punteros que apuntan a bloques de punteros que apuntan a bloques de
  datos.  Esto significa que ficheros menores o iguales a doce bloques
  de datos en longitud son m�s facilmente accedidos que ficheros m�s
  grandes.
\end{description}
Indicar que los inodos EXT2 pueden describir ficheros de dispositivo
especiales.  No son ficheros reales pero permiten que los programas
puedan usarlos para acceder a los dispositivos.  Todos los ficheros de
dispositivo de \fn{/dev} est�n ahi para permitir a los programas
acceder a los dispositivos de Linux.  Por ejemplo el programa
\eg{mount} toma como argumento el fichero de dispositivo que el
usuario desee montar.

\subsection{El superbloque EXT2}
\index{Superbloque EXT2} El Superbloque contiene una descripci�n del
tama�o y forma base del sistema de ficheros.  La informaci�n contenida
permite al administrador del sistema de ficheros usar y mantener el
sistema de ficheros.  Normalmente s�lo se lee el Superbloque del Grupo
de Bloque 0 cuando se monta el sistema de ficheros pero cada Grupo de
Bloque contiene una copia duplicada en caso de que se corrompa sistema
de ficheros.  Entre otra informaci�n contiene el:
\SeeModule{include/\-linux/\-ext2\_fs\_sb.h}
\begin{description}
\item [Magic Number] Esto permite al software de montaje comprobar que
  es realmente el Superbloque para un sistema de ficheros EXT2.  Para
  la versi�n actual de \texttt{EXT2} �ste es \hex{EF53}.
\item [Revision Level] Los niveles de revisi�n mayor y menor permiten
  al c�digo de montaje determinar si este sistema de ficheros soporta
  o no caracter�sticas que s�lo son disponibles para revisiones
  particulares del sistema de ficheros.  Tambi�n hay campos de
  compatibilidad que ayudan al c�digo de montaje determinar que nuevas
  caracter�sticas se pueden usar con seguridad en ese sistema de
  ficheros,
\item [Mount Count and Maximum Mount Count] Juntos permiten al sistema
  determinar si el sistema de ficheros fue comprobado correctamente.
  El contador de montaje se incrementa cada vez que se monta el
  sistema de ficheros y cuando es igual al contador m�ximo de montaje
  muestra el mensaje de aviso �maximal mount count reached, running
  e2fsck is recommended�,
\item [Block Group Number] El n�mero del Grupo de Bloque que tiene la
  copia de este Superbloque,
\item [Block Size] El tama�{o} de bloque para este sistema deficheros
  en bytes, por ejemplo 1024 bytes,
\item [Blocks per Group] El n�mero de bloques en un grupo.  Como el
  tama�o de bloque �ste se fija cuando se crea el sitema de ficheros,
\item [Free Blocks] EL n�mero de bloques libres en el sistema de
  ficheros,
\item [Free Inodes] El n�mero de Inodos libres en el sistema de
  ficheros,
\item [First Inode] Este es el n�mero de inodo del primer inodo en el
  sistema de ficheros.  El primer inodo en un sistema de ficheros EXT2
  ra�z seria la entrada directorio para el directorio '/'.
\end{description}

\subsection{The EXT2 Group Descriptor}
\index{EXT2 Group Descriptor} Cada Grupo de Bloque tiene una
estructura de datos que lo describe.  Como el Superbloque, todos los
descriptores de grupo para todos los Grupos de Bloque se duplican en
cada Grupo de Bloque en caso de corrupci�n del sistema de fichero.
\SeeCode{ext2\_group\_desc}{include/\-linux/\-ext2\_fs.h} Cada
Descriptor de Grupo contiene la siguiente informaci�n:
\begin{description}
\item [Blocks Bitmap] El n�mero de bloque del mapa de bits de bloques
  reservados para este Grupo de Bloque.  Se usa durante la reseva y
  liberaci�n de bloques,
\item [Inode Bitmap] El n�mero de bloque del mapa de bits de inodos
  reservados para este Grupo de Bloques.  Se usa durante la reserva y
  liberaci�n de inodos,
\item [Inode Table] El n�mero de bloque del bloque inicial para la
  tabla de inodos de este Grupo de Bloque.  Cada inodo se representa
  por la estructura de datos inodo EXT2 descrita abajo.
        \item [Free blocks count, Free Inodes count, Used directory count]
\end{description}
Los descriptores de grupo se colocan uno detr�s de otro y juntos hacen
la tabla de descriptor de grupo.  Cada Grupo de Bloques contiene la
tabla entera de descriptores de grupo despues de su copia del
Superbloque.  S�lo la primera copia (en Grupo de Bloque 0) es usada
por el sistema de ficheros EXT2.  Las otras copias est�n ahi, como las
copias del Superbloque, en caso de que se corrompa la principal.

\subsection{EXT2 Directories}
\index{EXT2 Directories}
\index{Data structures, EXT2 Directory}
\begin{figure}
\begin{center}
{\centering \includegraphics{fs/ext2_dir.eps} \par}
\end{center}
\caption{EXT2 Directory}
\label{ext2fs-dir-figure}
\end{figure}
En el sistema de ficheros EXT2, los directorios son ficheros
especiales que se usan para crear y mantener rutas de acceso a los
ficheros en el sistema de ficheros.  La figura�\ref{ext2fs-dir-figure}
muestra la estructura de una estrada directorio en memoria.
\SeeCode{ext2\_dir\_entry}{include/\-linux/\-ext2\_fs.h} Un fichero
directorio es una lista de entradas directorio, cada una conteniendo
la siguiente informaci�n:
\begin{description}
\item [inode] El inodo para esta entrada directorio.  Es un �ndice al
  vector de inodos guardada en la Tabla de Inodos del Grupo de Bloque.
  En la figura�\ref{ext2fs-dir-figure}, la entrada directorio para el
  fichero llamado \fn{file} tiene una referencia al n�mero de inodo
  \texttt{i1},
\item [name length] La longitud de esta entrada directorio en bytes,
        \item [name] El nombre de esta entrada directorio.
\end{description}
Las dos primeras entradas para cada directorio son siempre las
entradas estandar �.� y �..� significando �este directorio� y �el
directorio padre� respectivamente.

\subsection{Finding a File in an EXT2 File System}
Un nombre de fichero Linux tiene el mismo formato que los nombres de
ficheros de todos los Unix\tm\.  Es una serie de nombres de
directorios separados por contra barras (�\fn{/}�) y acabando con el
nombre del fichero.  Un ejemplo de nombre de fichero podria ser
\fn{/home/rusling/.cshrc} donde \fn{/home} y \fn{/rusling} son nombres
de directorio y el nombre del fichero es \fn{.cshrc}.  Como todos los
demas sistemas Unix\tm� Linux no tiene encuenta el formato del nombre
del fichero; puede ser de cualquier longitud y cualquier caracter
imprimible.  Para encontrar el inodo que representa a este fichero
dentro de un sistema de ficheros \texttt{EXT2} el sistema debe
analizar el nombre del fichero directorio a directorio hasta encontrar
el fichero en si.  El primer inodo que se necesita es el inodo de la
ra�z del sistema de ficheros, que est� en el superbloque del sistema
de ficheros.  Para leer un inodo EXT2 hay que buscarlo en la tabla de
inodos del Grupo de Bloque apropiado.  Si, por ejemplo, el n�mero de
inodo de la ra�z es 42, entonces necesita el inodo 42avo de la tabla
de inodos del Grupo de Bloque 0.  El inodo ra�z es para un directorio
EXT2, en otras palabras el modo del inodo lo describe como un
directorio y sus bloques de datos contienen entradas directorio EXT2.

\fn{home} es una de las muchas entradas directorio y esta entrada
directorio indica el n�mero del inodo que describe al directorio
\fn{/home}.  Hay que leer este directorio (primero leyendo su inodo y
luego las entradas directorio de los bloques de datos descritos por su
inodo) para encontrar la entrada \fn{rusling} que indica el numero del
inodo que describe al directorio \fn{/home/rusling}.  Finalmente se
debe leer las entradas directorio apuntadas por el inodo que describe
al directorio \fn{/home/rusling} para encontrar el n�mero de inodo del
fichero \fn{.cshrc} y desde ahi leer los bloques de datos que
contienen la informaci�n del fichero.

\subsection{Changing the Size of a File in an EXT2 File System}
Un problema com�n de un sistema de ficheros es la tendencia a
fragmentarse.  Los bloques que contienen los datos del fichero se
esparcen por todo el sistema de ficheros y esto hace que los accesos
secuenciales a los bloques de datos de un fichero sean cada vez m�s
ineficientes cuanto m�s alejados est�n los bloques de datos.  El
sistema de ficheros EXT2 intenta solucionar esto reservando los nuevos
bloques para un fichero, fisicamente juntos a sus bloques de datos
actuales o al menos en el mismo Grupo de Bloque que sus bloques de
datos.  S�lo cuando esto falla, reserva bloques de datos en otros
Grupos de Bloque.

Siempre que un proceso intenta escribir datos a un fichero, el sistema
de ficheros Linux comprueba si los datos exceden el final del �ltimo
bloque para el fichero.  Si lo hace, entonces tiene que reservar un
nuevo bloque de datos para el fichero.  Hasta que la reserva no haya
acabado, el proceso no puede ejecutarse; debe esperarse a que el
sistema de ficheros reserve el nuevo bloque de datos y escriba el
resto de los datos antes de continuar.  La primera cosa que hacen las
rutinas de reserva de bloques EXT2 es bloquear el Superbloque EXT2 de
ese sistema de ficheros.  La reserva y liberaci�n cambia campos del
superbloque, y el sistema de ficheros Linux no puede permitir m�s de
un proceso haciendo �sto a la vez.  Si otro proceso necesita reservar
m�s bloques de datos, debe esperarse hasta que el otro proceso acabe.
Los procesos que esperan el superbloque son suspendidos, no se pueden
ejecutar, hasta que el control del superbloque lo abandone su usuario
actual.  El acceso al superbloque se garantiza mediante una pol�tica
�el primero que llega se atiende primero�, y cuando un proceso tiene
control sobre el superbloque le pone cerrojo hasta que no lo necesita
m�s. \footnote{REVISAR!!!} bloqueado el superbloque, el proceso
comprueba que hay suficientes bloques libres en ese sistema de
ficheros.  Si no es as�, el intento de reservar m�s bloques falla y el
proceso ceder� el control del superbloque del sistema de ficheros.

Si hay suficientes bloques en el sistema de ficheros, el proceso
intenta reservar uno.
\SeeCode{ext2\_new\_block()}{fs/\-ext2/\-balloc.c} Si el sistema de
ficheros EXT2 se ha compilado para prereservar bloques de datos
entonces se podr� usar uno de estos.  La prereserva de bloques no
existe realmente, s�lo se reservan dentro del mapa de bits de bloques
reservados.  El inodo VFS que representa el fichero que intenta
reservar un nuevo bloque de datos tiene dos campos EXT2 espec�ficos,
\field{prealloc\_block} y \field{prealloc\_count}, que son el numero
de bloque del primer bloque de datos prereservado y cuantos hay,
respectivamente.  Si no habian bloques prereservados o la reserva
anticipada no est� activa, el sistema de ficheros EXT2 debe reservar
un nuevo bloque.  El sistema de ficheros EXT2 primero mira si el
bloque de datos despues del �ltimo bloque de datos del fichero est�
libre.  Logicamente, este es el bloque m�s eficiente para reservar ya
que hace el acceso secuencial mucho m�s r�pido.  Si este bloque no
est� libre, la b�squeda se ensancha y busca un bloque de datos dentro
de los 64 bloques del bloque ideal.  Este bloque, aunque no sea ideal
est� al menos muy cerca y dentro del mismo Grupo de Bloque que los
otros bloques de datos que pertenecen a ese fichero.

Si incluso ese bloque no est� libre, el proceso empieza a buscar en
los dem�s Grupos de Bloque hasta encontrar algunos bloques libres.  El
c�digo de reserva de bloque busca un cluster de ocho bloques de datos
libres en cualquiera de los Grupos de Bloque.  Si no puede encontrar
ocho juntos, se ajustar� para menos.  Si se quiere la prereserva de
bloques y est� activado, actualizar� \field{prealloc\_block} y
\field{prealloc\_count} pertinentemente.

Donde quiera que encuentre el bloque libre, el c�digo de reserva de
bloque actualiza el mapa de bits de bloque del Grupo de Bloque y
reserva un buffer de datos en el buffer cach�.  Ese buffer de datos se
identifica unequivocamente por el identificador de dispositivo del
sistema y el n�mero de bloque del bloque reservado.  El buffer de
datos se sobreescribe con ceros y se marca como �sucio� para indicar
que su contenido no se ha escrito al disco f�sico.  Finalmente, el
superbloque se marca como �sucio� para indicar que se ha cambiado y
est� desbloqueado.  Si hubiera otros procesos esperando, al primero de
la cola se le permitiria continuar la ejecuci�n y terner el control
exclusido del superbloque para sus operaciones de fichero.  Los datos
del proceso se escriben en el nuevo bloque de datos y, si ese bloque
se llena, se repite el proceso entero y se reserva otro bloque de
datos.

\section{The Virtual File System (VFS)}
\index{Virtual File System (VFS)}
\index{VFS}
\begin{figure}
\begin{center}
{\centering \includegraphics{fs/vfs.eps} \par}
\end{center}
\caption{A Logical Diagram of the Virtual File System}
\label{vfs-figure}
\end{figure}
La figura�\ref{vfs-figure} muestra la relaci�n entre el Sistema de
Ficheros Virtual del kernel de Linux y su sistema de ficheros real.
El sistema de ficheros vitual debe mantener todos los diferentes
sistemas de ficheros que hay montados en cualquier momento.  Para
hacer esto mantiene unas estructuras de datos que describen el sistema
de ficheros (virtual) por entero y el sistema de ficheros, montado,
real.  \SeeModule{fs/*} De forma m�s confusa, el VFS describe los
ficheros del sistema en t�rminos de superbloque e inodos de la misma
forma que los ficheros EXT2 usan superbloques e inodos.  Como los
inodos EXT2, los inodos VFS describen ficheros y directorios dentro
del sistema; los contenidos y topolog�a del Sistema de Ficheros
Virtual.  De ahora en adelante, para evitar confusiones, se escribir�
inodos CFS y superbloques VFS para distinguirlos de los inodos y
superbloques EXT2.

Cuando un sistema de ficheros se inicializa, se registra �l mismo con
el VFS.  Esto ocurre cuando el sistema operativo se inicializa en el
momento de arranque del sistema.  Los sistemas de ficheros reales
est�n compilados con el nucleo o como m�dulos cargables.  Los m�dulos
de Sistemas de Ficheros se cargan cuando el sistema los necesita, as�,
por ejemplo, si el sistema de ficheros \texttt{VFAT} est� implementado
como m�dulo del kernel, entonces s�lo se carga cuando se monta un
sistema de ficheros \texttt{VFAT}.  Cuando un dispositivo de bloque
base se monta, y �ste incluye el sistema de ficheros ra�z, el VFS debe
leer su superbloque.  Cada rutina de lectura de superbloque de cada
tipo de sistema de ficheros debe resolver la topolog�a del sistema de
ficheros y mapear esa informaci�n dentro de la estructura de datos del
superbloque VFS.  El VFA mantiene una lista de los sitema de ficheros
montados del sistema junto con sus superbloques VFS.  Cada superbloque
VFS contiene informaci�n y punteros a rutinas que realizan funciones
particulares.  De esta forma, por ejemplo, el superbloque que
representa un sistema de ficheros EXT2 montado contiene un puntero a
la rutina de lectura de inodos espec�fica.  Esta rutina, como todas
las rutinas de lectura de inodos del sistema de ficheros espe�fico,
rellena los campos de un inodo VFS.  Cada superbloque VFS contiene un
puntero al primer inodo VFS del sistema de ficheros.  Para el sistema
de ficheros ra�z, �ste es el inodo que representa el directorio
\fn{�/�}.  Este mapeo de informaci�n es muy eficiente para el sistema
de ficheros EXT2 pero moderadamente menos para otros sistema de
ficheros.

Ya que los procesos del sistema acceden a directorios y ficheros, las
rutinas del sistema se dice que recorren los inodos VFS del sistema.
\SeeModule{fs/�inode.c} Por ejemplo, escribir \eg{ls} en un directorio
o \eg{cat} para un fichero hacen que el Sistema de Ficheros Virtual
busque atrav�s de los inodos VFS que representan el sistema de
ficheros.  Como cada fichero y directorio del sistema se representa
por un inodo VFS, un n�mero de inodos ser�n accedidos repetidamente.
Estos inodos se mantienen en la antememoria, o cach�, de inodos que
hace el acceso mucho m�s r�pido.  Si un inodo no est� en la cach�,
entonces se llama a una rutina espec�fica del sistema de ficheros para
leer el inodo apropiado.  La acci�n de leer el inodo hace que se ponga
en la cach� de inodos y siguientes accesos hacen que se mantenga en la
cach�.  Los inodos VFS menos usados se quitan de la cach�.

Todos los sistemas de ficheros de Linux usan un buffer cach� com�n
para mantener datos de los dispositivos para ayudar a acelerar el
acceso por todos los sistemas de ficheros al dispositivo f�sico que
contiene los sistemas de ficheros.  \SeeModule{fs/�buffer.c} Este
buffer cach� es independiente del sistema de ficheros y se integra
dentro de los mecanismos que el n�cleo de Linux usa para reservar,
leer y escribir datos.  Esto tiene la ventaja de hacer los sistemas de
ficheros de Linux independientes del medio y de los controladores de
dispositivos que los soportan.  Tofos los dispositivos estructurados
de bloque se registran ellos mismos con el n�cleo de Linux y presentan
una interfaz uniforme, basada en bloque y normalmente as�ncrona.
Incluso dispositivos de bloque relativamente complejos como SCSI lo
hacen.

Cuando el sistema de ficheros real lee datos del disco f�sico realiza
una petici�n al controlador de dispositivo de bloque para leer los
bloques f�sicos del dispositivo que controla.  Integrado en este
interfaz de dispositivo de bloque est� el buffer cach�.  Al leer
bloques del sistema de ficheros se guardan en un el buffer cach�
global compartido por todos los sistemas de ficheros y el n�cleo de
Linux.  Los buffers que hay dentro se identifican por su n�mero de
bloque y un identificador �nico para el dispositivo que lo ha leido.
De este modo, si se necesitan muy a menudo los mismos datos, se
obtendr�n del buffer cach� en lugar de leerlos del disco, que tarda
m�s tiempo.  Algunos dispositivos pueden realizar lecturas anticipadas
(\textit{read ahead}), mediante lo cual se realizan lecturas antes de
necesitarlas, especulando con que se utilizar�n m�s
adelante.\footnote{REVISAR!!!!!} 

El VFS tambi�n mantiene una cach� de directorios donde se pueden
encontrar los inodos de los directorios que se usan de forma m�s
frecuente.  \SeeModule{fs/�dcache.c} Como experimento, probar a listar
un directorio al que no se haya accedido recientemente.  La primera
vez que se lista, se puede notar un peque�o retardo pero la segunda
vez el resultado es inmediato.  El cach� directorio no almacena
realmente los inodos de los directorios; �stos estar�n en el cach� de
inodos, el directorio cach� simplemente almacena el mapa entre el
nombre entero del directorio y sus n�meros de inodo.

\subsection{The VFS Superblock}
\index{Superblock, VFS} \index{VFS superblock} Cada sistema de
ficheros montado est� representado por un superbloque VFS; entre otra
informaci�n, el superbloque VFS contiene:
\SeeModule{include/�linux/�fs.h}
\begin{description}
\item [Device] Es el identificador de dispositivo para el dispositivo
  bloque que contiene este a este sistema de ficheros.  Por ejemplo,
  \fn{/dev/hda1}, el primer disco duro IDE del sistema tiene el
  identificador de dispositivo \hex{301},
\item [Inode pointers] El puntero de inodo \field{montado} apunta al
  primer inodo del sistema de ficheros.  El puntero de inodo
  \field{cubierto} apunta al inodo que representa el directorio donde
  est� montado el sistema de ficheros.  El superbloque VFS del sistema
  de ficheros ra�z no tiene puntero \field{cubierto},
\item [Blocksize] EL tama�o de bloque en bytes del sistema de
  ficheros, por ejemplo 1024 bytes,
\item [Superblock operations] Un puntero a un conjunto de rutinas de
  superbloque para ese sistema de ficheros.  Entre otras cosas, estas
  rutinas las usa el VFS para leer y escribir inodos y superbloques.
\item [File System type] Un puntero a la estructura de datos
  \ds{tipo\_sistema\_ficheros} del sistema de ficheros montado,
\item [File System specific] Un puntero a la informaci�n que necesaria
  este sistema de ficheros.
\end{description}

\subsection{The VFS Inode}
\index{inode, VFS} \index{VFS inode} Como el sistema de ficheros EXT2,
cada fichero, directorio y dem�s en el VFS se representa por uno y
solo un inodos VFS.  \SeeModule{include/�linux/�fs.h} La infomaci�n en
cada inodo VFS se construye a partir de informaci�n del sistema de
ficheros por las rutinas espec�ficas del sistema de ficheros.  Los
inodos VFS existen s�lo en la memoria del n�cleo y se mantienen en el
cach� de inodos VFS tanto tiempo como sean �tiles para el sistema.
Entre otra informaci�n, los inodos VFS contienen los siguientes
campos:
\begin{description}
\item [device] Este es el identificador de dispositivo del dispositivo
  que contiene el fichero o lo que este inodo VFS represente,
\item [inode number] Este es el n�mero del inodo y es �nico en este
  sistema de ficheros.  La combinaci�n de \field{device} y
  \field{inode number} es �nica dentro del Sistema de Ficheros
  Virtual,
\item [mode] Como en EXT2 este campo describe que representa este
  inodo VFS y sus permisos de acceso,
\item [user ids] Los identificadores de propietario,
\item [times] Los tiempos de creaci�n, modificaci�n y escritura,
\item [block size] El tama�o de bloque en bytes para este fichero, por
  ejemplo 1024 bytes,
\item [inode operations] Un puntero a un bloque de direcciones de
  rutina.  Estas rutinas son espef�ficas del sistema de ficheros y
  realizan operaciones para este inodo, por ejemplo, truncar el
  fichero que representa este inodo.
\item [count] El n�mero de componentes del sistema que est�n usando
  actualmente este inodo VFS.  Un contador de cero indica que el inodo
  est� libre para ser descartado o reusado,
\item [lock] Este campo se usa para bloquear el inodo VFS, por
  ejemplo, cuando se lee del sistema de ficheros,
\item [dirty] Indica si se ha escrito en este inodo, si es as�{i} el
  sistema de ficheros necesitar� modificarlo,
\item [file system specific information]
\end{description}

\subsection{Registering the File Systems}
\begin{figure}
\begin{center}
{\centering \includegraphics{fs/file-systems.eps} \par}
\end{center}
\caption{Registered File Systems}
\label{file-systems-figure}
\end{figure}
\index{File System, registering} \index{Registering a file system}
Cuando se compila el n�cleo de Linux se pregunta si se quiere cada uno
de los sistemas de ficheros soportados.  Cuando el n�cleo est�
compilado, el c�difo de arranque del sistema de ficheros contiene
llamadas a las rutinas de inicializaci�n de todos los sistemas de
ficheros compilados.  \SeeCode{sys\_setup()}{fs/\-filesystems.c} Los
sistemas de ficheros de Linux tambi�n se pueden compilar como m�dulos
y, en este caso, pueden ser cargados cuando se les necesita o
cargarlos a mano usando \eg{insmod}.  Siempre que un m�dulo de sistema
de ficheros se carga se registra �l mismo con el n�cleo y se borra �l
mismo cuando se descarga.  Cada rutina de inicializaci�n del sistema
de ficheros se registra con el Sistema de Ficheros Virtual y se
representa por una estructura de datos \ds{tipo\_sistema\_ficheros}
que contiene el nombre del sistema de ficheros y un puntero a su
rutina de lectura de superbloque VFS.  La
figura�\ref{file-systems-figure} muestra que las estructuras de datos
\ds{tipo\_sistems\_ficheros} se ponen en una lista apuntada por el
puntero \ds{sistemas\_ficheros}.  Cada estructura de datos
\ds{tipo\_sistema\_ficheros} contiene la siguiente informaci�n:
\SeeCode{file\_system\_type}{include/\-linux/\-fs.h}
\begin{description}
\item [Superblock read routine] Esta rutina se llama por el VFS cuando
  se monta una instancia del sistema de ficheros,
\item [File System name] El nombre de este sistema de ficheros, por
  ejemplo \texttt{ext2},
\item [Device needed] Necesita soportar este sistema de ficheros un
  dispositivo?  No todos los sistemas de ficheros necesitan un
  dispositivo.  El sistema de fichero \texttt{/proc}, por ejemplo, no
  requiere un dispositivo de bloque,
\end{description}
Se puede ver que sistemas de ficheros hay rgistrados mirando en
\fn{/proc/filesystems}.  Por ejemplo:
\begin{verbatim}
      ext2
nodev proc
      iso9660
\end{verbatim}

\subsection{Mounting a File System}
\index{Mounting a File System} \index{File System, mounting} Cuando el
superusuario intenta montar un sistema de ficheros, el n�cleo de Linux
debe primero validar los argumentos pasados en la llamada al sistema.
Aunque \eg{mount} hace una comprobaci�n b�sica, no conoce que sistemas
de ficheros soporta el kernel o si existe el punto de montaje
propuesto.  Considerar el siguiente comando \eg{mount}:
\begin{verbatim}
$ mount -t iso9660 -o ro /dev/cdrom /mnt/cdrom
\end{verbatim}% $ para que no moleste el modo AUC-TeX de Emacs
Este comando \eg{mount} pasa al n�cleo tres trozos de informaci�n; el
nombre del sistema de ficheros, el dispositivo de bloque f�sico que
contiene al sistema de fichros y, por �ltimo, donde, en la topolog�a
del sistema de ficheros existente, se montar� el nuevo sistema de
ficheros.

La primera cosa que debe hacer el Sistema de Ficheros Virtual es
encontrar el sistema de ficheros.  \SeeCode{do\_mount()}{fs/\-super.c}
Para hacer �{es}to busca a trav�s de la lista de sistemas de ficheros
conocidos y mira cada estructura de datos ds{tipo\_sistema\_ficheros}
en la lista apuntada por \ds{sistema\_ficheros}.
\SeeCode{get\_fs\_type()}{fs/\-super.c} Si encuentra una coincidencia
del nombre ahora sabe que ese tipo de sistema de ficheros es soportado
por el n�cleo y tiene la direcci�n de la rutina espec�fica del sistema
de ficheros para leer el superbloque de ese sistema de ficheros.  Si
no puede encontrar ninguna coindidencia no todo est� perdido si el
n�cleo puede cargar m�dulos por demanda (ver
Cap�tulo�\ref{modules-chapter}).  En este caso el n�cleo piede al
demonio del n�cleo que cargue el m�dulo del sistema de ficheros
apropiado antes de continuar como anteriormente.

Si el dispositivo f�sico pasado por \eg{mount} no est� ya montado,
debe encontrar el inodo VFS del directorio que ser� el punto de
montaje del nuevo sistema de ficheros.  Este inodo VFS debe estar en
el cach� de inodos o se debe leer del dispositivo de bloque que
soporta el sistema de ficheros del punto de montaje.  Una vez que el
inodo se ha encontrado se comprueba para ver que sea un directorio y
que no contenga ya otro sistema de ficheros montado.  El mismo
directorio no se puede usar como punto de montaje para m�s de un
sistema de ficheros.

En este punto el c�difo de montaje VFS reserva un superbloque VFS y le
pasa la informaci�n de montahe a la rutina de lectura de superblque
para este sistema de ficheros.  Todos los superbloques VFS del sistema
se mantienen en el vector \ds{super\_bloques} de las estructuras de
datos \ds{super\_bloque} y se debe reservar una para este montaje.  La
rutina de lectura de superbloque debe rellenar los campos b�sicos del
superbloque VFS con informaci�n que lee del dispositivo f�sico.  Para
el sistema de ficheros EXT2 este mapeo o traducci�n de informaci�n es
bastante facil, simplemente lee el superbloque EXT2 y rellena el
superbloque VFS de ah�.  Para otros sistemas de ficheros, como el MS
DOS, no es una tarea t�n facil.  Cualquiera que sea el sistema de
ficheros, rellenar el superbloque VFS significa que el sistema de
ficheros debe leer todo lo que lo describe del dispositivo de bloque
que lo soporta.  Si el dispositivo de bloque no se puede leer o si no
contiene este tipo de sistema de ficheros entonces el comando
\eg{mount} fallar�.

\begin{figure}
\begin{center}
{\centering \includegraphics{fs/mounted.eps} \par}
\end{center}
\caption{A Mounted File System}
\label{mounted-figure}
\end{figure}
Cada sistema de ficheros montado es descrito por una estructura de
datos \ds{vfsmount}; ver figura�\ref{mounted-figure}.  Estos son
puestos en una cola de una lista apuntada por \ds{vfsmntlist}.
\SeeCode{add\_vfsmnt()}{fs/�super.c} Otro puntero, \ds{vfsmnttail}
apunta a la �ltima entrada de la lista y el puntero
\dsni{mru\_vfsmnt}\index{mru\_vfsmnt pointer} apunta al sistemas de
ficheros m�s recientemente usado.  Cada estructura \ds{vfsmount}
contiene el n�mero de dispositivo del dispositivo de bloque que
contiene al sistema de ficheros, el directorio donde el sistema de
ficheros est� montado y un puntero al superbloque VFS reservado cuando
se mont�.  En cambio el superbloque VFS apunta a la estructura de
datos \ds{tipo\_sistema\_ficheros} para este tipo de sisetma de
ficheros y al inodo ra�{i}z del sistema de ficheros.  Este inodo se
mantiene residente en el cach� de inodos VFS todo el tiempo que el
sistema de ficheros est� cargado.

\subsection{Finding a File in the Virtual File System}
\index{Finding a File} \index{Files, finding} Para encontrar el inodo
VFS de un fichero en el Sistema de Ficheros Virtual, VFS debe resolver
el nombre del directorio, mirando el inodo VFS que representa cada uno
de los directorios intermedios del nombre.  Mirar cada directorio
envuelve una llamada al sistema de ficheros espec�fico cuya direcci�n
se mantiene en el inodo VFS que representa al directorio padre.  Esto
funciona porque siempre tenemos el inodo VFS del ra�z de cada sistema
de ficheros disponible y apuntado por el superbloque VFS de ese
sistema.  Cada vez que el sistema de ficheros real mira un inodo
comprueba el cach� de directorios.  Si no est� la entrada en el cach�
de directorios, el sistema de ficheros real toma el inodo VFS tanto
del sistema de ficheros como del cach� de inodos.

\subsection{Creating a File in the Virtual File System}
\index{Creating a file}
\index{Files, creating}

\subsection{Unmounting a File System}
\index{Unmounting a File System} \index{File System, unmounting} The
workshop manual for my MG usually describes assembly as the reverse of
disassembly and the reverse is more or less true for unmounting a file
system.  \SeeCode{do\_umount()}{fs/�super.c} A file system cannot be
unmounted if something in the system is using one of its files.  So,
for example, you cannot umount \fn{/mnt/cdrom} if a process is using
that directory or any of its children.  If anything is using the file
system to be unmounted there may be VFS inodes from it in the VFS
inode cache, and the code checks for this by looking through the list
of inodes looking for inodes owned by the device that this file system
occupies.  If the VFS superblock for the mounted file system is dirty,
that is it has been modified, then it must be written back to the file
system on disk.  Once it has been written to disk, the memory occupied
by the VFS superblock is returned to the kernel's free pool of memory.
Finally the \ds{vfsmount} data structure for this mount is unlinked
from \ds{vfsmntlist} and freed.
\SeeCode{remove\_vfsmnt()}{fs/�super.c}

\subsection{The VFS Inode Cache}
\index{Inode cache} \index{Caches, VFS inode} As the mounted file
systems are navigated, their VFS inodes are being continually read
and, in some cases, written.  The Virtual File System maintains an
inode cache to speed up accesses to all of the mounted file systems.
Every time a VFS inode is read from the inode cache the system saves
an access to a physical device.  \SeeModule{fs/�inode.c}

The VFS inode cache is implmented as a hash table whose entries are
pointers to lists of VFS inodes that have the same hash value.  The
hash value of an inode is calculated from its inode number and from
the device identifier for the underlying physical device containing
the file system.  Whenever the Virtual File System needs to access an
inode, it first looks in the VFS inode cache.  To find an inode in the
cache, the system first calculates its hash value and then uses it as
an index into the inode hash table.  This gives it a pointer to a list
of inodes with the same hash value.  It then reads each inode in turn
until it finds one with both the same inode number and the same device
identifier as the one that it is searching for.

If it can find the inode in the cache, its count is incremented to
show that it has another user and the file system access continues.
Otherwise a free VFS inode must be found so that the file system can
read the inode from memory.  VFS has a number of choices about how to
get a free inode.  If the system may allocate more VFS inodes then
this is what it does; it allocates kernel pages and breaks them up
into new, free, inodes and puts them into the inode list.  All of the
system's VFS inodes are in a list pointed at by \ds{first\_inode} as
well as in the inode hash table.  If the system already has all of the
inodes that it is allowed to have, it must find an inode that is a
good candidate to be reused.  Good candidates are inodes with a usage
count of zero; this indicates that the system is not currently using
them.  Really important VFS inodes, for example the root inodes of
file systems always have a usage count greater than zero and so are
never candidates for reuse.  Once a candidate for reuse has been
located it is cleaned up.  The VFS inode might be dirty and in this
case it needs to be written back to the file system or it might be
locked and in this case the system must wait for it to be unlocked
before continuing.  The candidate VFS inode must be cleaned up before
it can be reused.

However the new VFS inode is found, a file system specific routine
must be called to fill it out from information read from the
underlying real file system.  Whilst it is being filled out, the new
VFS inode has a usage count of one and is locked so that nothing else
accesses it until it contains valid information.

To get the VFS inode that is actually needed, the file system may need
to access several other inodes.  This happens when you read a
directory; only the inode for the final directory is needed but the
inodes for the intermediate directories must also be read.  As the VFS
inode cache is used and filled up, the less used inodes will be
discarded and the more used inodes will remain in the cache.

\subsection{The Directory Cache}
\index{Directory cache} \index{Caches, directory} To speed up accesses
to commonly used directories, the VFS maintains a cache of directory
entries.  \SeeModule{fs/�dcache.c} As directories are looked up by the
real file systems their details are added into the directory cache.
The next time the same directory is looked up, for example to list it
or open a file within it, then it will be found in the directory
cache.  Only short directory entries (up to 15 characters long) are
cached but this is reasonable as the shorter directory names are the
most commonly used ones.  For example, \fn{/usr/X11R6/bin} is very
commonly accessed when the X server is running.

The directory cache consists of a hash table, each entry of which
points at a list of directory cache entries that have the same hash
value.  The hash function uses the device number of the device holding
the file system and the directory's name to calculate the offset, or
index, into the hash table.  It allows cached directory entries to be
quickly found.  It is no use having a cache when lookups within the
cache take too long to find entries, or even not to find them.

In an effort to keep the caches valid and up to date the VFS keeps
lists of Least Recently Used (LRU) directory cache entries.  When a
directory entry is first put into the cache, which is when it is first
looked up, it is added onto the end of the first level LRU list.  In a
full cache this will displace an existing entry from the front of the
LRU list.  As the directory entry is accessed again it is promoted to
the back of the second LRU cache list.  Again, this may displace a
cached level two directory entry at the front of the level two LRU
cache list.  This displacing of entries at the front of the level one
and level two LRU lists is fine.  The only reason that entries are at
the front of the lists is that they have not been recently accessed.
If they had, they would be nearer the back of the lists.  The entries
in the second level LRU cache list are safer than entries in the level
one LRU cache list.  This is the intention as these entries have not
only been looked up but also they have been repeatedly referenced.

\ReviewNotes{Do we need a diagram for this?}

\section{The Buffer Cache}
\index{Buffer caches}
\index{Caches, buffer}
\begin{figure}
\begin{center}
{\centering \includegraphics{fs/buffer-cache.eps} \par}
\end{center}
\caption{The Buffer Cache}
\label{buffer-cache-figure}
\end{figure}
As the mounted file systems are used they generate a lot of requests
to the block devices to read and write data blocks.  All block data
read and write requests are given to the device drivers in the form of
\ds{buffer\_head} data structures via standard kernel routine calls.
These give all of the information that the block device drivers need;
the device identifier uniquely identifies the device and the block
number tells the driver which block to read.  All block devices are
viewed as linear collections of blocks of the same size.  To speed up
access to the physical block devices, Linux maintains a cache of block
buffers.  All of the block buffers in the system are kept somewhere in
this buffer cache, even the new, unused buffers.  This cache is shared
between all of the physical block devices; at any one time there are
many block buffers in the cache, belonging to any one of the system's
block devices and often in many different states.  If valid data is
available from the buffer cache this saves the system an access to a
physical device.  Any block buffer that has been used to read data
from a block device or to write data to it goes into the buffer cache.
Over time it may be removed from the cache to make way for a more
deserving buffer or it may remain in the cache as it is frequently
accessed.

Block buffers within the cache are uniquely identfied by the owning
device identifier and the block number of the buffer.  The buffer
cache is composed of two functional parts.  The first part is the
lists of free block buffers.  There is one list per supported buffer
size and the system's free block buffers are queued onto these lists
when they are first created or when they have been discarded.  The
currently supported buffer sizes are 512, 1024, 2048, 4096 and 8192
bytes.  The second functional part is the cache itself.  This is a
hash table which is a vector of pointers to chains of buffers that
have the same hash index.  The hash index is generated from the owning
device identifier and the block number of the data block.
Figure�\ref{buffer-cache-figure} shows the hash table together with a
few entries.  Block buffers are either in one of the free lists or
they are in the buffer cache.  When they are in the buffer cache they
are also queued onto Least Recently Used (LRU) lists.  There is an LRU
list for each buffer type and these are used by the system to perform
work on buffers of a type, for example, writing buffers with new data
in them out to disk.  The buffer's type reflects its state and Linux
currently supports the following types:
\begin{description}
\item [clean] Unused, new buffers,
\item [locked] Buffers that are locked, waiting to be written,
\item [dirty] Dirty buffers.  These contain new, valid data, and will
  be written but so far have not been scheduled to write,
\item [shared] Shared buffers,
\item [unshared] Buffers that were once shared but which are now not
  shared,
\end{description}

Whenever a file system needs to read a buffer from its underlying
physical device, it trys to get a block from the buffer cache.  If it
cannot get a buffer from the buffer cache, then it will get a clean
one from the appropriate sized free list and this new buffer will go
into the buffer cache.  If the buffer that it needed is in the buffer
cache, then it may or may not be up to date.  If it is not up to date
or if it is a new block buffer, the file system must request that the
device driver read the appropriate block of data from the disk.

Like all caches, the buffer cache must be maintained so that it runs
efficiently and fairly allocates cache entries between the block
devices using the buffer cache.  Linux uses the \texttt{bdflush}
\index{bdflush, kernel daemon} \index{Kernel daemons, bdflush} kernel
daemon to perform a lot of housekeeping duties on the cache but some
happen automatically as a result of the cache being used.

\subsection{The \texttt{bdflush} Kernel Daemon}
\index{bdflush, kernel daemon} \index{Kernel daemons, bdflush}
\SeeCode{bdflush()}{fs/�buffer.c} The \texttt{bdflush} kernel daemon
is a simple kernel daemon that provides a dynamic response to the
system having too many dirty buffers; buffers that contain data that
must be written out to disk at some time.  It is started as a kernel
thread at system startup time and, rather confusingly, it calls itself
�kflushd� and that is the name that you will see if you use the
\eg{ps} command to show the processes in the system.  Mostly this
daemon sleeps waiting for the number of dirty buffers in the system to
grow too large.  As buffers are allocated and discarded the number of
dirty buffers in the system is checked.  If there are too many as a
percentage of the total number of buffers in the system then
\texttt{bdflush} is woken up.
The default threshold is 60% but, if the system is desperate for buffers, \texttt{bdflush}
will be woken up anyway.
This value can be seen and changed using the \eg{update} command:
\begin{verbatim}

# update -d

bdflush version 1.4
0:    60 Max fraction of LRU list to examine for dirty blocks
1:   500 Max number of dirty blocks to write each time bdflush activated
2:    64 Num of clean buffers to be loaded onto free list by refill_freelist
3:   256 Dirty block threshold for activating bdflush in refill_freelist
4:    15 Percentage of cache to scan for free clusters
5:  3000 Time for data buffers to age before flushing
6:   500 Time for non-data (dir, bitmap, etc) buffers to age before flushing
7:  1884 Time buffer cache load average constant
8:     2 LAV ratio (used to determine threshold for buffer fratricide).

\end{verbatim}
All of the dirty buffers are linked into the \texttt{BUF\_DIRTY} LRU
list whenever they are made dirty by having data written to them and
\texttt{bdflush} tries to write a reasonable number of them out to
their owning disks.  Again this number can be seen and controlled by
the \eg{update} command and the default is 500 (see above).

\subsection{The \eg{update} Process}
\index{update process} The \eg{update} command is more than just a
command; it is also a daemon.  When run as superuser (during system
initialisation) it will periodically flush all of the older dirty
buffers out to disk.  It does this by calling a system service routine
\SeeCode{sys\_bdflush()}{fs/�buffer.c} that does more or less the same
thing as \texttt{bdflush}.  Whenever a dirty buffer is finished with,
it is tagged with the system time that it should be written out to its
owning disk.  Every time that \eg{update} runs it looks at all of the
dirty buffers in the system looking for ones with an expired flush
time.  Every expired buffer is written out to disk.

\section{The /proc File System}
\index{/proc file system} The \texttt{/proc} file system really shows
the power of the Linux Virtual File System.  It does not really exist
(yet another of Linux's conjuring tricks), neither the \texttt{/proc}
directory nor its subdirectories and its files actually exist.  So how
can you \eg{cat} \fn{/proc/devices}?  The {/proc} file system, like a
real file system, registers itself with the Virtual File System.
However, when the VFS makes calls to it requesting inodes as its files
and directories are opened, the \texttt{/proc} file system creates
those files and directories from information within the kernel.  For
example, the kernel's \fn{/proc/devices} file is generated from the
kernel's data structures describing its devices.

The \texttt{/proc} file system presents a user readable window into
the kernel's inner workings.  Several Linux subsystems, such as Linux
kernel modules described in chapter�\ref{modules-chapter}, create
entries in the the \texttt{/proc} file system.

\section{Ficheros especiales de dispostivo}
\index{Device Special Files} Linux, like all versions of Unix\tm\ 
presents its hardware devices as special files.  So, for example,
/dev/null is the null device.  A device file does not use any data
space in the file system, it is only an access point to the device
driver.  The EXT2 file system and the Linux VFS both implement device
files as special types of inode.  There are two types of device file;
character and block special files.  Within the kernel itself, the
device drivers implement file semantices: you can open them, close
them and so on.  Character devices allow I/O operations in character
mode and block devices require that all I/O is via the buffer cache.
When an I/O request is made to a device file, it is forwarded to the
appropriate device driver within the system.  Often this is not a real
device driver but a pseudo-device driver for some subsystem such as
the SCSI device driver layer.  Device files are referenced by a major
number, which identifies the device type, and a minor type, which
identifies the unit, or instance of that major type.  For example, the
IDE disks on the first IDE controller in the system have a major
number of 3 and the first partition of an IDE disk would have a minor
number of 1.  So, \texttt{ls -l}\index{ls command} of \fn{/dev/hda1}
gives: \marginnote{see \fn{/include/linux/\\major.h} for all of
  Linux's major device numbers.}
\begin{verbatim}
$ brw-rw----   1 root    disk       3,    1  Nov 24  15:09 /dev/hda1
\end{verbatim}
Within the kernel, every device is uniquely described by a
\dsni{kdev\_t}\index{kdev\_t data type} data type, this is two bytes
long, the first byte containing the minor device number and the second
byte holding the major device number.
\SeeModule{include/�linux/�kdev\_t.h} The IDE device above is held
within the kernel as \hex{0301}.  An EXT2 inode that represents a
block or character device keeps the device's major and minor numbers
in its first direct block pointer.  When it is read by the VFS, the
VFS inode data structure representing it has its \field{i\_rdev} field
set to the correct device identifier.
  %% rak
% Linux Installation and Getting Started    -*- TeX -*-
% swapfile.tex
% Copyright (c) 1992, 1993 by Matt Welsh <mdw@sunsite.unc.edu>
%
% This file is freely redistributable, but you must preserve this copyright 
% notice on all copies, and it must be distributed only as part of "Linux 
% Installation and Getting Started". This file's use is covered by
% the copyright for the entire document, in the file "copyright.tex".
%
% Copyright (c) 1998 by Specialized Systems Consultants Inc. 
% <ligs@ssc.com>
%Traducido por Sebasti�n Gurin, Cancerbero <anon@adinet.com.uy>, el 09/01/01 
%Revisi�n 1 7/7/2002 por Francisco Javier Fernandez <serrador@arrakis.es>

\section{Usando un fichero de intercambio}
\label{sec-swap-file}
\markboth{Administraci�n del Sistema}{Usando un fichero de intercambio}

En lugar de reservar una partici�n separada para el espacio de
intercambio, se puede usar un fichero de intercambio. Sin embargo,
ser� necesario instalar {\linux} y conseguir que todo funcione antes de crearlo.

Teniendo {\linux} ya instalado, se puede usar las siguientes
instrucciones para crear el fichero de intercambio. La orden de abajo,
crea un fichero de intercambio de 8208 bloques de tama�o, (aproximadamente 8 Mb).


\begin{tscreen}
\# dd if=/dev/zero of=/swap bs=1024 count=8208 
\end{tscreen}

Esta orden crea el fichero de intercambio, {\tt /swap}. El par�metro
``{\tt count=}'', es el tama�o del fichero de intercambio en bloques.
\begin{tscreen}
\# mkswap /swap 8208
\end{tscreen}
Esta orden inicia el fichero de intercambio. Una vez m�s, ser�
necesario reemplazar el nombre y el tama�o del fichero de intercambio
con los valores apropiados. 

\begin{tscreen}
\# sync \\
\# swapon /swap
\end{tscreen}
Ahora el sistema est� realizando el intercambio en el fichero {\tt
  /swap}. La instrucci�n {\tt sync} garantiza que el fichero haya sido escrito en el disco. 

Una desventaja importante de usar un fichero de intercambio, es que todo acceso al fichero, es hecho a trav�s del sistema de ficheros. Esto significa que
los bloques que constituyen el fichero de intercambio pueden no ser contiguos. Como consecuencia, el rendimiento puede no ser tan bueno como el de una partici�n de 
intercambio, en donde los bloques son siempre contiguos y las demandas de entrada/salida son realizadas directamente al dispositivo. 
Otra desventaja de los ficheros de intercambio largos es el gran peligro de que el sistema de ficheros se corrompa si algo sale mal. 
Conservar los ficheros normales, separados de las particiones de intercambio previene que esto pase. 
Los ficheros de intercambio pueden ser �tiles si, por ejemplo, se
necesita usar, temporalmente, m�s espacio de intercambio. Si se est�
compilando un programa extenso y se quisiera  acelerar las cosas un
tanto, se puede crear un fichero de intercambio temporal y usarlo
adem�s del espacio de intercambio regular. 
Para eliminar un fichero de intercambio, usa primero {\tt swapoff}, como en
\begin{tscreen}
\# swapoff /swap
\end{tscreen}
Luego, el fichero puede ser eliminado
\begin{tscreen}
\# rm /swap
\end{tscreen}


Cada fichero o partici�n de intercambio puede tener un tama�o m�ximo
de 128 megabytes, pero se puede usar hasta 8 ficheros o particiones de
intercambio en el sistema. 

%Traducido por Sebasti�n Gurin (Cancerbero), el 12/01/01  %% rak
% \linux Installation and Getting Started    -*- TeX -*-
% users.tex
% Copyright (c) 1993 by Matt Welsh and Lars Wirzenius
%
% This file is freely redistributable, but you must preserve this copyright 
% notice on all copies, and it must be distributed only as part of "\linux 
% Installation and Getting Started". This file's use is covered by
% the copyright for the entire document, in the file "copyright.tex".
%
% Este fichero es de distribuci�n libre, pero debe mantenerse esta 
% informaci�n de Copyright en todas las copias, y debe distribuirse solo como
% parte de "Instalaci�n y Primeros Pasos en \linux". El uso de este fichero esta
% cubierto por el Copyright del documento completo, en el fichero "copyright.tex"
% Copyright (c) 1995 por Gerardo Izquierdo para la versi�n al Castellano
%

% 
% Versi�n para lipp 2.0 por Alberto Molina. Comentarios a:
%            alberto@nucle.us.es 
%Revisi�n1 por Javier Fernandez <serrador@arrakis.es>
%Gold


\section{Gesti�n de Usuarios}
\label{sec-manage-users}
\label{sec-add-user}

\index{administraci�n de usuarios!a�adiendo usuarios}
\index{a�adiendo usuarios}
\index{usuarios!a�adiendo}
Independientemente de que haya muchos usuarios o no en el sistema, es
importante comprender los aspectos de la gesti�n de usuarios bajo \linux.
Incluso si se es el �nico usuario, se debe tener, presumiblemente, una cuenta
distinta de {\tt root} para hacer la mayor parte del trabajo.

        Cada persona que utilice el sistema debe tener su propia cuenta.
        Raramente es una buena idea el que varias personas compartan la misma
        cuenta. No s�lo es un problema de seguridad, sino que las cuentas
        se utilizan para identificar un�vocamente a los usuarios al sistema.
        Es necesario saber qui�n est� haciendo qu� en cada momento.

\subsection{Conceptos de gesti�n de usuarios}
El sistema mantiene una cierta cantidad de informaci�n acerca de cada usuario.
Dicha informaci�n se resume a continuaci�n.
\begin{dispitems}

\index{usuarios!nombre de }
\index{nombre de usuario!definici�n}
\ditem{{\bf nombre de usuario}}
El nombre de usuario es el identificador �nico dado a cada usuario del 
sistema. Ejemplos de nombres de usuario son {\tt manolo}, {\tt pepe} y 
{\tt mdw}. Se pueden utilizar letras y d�gitos junto a los car�cteres 
``{\tt \_}'' (subrayado) y ``{\tt .}'' (punto). Los nombres de usuario se
limitan normalmente a 8 car�cteres de longitud.

\index{usuarios!user ID de}
\index{user ID!definici�n}
\index{UID!definici�n}
\ditem{{\bf ID de usuario}}
El ID de usuario, o UID en sus siglas en ingl�s, es un n�mero
�nico dado a cada usuario del sistema. El sistema normalmente
le sigue la pista a los usuarios por su UID, no por el nombre de usuario.

\index{usuarios!group ID de}
\index{group ID!definici�n}
\index{GID!definici�n}
\ditem{{\bf ID de grupo}}
El ID de grupo, o GID en sus siglas en ingl�s, es la identificaci�n
del grupo del usuario predeterminado. En la secci�n~\ref{sec-perms}
discutimos los permisos de grupo; cada usuario pertenece a uno o m�s
grupos definidos por el administrador del sistema.

\index{usuarios!clave de}
\ditem{{\bf clave}}
El sistema tambi�n almacena la clave cifrada del usuario. La orden
{\tt passwd} se utiliza para poner y cambiar las claves de los usuarios.


\index{usuarios!nombre completo de}
\ditem{{\bf nombre completo}}
El ``nombre real'' o ``nombre completo'' del usuario se almacena junto con el
nombre de usuario. Por ejemplo, el usuario {\tt jperez} puede tener el nombre
``Jos� P�rez'' en la vida real.

\index{usuarios!directorio inicial de}
\index{directorio inicial!definido}
\ditem{{\bf directorio inicial}}
El directorio inicial es el directorio en el que se coloca inicialmente al
usuario en tiempo de conexi�n. Cada usuario debe tener su propio directorio
inicial, normalmente situado bajo {\tt /home}.

\index{usuarios!Int�rprete de conexi�n de}
\index{int�rprete de conexi�n!definici�n}
\ditem{{\bf int�rprete al registrarse}}
El int�rprete al registrarse  es el int�rprete de �rdenes que se ejecutar�
cuando se registre el usuario. Ejemplos pueden ser {\tt /bin/bash} y {\tt /bin/tcsh}.
\end{dispitems}

\index{/etc/passwd@{\tt /etc/passwd}}
\index{fichero de constrase�as!formato de}
El fichero {\tt /etc/passwd} contiene la informaci�n anterior acerca de los
usuarios. Cada l�nea del fichero contiene informaci�n acerca de un �nico 
usuario; el formato de cada l�nea es
\begin{tscreen}
nombre:clave cifrada:UID:GID:nombre completo:dir.inicio:int�rprete
\end{tscreen}
Un ejemplo puede ser:
\begin{tscreen}
kiwi:Xv8Q981g71oKK:102:100:Laura Villa:/home/kiwi:/bin/bash
\end{tscreen}

Como puede verse, el primer campo , ``{\tt kiwi}'', es el nombre de usuario.

El siguiente campo, ``{\tt Xv8Q981g71oKK}'', es la clave cifrada.
Las claves no se almacenan en el sistema en ning�n formato legible por
el hombre. Las claves se cifran utiliz�ndose ellas mismas como clave
secreta. En otras palabras, s�lo si se conoce la clave, �sta puede ser
descifrada. Esta forma de cifrado es bastante segura.

Algunos sistemas utilizan ``claves en sombra'' en la que la informaci�n
de las claves se relega al fichero {\tt /etc/shadow}. Puesto que 
{\tt /etc/passwd} es legible por todo el mundo, {\tt /etc/shadow} suministra
un grado extra de seguridad, puesto que �ste no lo es. Las claves en 
sombra suministran algunas otras funciones como puede ser la expiraci�n de 
claves; no entraremos a detallar estas funciones aqu� .

El tercer campo ``{\tt 102}'', es el UID. Este debe ser �nico para cada 
usuario. El cuarto campo, ``{\tt 100}'', es el GID. Este usuario pertenece
al grupo numerado 100. La informaci�n de grupos, como la informaci�n de
usuarios, se almacena en el fichero {\tt /etc/group}. V�ase la 
secci�n~\ref{sec-manage-groups} para m�s informaci�n.

El quinto campo es el nombre completo del usuario. ``{\tt Laura Villa}''. Los
dos �ltimos campos son el directorio inicial del usuario ({\tt /home/kiwi}) y
la shell de ingreso ({\tt /bin/bash}), respectivamente. No es necesario
que el directorio inicial de un usuario tenga el mismo nombre que el del
nombre de usuario. Sin embargo, ayuda a identificar el directorio.

\subsection{A�adir usuarios}
Cuando se a�ade un usuario hay varios pasos a seguir. Primero, 
se le debe crear una entrada en {\tt /etc/passwd}, con un nombre de
usuario y UID �nicos. Se debe especificar el GID, nombre completo y resto
de informaci�n. Se debe crear el directorio inicial, y poner los permisos
en el directorio para que el usuario sea el due�o. Se deben suministrar 
ficheros de ordenes de inicializaci�n en el nuevo directorio y se debe 
hacer alguna otra configuraci�n del sistema (por ejemplo, preparar
un buz�n para el correo electr�nico entrante para el nuevo usuario).

Aunque no es dif�cil el a�adir usuarios a mano (yo lo hago), cuando 
se est� ejecutando un sistema con muchos usuarios, es f�cil el 
olvidarse de algo. La manera m�s simple de a�adir usuarios es 
utilizar un programa interactivo que vaya preguntando por la 
informaci�n necesaria y actualice todos los ficheros del sistema 
autom�ticamente. El nombre de este programa es {\tt useradd} o {\tt 
adduser} dependiendo del software que est� instalado.

Un fichero t�pico {\tt /etc/adduser.conf} se muestra a continuaci�n: 
\begin{tscreen}\begin{verbatim}
# /etc/adduser.conf: Configuraci�n de `adduser'.
# Vea adduser(8) y adduser.conf(5) para m�s informaci�n.

# La variable DSHELL especifica la shell de ingreso asumida en 
# el sistema.
DSHELL=/bin/bash

# La variable DHOME especifica el directorio que contendr� los
# directorios iniciales de los usuarios.
DHOME=/home

# Si en GROUPHOMES pone "yes", entonces los directorios iniciales
# estar�n en /home/nombre_grupo/usuario.
GROUPHOMES=no

# Si en LETTERHOMES pone "yes", entonces los directorios iniciales
# tendr�n un directorio extra correspondiente a la primera letra del
# nombre de usuario, como por ejemplo: /home/u/user.
LETTERHOMES=no

# La variable SKEL especifica el directorio que contiene los ficheros
# inciales configurables de cada usuario, como el fichero .profile que
# se copiar� al directorio de inicio de un usuaio cuando sea creado.
SKEL=/etc/skel

# De FIRST_SYSTEM_UID a LAST_SYSTEM_UID ambos inclusive, va el rango de
# UID para cuentas del sistema y administraci�n.
FIRST_SYSTEM_UID=100
LAST_SYSTEM_UID=999

# DE FIRST_UID a LAST_UID ambos inclusive, va el rango de UID para
# cuentas de usuarios. 
FIRST_UID=1000
LAST_UID=29999

# La variable USERGROUPS puede estar en "yes" o "no". Si est� en "yes"
# cada usuario tendr� que usar como asumido su propio grupo y su
# directorio inicial ser� g+s. Si est� en "no", cada usuario a�adido
# ser� colocado en el grupo con gid igual a USERS_GID (ver m�s abajo).
USERGROUPS=yes

# Si USERGROUPS est� en "no", entonces USERS_GID ser� el GID del grupo
# `users' del sistema.
USERS_GID=100

# Si se especifica QUOTAUSER, se limitar� el espacio para el
# directorio inicial de un usuario (cuota) mediante:
# `edquota -p QUOTAUSER newuser'
QUOTAUSER=""
\end{verbatim}\end{tscreen}

Adem�s de definir las variables predefinidas que la orden adduser
utiliza, {\tt /etc/adduser.conf} tambi�n especifica d�nde se localizan
los ficheros de configuraci�n del sistema de cada usuario. En este
ejemplo, est�n en {\tt /etc/skel}, definido por la l�nea {\tt
  SKEL=}. Los ficheros que se coloquen en este directorio, como {\tt
  .profile}, {\tt .tcshrc} o {\tt .bashrc} se copiar�n
autom�ticamente al directorio de inicio de un usuario al a�adirlo con
la orden {\tt adduser}.

\subsection{Borrando usuarios}

De forma parecida, se pueden borrar usuarios mediante la orden 
{\tt userdel} o {\tt deluser} dependiendo de qu� software est� instalado 
en el sistema.

\index{usuarios!deshabilitando}
\index{deshabilitando usuarios}
Si se desea ``deshabilitar'' temporalmente un usuario para que no se conecte
al sistema (sin borrar la cuenta del usuario), se puede anteponer un
asterisco (``{\tt *}'') al campo de la clave en {\tt /etc/passwd}. Por 
ejemplo, cambiando la l�nea de {\tt /etc/passwd} correspondiente a 
{\tt kiwi} a 
\begin{tscreen}
kiwi:*Xv8Q981g71oKK:102:100:Laura Villa:/home/kiwi:/bin/bash
\end{tscreen}
evitar� que {\tt kiwi} se conecte.

\subsection{Poniendo atributos de usuario}

Despu�s de que haya creado un usuario, puede necesitar cambiar alg�n 
atributo de dicho usuario, como puede ser el directorio inicial o la 
clave. La forma m�s simple de hacer �sto es cambiar los valores 
directamente en {\tt /etc/passwd}. Para poner clave a un usuario, utilice 
la orden {\tt passwd}. Por ejemplo,
\begin{tscreen}
\# {\em passwd manuel}
\end{tscreen}
cambiar� la clave de {\tt manuel}. S�lo el administrador `` {\tt root}'' puede cambiar la 
clave de otro usuario de esta forma. Los usuarios pueden cambiar su propia
clave con {\tt passwd} tambi�n.

En algunos sistemas, las instrucciones {\tt chfn} y {\tt chsh} est�n 
disponibles, permitiendo a los usuarios cambiar su nombre completo y
la shell de ingreso. Si no, deben pedirle al administrador del sistema
que los modifique.

\subsection{Grupos}\label{sec-manage-groups}

Como hemos citado anteriormente, cada usuario pertenece a uno o m�s grupos.
La �nica importancia real de las relaciones de grupo es la perteneciente a
los permisos de ficheros, como dijimos en la secci�n~\ref{sec-perms}, cada
fichero tiene un ``grupo propietario'' y un conjunto de permisos de grupo
que define de qu� forma pueden acceder al fichero los usuarios del grupo.

Hay varios grupos definidos en el sistema, como pueden ser {\tt bin}, 
{\tt mail}, y {\tt sys}. Los usuarios no deben pertenecer a ninguno de estos
grupos; se utilizan para permisos de ficheros del sistema. En su lugar, los
usuarios deben pertenecer a un grupo individual, como {\tt users}. Si se 
quiere ser detallista, se pueden mantener varios grupos de usuarios como por
ejemplo {\tt estudiantes}, {\tt mantenimiento} y {\tt secretar�a}.

\index{/etc/group@{\tt /etc/group}!formato de}
El fichero {\tt /etc/group} contiene informaci�n acerca de los grupos.
El formato de cada l�nea es
\begin{tscreen}
nombre de grupo:clave:GID:otros miembros
\end{tscreen}
Algunos ejemplos de grupos pueden ser:
\begin{tscreen}
root:*:0: \\
usuarios:*:100:mdw,pepe \\
invitados:*:200: \\
otros:*:250:kiwi
\end{tscreen}

El primer grupo, {\tt root}, es un grupo especial del sistema reservado para
la cuenta {\tt root}. El siguiente grupo, {\tt users}, es para usuarios 
normales. Tiene un GID de 100. Los usuarios {\tt mdw} y {\tt pepe} tienen
acceso a este grupo. Recu�rdese que en {\tt /etc/passwd} cada usuario tiene
un GID predeterminado. Sin embargo, los usuarios pueden pertenecer a m�s de un
grupo, a�adiendo sus nombres de usuario a otras l�neas de grupo en 
{\tt /etc/group}. La orden {\tt groups} nos dice a qu� grupos se tiene 
acceso.

El tercer grupo, {\tt invitados}, es para usuarios invitados, y {\tt otros}
es para ``otros'' usuarios. El usuario {\tt kiwi} tiene acceso a �ste 
grupo.

Como se puede ver, el campo ``clave'' de {\tt /etc/group} raramente se 
utiliza. A veces se utiliza para dar una clave para acceder a un grupo.
Esto es raras veces necesario. Para evitar el que los usuarios cambien a
grupos privilegiados (con la orden {\tt newgroup}), se pone el campo de
la clave a ``{\tt *}''.

Se pueden usar las ordenes {\tt addgroup} o {\tt groupadd} para a�adir
grupos a su sistema. Normalmente es m�s sencillo a�adir l�neas a 
{\tt /etc/group} uno mismo, puesto que no se necesitan m�s 
configuraciones para a�adir un grupo. Para borrar un grupo, s�lo hay 
que borrar su entrada de {\tt /etc/group}.



% Linux Installation and Getting Started    -*- TeX -*-
% hats.tex
% Copyright (c) 1993 by Matt Welsh and Lars Wirzenius
%
% This file is freely redistributable, but you must preserve this copyright 
% notice on all copies, and it must be distributed only as part of "Linux 
% Installation and Getting Started". This file's use is covered by
% the copyright for the entire document, in the file "copyright.tex".
%
% Este fichero es de distribuci\'on libre, pero debe mantenerse esta 
% informaci\'on de Copyright en todas las copias, y debe distribuirse solo como
% parte de "Instalaci\'on y Primeros Pasos en Linux". El uso de este fichero esta
% cubierto por el Copyright del documento completo, en el fichero "copyright.tex"
% Copyright (c) 1995 por Gerardo Izquierdo para la versi\'on al Castellano
% $Log: hats.tex,v $
% Revision 1.6  2003/07/19 06:32:34  joseluis.ranz
% Correcciones varias.
%
% Revision 1.5  2002/07/20 22:24:29  pakojavi2000
% Beta2
%
% Revision 1.4  2002/07/07 21:40:49  pakojavi2000
%  Traducci�n de fragmentos incompletos
%
% Revision 1.3  2001/04/18 16:29:10  amolina
% Segunda revisi�n de los ficheros
%
% Revision 1.2  2001/01/16 15:10:36  amolina
%
% Primera traducci�n de sysadm/hats,tex
%
% Revision 0.5.0.1  1996/02/10 23:45:12  rcamus
% Primera beta publica

% 
% Versi�n para lipp 2.0 por Alberto Molina. Comentarios a:
%            alberto@nucle.us.es 
%
%

\subsection{Responsabilidades de la Administraci�n del Sistema}

Puesto que el administrador de sistema tiene mucho m�s poder y
responsabilidad, cuando algunos usuarios tienen la oportunidad de
ingresar por primera vez como {\tt root}, tanto en sistemas GNU/Linux como
en otros, tienden a abusar de los privilegios de {\tt root}. Existen
``administradores de sistema'' que leen el correo de otros usuarios,
borran ficheros sin avisar y se comportan como ni�os con un poderoso
juguete entre sus manos.

Como el administrador tiene tanto poder sobre el sistema, se requiere
cierta madurez y autocontrol para utilizar la cuenta {\tt
  root}. Existe un c�digo de honor no escrito que establece las normas
de comportamiento del administrador del sistema para con el resto de
usuarios. �C�mo se sentir�a si el administrador de su sistema se
dedicase a leer su correo electr�nico o a mirar en sus
ficheros?. Existe un cierto vac�o legal en estos asuntos. En los
sistemas UNIX, el usuario {\tt root} tiene la posibilidad de saltarse
todos los mecanismos de seguridad y privacidad. Es importante que el
administrador de sistema establezca una relaci�n de confianza con sus
usuarios.

\subsection{C�mo proceder con los usuarios}

Los administradores de sistemas pueden tomar dos posturas cuando traten con
usuarios abusivos: ser paranoicos o confiados. El administrador de 
sistemas paranoico normalmente causa m\'as da\~no que el que previene. Una de
mis citas favoritas es: ``Nunca atribuyas a la malicia nada que pueda ser
atribuido a la estupidez''. Dicho de otra forma, muchos usuarios no tienen
la habilidad o el conocimiento para hacer da\~no real al sistema. El 90\% del
tiempo, cuando un usuario causa problemas en el sistema (por ejemplo, 
rellenando la partici\'on de usuarios con grandes ficheros, o ejecutando
m\'ultiples veces simult�neamente un gran programa), el usuario simplemente desconoce 
que est\'a causando un problema. He ido a ver a usuarios que estaban
causando una gran cantidad de problemas, pero su actitud estaba causada por 
la ignorancia, no por la malicia.

Cuando se encuentre con usuarios que puedan causar problemas potenciales
no sea hostil. La antigua regla de ``inocente hasta que se demuestre lo
contrario'' sigue siendo v\'alida. Es mejor una simple charla con el usuario,
pregunt\'andole acerca del problema, en lugar de causar una confrontaci\'on. Lo
\'ultimo que se desea es estar entre los malos desde el punto de vista del
usuario. Esto levantar\'\i a un mont\'on de sospechas acerca de si el
administrador de sistemas tiene el sistema correctamente 
configurado. Si un usuario cree que uno le disgusta o no le tiene 
confianza, le puede acusar de borrar ficheros o romper la privacidad del 
sistema. Esta no es, ciertamente, el tipo de situaci\'on en la que se
quisiera estar.

Si se encuentra que un usuario ha estado intentando ``romper'' el sistema,
o ha estado haciendo da\~no al sistema de forma intencionada, no hay
que devolver el comportamiento malicioso a su vez. En vez de ello,
simplemente, es recomendable darle un  aviso ---pero siendo
flexible. En muchos casos, se puede cazar a un usuario
``con las manos en la masa'', da\~nando al sistema, lo correcto es
advertirle y decirle que no lo vuelva a repetir. Sin embargo, si le
{\em vuelve\/} a cazar haciendo da\~no, entonces se puede estar
absolutamente seguro de que es intencionado.
Ni siquiera puedo empezar a describir la cantidad de veces que parec\'\i a que
hab\'\i a un usuario causando problemas al sistema, cuando de hecho, era o un
accidente o un fallo m\'\i o.

\subsection{Fijando las reglas}

La mejor forma de administrar un sistema no es con un pu\~no de hierro. 
As\'{\i} puede ser como se haga lo militar, pero UNIX no fue dise\~nado para 
ese tipo de disciplinas. Tiene sentido el escribir un conjunto sencillo y 
flexible de reglas para los usuarios, pero hay que recordar que cuantas menos 
reglas tenga, menos posibilidades habr\'a de romperlas. Incluso si las 
reglas para utilizar el sistema son perfectamente razonables y claras, 
siempre habr\'a momentos en que los usuarios romper\'an dichas reglas sin 
pretenderlo. Esto es especialmente cierto en el caso de usuarios UNIX 
nuevos, que est\'an aprendiendo los entresijos del sistema. No esta 
suficientemente claro, por ejemplo, que uno no debe bajarse un gigabyte de 
ficheros y envi\'arselo por correo a todos los usuarios del sistema. Los 
usuarios necesitan comprender las reglas y por qu� est\'an establecidas.

Si especifica reglas de uso para su sistema, hay que asegurarse de que el motivo 
detr\'as de cada regla particular est\'e claro. Si no se hace, los usuarios
encontrar\'an toda clase de formas creativas de salt\'arsela y no saber que en
realidad la est\'an rompiendo.

\subsection{Lo que todo esto significa}

No podemos decir c�mo ejecutar su sistema al \'ultimo detalle. Mucha de la
filosof\'\i a depende de c�mo se use el sistema. Si se tienen muchos 
usuarios, las cosas son muy diferentes de si s�lo tiene unos pocos o si 
se es el \'unico usuario del sistema. Sin embargo, siempre es una buena 
idea, en cualquier situaci\'on, comprender lo que ser administrador 
de sistema significa en realidad.

Ser el administrador de un sistema no le hace a uno un mago del UNIX. Hay
muchos administradores de sistemas que conocen muy poco acerca de UNIX.
Igualmente, hay muchos usuarios ``normales'' que saben m\'as acerca de 
UNIX que cualquier administrador de sistema. Tambi\'en, ser 
administrador de sistemas no le permite el utilizar la malicia contra sus 
usuarios. Aunque el sistema le d\'e el privilegio de enredar en los 
ficheros de los usuarios, no significa que se tenga ning\'un derecho a 
hacerlo.

Por \'ultimo, ser el administrador del sistema no es realmente una gran cosa.
No importa si el sistema es un peque\~no 386 o un super ordenador Cray. La
ejecuci\'on del sistema es la misma. El saber la clave de {\tt root} no
significa ganar dinero o fama. Tan solo le permitir\'a ejecutar el sistema
y mantenerlo funcionando. Eso es todo.

  %% rak
% Linux Installation and Getting Started    -*- TeX -*-
% tar.tex
% Copyright (c) 1993 by Matt Welsh and Lars Wirzenius
%
% This file is freely redistributable, but you must preserve this copyright 
% notice on all copies, and it must be distributed only as part of "Linux 
% Installation and Getting Started". This file's use is covered by
% the copyright for the entire document, in the file "copyright.tex".
%
% Copyright (c) 1998 by Specialized Systems Consultants Inc. 
% <ligs@ssc.com>
%

%Traducido por Sebasti�n Gurin, Cancerbero <anon@adinet.com.uy> el 12/01/01

%Revisado por Sebasti�n Gurin, Cancerbero <anon@adinet.com.uy> el 18/01/01
%Revisi�n 1 por Francisco Javier Mart�nez <serrador@arrakis.es>   el 7/7/02


 

\section{Almacenamiento y compresi�n de ficheros}
\markboth{Administraci�n de Sistema}{Almacenando y Comprimiendo ficheros}
\subsection*{Pre�mbulo a la traducci�n al castellano}
{\em En espa�ol existe cierta confusi�n entre los t�rminos fichero y archivo, los cu�les se toman como sin�nimos. En inform�tica y en este texto, 
existe una sutil pero importante diferencia entre los t�rminos. Cuando nos referimos a un fichero, nos referimos a cualquier tipo de documento,
imagen, sonido almacenado en un soporte l�gico. Sin embargo, un archivo es una clase especial de fichero que contiene otros ficheros. El origen
de dicha confusi�n parece ser la traducci�n err�nea de fichero por archivo en los sistemas operativos de Microsoft. No cometeremos el mismo error aqu�,
Por lo tanto aqu� se llamar�n archivos a los ficheros .tar y similares cuyo prop�sito es contener otros ficheros.}
{\em(Nota del Revisor)}

\subsection{Usando {\tt tar}}
Antes de que podamos hablar de copias de seguridad, necesitamos realizar una presentaci�n de las herramientas utilizadas para almacenar ficheros en sistemas UNIX. 

La orden {\tt tar} es la m�s usada para almacenar ficheros. Su sintaxis es:
\begin{tscreen}
tar \cparam{opciones} \cparam{ficheros} 
\end{tscreen}
en donde \textsl{opciones} es la lista de opciones para {\tt tar},
y \textsl{ficheros} es la lista de ficheros a agregar o extraer del archivo tar.  
Por ejemplo, la orden 
\begin{tscreen}
\# tar cvf backup.tar /etc
\end{tscreen}
empaqueta todos los ficheros del directorio {\tt /etc} en el archivo tar {\tt
backup.tar}. El primer par�metro que se le entrega a {\tt tar}, ``{\tt cvf}'', es la orden que le transmitimos a {\tt tar}.
``{\tt c}'' le dice a tar que cree un nuevo archivo. La opci�n ``{\tt v}''  fuerza a tar en el modo detallado,
imprimiendo los nombres de los ficheros seg�n se archivan. La opci�n ``{\tt f}'' le informa a {\tt tar}, que el pr�ximo argumento, 
{\tt backup.tar}, es el nombre del archivo a crear. El resto de los argumentos para {\tt tar} son el/los nombre(s) de ficheros(s) y
directorio(s) para agregar al archivo tar.

La instrucci�n
\begin{tscreen}
\# tar xvf backup.tar
\end{tscreen}
extraer� todos los ficheros archivados dentro de {\tt backup.tar} en el directorio actual.

\blackdiamond Los ficheros antiguos con el mismo nombre son sobrescritos cuando se extraen en un directorio existente. 
Antes de extraer ficheros de un archivo tar, es importante saber d�nde deben ser desempaquetados los ficheros.
Digamos que se han archivado los siguientes ficheros: {\tt /etc/hosts}, {\tt /etc/group}, y {\tt /etc/passwd}. Si se us� la orden

\begin{tscreen}
\# tar cvf backup.tar /etc/hosts /etc/group /etc/passwd
\end{tscreen}
el nombre del directorio {\tt /etc/} se a�adir� al principio de los nombres de cada fichero. Para extraer los ficheros en su ubicaci�n correcta, debe usarse
\begin{tscreen}
\# cd / \\
\# tar xvf backup.tar
\end{tscreen}
porque los ficheros son extra�dos con el nombre de ruta, guardado, en el archivo tar.
Sin embargo, si se han archivado los ficheros con la orden
\begin{tscreen}
\# cd /etc \\
\# tar cvf hosts group passwd
\end{tscreen}
el nombre del directorio no se conserva en el archivo tar. En consecuencia, necesitar�s hacer un ``{\tt cd /etc}'', 
antes de extraer los ficheros. Como puedes ver, el c�mo haya sido creado un fichero tar, marca una gran diferencia en c�mo se extrae; 
o dicho de otra modo: la manera en la que ser�n extra�dos los ficheros de un archivo tar, est� estrechamente relacionada con la manera en c�mo han sido archivados.
La orden
\begin{tscreen}
\# tar tvf backup.tar
\end{tscreen}
se puede usar para mostrar una lista de los ficheros del archivo tar, pero sin extraerlos. De esta forma se puede ver qu�
directorio se utiliz� como origen de los nombres de los ficheros, y se puede extraer el fichero desde la localizaci�n correcta.

\subsection{{\tt gzip} y {\tt compress}}
A diferencia de los de archivado para MS-DOS, {\tt tar} no comprime
los ficheros autom�ticamente seg�n los archiva. Por ejemplo: si se
archivan  dos ficheros de 1 Mega byte cada uno, en un archivo tar, el
tama�o de este �ltimo ser� de 2 Mega bytes. En \linux, la orden {\tt
gzip}, puede utilizarse para comprimir un archivo, (no tiene por que
ser un archivo tar). La instrucci�n
\begin{tscreen}
\# gzip -9 backup.tar
\end{tscreen}
comprime {\tt backup.tar}, dej�ndonos el fichero {\tt backup.tar.gz}, una versi�n
comprimida del archivo. El par�metro {\tt -9}, hace que {\tt gzip}, utilice el mayor factor de compresi�n.
La orden gunzip puede ser utilizado para descomprimir un fichero comprimido con gzip. La orden {\tt gzip -d} es equivalente a {\tt gunzip}.
{\tt gzip} es una herramienta relativamente nueva en la comunidad
UNIX. Durante muchos a�os, se utiliz� en su lugar {\tt compress}. Sin
embargo, debido a varios factores, incluyendo una disputa por una
patente de software contra su algoritmo de compresi�n, y el hecho de
que {\tt gzip} es mucho m�s eficiente, {\tt compress} se est�
volviendo anticuado.



\subsection{Aplic�ndolos en conjunto}
Para archivar un grupo de ficheros y comprimir el resultado, use las �rdenes
\begin{tscreen}
\# tar cvf backup.tar /etc \\
\# gzip -9 backup.tar
\end{tscreen}

Como resultado obtendr� {\tt backup.tar.gz}. Para descomprimir este
archivo, use las �rdenes inversas:

\begin{tscreen}
\# gunzip backup.tar.gz \\
\# tar xvf backup.tar
\end{tscreen}

Recordatorio: Siempre hay que estar seguro de encontrarse en el
directorio correcto antes de descomprimir un archivo tar.
Tambi�n se puede usar algunas de las ingeniosidades de {\linux} para
realizar esto, pero en una sola l�nea de ordenes: 
\begin{tscreen}
\# tar cvf - /etc $\mid$ gzip -9c $>$ backup.tar.gz
\end{tscreen}
Aqu�, enviamos el fichero tar a ``{\tt -}'', que representa la salida est�ndar de {\tt
tar}. Esto es canalizado a {\tt gzip}, quien comprime el archivo tar entrante. El producto es guardado en {\tt backup.tar.gz}.
La opci�n {\tt -c} le ordena a {\tt gzip} que env�e su salida a la salida est�ndar, que es reencauzada a {\tt backup.tar.gz}.
Una simple orden para descomprimir este archivo ser�a:
\begin{tscreen}
\# gunzip -c backup.tar.gz $\mid$ tar xvf -
\end{tscreen}
Nuevamente, {\tt gunzip} descomprime el contenido de {\tt backup.tar.gz} y env�a el archivo tar resultante a la salida est�ndar. �sta es canalizada a {\tt tar},
quien lee ``{\tt -}'', lo cual representa, esta vez, la entrada est�ndar de {\tt tar}.

Felizmente, la orden {\tt tar} incluye tambi�n la opci�n {\tt z} que, autom�ticamente realiza los procesos  de comprimir/descomprimir ficheros, e
invoca el programa, usando el algoritmo de compresi�n de {\tt gzip}.
La orden
\begin{tscreen}
\# tar cvfz backup.tar.gz /etc
\end{tscreen}
es equivalente a 
\begin{tscreen}
\# tar cvf backup.tar /etc \\
\# gzip backup.tar
\end{tscreen}
Tal como la orden
\begin{tscreen}
\# tar xvfz backup.tar.Z
\end{tscreen}
puede ser usado en lugar de 
\begin{tscreen}
\# uncompress backup.tar.Z  \\
\# tar xvf backup.tar
\end{tscreen}
Indagando en las p�ginas man se puede obtener mas informaci�n acerca de tar y gzip. 

% Linux Installation and Getting Started    -*- TeX -*-
% backups.tex
% Copyright (c) 1993 by Matt Welsh and Lars Wirzenius
%
% This file is freely redistributable, but you must preserve this copyright 
% notice on all copies, and it must be distributed only as part of "Linux 
% Installation and Getting Started". This file's use is covered by
% the copyright for the entire document, in the file "copyright.tex".
%
% Este fichero es de distribuci�n libre, pero debe mantenerse esta 
% informaci�n de Copyright en todas las copias, y debe distribuirse solo como
% parte de "Instalaci�n y Primeros Pasos en Linux". El uso de este fichero esta
% cubierto por el Copyright del documento completo, en el fichero "copyright.tex"
% Copyright (c) 1995 por Gerardo Izquierdo para la versi�n al Castellano
% $Log: backups.tex,v $
% Revision 1.9  2003/07/19 06:41:04  joseluis.ranz
% Correcciones varias.
%
% Revision 1.8  2002/10/12 19:53:23  montuno
% quitando defectos y comandos
%
% Revision 1.7  2002/09/09 16:50:46  pakojavi2000
% Correcci�n de fallos peque�os
%
% Revision 1.6  2002/07/25 02:08:22  pakojavi2000
% Beta 2.1
%
% Revision 1.5  2002/07/21 00:56:46  pakojavi2000
% Beta2.1
%
% Revision 1.4  2002/07/20 17:41:16  pakojavi2000
% beta2
%
% Revision 1.3  2002/07/07 20:47:31  pakojavi2000
% Terminado de traducir el fragmento que quedaba
%
% Revision 1.2  2001/05/09 18:29:17  amolina
% Primera versi�n de sysadm/backups.tex, (falta la secci�n correspondiente
% a las unidades de cinta).
%
% Revision 0.5.0.1  1996/02/10 23:45:12  rcamus
% Primera beta publica
%Revisi�nn 0.6 2002/07/07 22:11:12 <serrador@arrakis.es>
%


%
% Versi�n para lipp 2.0 por Alberto Molina. Comentarios a:
%            alberto@nucle.us.es 
%

\section{Usando Disquetes y Haciendo Copias de Seguridad}
\index{ficheros!salvaguarda|(}
\index{copias de seguridad|(}
Los disquetes son utilizados normalmente como medio para copias de seguridad.
Si no se tiene una unidad de cinta conectada al sistema, se pueden utilizar 
disquetes (a pesar de que sean m�s lentos y algo menos seguros).

Como se mencion� anteriormente, los disquetes se pueden formatear con
los programas {\tt FORMAT.COM} de MS-DOS o {\tt fdformat} de
{\linux}. Esto graba la informaci�n apropiada de la capacidad del
disquete.

Algunos de los nombres de dispositivos y formatos accesibles por
{\linux}, se dan en la tabla~\ref{table-disk-formats}.

\begin{table}[ht]\begin{center}
\small\begin{tabular}{ll}
\hline
Controlador del disquete              & Formato \\
\hline
/dev/fd0d360                    & Double densidad, 360 Kb, 5.25 pulgadas.  \\
/dev/fd0h1200                   & Alta densidad, 1.2 MB, 5.25 pulgadas. \\
/dev/fd0h1440                   & Alta densidad, 1.44 MB, 3.5 pulgadas.
\end{tabular}\normalsize\rm
\caption{Formatos de disquete {\linux}}
\label{table-disk-formats}
\end{center}\end{table}

El dispositivo que empieza con {\tt fd0} es la primera unidad de
disquete, que se corresponde con la {\tt A:} de MS-DOS. Los nombres de
los ficheros del controlador de la segunda unidad de disquete empiezan
con {\tt fd1}. Normalmente, el n�cleo de {\linux} puede detectar el
formato del disquete, basta con usar {\tt /dev/fd0} y dejar que el
sistema reconozca el formato. Pero cuando se utilizan disquetes nuevos
sin formato, puede ser necesario especificar el formato si el sistema
no logra detectar el tipo de disquete que es.
 
Una lista completa de los dispositivos {\linux} y los nombres de los
controladores de las unidades viene en {\em {\linux} Allocated Devices,}
de H. Peter Anvin (ver Apendice~\ref{app-info}).

\subsection{Utilizando disquetes para copias de seguridad}
\index{disquetes!como medio de copias de seguridad}
\index{copias de seguridad!a disquete}
\index{copias de seguridad!multi-volumen}
La forma m�s simple de hacer una copia de seguridad es con {\tt tar}.
La orden
\begin{tscreen}
\# {\em tar cvfzM /dev/fd0 /}
\end{tscreen}
har� una copia de seguridad completa del sistema utilizando el disquete 
{\tt /dev/fd0}. La opci�n ``{\tt M}'' de {\tt tar} permite que la copia de
seguridad sea una copia multi-volumen; esto es, cuando un disquete est� lleno,
{\tt tar} pedir� el siguiente. La orden
\begin{tscreen}
\# {\em tar xvfzM /dev/fd0}
\end{tscreen}
puede ser utilizada para recuperar la copia de seguridad completa.
Este m�todo puede ser utilizado tambi�n si se tiene una unidad de cinta
({\tt /dev/rmt0}) conectada al sistema.

\index{backflops@{\tt backflops}}
\index{afio@{\tt afio}}
\label{sec-backfloppy}
Existen otros programas para hacer copias de seguridad multi-volumen; 
el programa {\tt backflops} disponible en {\tt tsx-11.mit.edu} puede ser 
�til.

Hacer una copia de seguridad completa del sistema puede ser costoso en 
tiempo y recursos.
Muchos administradores de sistemas utilizan una pol�tica de copias de 
seguridad incrementales, en la que cada mes se hace una copia de seguridad 
completa, y cada semana s�lo se copian aquellos ficheros que hayan sido 
modificados en esa semana. En este caso, si el sistema se viene abajo a 
mitad de mes, s�lo tiene que restaurar la �ltima copia de seguridad 
mensual 
completa y, despu�s, las �ltimas copias semanales seg�n el caso.

\index{copias de seguridad!incremental}
\index{find@{\tt find}!para copias de seguridad incrementales}
La instrucci�n {\tt find} puede ser �til para localizar ficheros que hayan 
cambiado desde una cierta fecha. Se pueden encontrar varios ficheros de 
ordenes para manejar copias de seguridad incrementales en 
{\tt sunsite.unc.edu}. 
\index{copias de seguridad|)}
\index{ficheros!salvaguarda|)}
\subsection{Copias de seguridad con unidades Zip} \label{sec-zip-backup}

Las copias de seguridad sobre unidades Zip son muy parecidas a las de
disquetes, pero puesto que los Zip tienen una capacidad de 98 Mb,
muchas veces s�lo se necesita uno para la copia de seguridad.

Las unidades Zip est�n disponibles con tres interfaces de hardware:
una interfaz SCSI, una interfaz IDE y una interfaz PPA de puerto
paralelo. El soporte de unidades Zip no est� incluido como opci�n de
pre-compilado en {\linux}, pero se puede especificar cuando se
personaliza el n�cleo del sistema. En la
p�gina~\pageref{kernel-ppa-driver} se describe la instalaci�n de este
tipo de unidades. 

Las unidades Zip con interfaz SCSI y PPA, usan la interfaz SCSI y
siguen las convenciones de nombres de los dispositivos SCSI que se
describen en la p�gina~\pageref{device-driver-names}. 

Los discos Zip vienen normalmente pre-formateados como tipo
MS-DOS. Hay dos opciones a la hora de usarlos: Usar el Zip como
sistema de ficheros MS-DOS, que debe soportar el n�cleo del sistema o
usar {\tt mke2fs} o alg�n programa similar para escribir un sistema de
ficheros GNU/{\linux}.

Un disco Zip, cuando est� montado como el primer dispositivo SCSI,
est� en {\tt /dev/sda4}.
\begin{tscreen}
\# mount /dev/sda4 /mnt
\end{tscreen}

Muchas veces conviene proporcionar un directorio diferente para montar
sistemas de fichero Zip, por ejemplo, {\tt /zip}. Los siguientes
pasos, que deben realizarse como {\tt root}, montar�n la unidad en
este directorio:
\begin{tscreen}
\# mkdir /zip \\
\# chmod 0755 /zip
\end{tscreen}
Entonces ya se puede utilizar {\tt zip} para montar el sistema de
ficheros Zip.

Escribir archivos a discos Zip es parecido a hacerlo en
disquetes. Para comprimir el directorio {\tt /etc} a una unidad Zip ya
montada, se debe utilizar la instrucci�n
\begin{tscreen}
\# tar zcvf /zip/etc.tgz /etc
\end{tscreen}

Que se puede ejecutar desde cualquier directorio, puesto que
especifica completamente el {\em path}. El nombre del archivo {\tt
  etc.tgz} es necesario si la unidad Zip contiene un sistema de
ficheros MS-DOS, puesto que todos los ficheros que se graben en el
disco deben seguir la convenci�n de nombres de MS-DOS de 8+3. En caso
contrario, se truncar� el nombre del fichero.

De forma similar, se extrae el contenido del archivo con la instrucci�n
\begin{tscreen}
\# cd /  \\
\# tar zxvf /zip/etc.tgz
\end{tscreen}

Para crear, por ejemplo, un sistema de ficheros ext2 en una unidad
Zip, se debe introducir la orden (para un disco Zip {\em desmontado})
\begin{tscreen}
\# mke2fs /dev/sda4
\end{tscreen}

Con una unidad Zip montada de esta manera, con un sistema de ficheros
ext2, es posible hacer una copia de seguridad de sistema de ficheros
completo con la simple instrucci�n
\begin{tscreen}
\# tar zcvf /zip/local.tar.gz /usr/local
\end{tscreen}

Hay que notar que el hacer copias de seguridad con {\tt tar} es m�s
aconsejable en muchos casos que hacer un simple {\tt cp -a}, porque
{\tt tar} conserva las fechas originales de modificaci�n de ficheros. 


% !Falta por traducir la siguiente subsecci�n!
% Me encargo de traducirla ya
%
\subsection{Hacer copias de seguridad a dispositivos de cinta} \label{sec-tape-backups}


%Archiving to a streaming tape drive is similar to making a backup to a
%floppy file system, only to a different device driver. Tapes are also
%formatted and handled differently that floppy diskettes. Some
%representative tape device drivers for {\linux} systems are listed in
%Table~\ref{table-tape-devices}.

Archivar a un dispositivo de cinta es similiar a hacer una copia de seguridad a un sistema de ficheros en disquetes,
solo que a un dispositivo diferente. La cintas se formatean y se manipulan de manera distinta que los  disquetes.
Algunos controladores de dispositivo para {\linux} se muestran en la tabla~\ref{table-tape-devices}
\begin{table}[ht]\begin{center}
\small\begin{tabular}{ll}
\hline
Controlador de dispositivo de cinta              & Formato \\
\hline
{\tt /dev/rft0}                 & Cinta QIC-117, rebobinar al cierre. \\
{\tt /dev/nrft0}                & Cinta QIC-117, no rebibonar al cierre.\\
{\tt /dev/tpqic11}              & Cinta QIC-11, rebobinar al cierre. \\
{\tt /dev/ntpqic11}             & Cinta QIC-11, no rebobinar al cierre. \\
{\tt /dev/qft0}                 & Unidad de cinta ``floppy'', rebobinar al cerrar. \\%rewind on close. \\
{\tt /dev/nqft0}                & Unidad de cinta ``floppy'', no rebobinar al cerrar. \\%Floppy tape drive, no rewind on close.\\
\end{tabular}\normalsize\rm
\caption{Controladores de dispositivos de cinta.}%Tape device drivers.}
\label{table-tape-devices}
\end{center}\end{table}
%Floppy tape drives use the floppy drive controller interface and are
%controlled by the ftape device driver, which is covered below.
%Installation of the ftape device driver module is described on
%page~\pageref{ftape-module}. SCSI tape devices are listed in
Las unidades de cinta ``floppy'' utilizan el interfaz del controlador de dispositivo de la unidad de disquete y se
controlan por el controlador de dispositivo ftape, del que se habla m�s abajo.
La instalaci�n del m�dulo del controlador de dispositivo ftape se describe en la p�gina~\pageref{ftape-module}. 
Los dispositivos de cinta SCSI se muestran en la tabla~\ref{table-scsi-devices}.

%To archive the {\tt /etc} directory a tape device with {\tt tar}, use 
%the command

Para archivar el directorio {\tt /etc} a una cinta mediante {\tt tar}, 
se usar� la orden:
\begin{tscreen}
\# tar cvf /dev/qft0 /etc
\end{tscreen}

%Similarly, to extract the files from the tape, use the commands
Similarmente para extraer los ficheros desde la cinta, se utiliza la orden:
\begin{tscreen}
\# cd / \\
\# tar xvf /dev/qft0
\end{tscreen}

%These tapes, like diskettes, must be formatted before they can be used. The
%ftape driver can format tapes under {\linux}. To format a QIC-40 format
%tape, use the command
Estas cintas, como los disquetes, deben ser formateados antes de que puedan usarse.
El controlador ftape puede formatear cintas en GNU/{\linux}. Para formatear una cinta QIC-40, se
utilizar� la orden
\begin{tscreen}
\# ftformat --format-parameter qic40-205ft --mode-auto --omit-erase --discard-header
\end{tscreen}
%Other tape drives have their own formatting software. Check the
%hardware documentation for the tape drive or the documentation of the
%{\linux} device driver associated with it.

Otros dispositivos de cinta tienen su propio software para darles  formato.
Revisa la documentaci�n del hardware de la unidad de cinta o  la documentaci�n
del controlador de dispositivo asociado a �l.

%Before tapes can be removed from the drive, they must be rewound and
%the I/O buffers written to the tape. This is analogous to unmounting
%a floppy before ejecting it, because the tape driver also caches data
%in memory. The standard UNIX command to control tape drive operations
%is {\tt mt}. Your system may not provide this command, depending on
%whether it has tape drive facilities. The ftape driver has a similar
%command, {\tt ftmt}, which is used to control tape operations.

Antes de que se puedan extraer las cintas de la unidad, se debe escribir
los b�ffers de E/S y rebobinar la cinta. Esto es an�logo a desmontar un
disquete antes de extraerlo, porque el controlador de dispositivo tambi�n 
cachea datos en la memoria. La orden estandar de Unix para controlar las
operaciones de la unidad es {\tt mt}. El sistema puede no proporcionar
esta orden, dependiendo de si tiene soporte para unidades de cinta. El controlador ftape
tiene una orden similar, {\tt ftmt}, que se usa para controlar las operaciones de cinta.
 
%To rewind a tape before removing it, use the command
Para rebobinar una cinta antes de retirarla, se usa la orden
\begin{tscreen}
\# ftmt -f /dev/qft0 rewoffl
\end{tscreen}
%Of course, substitute the correct tape device driver for your system.
Desde luego, sustituya el controlador de dispositivo correcto para su sistema.

%It is also a good idea to retension a tape after writing to it,
%because magnetic tapes are susceptible to stretch. The command
Tambi�n es una buena idea hacer una retensi�n de la cinta despu�s de escribir
en ella, porque las cintas magn�ticas son susceptibles de arrugarse. La orden
\begin{tscreen}
\# ftmt -f /dev/qft0 retension
\end{tscreen}
realizar� la operaci�n.

%To obtain the status of the tape device, with a formatted tape
%in the drive, give the command
Para obtener el estado de un dispositivo de cinta, con una cinta
dentro utilice la orden
\begin{tscreen}
\# ftmt -f /dev/qft0 status
\end{tscreen}
%
%
% Hasta aqu�

\subsection{Utilizando disquetes como sistemas de ficheros}\label{sec-floppy}
\index{sistemas de ficheros!en disquete}
\index{disquetes!sistemas de ficheros en}
\index{mke2fs@{\tt mke2fs}!para disquete}
Puede crearse un sistema de ficheros en un disquete igual que lo har�a en
una partici�n de un disco duro. Por ejemplo,
\begin{tscreen}
\# {\em mke2fs /dev/fd0 1440}
\end{tscreen}
crea un sistema de ficheros en el disquete en {\tt /dev/fd0}. El tama�o del
sistema de ficheros debe corresponder al tama�o del disquete. Los disquetes 
de alta densidad de 3.5" tienen un tama�o de 1.44 megabytes, o 1440 bloques.
Los disquetes de alta densidad de 5.25" tienen 1200 bloques.

\index{mount@{\tt mount}!montando disquetes con}
Para poder acceder a un disquete, se debe montar {\bf mount} el sistema de 
ficheros que contiene. La instrucci�n
\begin{tscreen}
\# {\em mount -t ext2 /dev/fd0 /mnt}
\end{tscreen}
montar� el disquete en {\tt /dev/fd0} en el directorio {\tt /mnt}.
Ahora todos los ficheros del disquete aparecer�n bajo {\tt /mnt} en su 
unidad. ``{\tt -t ext2}'' especifica el tipo de sistema de ficheros como 
ext2fs. Si crea otro tipo de sistema de ficheros en el disquete, 
necesitar� especific�rselo a la instrucci�n {\tt mount}.

\index{punto de montaje!definici�n}
El ``punto de montaje'' (el directorio donde est� montando el sistema de 
ficheros) debe existir en el momento de utilizar la orden {\tt mount}. Si
no existiese, se debe crear con la instrucci�n {\tt mkdir}.

Para m�s informaci�n sobre sistemas de ficheros, montaje y puntos de 
montaje, ver secci�n~\ref{sec-manage-fs}.

\index{disquetes!desmontando}
\index{umount@{\tt umount}!desmontando disquetes con}
\blackdiamond Tenga en cuenta que cualquier entrada/salida al disquete se 
gestiona con buffers igual que si fuese de disco duro.
Si cambia datos en el disquete, puede que no vea encenderse la luz de la 
unidad hasta que el n�cleo decida vaciar sus buffers. Es importante que no
quite un disquete antes de haberlo desmontado; esto puede hacerse con
la instrucci�n
\begin{tscreen}
\# {\em umount /dev/fd0}
\end{tscreen}
No cambie los disquetes como se hace en un sistema MS-DOS; siempre que 
cambie disquetes, desmonte {\tt umount} el primero y monte {\tt mount} el
siguiente.










% Linux Installation and Getting Started    -*- TeX -*-
% upgrade.tex
% Copyright (c) 1992, 1993 by Matt Welsh <mdw@sunsite.unc.edu>
%
% This file is freely redistributable, but you must preserve this copyright 
% notice on all copies, and it must be distributed only as part of "Linux 
% Installation and Getting Started". This file's use is covered by the 
% copyright for the entire document, in the file "copyright.tex".
%
% Copyright (c) 1998 by Specialized Systems Consultants Inc. 
% <ligs@ssc.com>
%
% Traducci�n al espa�ol:
% Sebasti�n Gurin, <Cancerbero>, anon@adinet.com.uy
% Traducido el 28/02/01  
% Revisado el 7/7/2002 por Francisco Javier Fernandez <serrador@arrakis.es>
% La revisi�n exige la contrastaci�n intensiva con los originales para resolver ciertos pasajes dudosos o de dif�cil soluci�n.

\section{Actualizando e instalando software nuevo}
\markboth{Administraci�n del sistema}{Actualizando e instalando software nuevo}
\label{sec-sysadm-upgrade}

\index{software!actualizar|(}
\index{software!instalar|(}
Otra de las responsabilidades  del administrador del sistema, es la actualizaci�n e instalaci�n de nuevo software. 

El desarrollo del sistema {\linux} es r�pido. Cada pocas semanas aparecen versiones nuevas del
n�cleo, y los dem�s programas se actualizan casi tan a menudo. Por esto, los usuarios nuevos de {\linux},
sienten la necesidad de actualizar sus sistemas constantemente, para
mantenerse, as�, al r�pido paso de los cambios. Esto es innecesario y
una p�rdida de tiempo: si estuvieras todo el tiempo siguiendo el ritmo
de los cambios que ocurren en el mundo de GNU/Linux, se gastar�a todo el
tiempo actualizando y nada del tiempo usando el sistema. 

Algunas personas consideran que se deber�a actualizar el sistema,
solamente cuando una nueva distribuci�n es mostrada al p�blico; por
ejemplo, cuando se presenta una nueva versi�n de Slackware. Entonces,
muchos usuarios de {\linux}, a la hora de actualizar sus sistemas,
reinstalan todo el software, usando la distribuci�n Slackware m�s nueva. 

La mejor manera de actualizar el sistema, depende del tipo de
distribuci�n que se posea. Debian\tm, S.u.S.E.\tm, Caldera\tm y Red Hat
Linux\tm tienen, todos, gestores inteligentes de paquetes de software,
los cuales permiten realizar las actualizaciones mucho m�s f�cilmente,
instalando paquetes nuevos. Por ejemplo, el compilador de C, {\tt gcc},
viene en un paquete binario, pre-compilado. Cuando se instala,
todos los ficheros de la versi�n antigua se sobreecriben o se eliminan. 

Como casi siempre pasa, actualizar insensatamente para "mantenerse a
la moda", no es importante en absoluto. �Esto no es MS-DOS o Microsoft
Windows!. No existe ninguna raz�n importante, para usar la versi�n m�s
reciente de todo el software. Ahora bien, si se siente que se quieren o 
necesitan caracter�sticas que una versi�n nueva ofrece, entonces hay 
que actualizar. Si no, no actualice. En otras palabras, actualizar s�lo
lo que se deba, cuando se deba. No actualizar s�lo por actualizar. 
Esto consume mucho tiempo y esfuerzo. 

\subsection{Actualizando el n�cleo}
\index{n�cleo!actualizar}

Actualizar el n�cleo es s�lo un asunto de obtener las fuentes del n�cleo y
compilarlas. Esto es generalmente un proceso sin dificultad; 
sin embargo, uno puede tener problemas si trata de actualizar a
un n�cleo en desarrollo, o actualizarlo a una nueva versi�n. 
La versi�n de un n�cleo tiene dos partes: la
versi�n del n�cleo, y el nivel del parche. Cuando esto fue escrito, la
�ltima versi�n estable del n�cleo era la {\tt 2.0.33}. Los n�meros
{\tt 2.0} representan la versi�n del n�cleo, y los n�meros {\tt 33} es
el nivel del parche. Las versiones del n�cleo se�aladas con n�meros
impares, por ejemplo {\tt 2.1} son n�cleos en desarrollo. Mant�ngase
lejos de este tipo de n�cleos, �a menos que quiera vivir peligrosamente!
Como regla general, uno deber�a ser capaz
de actualizar su n�cleo f�cilmente a otro nivel de parche; sin
embargo, actualizar a una nueva versi�n requiere, a su vez, la
actualizaci�n de las utilidades del sistema  que interact�an �ntimamente con el n�cleo. 

\index{n�cleo!fuentes del}
El c�digo fuente del n�cleo Linux puede ser obtenido de cualquiera
de los servidores FTP de {\linux}, (ver la p�gina~\pageref{app-ftp} para una lista de ellos).
En {\tt sunsite.unc.edu}, por ejemplo, las fuentes del n�cleo se encuentran en 
{\tt /pub/Linux/kernel}, organizado
en subdirectorios por n�mero de versi�n. 

El c�digo fuente del n�cleo es publicado en un fichero tar comprimido
con gzip. Por ejemplo, el fichero que contiene el c�digo fuente del
n�cleo 2.0.33 es {\tt linux-2.0.33.tar.gz}

Las fuentes del n�cleo deber�n descomprimirse y desempaquetarse en
el directorio {\tt /usr/src}, creando el directorio {\tt
  /usr/src/linux}. Es una costumbre com�n que, {\tt /usr/src/linux} sea
un enlace blando a otro directorio que contenga el n�mero de versi�n
del n�cleo, tal como {\tt /usr/src/GNU/Linux-2.0.33}. De esta manera, se
podr�n instalar nuevos c�digos fuente y verificar su correcto
funcionamiento, antes de eliminar los fuentes del n�cleo antiguo. Las
ordenes para crear el enlace al directorio donde se aloja el c�digo
fuente del n�cleo son:

\begin{tscreen}
\# cd /usr/src \\
\# mkdir linux-2.0.33 \\
\# rm -r linux \\
\# ln -s linux-2.0.33 linux \\
\# tar xzf linux-2.0.33.tar.gz
\end{tscreen}

Cuando se actualiza a un nuevo nivel de parche de la misma versi�n del
n�cleo, un fichero de nivel de parche puede resultar en un ahorro de 
tiempo en la transferencia de ficheros, puesto que las fuentes del 
n�cleo tienen un tama�o alrededor de los 7 Mb. tras ser comprimidas 
con {\tt gzip}. Para actualizar del
n�cleo 2.0.31 al n�cleo 2.0.33, habr�a que descargar los parches
{\tt patch-2.0.32.gz} y {\tt patch-2.0.33.gz}, los cuales pueden encontrarse 
en el mismo servidor FTP de las fuentes del n�cleo. Tras haber
ubicado los parches en el directorio {\tt /usr/src/}, se deben
aplicar en las fuentes del n�cleo, uno tras otro para actualizar el
c�digo fuente. Una forma de hacer esto ser�a

\begin{tscreen}
\# cd /usr/src \\
\# gzip -cd patch-2.0.32.gz $\mid$ patch -p0 \\
\# gzip -cd patch-2.0.33.gz $\mid$ patch -p0
\end{tscreen}

Despu�s de desempaquetar los fuentes y de aplicar los parches,
necesitar� asegurarse de que existan tres enlaces
simb�licos en {\tt /usr/include}, los cu�les son justo los que
necesita el n�cleo de su distribuci�n. Para crear dichos enlaces,
se podr�n usar las �rdenes
\begin{tscreen}
\# cd /usr/include \\
\# rm -rf asm linux scsi \\
\# ln -s /usr/src/linux/include/asm-i386 asm \\
\# ln -s /usr/src/linux/include/linux linux \\
\# ln -s /usr/src/linux/include/scsi scsi
\end{tscreen}

Despu�s de que haya creado los enlaces, no existe ninguna raz�n para
que deba crearlos nuevamente la pr�xima vez que se instale el siguiente
parche, o una nueva versi�n del n�cleo. (Para m�s informaci�n sobre
enlaces simb�licos: ver secci�n~\ref{sec-manage-links})


\index{n�cleo!compilaci�n}
A fin de compilar el n�cleo, habr� que tener el compilador de C {\tt
  gcc} instalado en su sistema. Para compilar la versi�n 2.0 del
n�cleo, se requiere el {\tt gcc}, versi�n 2.6.3 o m�s reciente. 

Primero cambie de directorio a  {\tt /usr/src/linux}. La orden
{\tt make config} ir� preguntando por un n�mero de opciones de
configuraci�n. �ste es el paso d�nde se selecciona el hardware al que
el n�cleo podr� dar soporte. La equivocaci�n m�s grande a evitar, es
no incluir soporte parar el/los controlador/es del/los disco/s
duro/s. Sin el correcto soporte para el disco duro en el n�cleo, el
sistema ni siquiera se iniciar�. Si en el proceso, se siente inseguro
sobre lo que significa una de las opciones del n�cleo, est� disponible
una corta descripci�n pulsando \key{?} y \key{Enter}

Lo siguiente ser� ejecutar la orden {\tt make dep} para actualizar
todas las dependencias del c�digo fuente. �ste es, tambi�n, un paso
importante. {\tt make clean} eliminar� los ficheros binarios antiguos
del �rbol del n�cleo. 

\index{n�cleo!compilando una imagen comprimida}

La instrucci�n {\tt make zImage} compila el n�cleo y lo escribe en el
fichero {\tt /usr/src/linux/arch/i386/boot/zImage}. Los n�cleos de
Linux en los sistemas Intel, est�n siempre comprimidos. Algunas veces,
el n�cleo que se desea compilar es demasiado grande para ser comprimido
por el sistema de compresi�n que usa {\tt make zImage}. Un n�cleo
excesivamente grande para dicho sistema de compresi�n, retornar�
del proceso de compilaci�n del n�cleo con el siguiente mensaje de
error: {\tt Kernel Image Too Large}. Si esto llegara a pasar, se
debe tratar con la orden {\tt make bzImage}. Esta orden usa un
sistema de compresi�n que soporta los n�cleos grandes. El n�cleo ser�
escrito en {\tt /usr/src/linux/arch/i386/boot/bzImage}.

Una vez que se tenga el n�cleo compilado, se podr� copiar a un disquete
de arranque, (por ejemplo, con la orden ``{\tt cp zImage
  /dev/fd0}''), o se podr� instalar la imagen, y as�, LILO iniciar� el
sistema desde el disco duro. Para m�s informaci�n, ver la p�gina~\pageref{sec-lilo}. 

\subsection{Agregando un controlador de dispositivo al n�cleo}
\label{kernel-ppa-driver}
\index{n�cleo!arraglar controlador de dispositivo}

La p�gina~\pageref{sec-zip-backup} describe c�mo usar una unidad Zip
Iomega, para efectuar copias de seguridad. El soporte para este tipo
de unidades, como para muchos otros dispositivos, no son generalmente
compilados en los n�cleos comunes y corrientes de las distribuciones {\linux}
---la variedad de dispositivos es simplemente demasiado extensa como
para poder respaldarlos a todos en un s�lo n�cleo utilizable. No
obstante, el c�digo fuente para el dispositivo de la unidad Zip en
puerto paralelo, est� incluido como una parte de c�digo fuente del
n�cleo de la distribuci�n. Esta secci�n describe c�mo agregar el
soporte para una unidad de puerto paralelo Iomega Zip, y c�mo hacer
para que conviva con una impresora conectada a otro puerto paralelo.

Para esto, usted deber� tener instalado, y haber construido
exitosamente un n�cleo, como el descrito en la secci�n anterior. 

El poder elegir un dispositivo unidad Zip {\tt ppa}, como una de las
opciones del n�cleo, requiere que se conteste {\tt Y} a las
respuestas apropiadas, durante el proceso {\tt make config}, o sea,
cuando se determina la configuraci�n del n�cleo a construir. En
particular, el dispositivo {\tt ppa}, requiere que se conteste
``{\tt Y}'' a tres opciones:

\begin{tscreen}
SCSI support? [Y/n/m] Y \\
SCSI disk support? [Y/n/m] Y \\
IOMEGA Parallel Port Zip Drive SCSI support? [Y/n/m] Y
\end{tscreen}

Despu�s de haber ejecutado exitosamente {\tt make config}, con todas
las opciones que quiere incluir en su n�cleo, ejecutar {\tt make dep},
{\tt make clean}, y {\tt make zImage}, para construirlo. Adem�s, hay que
decirle al n�cleo, de qu� manera instalar el controlador. Esto se
efect�a a trav�s de una l�nea de ordenes al LILO. Como se ha
descrito en la secci�n~\ref{sec-lilo}, el archivo de configuraci�n
del LILO {\tt /etc/lilo.conf} tiene ``estrofas'', una para cada
sistema operativo que domina y tambi�n directivas para ofrecer al
usuario estas opciones, en el momento de arrancar el sistema. 

Una de las directivas que LILO acepta es ``{\tt append=}'', la cual
permite a�adir informaci�n requerida por varios controladores a la
l�nea de ordenes. En este caso, el controlador de la unidad Iomega
Zip {\tt ppa}, requiere de una interrupci�n y una direcci�n del puerto
de entrada/salida, sin uso. Esto es exactamente an�logo a especificar
dispositivos de impresoras separados, como {\tt LPT1:} y {\tt LPT2:}
en MS-DOS. 

Por ejemplo, si la impresora usa la direcci�n del puerto hexadecimal
(en base 16), {\tt 0x378} (ver el manual de instalaci�n de la tarjeta
del puerto paralelo, si no se sabe cu�l es la direcci�n), y est�
sondeada\NT{``polled'' en el original,}, (esto
es, no requiere de una l�nea IRQ, una configuraci�n com�n de {\linux},
se deber�a colocar la siguiente l�nea, en el archivo {\tt
  /etc/lilo.conf} del sistema:

\begin{tscreen}
append="lp=0x378,0"
\end{tscreen}

Es digno de observar que Linux reconoce autom�ticamente un puerto {\tt
  /dev/lp} al arrancar el sistema, pero al especificar algunas otras
configuraciones para los puertos, las instrucci�nes al inicio del
  sistema, son requeridas. 

El ``{\tt 0}'' que se encuentra despu�s de la direcci�n del puerto, le
dice al n�cleo que {\em no} use una l�nea IRQ (pedido de
interrupci�n), para la impresora. Esto es generalmente aceptable, ya
que las impresoras son mucho m�s lentas que la CPU, y tan as� que un
m�todo m�s lento de acceso a los dispositivos E/S, conocido como {\bf
  sondeo}\footnote{``polling'' en el original, (Nota del T.)}, en el
cual el n�cleo comprueba, peri�dicamente, el estado de la impresora,
todav�a permite al computador supervisar este dispositivo.

Sin embargo, los dispositivos que operan a mayores velocidades, como
las l�neas en serie y los discos, requieren, cada uno, de una l�nea
{\bf IRQ,} o {\bf petici�n de interrupci�n (Interrupt ReQuest)}. Esta,
es una se�al del hardware, enviada por el dispositivo hacia el
microprocesador, cada vez que dicho dispositivo requiere la atenci�n
del procesador; por ejemplo: si el dispositivo tiene datos esperando a
ser despachados por el procesador. El procesador, interrumpe lo que
est� haciendo en ese momento para obedecer al pedido de interrupci�n
del dispositivo. El dispositivo unidad Zip {\bf ppa}, exige una
l�nea de interrupci�n libre, la cual debe corresponder con la de la tarjeta de la
impresora a la cual se conecta la unidad Zip. En el momento en que
esto se escrib�a, el controlador del dispositivo {\bf ppa} para GNU/Linux,
no soportaba ``sucesiones'' de puertos paralelos, y se deb�an emplear
puertos paralelos separados para usar el dispositivo Zip {\bf ppa} y
cada impresora. 

Para saber qu� interrupciones est�n actualmente utilizadas por su
sistema, la orden

\begin{tscreen}
\# cat /proc/interrupt
\end{tscreen}

muestra una lista de dispositivos y las l�neas IRQ que usan. Sin
embargo, tambi�n se deber� tener cuidado de no usar ninguna interrupci�n
de ning�n puerto en serie configurada autom�ticamente; la cual puede
no estar listada en el archivo {\tt /proc/interrupt}. El Proyecto de
Documentaci�n de Linux, serial HOWTO, el cual est� disponible en los
recursos listados en el Ap�ndice~\ref{app-sources-num}, describe
detalladamente, la configuraci�n de los puertos en serie. 

\blackdiamond Uno tambi�n deber�a realizar un chequeo de la
configuraci�n de la interfaz de varias tarjetas, abriendo la carcasa
de su m�quina y verificando visualmente la configuraci�n de los
puentes si es necesario, para asegurarse, as�, de no estar asociando
una l�nea IRQ usada por otro dispositivo. La lucha de m�ltiples
dispositivos por una l�nea de interrupci�n es quiz� el problema m�s
sencillo y com�n que causa que los sistemas GNU/Linux no funcionen. 

Un t�pico archivo {\tt /proc/interrupt} suele ser como
\begin{tscreen}
 0:    6091646   timer \\
 1:      40691   keyboard \\
 2:          0   cascade \\
 4:     284686 + serial \\
13:          1   math error \\
14:     192560 + ide0 \\
\end{tscreen}

Aqu�, la primera columna nos es de inter�s. Estos son los n�meros de
las l�neas IRQ usadas por el sistema. Para el controlador {\tt ppa},
necesitamos escoger una l�nea que {\tt no} est� listada. La l�nea IRQ
7 es, a menudo, una buena elecci�n ya que rara vez es usada en las
configuraciones predeterminadas del sistema. Tambi�n necesitamos
especificar la direcci�n del puerto que usar� el dispositivo {\tt
  ppa}. Esta direcci�n necesita estar configurada f�sicamente con la
interfaz de la tarjeta. A los puertos paralelos de E/S se les deben
asignar direcciones espec�ficas, por lo que usted tendr� que leer la
documentaci�n de la tarjeta de su puerto paralelo. En este ejemplo
usaremos, para el puerto de E/S, la direcci�n {\tt 0x278}, la cual
corresponde al puerto {\tt LPT2:} de la impresora, en MS-DOS. Para
a�adir la l�nea IRQ y la direcci�n del puerto en una l�nea de ordenes
cuando arranca el sistema, necesitamos agregar la siguiente
expresi�n a la ``estrofa'' apropiada del archivo {\tt /etc/lilo.conf}:

\begin{tscreen}
append="lp=0x378,0 ppa=0x278,7"
\end{tscreen}

Estas expresiones son a�adidas a los par�metros de arranque del
n�cleo, cuando se inicia el sistema. Aseguran que cualquier impresora
conectada al sistema no interfiera con el funcionamiento de la unidad
Zip. Por supuesto, si el sistema no tiene ninguna impresora instalada
la directiva ``{\tt lp=}'' puede y deber�a ser omitida. 

Despu�s de que haya instalado el n�cleo, como se describi� en la
secci�n~\ref{sec-lilo}, y antes de reiniciar el sistema, hay que
asegurarse de ejecutar la instrucci�n
\begin{tscreen}
\# /sbin/lilo
\end{tscreen}
para as�, instalar la nueva configuraci�n de LILO en el sector de arranque del disco duro. 

\subsection{Instalando controladores en m�dulos}
\label{ftape-module}

La p�gina~\pageref{sec-tape-backups} describe c�mo realizar copias de
seguridad en un accionador de cinta magn�tica. Linux da soporte a una
gran variedad de accionadores de cinta con interfaces IDE, SCSI y
algunas interfaces del propietario. Otro tipo corriente de
accionadores de cinta son aquellos que se conectan directamente al
controlador de la disquetera. Linux suministra el controlador para la
unidad ftape como un m�dulo. 

Cuando esto se estaba escribiendo, la versi�n m�s reciente de ftape
era la 3.04d. Se puede obtener el controlador en el servidor FTP
{\tt sunsite.unc.edu}, (para m�s informaci�n, ver el
Ap�ndice~\ref{app-ftp}). El archivo ftape se encuentra en el
directorio {\tt /pub/Linux/n�cleo/tapes}. Hay que asegurarse de
procurarse la versi�n m�s reciente, la cual, cuando este documento se estaba
editando, era {\tt ftape-3.04d.tar.gz}.

Despu�s de desempaquetar el archivo ftape en el directorio {\tt
  /usr/src}, al escribir {\tt make install} en el directorio padre de
ftape, se compilar�n el m�dulo del controlador ftape y sus utilidades,
si son necesarias, y luego se instalar�n. Si experimenta problemas de
compatibilidad entre los ficheros de la distribuci�n ejecutable ftape
y su n�cleo o las bibliotecas de su sistema, ejecute las �rdenes {\tt
  make~clean} y {\tt make~install}, y se asegurar�, de que los m�dulos
sean compilados en su sistema. 

\blackdiamond Para usar esta versi�n del controlador ftape, usted
deber� tener el soporte para m�dulos en el n�cleo, como tambi�n
soporte para el demonio {\tt n�cleod}. Sin embargo, {\em no} deber�
incluir el c�digo interno del n�cleo para ftape como una opci�n del
n�cleo, ya que las versi�nes m�s recientes del m�dulo ftape remplazan
completamente este c�digo. 

{\tt make install}, tambi�n instalar� el controlador del dispositivo
en el directorio correcto. En un sistema GNU/Linux est�ndar, los m�dulos
se encuentran en el directorio
\begin{tscreen}
/lib/modules/\cparam{n�cleo-version}
\end{tscreen}
Si la versi�n de tu n�cleo es 2.0.30, los m�dulos de su sistema se
encuentran en el directorio {\tt /lib/modules/2.0.30}. {\tt make
  install} tambi�n asegura que estos m�dulos puedan ser localizados en
cualquier momento, agregando las expresi�nes apropiadas en el archivo
{\tt modules.dep}, que se encuentra en el directorio ra�z de los
ficheros m�dulo, en este caso, {\tt /lib/modules/2.0.30}. La
instalaci�n de ftape, a�ade los siguientes m�dulos a su sistema,
(usando, en este ejemplo, la versi�n 2.0.30 del n�cleo):
\begin{tscreen}
/lib/modules/2.0.30/misc/ftape.o \\
/lib/modules/2.0.30/misc/zft-compressor.o \\
/lib/modules/2.0.30/misc/zftape.o
\end{tscreen}
Tambi�n se necesitan agregar las instrucciones para cargar los m�dulos
al archivo de la configuraci�n de m�dulos de su sistema. En muchos
sistemas, este es el archivo {\tt /etc/conf.modules}. Para cargar
autom�ticamente los m�dulos ftape a pedido, agregue las siguientes
l�neas en el archivo {\tt /etc/conf.modules}:
\begin{tscreen}
alias char-major-27 zftape \\
pre-install ftape /sbin/swapout 5
\end{tscreen}

La primera declaraci�n carga los m�dulos relacionados con ftape cuando
un dispositivo con el n�mero principal 27\NT{``majornumber 27''
  en el original} (el dispositivo ftape), es accedido
por el n�cleo. Debido a que el m�dulo de soporte para zftape, (el cual
provee compresi�n autom�tica para los dispositivos ftape) requiere el
soporte de los dem�s m�dulos ftape, todos ellos son cargados en el
momento en que el n�cleo efect�a la demanda. La segunda l�nea
especifica los par�metros que ser�n dados a los m�dulos al iniciarse
el sistema. En este caso, la utilidad {\tt /sbin/swapout}, la cual
viene incorporada en el paquete de software ftape, asegura que hay
suficiente memoria DMA, para el correcto funcionamiento del
controlador ftape. 

Para tener acceso al dispositivo ftape driver se deber� primero, colocar
una cinta formateada en la unidad. Las instrucci�nes para formatear
cintas y operar correctamente la unidad de cintas son dadas en la secci�n~\ref{sec-tape-backups}.


\subsection{Actualizando las bibliotecas compartidas}\label{sec-upgrade-libs}
\index{bibliotecas!actualizaci�n}

Como se mencion� antes, la mayor parte del software del sistema est�
compilado para que utilice las bibliotecas compartidas, las cuales
contienen subrutinas comunes compartidas entre distintos programas.
Si aparece el mensaje

\begin{tscreen}
Incompatible library version
\end{tscreen}

cuando se intenta ejecutar un programa, entonces necesita actualizar a
la versi�n de las bibliotecas que el programa requiere. Las bibliotecas
son compatible-ascendentes; esto es, un programa compilado para
utilizar una versi�n antigua de las bibliotecas, deber�a ser capaz de
trabajar con la nueva versi�n de las bibliotecas instalada. Sin embargo,
esto no se da en sentido contrario. 

La ultima versi�n de las bibliotecas se puede encontrar en los
servidores FTP de GNU/Linux. En {\tt sunsite.unc.edu}, est�n disponibles
en {\tt /pub/GNU/Linux/GCC}. Los ficheros a descargar deber�an explicar
qu� ficheros se necesita obtener y como instalarlos. Deber�a ser
capaz de coger r�pidamente los ficheros {\tt image-{\em
    versi�n}.tar.gz} e {\tt inc-{\em versi�n}.tar.gz}, donde {\em
  versi�n} es la versi�n de las bibliotecas a instalar, por ejemplo {\tt
  4.4.1}. Estos son ficheros tar, comprimidos con {\tt gzip}. El
fichero {\tt imagen} contiene las im�genes de las bibliotecas a instalar
en {\tt /lib} y {\tt /usr/lib}. El fichero {\tt inc} contiene los
ficheros de inclusi�n, a instalar en {\tt /usr/include}.

El fichero {\tt release-}{\em versi�n}{\tt .tar.gz} deber�a explicar
el procedimiento de instalaci�n en detalle (las instrucci�nes exactas
cambian seg�n la versi�n). Generalmente se necesitar� instalar los
ficheros de bibliotecas {\tt .a} y {\tt .sa} en {\tt /usr/lib}. Estas
son las utilizadas al compilar.

Adem�s, los ficheros imagen de las bibliotecas compartidas {\tt
  lib.so.}{\em versi�n} se instalan en {\tt /lib}. Estas son las
im�genes de las bibliotecas compartidas que son cargadas en tiempo de
ejecuci�n por los programas que las utilizan. Cada biblioteca tiene un
enlace simb�lico utilizando el numero de versi�n
principal \NT{``major version number'' en el original.} de la biblioteca en {\tt /lib}.

La versi�n 4.4.1 de la biblioteca {\tt libc} tiene un n�mero de
versi�n principal {\tt 4}. El archivo que contiene a la
biblioteca es el {\tt libc.so.4.4.1}. Existe un enlace simb�lico con
el nombre {\tt libc.so.4} en {\tt/lib} apuntando a este fichero. Es
por esto que se debe cambiar estos enlaces simb�licos cuando se
actualizan las bibliotecas. Por ejemplo, cuando se actualiza de {\tt
  libc.so.4.4} a {\tt libc.so.4.4.1}, se debe cambiar el enlace simb�lico
de tal modo que apunte a la nueva versi�n. 

\blackdiamond Se deber� cambiar el enlace simb�lico de un solo paso, como
se describir� m�s abajo. Si se borra el enlace simb�lico {\tt libc.so.4},
los programas que dependen de �l (incluyendo utilidades b�sicas como
{\tt ls} y {\tt cat}), dejar�n de funcionar. Por lo tanto, es
recomendable usar la siguiente orden para actualizar el enlace
simb�lico {\tt libc.so.4} y hacer que apunte al archivo {\tt libc.so.4.4.1}:

\begin{tscreen}
\# ln -sf /lib/libc.so.4.4.1 /lib/libc.so.4
\end{tscreen}

Tambi�n se necesitar� cambiar el enlace simb�lico {\tt libm.so.}{\em
  versi�n} de la misma manera. Si se est� actualizando a una
versi�n de biblioteca diferente, substituir apropiadamente los nombres
de arriba. Las notas que vienen con el paquete de la biblioteca,
deber�an explicar los detalles. (Mirar en la
p�gina~\pageref{sec-manage-links} para m�s informaci�n sobre los
enlaces simb�licos.) 

\subsection{Actualizando el {\tt gcc}}
\label{sec-upgrade-gcc}
\index{gcc@{\tt gcc}!actualizaci�n}

El compilador de C y C++ {\tt gcc}, es usado para compilar el software
de su sistema, siendo lo m�s importante, el n�cleo. La �ltima versi�n
del {\tt gcc} se puede obtener en los servidores de GNU/Linux FTP. En {\tt
  sunsite.unc.edu}, se encuentra en el directorio {\tt
  /pub/GNU/Linux/GCC} (junto con las bibliotecas). Deber�a de haber un
{\tt fichero de entrega}\footnote{``release file'' en el
  original. Nota del T. } en la distribuci�n del {\tt gcc}, el cual
explique qu� ficheros necesita obtener y c�mo instalarlos. 

La mayor�a de las distribuci�nes de GNU/Linux tienen versiones para
actualizar el {\tt gcc} que trabajan con su propia gesti�n de paquetes
de software. En general, estos paquetes son mucho m�s f�ciles de
instalar que las distribuci�nes ``gen�ricas''. 

\subsection{Actualizando otro software}

Actualizar otro software suele ser simplemente materia de obtener los
ficheros apropiados e instalarlos. La mayor parte del software de
GNU/Linux se distribuye bajo la forma de ficheros tar comprimidos que
pueden incluir los fuentes, los binarios, o ambos. Si los binarios no
est�n incluidos en ese paquete, puede que sea necesario que usted los
compile. Esto significa que, por lo menos, tenga que teclear {\tt
  make} dentro del directorio donde se encuentran los ficheros fuente. 

\index{software!d�nde encontrar versiones}
Leer el grupo de noticias de Usenet {\tt comp.os.GNU/Linux.announce} en
busca de anuncios de nuevas versi�nes de software, es la manera m�s
simple para enterarse de la aparici�n de nuevo software. Cada vez que
se busque software en un servidor FTP, obtener el fichero de
�ndice {\tt ls-lR} del servidor FTP y utilizar {\tt grep} para
encontrar los ficheros en cuesti�n, es la forma mas simple de
localizar software. Si se dispone de {\tt archie}, este puede servir de
ayuda; o de otra manera \footnote{Si no se tiene {\tt archie}}, es posible
conectarse v�a telnet a un servidor {\tt archie} como puede ser
archie.rutgers.edu, identificarse como ``{\tt archie}'' y utilizar la
orden "help". Tambi�n se puede encontrar otros recursos en Internet,
los cuales son consagrados espec�ficamente para GNU/Linux. Mirar el
Ap�ndice~\ref{app-info} para obtener informaci�n m�s detallada.

%% One handy source of GNU/Linux software is the Slackware distribution disk
%% images. Each disk contains a number of {\tt .tgz} files which are
%% simply gzipped tar files. Instead of downloading the disks, you can
%% download the desired {\tt .tgz} files from the Slackware directories
%% on the FTP site and install them directly. If you run the Slackware
%% distribution, the {\tt setup} command can be used to automatically
%% load and install a complete series of disks. 

%% Again, it's usually not a good idea to upgrade by reinstalling with
%% the newest version of Slackware, or another distribution. If you reinstall 
%% in this way, you will no doubt wreck your current installation, including 
%% user directories and all of your customized configuration. The best way
%% to upgrade software is piecewise; that is, if there is a program that
%% you use often that has a new version, upgrade it. Otherwise, don't bother.
%% Rule of thumb: If it ain't broke, don't fix it. If your current software
%% works, there's no reason to upgrade. 
\index{software!actualizar|)}
\index{software!instalar|)}

%% Traducci�n terminada el 11/02/01 por Sebasti�n Gurin, Cancerbero <anon@adinet.com.uy> 





% {\linux} Installation and Getting Started    -*- TeX -*-
% misc.tex
% Copyright (c) 1992-1994 by Matt Welsh <mdw@sunsite.unc.edu>
%
% This file is freely redistributable, but you must preserve this copyright 
% notice on all copies, and it must be distributed only as part of "{\linux} 
% Installation and Getting Started". This file's use is covered by the 
% copyright for the entire document, in the file "copyright.tex".
%
% Copyright (c) 1998 by Specialized Systems Consultants Inc. 
% <ligs@ssc.com>
% Revisi�n 1 por Francisco javier Fernandez --sin fallos--
%Gold
\subsection{Otras aplicaciones}

Existen para {\linux} multitud de aplicaciones y utilidades de todo tipo, 
como cabe esperar de un sistema tan variado. El principal objetivo de
{\linux} es la inform�tica personal con UNIX, pero no es �ste el �nico campo
en donde sobresale. El cat�logo de programas cient�ficos y para empresas,
sigue creciendo y los desarrolladores de software comercial hace tiempo que 
han comenzado a contribuir al creciente fondo de aplicaciones para {\linux}.

Hay disponibles en {\linux} varias bases de datos relacionales, por ejemplo
Postgres, Ingres, Mbase, Oracle, IBM DB2, Interbase, Sybase y ADABAS.
Se trata de aplicaciones de bases datos profesionales, con todo tipo 
de caracter�sticas avanzadas, y de arquitectura cliente/servidor, 
semejantes a las que se encuentran en otras plataformas UNIX. 
Existen igualmente otros sistemas comerciales de bases de datos para {\linux}.

Entre las aplicaciones cient�ficas se cuentan FELT (finite element
analysis, an�lisis de elementos finitos); {\tt gnuplot} (representaci�n 
y an�lisis de datos); Octave (un paquete de matem�tica simb�lica similar
a MATLAB); {\tt xspread} (calculadora y hoja de c�lculo); {\tt xfractint}
(una adaptaci�n a X~Window del conocido generador de fractales Fractint)
y {\tt xlispstat} (para estad�sticas). Otras aplicaciones: SPICE 
(dise�o y an�lisis de circuitos) y Khoros (dise�o y visualizaci�n de im�genes
y se�ales digitales). Tambi�n existen aplicaciones comerciales como Maple y 
Matlab.

Se han adaptado a {\linux} muchas m�s aplicaciones, y �ltimamente el n�mero
crece vertiginosamente. Si de ninguna manera encuentra lo que busca,
siempre puede intentar migrar usted mismo la aplicaci�n desde otra
plataforma. Migrar aplicaciones UNIX, del campo que sea, a {\linux} no suele
presentar problemas. El completo entorno de programaci�n UNIX del que dispone
{\linux} sirve de base para cualquier aplicaci�n cient�fica.

{\linux} cuenta tambi�n con un creciente n�mero de juegos. Existen los cl�sicos
juegos de dragones y mazmorras basados en texto, como Nethack y Moria;
luego est�n los {\bf MUDs} (multi-user dungeons, dragones y mazmorras multiusuario) 
que permiten que muchos usuarios interact�en en una aventura basada en texto, como
DikuMUD y TinyMUD; y una pl�yade de juegos para las X~Window, como {\tt xtetris}, 
{\tt netrek}, y {\tt xboard}, la versi�n X11 de {\tt gnuchess}. El popular 
juego de disparar a todo lo que se mueva, Doom, y los arcades que lo continuaron,
Quake, Quake II y Quake III, han sido portados a {\linux}. �ste �ltimo ha salido para 
{\linux} antes que para algunas plataformas mayoritarias, 

Para los mel�manos, {\linux} soporta gran variedad de tarjetas de sonido
y programas asociados, como CDplayer, que convierte su unidad de CD-Rom
en un reproductor de CD's, secuenciadores y editores MIDI, que permiten
componer m�sica para su reproducci�n en un sintetizador u otro instrumento
controlado por MIDI, editores de sonido para sonidos digitalizados, y 
codificadores y reproductores de ficheros en formato MP3\NT{Durante la traducci�n
de este documento, el formato libre Ogg Vorbis ha alcanzado la versi�n 1.0}.

�No encuentra la aplicaci�n que busca? El Mapa de Software {\linux}, que
se describe en el Ap�ndice~\ref{app-sources-num}, enumera los paquetes 
de software que se han escrito o migrado a {\linux}. Otra manera de encontrar
aplicaciones para {\linux} es buscar en los ficheros {\tt INDEX} que se 
encuentran en los sitios FTP con programas para {\linux}, en el caso de que tenga
acceso a Internet.

La mayor parte del software libremente redistribuible disponible para UNIX
compila sin problemas en {\linux}, o al menos con poca dificultad.
Pero si todo lo dem�s fallara, siempre puede programarse usted mismo la
aplicaci�n. Si anda buscado una aplicaci�n comercial, puede existir un 
cl�nico libre. Incluso puede considerar la posibilidad de animar a su
compa��a proveedora de software a que lance una versi�n de su programa
para {\linux}. Muchos individuos y organizaciones han contactado ya con 
compa��as de software y les han pedido que porten sus aplicaciones a {\linux},
con diferentes grados de �xito.


%%% Local Variables: 
%%% mode: plain-tex
%%% TeX-master: t
%%% End: 


% Linux Installation and Getting Started    -*- TeX -*-
% emergency.tex
% Copyright (c) 1993 by Matt Welsh and Lars Wirzenius
%
% This file is freely redistributable, but you must preserve this copyright 
% notice on all copies, and it must be distributed only as part of "Linux 
% Installation and Getting Started". This file's use is covered by
% the copyright for the entire document, in the file "copyright.tex".
%
% Copyright (c) 1998 by Specialized Systems Consultants Inc. 
% <ligs@ssc.com>
%
% Este fichero es de distribuci�n libre, pero debe mantenerse esta 
% informaci�n de Copyright en todas las copias, y debe distribuirse solo como
% parte de "Instalaci�n y Primeros Pasos en Linux". El uso de este fichero esta
% cubierto por el Copyright del documento completo, en el fichero "copyright.tex"
% Copyright (c) 1995 por Gerardo Izquierdo para la versi�n al Castellano
% $Log: emergency.tex,v $
% Revision 1.9  2003/07/19 06:55:42  joseluis.ranz
% Correcciones varias.
%
% Revision 1.8  2002/10/12 19:53:23  montuno
% quitando defectos y comandos
%
% Revision 1.7  2002/07/30 16:23:05  pakojavi2000
% Beta 2.2 Formatos de p�rrafo
%
% Revision 1.6  2002/07/21 00:56:46  pakojavi2000
% Beta2.1
%
% Revision 1.5  2002/07/20 17:41:16  pakojavi2000
% beta2
%
% Revision 1.4  2002/07/12 10:38:33  pakojavi2000
% Corregidos errores de compilaci�n
%
% Revision 1.3  2002/07/07 21:03:13  pakojavi2000
%  Errores de deletreo
%
% Revision 1.2  2001/05/17 12:33:17  amolina
% traducci�n de emergency.tex: Ya acabamos sysadm/ :-)
%
% Revision 0.5.0.1  1996/02/10 23:45:12  rcamus
% Primera beta publica
%
%

%
% Versi�n para lipp 2.0 por Alberto Molina. Comentarios a:
%            alberto@nucle.us.es 
%

\section{Qu� hacer en caso de emergencia}

\index{emergencias!recuperaci�n de|(}
\index{desastres!recuperaci�n de|(}
En algunas ocasiones, el administrador de sistemas se encuentra con el 
problema de recuperarse de un desastre completo, como puede ser el 
olvidarse la palabra clave del usuario root, o el enfrentarse 
con sistemas de ficheros da�ados. El mejor 
consejo es, {\em obrar sin p�nico}. Todo el mundo comete errores 
est�pidos, �sta es la mejor forma de aprender sobre 
administraci�n de sistemas: la forma dif�cil.

Linux no es una versi�n inestable de UNIX. De hecho, he tenido menos problemas
con ``cuelgues'' de sistemas Linux que con versiones comerciales de UNIX en 
muchas plataformas. Linux tambi�n se beneficia de un fuerte complemento de 
asistentes que pueden ayudar a salir del agujero.

El primer paso al investigar cualquier problema es intentar arreglarlo 
uno mismo. Hay que echar un vistazo y ver c�mo funcionan las cosas. Demasiadas 
veces, un administrador de sistemas pone un mensaje desesperado 
rogando ayuda antes de investigar el problema. Muchas de las veces, 
arreglar problemas por uno mismo es realmente muy 
f�cil. Este es el camino que debe seguir para convertirse en un gur�.

Hay pocos casos en los que sea necesario reinstalar el sistema desde cero.
Muchos nuevos usuarios borran accidentalmente alg�n fichero esencial del 
sistema, e inmediatamente acuden a los discos de instalaci�n. Esta no 
es una buena idea. Antes de tomar medidas dr�sticas como esa, 
investigar el problema y preguntar a otros ayudar� a solucionar las 
cosas. En pr�cticamente todos los casos, podr� recuperar el sistema 
desde un disquete de mantenimiento.

\subsection{Recuperaci�n utilizando un disquete de mantenimiento}
\label{sec-maint-diskette}
\index{desastres!recuperaci�n de!con disquete de mantenimiento}
\index{emergencias!recuperaci�n de!con disquete de mantenimiento}
\index{arrancando!de un disquete de mantenimiento}
\index{disquete de mantenimiento}
\index{disquete de arranque}
\index{disquette!arranque/ra�z}
\index{disquette!de mantenimiento}
Una herramienta indispensable para el administrador de sistemas es el 
llamado ``disco arranque/ra�z'' (``boot/root disk'') ---un disquete 
desde el que se puede arrancar un sistema GNU/Linux completo, independiente 
del disco duro. Los discos de arranque/ra�z son realmente muy
simples, se crea un sistema de ficheros ra�z en el disquete, se ponen todas 
las utilidades necesarias en �l y se instala LILO y un n�cleo 
arrancable en el disquete. Otra t�cnica es usar un disquete para el 
n�cleo y otro para el sistema de ficheros ra�z. En cualquier caso, 
el resultado es el mismo: Ejecutar un sistema Linux completamente desde 
disquete.

El ejemplo m�s claro de un disco de arranque/ra�z son los discos de 
arranque Slackware\footnote{V�ase la 
Secci�n~\ref{sec-getting-internet} para la informaci�n sobre c�mo 
obtener �sta desde Internet. Para este procedimiento, no se necesita 
obtener la versi�n completa de Slackware, s�lo los disquetes de 
arranque y ra�z.}. Estos disquetes contienen un n�cleo capaz de iniciar y 
un sistema de ficheros ra�z, todo en disquete. Est�n dise�ados 
para usarse en la instalaci�n de la distribuci�n Slackware, pero 
vienen muy bien cuando hay que hacer mantenimiento del sistema.

El disco de arranque/ra�z de H.J Lu, disponible en 
{\tt /pub/Linux/GCC/rootdisk} en {\tt sunsite.unc.edu}, es otro ejemplo de 
este tipo de discos de mantenimiento. O, si se es ambicioso, se puede crear
uno su propio disco. En muchos casos, sin embargo, la utilizaci�n de un 
disco de arranque/ra�z prefabricado es mucho m�s simple y 
probablemente ser� m�s completo.

La utilizaci�n de un disco de arranque/ra�z es muy simple. Tan 
s�lo arranque
el sistema con el disco, y haga login como {\tt root} (normalmente sin 
clave). Para poder acceder a los ficheros del disco duro, se necesitar� 
montar el sistema de ficheros a mano. Por ejemplo, la orden
\begin{tscreen}
\# {\em mount -t ext2 /dev/hda2 /mnt}
\end{tscreen}
montar� un sistema de ficheros ext2fs existente en {\tt /dev/hda2} bajo
{\tt /mnt}. Recuerde que {\tt /} es ahora el propio disco de arranque/ra�z;
se necesitar� montar los sistemas de ficheros de su disco duro bajo alg�n
directorio para poder acceder a los ficheros. Por lo tanto, el fichero {\tt 
/etc/passwd} de su disco duro es ahora {\tt /mnt/etc/passwd} si se 
mont� el sistema de ficheros ra�z bajo {\tt /mnt}.

\subsection{Arreglando la clave de root}
\index{contrase�a!arreglando la de root}
\index{root!arreglando la password de}
Si se olvida de la clave de root, no hay problema. S�lo hay que
arrancar del disco de arranque/ra�z, montar su sistema de ficheros ra�z
en {\tt /mnt}, y eliminar el campo de la clave de {\tt /root} en
{\tt /mnt/etc/passwd}, como por ejemplo:
\begin{tscreen}
root::0:0:root:/:/bin/sh
\end{tscreen}
Ahora {\tt root} no tiene clave; al reiniciar desde el disco duro 
deber�a ser capaz de hacer login como {\tt root} y poner la clave
que desee utilizando {\tt passwd}.

�No quiso aprender a utilizar {\tt vi}? En el disco de 
arranque/ra�z probablemente no estar�n disponibles otros editores como 
pueda ser Emacs, pero {\tt vi} deber�a estarlo.

\subsection{Arreglando sistemas de ficheros corrompidos}
\index{sistemas de ficheros!arreglando corrompidos}
\index{e2fsck@{\tt e2fsck}}
\index{fsck@{\tt fsck}}
Si se corrompiese de alguna forma el sistema de ficheros, se puede ejecutar
{\tt e2fsck} o la forma apropiadad de {\tt fsck} para el tipo de
sistema de ficheros (vease la
p�gina~\pageref{sec-checking-file-system}). En muchos casos, es m�s
seguro corregir cualquier dato da�ado en el sistema de ficheros del
disco duro desde un disquete.

\index{super bloque!definici�n}
\index{super bloque!corrompido, arreglo}
Una causa com�n de da�o en un sistema de ficheros es la
corrupci�n del super bloque. 
El {\bf super bloque\/} es la ``cabecera'' del sistema de ficheros que 
contiene informaci�n acerca del estado del sistema de ficheros, tama�o, 
bloques libres, y dem�s. Si se corrompe el super bloque (por ejemplo,
escribiendo accidentalmente datos directamente a la partici�n del sistema 
de ficheros), el sistema no puede reconocer nada del sistema de ficheros.
Cualquier intento de montar el sistema de ficheros fallar� y {\tt e2fsck} no
ser� capaz de arreglar el problema.

Afortunadamente, el tipo de sistema de ficheros {\em ext2fs} salva copias del
super bloque en los l� mites de ``grupos de bloques'' en el disco 
---normalmente cada 8K bloques. Para poder decirle al {\tt e2fsck} que 
utilice una copia del super bloque, se puede utilizar una orden tal que
\begin{tscreen}
\# {\em e2fsck -b 8193 \cparam{partici�n}}
\end{tscreen} 
donde \cparam{partici�n} es la partici�n en la que reside el sistema de 
ficheros. La opci�n {\tt -b 8193} le dice al {\tt e2fsck} que utilice la copia
del super bloque almacenada en el bloque 8193 del sistema de ficheros.

\subsection{Recuperando ficheros perdidos}
\index{ficheros!recuperaci�n}
Si accidentalmente se borran ficheros importantes del sistema no 
hay forma de recuperarlos. Sin embargo, se pueden copiar los ficheros
relevantes desde el disquete al disco duro. Por ejemplo, si se hubiese borrado
{\tt /bin/login} de su sistema (que le permite registrarse en el sistema), simplemente
arranque del disquete de arranque/ra�z, monte el sistema de ficheros ra�z
en {\tt /mnt}, y use la orden
\begin{tscreen}
\# {\em cp -a /bin/login /mnt/bin/login}
\end{tscreen}
La opci�n {\tt -a} le dice a {\tt cp} que conserve los permisos en los
ficheros que se est�n copiando.

Por supuesto, si los ficheros que se borraron no fuesen ficheros esenciales 
del sistema que tengan contrapartidas en el disquete de arranque/ra�z, se
habr� acabado la suerte. Si se hicieron copias de seguridad, siempre se
podr� recuperar de ellas.

\subsection{Arreglando bibliotecas corrompidas}
\index{bibliotecas!arreglando corrompidas}
Si accidentalmente se llegasen a corromper las bibliotecas de enlaces 
simb�licos en {\tt /lib}, es m�s que seguro que instrucciones que dependan de 
estas bibliotecas no vuelvan a funcionar (V�ase la 
Secci�n~\ref{sec-upgrade-libs}). La soluci�n m�s simple es arrancar 
del disquete de arranque/ra�z, montar el sistema de ficheros ra�z 
y arreglar las bibliotecas en {\tt /mnt/lib}.

En la p�gina~\pageref{sec-upgrade-libs} se describe c�mo instalar este
tipo de bibliotecas y sus enlaces simb�licos.

\index{emergencias!recuperaci�n de|)}
\index{desastres!recuperaci�n de|)}



