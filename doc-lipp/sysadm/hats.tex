% Linux Installation and Getting Started    -*- TeX -*-
% hats.tex
% Copyright (c) 1993 by Matt Welsh and Lars Wirzenius
%
% This file is freely redistributable, but you must preserve this copyright 
% notice on all copies, and it must be distributed only as part of "Linux 
% Installation and Getting Started". This file's use is covered by
% the copyright for the entire document, in the file "copyright.tex".
%
% Este fichero es de distribuci\'on libre, pero debe mantenerse esta 
% informaci\'on de Copyright en todas las copias, y debe distribuirse solo como
% parte de "Instalaci\'on y Primeros Pasos en Linux". El uso de este fichero esta
% cubierto por el Copyright del documento completo, en el fichero "copyright.tex"
% Copyright (c) 1995 por Gerardo Izquierdo para la versi\'on al Castellano
% $Log: hats.tex,v $
% Revision 1.6  2003/07/19 06:32:34  joseluis.ranz
% Correcciones varias.
%
% Revision 1.5  2002/07/20 22:24:29  pakojavi2000
% Beta2
%
% Revision 1.4  2002/07/07 21:40:49  pakojavi2000
%  Traducci�n de fragmentos incompletos
%
% Revision 1.3  2001/04/18 16:29:10  amolina
% Segunda revisi�n de los ficheros
%
% Revision 1.2  2001/01/16 15:10:36  amolina
%
% Primera traducci�n de sysadm/hats,tex
%
% Revision 0.5.0.1  1996/02/10 23:45:12  rcamus
% Primera beta publica

% 
% Versi�n para lipp 2.0 por Alberto Molina. Comentarios a:
%            alberto@nucle.us.es 
%
%

\subsection{Responsabilidades de la Administraci�n del Sistema}

Puesto que el administrador de sistema tiene mucho m�s poder y
responsabilidad, cuando algunos usuarios tienen la oportunidad de
ingresar por primera vez como {\tt root}, tanto en sistemas GNU/Linux como
en otros, tienden a abusar de los privilegios de {\tt root}. Existen
``administradores de sistema'' que leen el correo de otros usuarios,
borran ficheros sin avisar y se comportan como ni�os con un poderoso
juguete entre sus manos.

Como el administrador tiene tanto poder sobre el sistema, se requiere
cierta madurez y autocontrol para utilizar la cuenta {\tt
  root}. Existe un c�digo de honor no escrito que establece las normas
de comportamiento del administrador del sistema para con el resto de
usuarios. �C�mo se sentir�a si el administrador de su sistema se
dedicase a leer su correo electr�nico o a mirar en sus
ficheros?. Existe un cierto vac�o legal en estos asuntos. En los
sistemas UNIX, el usuario {\tt root} tiene la posibilidad de saltarse
todos los mecanismos de seguridad y privacidad. Es importante que el
administrador de sistema establezca una relaci�n de confianza con sus
usuarios.

\subsection{C�mo proceder con los usuarios}

Los administradores de sistemas pueden tomar dos posturas cuando traten con
usuarios abusivos: ser paranoicos o confiados. El administrador de 
sistemas paranoico normalmente causa m\'as da\~no que el que previene. Una de
mis citas favoritas es: ``Nunca atribuyas a la malicia nada que pueda ser
atribuido a la estupidez''. Dicho de otra forma, muchos usuarios no tienen
la habilidad o el conocimiento para hacer da\~no real al sistema. El 90\% del
tiempo, cuando un usuario causa problemas en el sistema (por ejemplo, 
rellenando la partici\'on de usuarios con grandes ficheros, o ejecutando
m\'ultiples veces simult�neamente un gran programa), el usuario simplemente desconoce 
que est\'a causando un problema. He ido a ver a usuarios que estaban
causando una gran cantidad de problemas, pero su actitud estaba causada por 
la ignorancia, no por la malicia.

Cuando se encuentre con usuarios que puedan causar problemas potenciales
no sea hostil. La antigua regla de ``inocente hasta que se demuestre lo
contrario'' sigue siendo v\'alida. Es mejor una simple charla con el usuario,
pregunt\'andole acerca del problema, en lugar de causar una confrontaci\'on. Lo
\'ultimo que se desea es estar entre los malos desde el punto de vista del
usuario. Esto levantar\'\i a un mont\'on de sospechas acerca de si el
administrador de sistemas tiene el sistema correctamente 
configurado. Si un usuario cree que uno le disgusta o no le tiene 
confianza, le puede acusar de borrar ficheros o romper la privacidad del 
sistema. Esta no es, ciertamente, el tipo de situaci\'on en la que se
quisiera estar.

Si se encuentra que un usuario ha estado intentando ``romper'' el sistema,
o ha estado haciendo da\~no al sistema de forma intencionada, no hay
que devolver el comportamiento malicioso a su vez. En vez de ello,
simplemente, es recomendable darle un  aviso ---pero siendo
flexible. En muchos casos, se puede cazar a un usuario
``con las manos en la masa'', da\~nando al sistema, lo correcto es
advertirle y decirle que no lo vuelva a repetir. Sin embargo, si le
{\em vuelve\/} a cazar haciendo da\~no, entonces se puede estar
absolutamente seguro de que es intencionado.
Ni siquiera puedo empezar a describir la cantidad de veces que parec\'\i a que
hab\'\i a un usuario causando problemas al sistema, cuando de hecho, era o un
accidente o un fallo m\'\i o.

\subsection{Fijando las reglas}

La mejor forma de administrar un sistema no es con un pu\~no de hierro. 
As\'{\i} puede ser como se haga lo militar, pero UNIX no fue dise\~nado para 
ese tipo de disciplinas. Tiene sentido el escribir un conjunto sencillo y 
flexible de reglas para los usuarios, pero hay que recordar que cuantas menos 
reglas tenga, menos posibilidades habr\'a de romperlas. Incluso si las 
reglas para utilizar el sistema son perfectamente razonables y claras, 
siempre habr\'a momentos en que los usuarios romper\'an dichas reglas sin 
pretenderlo. Esto es especialmente cierto en el caso de usuarios UNIX 
nuevos, que est\'an aprendiendo los entresijos del sistema. No esta 
suficientemente claro, por ejemplo, que uno no debe bajarse un gigabyte de 
ficheros y envi\'arselo por correo a todos los usuarios del sistema. Los 
usuarios necesitan comprender las reglas y por qu� est\'an establecidas.

Si especifica reglas de uso para su sistema, hay que asegurarse de que el motivo 
detr\'as de cada regla particular est\'e claro. Si no se hace, los usuarios
encontrar\'an toda clase de formas creativas de salt\'arsela y no saber que en
realidad la est\'an rompiendo.

\subsection{Lo que todo esto significa}

No podemos decir c�mo ejecutar su sistema al \'ultimo detalle. Mucha de la
filosof\'\i a depende de c�mo se use el sistema. Si se tienen muchos 
usuarios, las cosas son muy diferentes de si s�lo tiene unos pocos o si 
se es el \'unico usuario del sistema. Sin embargo, siempre es una buena 
idea, en cualquier situaci\'on, comprender lo que ser administrador 
de sistema significa en realidad.

Ser el administrador de un sistema no le hace a uno un mago del UNIX. Hay
muchos administradores de sistemas que conocen muy poco acerca de UNIX.
Igualmente, hay muchos usuarios ``normales'' que saben m\'as acerca de 
UNIX que cualquier administrador de sistema. Tambi\'en, ser 
administrador de sistemas no le permite el utilizar la malicia contra sus 
usuarios. Aunque el sistema le d\'e el privilegio de enredar en los 
ficheros de los usuarios, no significa que se tenga ning\'un derecho a 
hacerlo.

Por \'ultimo, ser el administrador del sistema no es realmente una gran cosa.
No importa si el sistema es un peque\~no 386 o un super ordenador Cray. La
ejecuci\'on del sistema es la misma. El saber la clave de {\tt root} no
significa ganar dinero o fama. Tan solo le permitir\'a ejecutar el sistema
y mantenerlo funcionando. Eso es todo.

