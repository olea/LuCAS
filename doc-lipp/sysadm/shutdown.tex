% Linux Installation and Getting Started    -*- TeX -*-
% shutdown.tex
% Copyright (c) 1993 by Matt Welsh and Lars Wirzenius
%
% This file is freely redistributable, but you must preserve this copyright 
% notice on all copies, and it must be distributed only as part of "Linux 
% Installation and Getting Started". This file's use is covered by
% the copyright for the entire document, in the file "copyright.tex".
%
% Copyright (c) 1998 by Specialized Systems Consultants Inc. 
% <ligs@ssc.com>
%
% Este fichero es de distribuci�n libre, pero debe mantenerse esta 
% informaci�n de Copyright en todas las copias, y debe distribuirse solo como
% parte de "Instalaci�n y Primeros Pasos en Linux". El uso de este fichero esta
% cubierto por el Copyright del documento completo, en el fichero "copyright.tex"
% Copyright (c) 1995 por Gerardo Izquierdo para la versi�n al Castellano
% $Log: shutdown.tex,v $
% Revision 1.9  2003/07/19 20:28:24  pakojavi2000
% Arreglando un peque�o destrozo de macros
%
% Revision 1.8  2003/07/19 06:20:33  joseluis.ranz
% Correcciones varias.
%
% Revision 1.7  2002/07/30 16:23:05  pakojavi2000
% Beta 2.2 Formatos de p�rrafo
%
% Revision 1.6  2002/07/20 22:24:29  pakojavi2000
% Beta2
%
% Revision 1.5  2002/07/20 17:41:16  pakojavi2000
% beta2
%
% Revision 1.4  2002/07/13 12:50:07  pakojavi2000
% Gold
%
% Revision 1.3  2001/04/18 16:29:10  amolina
% Segunda revisi�n de los ficheros
%
% Revision 1.2  2000/12/20 16:51:28  amolina
%
% Primera versi�n traducida de sysadm/shutdown.tex
%
% Revision 0.5.0.1  1996/02/10 23:45:13  rcamus
% Primera beta publica
%

% Versi�n para lipp 2.0 por Alberto Molina
%         comentarios a alberto@nucle.us.es
%
%Revisi�n 1 13 de julio 2002 por JFS <serrador@arrakis.es
%Gold
\section{Parada del sistema}
\markboth{Administraci�n del Sistema}{Parada del Sistema}
\label{sec-sysadm-shutdown}

\index{administraci�n del sistema!cierre del sistema|(}
\index{cierre del sistema|(}
Cerrar un sistema {\linux} tiene algo de truco. Hay que recordar que nunca se debe
cortar la corriente o pulsar el bot�n de apagado mientras el sistema
est� ejecut�ndose. El n�cleo sigue la pista de la entrada/salida a disco
en ``buffers'' de memoria. Si se reinicializa el sistema sin darle al n�cleo
la oportunidad de escribir sus ``buffers'' a disco, puede corromper sus sistemas
de ficheros.

En tiempo de cierre se toman tambi�n otras precauciones. Todos los procesos
reciben una se�al que les permite morir airosamente (escribiendo y cerrando 
todos los ficheros y ese tipo de cosas). Los sistemas de ficheros se 
desmontan por seguridad. Si se desea, el sistema tambi�n puede alertar a los
usuarios de que se est� cerrando y darles la posibilidad de desconectarse.

\index{orden shutdown@comando {\tt shutdown}}
La forma m�s simple de cerrar el sistema es con la orden {\tt 
shutdown}. El formato es
\begin{tscreen}
shutdown \cparam{tiempo} \cparam{mensaje-de-aviso}
\end{tscreen}
El argumento \cparam{tiempo} es el momento de cierre del sistema (en el
formato {\em hh:mm:ss}), y \cparam{mensaje-de-aviso} es un mensaje mostrado
en todos los terminales de usuario antes de cerrar. Alternativamente, se
puede especificar el par�metro \cparam{tiempo} como ``{\tt now}'', para
cerrar inmediatamente. Se le puede suministrar la opci�n {\tt -r} a 
{\tt shutdown} para reinicializar el sistema tras el cierre.

Por ejemplo, para cerrar el sistema a las 8:00pm, se puede utilizar la
siguiente orden
\begin{tscreen}
\# {\em shutdown -r 20:00}
\end{tscreen}

\index{halt@{\tt halt}}
La orden {\tt halt} puede utilizarse para forzar un cierre inmediato, sin
ning�n mensaje de aviso ni periodo de gracia. {\tt halt} se utiliza si se
es el �nico usuario del sistema y se quiere cerrar el sistema y apagarlo.

\blackdiamond No apagar o reinicializar el sistema hasta que se vea el mensaje:
\begin{tscreen}
The system is halted
\end{tscreen}
Es muy importante que cierre el sistema ``limpiamente'' utilizando la
orden {\tt shutdown} o el {\tt halt}. En algunos sistemas, se reconocer� 
el pulsar \key{ctrl-alt-del}, que causar� un {\tt shutdown}; en otros 
sistemas, sin embargo, el utilizar el ``Apret�n de Cuello de Vulcano''
reinicializar� el sistema inmediatamente y puede causar un desastre.

\index{administraci�n del sistema!cierre del sistema|)}
\index{cierre del sistema|)}


