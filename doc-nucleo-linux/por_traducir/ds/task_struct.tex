\subsection*{task\_struct}
Cada estructura de datos \ds{task\_struct} describe un proceso o tarea
en el sistema.
\SeeModule{include/linux/\\sched.h}
\begin{tscreen}\begin{verbatim}
struct task_struct {
/* these are hardcoded - don't touch */
  volatile long        state;          /* -1 unrunnable, 0 runnable, >0 stopped */
  long                 counter;
  long                 priority;
  unsigned             long signal;
  unsigned             long blocked;   /* bitmap of masked signals */
  unsigned             long flags;     /* per process flags, defined below */
  int errno;
  long                 debugreg[8];    /* Hardware debugging registers */
  struct exec_domain   *exec_domain;
/* various fields */
  struct linux_binfmt  *binfmt;
  struct task_struct   *next_task, *prev_task;
  struct task_struct   *next_run,  *prev_run;
  unsigned long        saved_kernel_stack;
  unsigned long        kernel_stack_page;
  int                  exit_code, exit_signal;
  /* ??? */
  unsigned long        personality;
  int                  dumpable:1;
  int                  did_exec:1;
  int                  pid;
  int                  pgrp;
  int                  tty_old_pgrp;
  int                  session;
  /* boolean value for session group leader */
  int                  leader;
  int                  groups[NGROUPS];
  /* 
   * pointers to (original) parent process, youngest child, younger sibling,
   * older sibling, respectively.  (p->father can be replaced with 
   * p->p_pptr->pid)
   */
  struct task_struct   *p_opptr, *p_pptr, *p_cptr, 
                       *p_ysptr, *p_osptr;
  struct wait_queue    *wait_chldexit;  
  unsigned short       uid,euid,suid,fsuid;
  unsigned short       gid,egid,sgid,fsgid;
  unsigned long        timeout, policy, rt_priority;
  unsigned long        it_real_value, it_prof_value, it_virt_value;
  unsigned long        it_real_incr, it_prof_incr, it_virt_incr;
  struct timer_list    real_timer;
  long                 utime, stime, cutime, cstime, start_time;
/* mm fault and swap info: this can arguably be seen as either
   mm-specific or thread-specific */
  unsigned long        min_flt, maj_flt, nswap, cmin_flt, cmaj_flt, cnswap;
  int swappable:1;
  unsigned long        swap_address;
  unsigned long        old_maj_flt;    /* old value of maj_flt */
  unsigned long        dec_flt;        /* page fault count of the last time */
  unsigned long        swap_cnt;       /* number of pages to swap on next pass */
/* limits */
  struct rlimit        rlim[RLIM_NLIMITS];
  unsigned short       used_math;
  char                 comm[16];
/* file system info */
  int                  link_count;
  struct tty_struct    *tty;           /* NULL if no tty */
/* ipc stuff */
  struct sem_undo      *semundo;
  struct sem_queue     *semsleeping;
/* ldt for this task - used by Wine.  If NULL, default_ldt is used */
  struct desc_struct *ldt;
/* tss for this task */
  struct thread_struct tss;
/* filesystem information */
  struct fs_struct     *fs;
/* open file information */
  struct files_struct  *files;
/* memory management info */
  struct mm_struct     *mm;
/* signal handlers */
  struct signal_struct *sig;
#ifdef __SMP__
  int                  processor;
  int                  last_processor;
  int                  lock_depth;     /* Lock depth. 
                                          We can context switch in and out
                                          of holding a syscall kernel lock... */  
#endif   
};
\end{verbatim}\end{tscreen}

