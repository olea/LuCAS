\subsection*{pci\_dev}

Cada disositivo PCI del sistema, incluso los puentes PCI-PCI y PCI-ISA
se representan mediante una estructura de datos \ds{pci\_dev}.
\SeeModule{include/linux/\\pci.h}
\begin{tscreen}\begin{verbatim}
/*
 * Existe una estructura pci_dev para cada combinaci�n
 * n�mero-de-ranura/n�mero-de-funci�n:
 */
struct pci_dev {
  struct pci_bus  *bus;      /* bus sobre el cual est� este dispositivo */
  struct pci_dev  *sibling;  /* pr�ximo dispositivo en este bus  */
  struct pci_dev  *next;     /* cadena de todos los dispositivos */

  void    *sysdata;          /* gancho para extensiones dependientes
                                del sistema */

  unsigned int  devfn;       /* �ndice de dispositivo y funci�n codificado */
  unsigned short  vendor;
  unsigned short  device;
  unsigned int  class;       /* 3 bytes: (base,sub,prog-if) */
  unsigned int  master : 1;  /* puesto a uno si el dispositivo es
                                capaz de ser maestro */
  /*
   * En teor�a, el nivel de irq se puede leer desde el espacio de
   * configuraci�n y todo debe ir bien.  Sin embargo, los viejos chips
   * PCI no manejan esos registros y retornan 0.  Por ejemplo, el chip 
   * Vision864-P rev 0 puede usar  INTA, pero retorna 0 en la l�nea de
   * interrupci�n y registro pin.  pci_init() inicializa este campo
   * con el valor  PCI_INTERRUPT_LINE y es trabajo de  pcibios_fixup()
   * cambiearlo si es necesario.  El campo no debe ser 0 a menos que
   * el dispositivo no tenga modo de generar interrupciones.
   */
  unsigned char  irq;        /* irq generada por este dispositivo */
};
\end{verbatim}\end{tscreen}
