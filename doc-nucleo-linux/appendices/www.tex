\chapter{Lugares �tiles en Web y FTP}
\label{www-appendix}
Los siguientes lugares de la Red WWW y de FTP le ser�n de utilidad:
\begin{description}
\item [http://www.azstarnet.com/\ $\tilde{}$axplinux] Este es el lugar
  en la web del \Linux\ para \axp\ de David
  Mosberger-Tang, y es el lugar donde debe ir a buscar todos los
  HOWTOs sobre \axp.  Tambi�n tiene una gran cantidad de punteros
  hacia informaci�n sobre \Linux\ y espec�fica sobre \axp, como por
  ejemplo las hojas de datos de la CPU.
  
\item [http://www.redhat.com/] El lugar de Red Ha en la web.  Hay aqu�
  un mont�n de punteros �tiles.
  
\item [ftp://sunsite.unc.edu] Este es el principal lugar para un
  mont�n de software libre.  El software espec�fico sobre \Linux\ se
  encuentra en \emph{pub/Linux}.
  
\item [http://www.intel.com] El lugar de Intel en la web, y un buen
  lugar donde encontrar informaci�n sobre chips Intel.
  
\item [http://www.ssc.com/lj/index.html] "<Linux Journal"> es una muy
  buena revista sobre \Linux\ y bien vale la pena el precio de la
  suscripci�n anual para leer sus excelentes art�culos.
  
\item [http://www.blackdown.org/java-linux.html] Este es el lugar
  principal con respecto a la informaci�n sobre Java para \Linux.
  
\item [ftp://tsx-11.mit.edu/\ $\tilde{}$ftp/pub/linux] El lugar FTP
  sobre \Linux\ del MIT.
  
\item [ftp://ftp.cs.helsinki.fi/pub/Software/Linux/Kernel] Fuentes del
  n�cleo de Linus.
  
\item [http://www.linux.org.uk] El "<UK Linux User Group">.
  
\item [http://sunsite.unc.edu/mdw/linux.html] P�gina principal para el
  "<Linux Documentation Project"> (Proyecto de documentaci�n para
  \Linux), al cual est� afiliado LuCAS ({\tt http://lucas.ctv.es/})
  que se dedica a traducir al castellano ---como en este caso--- dicha
  documentaci�n.
  
\item [http://www.digital.com] El lugar en la web de "<Digital
  Equipment Corporation">
  
\item [http://altavista.digital.com] La m�quina de b�squeda Altavista
  es un producto de la empresa Digital, y un muy buen lugar para
  buscar informaci�n dentrl de la web y los grupos de discusi�n.
  
\item [http://www.linuxhq.com] El lugar web "<Linux HQ"> contiene los
  actualizados parches tanto oficiales como no-oficiales, consejos, y
  punteros a la web que le ayudar�n a conseguir el mejor conjunto de
  fuentes posibles para su sistema.

\item [http://www.amd.com] El lugar de AMD en la web.
  
\item [http://www.cyrix.com] El lugar de Cyrix en la web.

\end{description}

