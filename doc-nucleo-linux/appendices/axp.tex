\chapter{El procesador AXP de Alpha}

\index{Alpha AXP, arquitectura}
La arquitectura \axp\ es una arquitectura RISC con
carga/almacenamiento de 64~bits, dise�ada con la mira puesta en la
velocidad.  Todos los registros son de 64~bits de longitud; hay
32~registros enteros y 32~de coma flotante.  El registro entero~31 y
el coma flotante~31 se utilizan para operaciones nulas: un lectura
desde ellos genera un valor cero, y la escritura en los mismos no
tiene ning�n efecto.  Todas las instrucciones son de 32~bits de
longitud, y las operaciones con memoria son bien lecturas, bien
escrituras.  La arquitectura permite diferentes implementaciones,
siempre que las mismas sigan las l�neas fijadas por la arquitectura. 

No existen instrucciones que operan directamente sobre valores
almacenados en memoria; toda la manipulaci�n de datos se hace sobre
los registros.  As� que si Ud.\ desea incrementar un contador en
memoria, primero lo debe leer en un registro, luego lo modifica y lo
escribe en la memoria.  Las instrucciones s�lo interact�an entre ellas
si una instrucci�n escribe en un registro o una posici�n de memoria, y
otra instrucci�n lee el mismo registro o posici�n de memoria.  Una
caracter�stica interesante del \axp\ es que hay instrucciones que
generan indicadores ("<flags">) ---por ejemplo al controlar si dos
registros son iguales--- pero en lugar de almacenar el resultado de la
comparaci�n en el registro de estado del procesador, lo almacenan en
un tercer registro.  Esto puede sonar extra�o a primera vista, pero al
quitar esta dependencia del registro de estado del procesador, es m�s
f�cil construir CPUs que puedan lanzar varias instrucciones en cada
ciclo. Las instrucciones que se realizan sobre registros distintos no
tienen que esperarse unas a otras, como ser�a el caso si dependieran
del registro de estado del procesador.  La ausencia de operaciones
directas sobre memoria, y la gran cantidad de registros tambi�n ayudan
al lanzamiento de m�ltiples instrucciones simult�neas.

\index{PALcode} La arquitectura \axp\ utiliza un conjunto de
subrutinas, denominadas "<Privileged architecture library code">
(C�digo de biblioteca de arquitectura privilegiada) (PALcode).  El
PALcode es espec�fico del sistema operativo, la implementaci�n en la
CPU de la arquitectura \axp, y el hardware del sistema.  Estas
subrutinas proveen primitivas del sistema operativo para el
intercambio de contextos, interrupciones, excepciones y administraci�n
de memoria.  Estas subrutinas pueden ser llamadas por el hardware o
mediante las instrucciones CALL\_PAL.  El PALcode est� escrito en
ensamblador \axp\ est�ndar con algunas extensiones espec�ficas de la
implementaci�n para proveer acceso directo a las funciones de hardware
de bajo nivel, por ejemplo a los registros internos del procesador.
El PALcode se ejecuta en modo PALmode, que es un modo privilegiado en
el cual se impide que sucedan ciertos eventos del sistema, y le
proporciona total control del hardware f�sico del sistema al c�digo
PALcode. 

