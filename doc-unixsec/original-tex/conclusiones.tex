\chapter{Conclusiones}
Si despu\'es de aproximadamente 500 hojas de trabajo, con m\'as de 300 
referencias
bibliogr\'aficas citadas, a\'un hay alguien que considere a Unix 
un sistema inseguro existen dos opciones: o se e\-qui\-vo\-ca \'el o me 
equivoco yo. 
Seguramente que me equivoque yo no ser\'{\i}a dif\'{\i}cil; lo realmente 
extra\~no es que se hayan equivocado todos los expertos que 
durante a\~nos -- algunos desde antes de que muchos de nosotros hubi\'eramos 
nacido -- han venido aportando su tiempo, su talento y sus conocimientos al 
mundo de la seguridad inform\'atica (por supuesto, hablo de expertos de
verdad, no de {\it hackers}, {\it crackers}, o como ahora se quiera llamar a los
piratas), una materia que d\'{\i}a a d\'{\i}a va demostrando su importancia en 
todo tipo de organizaciones. Como es bastante dif\'{\i}cil que toda esta gente 
se haya 
equivocado, ser\'{\i}a conveniente que el que a\'un a estas alturas dude de las 
posibilidades de Unix (en cuanto a seguridad se refiere, aunque podr\'{\i}amos
hablar de posibilidades en general) con respecto a otros sistemas se replantee
sus ideas.\\
\\En este proyecto se han revisado las bases m\'as importantes de la seguridad
en Unix y redes; evidentemente, muchas cosas se han quedado en el tintero, y
otras muchas no han sido comentadas con la profundidad que sin duda merecen. Se
han intentado ofrecer ejemplos aplicados a entornos que no precisan de una alta 
seguridad, pero s\'{\i} de una seguridad m\'{\i}nima, como es el caso de las 
redes de I+D, las de medianas empresas, y las de ISPs. El trabajo se ha 
dividido en seis grandes partes;
en la primera (seguridad del entorno de operaciones) se habla de las 
implicaciones de seguridad (e inseguridad) relacionadas con la simple existencia
de un sistema, Unix o no, en un entorno de trabajo: su ubicaci\'on f\'{\i}sica,
las personas que le rodean\ldots Una segunda parte es la relacionada con la
seguridad de la m\'aquina en s\'{\i}, sin conexi\'on a red, y todos los 
problemas que nos podemos encontrar en esta situaci\'on, y la tercera habla de
peculiaridades tambi\'en a nivel de {\it host} de diferentes clones de Unix; 
como los sistemas 
aislados son cada d\'{\i}a m\'as extra\~nos, la cuarta parte (seguridad de la
subred) introduce algunos de los peligros (y sus soluciones) que no 
exist\'{\i}an en m\'aquinas sin conectar a una red. A continuaci\'on, una 
quinta parte 
habla de otros aspectos relacionados con la seguridad de un equipo, algunos de 
los cuales son las bases para comprender muchas de las cosas que se explican en
el trabajo (por ejemplo, la criptolog\'{\i}a). Para terminar, en la sexta
parte del proyecto, ya como ap\'endices, se presenta un escueto resumen de 
normas de seguridad
a modo de `receta de cocina' para administradores, algunas normativas vigentes
en Espa\~na relacionadas con los sistemas inform\'aticos y su (in)seguridad, una
referencia de recursos relacionados con esta materia en Internet, y finalmente
un peque\~no glosario de t\'erminos anglosajones utilizados con frecuencia en
el mundo de la seguridad en Unix.\\
\\A pesar del elevado nivel de seguridad que Unix puede ofrecer (al menos espero
que haya quedado patente que Unix es el sistema operativo de prop\'osito general
m\'as seguro hoy en d\'{\i}a) cualquiera que se diera una vuelta, f\'{\i}sica o 
virtual, por la mayor\'{\i}a de entornos `normales' en Espa\~na
podr\'{\i}a comprobar que su seguridad es en la mayor parte de los casos pobre, 
cuando no inexistente. Si Unix es te\'oricamente tan seguro, >por qu\'e en la
pr\'actica cualquier aprendiz de pirata es capaz de `colarse' en servidores de 
todo tipo?, >d\'onde est\'a el problema? El problema no radica en Unix: radica 
en las personas que est\'an detr\'as del sistema operativo, generalmente 
administradores y usuarios de cualquier categor\'{\i}a. Unix ofrece los 
mecanismos suficientes como para conseguir un nivel de seguridad m\'as que 
aceptable, pero somos nosotros los que en muchos
casos no sabemos aprovecharlos. Para solucionar el problema, como ya hemos
comentado a lo largo del proyecto, existen dos soluciones que todos 
deber\'{\i}amos intentar aplicar: en primer lugar la {\bf concienciaci\'on} de
los problemas que nos pueden acarrear los fallos de seguridad (a muchos a\'un
les parece que el tema no va con ellos, que los piratas inform\'aticos s\'olo
existen en el cine, y que en su m\'aquina nada malo puede ocurrir). Tras la
concienciaci\'on, es necesaria una {\bf formaci\'on} adecuada a cada tipo de
persona (e\-vi\-den\-te\-men\-te no podemos exigir los mismos conocimientos a 
un administrador responsable de varias m\'aquinas que a un usuario que s\'olo 
conecta al sistema para lanzar simulaciones); no es necesario convertirse en un
experto, simplemente hay que leer un poco y conocer unas normas b\'asicas (por 
ejemplo, las presentadas en el ap\'endice A\ldots si alguien argumenta que no 
tiene tiempo para leer quince hojas, seguramente est\'a mintiendo). Con estos
dos pasos seguramente no pararemos a todos los piratas que nos intenten atacar,
pero s\'{\i} a la gran mayor\'{\i}a de ellos, que es lo que realmente interesa 
en el mundo de la seguridad.\\
\\Aparte del l\'ogico incremento en el nivel de seguridad que se 
conseguir\'{\i}a mediante una m\'{\i}nima concienciaci\'on y formaci\'on de los 
usuarios de Unix, existe un escollo que estas dos medidas dif\'{\i}cilmente nos
van a permitir superar: la simpat\'{\i}a que socialmente despiertan muchos 
piratas inform\'aticos; por desgracia, mucha gente a\'un considera a estos 
personajes una especie de h\'eroes. Si nadie aplaude al que roba un bolso 
en la calle, >por qu\'e a\'un existen defensores de los que roban contrase\~nas 
de un sistema? Mientras sigamos sin darnos cuenta de lo que realmente son los 
piratas (simplemente delincuentes) ser\'a dif\'{\i}cil que la seguridad 
inform\'atica sea tomada en serio.\\
\\No me gustar\'{\i}a acabar este trabajo sin una peque\~na reflexi\'on sobre
el panorama de la seguridad en Unix y redes que existe actualmente en Espa\~na;
s\'olo cabe una definici\'on: {\bf lamentable}. Lo \'unico que por suerte se 
toma en serio es la criptograf\'{\i}a, que cuenta con grupos de estudio y 
docencia en algunas universidades del pa\'{\i}s. Del resto, casi es mejor no 
hablar: no existe ning\'un grupo importante de investigaci\'on en ninguna 
universidad espa\~nola, el n\'umero de art\'{\i}culos pu\-bli\-ca\-dos en 
revistas
serias se reduce a cero, y la docencia universitaria a unas pocas asignaturas
gen\'ericas -- y que ni siquiera son obligatorias --; por supuesto, no existe
ning\'un programa de doctorado relacionado con la materia (excepto, una vez 
m\'as, y afortunadamente, con la criptograf\'{\i}a). De esta forma, si la mayor
parte de los inform\'aticos salen de las facultades sin conocer conceptos tan
b\'asicos como {\it sniffer} o caballo de Troya (ya no hablamos de cosas como 
esteganograf\'{\i}a o seguridad multinivel), no es de extra\~nar que la 
seguridad se encuentre actualmente (en la mayor parte de los casos) en manos de 
aficionados a la inform\'atica
con ciertos conocimientos pr\'acticos pero con una importante falta de bases 
te\'oricas sobre la materia. Si lo que queremos son sistemas inseguros y 
reportajes sensacionalistas sobre quincea\~neros que violan la seguridad de La
Moncloa, lo estamos consiguiendo\ldots pero quiz\'as deber\'{\i}amos 
plantearnos qu\'e ha de pasar para que esto cambie.
\vspace{1cm}\\
\makebox[13cm][l]{\parbox[t]{13cm}{\hspace{8cm}Valencia, mayo de 2002}}
