%\documentclass{article}
%\oddsidemargin 0in
%\evensidemargin 0in
%\textheight 685pt
%\textwidth 430pt
%\topmargin 0in 
%\headsep 9pt
%\parindent 9pt
%\makeindex
%\begin{document}
%\title{Mecanismos b\'asicos de seguridad para administradores de m\'aquinas 
%Unix}
%\author{Antonio Villal\'on Huerta {\tt $<$toni@aiind.upv.es$>$}}
%\date{Noviembre, 1999}
%\maketitle
\chapter{Seguridad b\'asica para administradores}
\section{Introducci\'on}
Lamentablemente, muchos administradores de equipos Unix no disponen de los
conocimientos, del tiempo, o simplemente del inter\'es necesario para 
conseguir sistemas m\'{\i}nimamente fiables. A ra\'{\i}z de esto, las 
m\'aquinas Unix se convierten en una puerta abierta a cualquier ataque, poniendo
en peligro no s\'olo la integridad del equipo, sino de toda su subred y a la 
larga de toda Internet.\\
\\Aunque esta situaci\'on se da en cualquier tipo de organizaci\'on, es en las
dedicadas a I+D donde se encuentran los casos m\'as extremos; se trata de redes 
y equipos Unix muy abiertos y con un elevado n\'umero de usuarios (incluidos 
externos al per\'{\i}metro f\'{\i}sico de la organizaci\'on) que precisan de 
una gran disponibilidad de los datos, primando este aspecto de la informaci\'on 
ante otros como la integridad o la privacidad. Esto convierte a los sistemas 
Unix de centros de I+D, especialmente de universidades, en un objetivo 
demasiado f\'acil incluso para los piratas menos experimentados.\\
\\Con el objetivo de subsanar esta situaci\'on, aqu\'{\i} se van a intentar 
marcar unas pautas para conseguir un nivel {\bf m\'{\i}nimo} de fiabilidad en 
los equipos Unix. No se va a entrar en detalles muy t\'ecnicos o en desarrollos 
te\'oricos sobre seguridad que muy pocos van a leer (para eso est\'a el resto
de este proyecto), sino que la idea es 
\'unicamente explicar los pasos b\'asicos para que incluso los administradores 
menos preocupados por la seguridad puedan aplicarlos en sus sistemas. A modo de
ilustraci\'on, hay peque\~nos ejemplos que han sido realizados sobre una 
plataforma Solaris 7 (SunOS 5.7); en otros clones de Unix quiz\'as sea 
necesario modificar las opciones de alg\'un comando o la localizaci\'on de 
ciertos ficheros.\\
\\Hay que recalcar que se trata de mecanismos {\bf b\'asicos} de seguridad, que
pueden evitar la acci\'on de algunos piratas casuales (si nuestra m\'aquina
ofrece una m\'{\i}nima protecci\'on abandonar\'an el ataque para dedicarse a
equipos menos protegidos) pero no de un atacante con cierta experiencia.
Lo ideal ser\'{\i}a que las pautas marcadas aqu\'{\i} se complementaran con 
todas las medidas de seguridad posibles, y que entre los libros habituales de 
un administrador se encontraran t\'{\i}tulos sobre seguridad en Unix; uno 
especialmente recomendado es {\it Practical Unix \& Internet Security}, de
Simson Garfinkel y Gene Spafford (Ed. O\'{}Reilly and Associates, 1996). 
Tambi\'en es muy recomendable que la persona encargada de la seguridad de cada
equipo permanezca atenta a los nuevos problemas que cada d\'{\i}a surgen; una
buena forma de conseguirlo es mediante listas de correo como {\sc bugtraq}.
\section{Prevenci\'on}
Los mecanismos de prevenci\'on han de ser los m\'as importantes para cualquier
administrador, ya que obviamente es mucho mejor evitar un ataque que detectar
ese mismo problema o tener que recuperar al sistema tras detectarlo.
\begin{itemize}
\item Cierre de servicios ofrecidos por {\tt inetd}\\
Cada servicio ofrecido en nuestro sistema se convierte en una potencial puerta
de acceso al mismo, por lo que hemos de minimizar su n\'umero: se recomienda 
cerrar cualquier servicio que no se vaya a utilizar, y todos aquellos de los 
que no conozcamos su utilidad (si m\'as tarde son necesarios, los podemos 
volver a abrir).\\ 
Para cerrar un servicio ofrecido desde {\tt inetd}, en el fichero {\tt 
/etc/inetd.conf} debemos comentar la l\'{\i}nea correspondiente a ese servicio,
de forma que una entrada como 
\tt
\begin{quote}
\begin{verbatim}
telnet  stream  tcp     nowait  root    /usr/sbin/in.telnetd
\end{verbatim}
\end{quote}
\rm
se convierta en una como 
\tt
\begin{quote}
\begin{verbatim}
#telnet  stream  tcp     nowait  root    /usr/sbin/in.telnetd
\end{verbatim}
\end{quote}
\rm
Tras efectuar esta operaci\'on, debemos reiniciar el demonio {\tt inetd} para
que relea su configuraci\'on; esto lo conseguimos, por ejemplo, con la orden
\tt
\begin{quote}
\begin{verbatim}
anita:/# pkill -HUP inetd
\end{verbatim}
\end{quote}
\rm
o, si no disponemos de un comando para enviar se\~nales a procesos a partir de
su nombre, con la orden
\tt
\begin{quote}
\begin{verbatim}
anita:/# kill -HUP `ps -ef|grep -w inetd|awk '{print $2}'`
\end{verbatim}
\end{quote}
\rm
\item Cierre de servicios ofrecidos en el arranque de m\'aquina\\
Existen una serie de demonios que ofrecen ciertos servicios, como {\tt 
sendmail}, que no se procesan a trav\'es de {\tt inetd} sino que se lanzan
como procesos independientes al arrancar la m\'aquina. Para detener este tipo
de demonios hemos de comentar las l\'{\i}neas de nuestros ficheros de arranque
encargadas de lanzarlos (generalmente en directorios como {\tt /etc/rc?.d/} o
{\tt /etc/rc.d/}): de esta forma conseguimos que la pr\'oxima vez que el sistema
se inicie, los demonios no se ejecuten. Aparte de esto, hemos de detener los
demonios en la sesi\'on actual, ya que en estos momentos seguramente est\'an
funcionando; para ello les enviamos la se\~nal {\sc sigkill} mediante el 
comando {\tt kill}.\\
Por ejemplo, en el caso de Solaris, {\tt sendmail} se lanza desde el archivo\\
{\tt /etc/rc2.d/S88sendmail}; en este fichero tendremos unas l\'{\i}neas 
similares a estas:
\tt
\begin{quote}
\begin{verbatim}
if [ -f /usr/lib/sendmail -a -f /etc/mail/sendmail.cf ]; then
   if [ ! -d /var/spool/mqueue ]; then
      /usr/bin/mkdir -m 0750 /var/spool/mqueue
      /usr/bin/chown root:bin /var/spool/mqueue
   fi
   /usr/lib/sendmail -bd -q15m &
fi
\end{verbatim}
\end{quote}
\rm
Podemos renombrar este archivo como {\tt disabled.S88sendmail} o comentar estas
l\'{\i}neas de la forma siguiente:
\tt
\begin{quote}
\begin{verbatim}
#if [ -f /usr/lib/sendmail -a -f /etc/mail/sendmail.cf ]; then
#   if [ ! -d /var/spool/mqueue ]; then
#      /usr/bin/mkdir -m 0750 /var/spool/mqueue
#      /usr/bin/chown root:bin /var/spool/mqueue
#   fi
#   /usr/lib/sendmail -bd -q15m &
#fi
\end{verbatim}
\end{quote}
\rm
Y a continuaci\'on eliminaremos el proceso {\tt sendmail} envi\'andole la 
se\~nal {\sc sigkill}:
\tt
\begin{quote}
\begin{verbatim}
anita:/# ps -ef |grep sendmail
root   215     1  0 01:00:38 ?        0:00 /usr/lib/sendmail -bd -q15m
anita:/# kill -9 215
\end{verbatim}
\end{quote}
\rm
\item Instalaci\'on de {\it wrappers}\\
A pesar de haber cerrado muchos servicios siguiendo los puntos anteriores, 
existen algunos que no podremos dejar de ofrecer, como {\tt telnet} o {\tt
ftp}, ya que los usuarios van a necesitar conectar al sistema de forma remota o
transferir ficheros. En estos casos es muy conveniente instalar {\it wrappers} 
para los demonios que sigan recibiendo conexiones; mediante el uso de estos
programas vamos a poder restringir los lugares desde los que nuestro equipo
va a aceptar peticiones de servicio. {\bf Especialmente recomendable} es el 
programa {\it TCP-Wrapper} para controlar las conexiones servidas por {\tt 
inetd} (incluso {\tt sendmail} se puede controlar por {\tt inetd}, lo cual es 
muy \'util si queremos restringir los lugares desde los que nos pueda llegar 
correo).\\
Por ejemplo, si no utilizamos {\it wrappers} para controlar el servicio de {\tt
telnet}, cualquier m\'aquina de Internet puede intentar el acceso a nuestro
sistema:
\tt
\begin{quote}
\begin{verbatim}
luisa:~$ telnet anita
Trying 192.168.0.3...
Connected to anita.
Escape character is '^]'.

SunOS 5.7

login:
\end{verbatim}
\end{quote}
\rm
Sin embargo, configurando {\it TCP-Wrapper} para que no admita conexiones desde 
fuera de la universidad, si alguien intenta lo mismo obtendr\'a un resultado
similar al siguiente:
\tt
\begin{quote}
\begin{verbatim}
luisa:~$ telnet anita
Trying 192.168.0.3...
Connected to anita.
Escape character is '^]'.
Connection closed by foreign host.
luisa:~$
\end{verbatim}
\end{quote}
\rm
De esta forma, incluso si el atacante conociera un nombre de usuario y su clave
le ser\'{\i}a m\'as dif\'{\i}cil acceder a nuestro equipo por {\it telnet}.
\item Ficheros setuidados y setgidados\\
En un sistema Unix reci\'en instalado podemos tener incluso m\'as de cincuenta
ficheros con los modos {\it setuid} o {\it setgid} activados; cualquiera de
estos programas representa un potencial agujero a la seguridad de nuestro 
sistema, y aunque muchos son necesarios (como {\tt /bin/passwd} en la 
mayor\'{\i}a de situaciones), de otros se puede prescindir. Para localizar los
ficheros setuidados podemos utilizar la orden
\tt
\begin{quote}
\begin{verbatim}
anita:/# find / -perm -4000 -type f -print
\end{verbatim}
\end{quote}
\rm
mientras que para localizar los setgidados podemos utilizar 
\tt
\begin{quote}
\begin{verbatim}
anita:/# find / -perm -2000 -type f -print
\end{verbatim}
\end{quote}
\rm
Es conveniente que reduzcamos al m\'{\i}nimo el n\'umero de estos archivos, pero
tampoco se recomienda borrarlos del sistema de ficheros; es mucho m\'as habitual
resetear el bit de {\it setuid} o {\it setgid}, y en caso de que sea necesario
volverlo a activar. Para desactivar estos bits podemos usar la orden {\tt
chmod -s}, mientras que para activarlos utilizaremos {\tt chmod u+s} o {\tt 
chmod g+s}.\\
Por ejemplo, si el fichero {\tt /usr/lib/fs/ufs/ufsdump} est\'a {\it setuidado},
un listado largo del mismo nos mostrar\'a una {\tt s} en el campo de ejecuci\'on
para propietario, mientras que si est\'a {\it setgidado} aparecer\'a una {\tt
s} en el campo de ejecuci\'on para grupo; podemos resetear los dos bits con
la orden vista anteriormente:
\tt
\begin{quote}
\begin{verbatim}
anita:/# ls -l /usr/lib/fs/ufs/ufsdump
-r-sr-sr-x  1 root  tty   144608 Oct 6 1998 /usr/lib/fs/ufs/ufsdump
anita:/# chmod -s /usr/lib/fs/ufs/ufsdump
anita:/# ls -l /usr/lib/fs/ufs/ufsdump
-r-xr-xr-x  1 root  tty   144608 Oct 6 1998 /usr/lib/fs/ufs/ufsdump
\end{verbatim}
\end{quote}
\rm
\item Cifrado de datos\\
El principal problema de las claves viajando en texto claro por la red es que
cualquier atacante puede leerlas: si usamos {\tt telnet}, {\tt rlogin} o {\tt
ftp}, cualquier persona situada entre nuestra estaci\'on de trabajo y el 
servidor al que conectamos puede `esnifar' los paquetes que circulan por la 
red y obtener as\'{\i} nuestro nombre de usuario y nuestro {\it password}.
Para evitar este problema es conveniente utilizar {\it software} que implemente
protocolos cifrados para conectar; el m\'as habitual hoy en d\'{\i}a es {\sc
ssh} ({\it Secure Shell}). Por una parte, tenemos el programa servidor {\tt 
sshd}, que se ha de instalar en el equipo al que conectamos, y por otra 
programas clientes ({\tt ssh} para sustituir a {\tt rsh/rlogin} y {\tt scp} 
para sustituir a {\tt rcp}).\\
Una vez instalado, este software es transparente al usuario: simplemente 
ha de recordar su clave, igual que si conectara por {\tt telnet} o {\tt rlogin}.
\item Relaciones de confianza\\
En el fichero {\tt /etc/hosts.equiv} se indican, una en cada l\'{\i}nea, las 
m\'aquinas confiables.
>Qu\'e significa {\it confiables}? B\'asicamente que confiamos en su seguridad
tanto como en la nuestra, por lo que para facilitar la compartici\'on de
recursos, no se van a pedir contrase\~nas a los usuarios que quieran conectar
desde estas m\'aquinas con el mismo {\it login}, utilizando las \'ordenes
{\sc bsd} {\tt r$\ast$} ({\tt rlogin, rsh, rcp}\ldots). Por ejemplo, si en el
fichero {\tt /etc/hosts.equiv} del servidor {\tt anita} hay una entrada
para el nombre
de {\it host} {\tt luisa}, cualquier usuario\footnote{Excepto el {\it
root}.} de este sistema puede ejecutar una orden como la siguiente para
conectar a {\tt anita} {\bf <sin necesidad de ninguna clave!}:
\tt
\begin{quote}
\begin{verbatim}
luisa:~$ rlogin anita
Last login: Sun Oct 31 08:27:54 from localhost
Sun Microsystems Inc.   SunOS 5.7       Generic October 1998
anita:~$
\end{verbatim}
\end{quote}
\rm
Obviamente, esto supone un gran problema de seguridad, por lo que lo m\'as
recomendable es que el fichero {\tt /etc/hosts.equiv} est\'e vac\'{\i}o o no
exista. De la misma forma, los usuarios pueden crear ficheros {\tt
\$HOME/.rhosts} para establecer un mecanismo de confiabilidad bastante similar
al de {\tt /etc/hosts.equiv}; es importante para la seguridad de nuestro
sistema el controlar la existencia y el contenido de estos archivos {\tt
.rhosts}. Por ejemplo, podemos aprovechar las facilidades de planificaci\'on de
tareas de Unix para, cada cierto tiempo, chequear los directorios {\tt \$HOME}
de los usuarios en busca de estos ficheros, elimin\'andolos si los
encontramos. Un {\it shellscript} que hace esto puede ser el siguiente:
\tt
\begin{quote}
\begin{verbatim}
#!/bin/sh
for i in `cat /etc/passwd |awk -F: '{print $6}'`; do
    cd $i
    if [ -f .rhosts ]; then
        echo "$i/.rhosts detectado"|mail -s "rhosts" root
        rm -f $i/.rhosts
    fi
done
\end{verbatim}
\end{quote}
\rm
Este programa env\'{\i}a un correo al {\it root} en caso de encontrar un
fichero {\tt .rhosts}, y lo elimina; podemos planificarlo mediante {\tt cron}
para que se ejecute, por ejemplo, cada cinco minutos. La forma de planificarlo
depende del clon de Unix en el que trabajemos, por lo que se recomienda
consultar la p\'agina del manual de {\tt cron} o {\tt crond}; en el caso de
Solaris, para que se ejecute cada vez que {\tt cron} despierte, y suponiendo
que el {\it script} se llame {\tt /usr/local/sbin/busca}, pondr\'{\i}amos
en nuestro {\it crontab} (con {\tt crontab -e}) una l\'{\i}nea como
\tt
\begin{quote}
\begin{verbatim}
* * * * *      /usr/local/sbin/busca 2>&1 >/dev/null
\end{verbatim}
\end{quote}
\rm
Hemos de estar atentos a la carga que este tipo de actividades peri\'odicas 
puede introducir en el sistema; la orden anterior se va a ejecutar cada vez
que {\tt cron} despierta, generalmente una vez por minuto, lo que implica que
en m\'aquinas con un gran n\'umero de usuarios puede introducir un factor 
importante de operaciones de I/O. Una soluci\'on m\'as adecuada en estas 
situaciones ser\'{\i}a planificar el programa para que se ejecute cada cinco o
diez minutos, o el tiempo que estimemos necesario en nuestro equipo. 
\item Pol\'{\i}tica de cuentas\\
Muchos clones de Unix se instalan con cuentas consideradas `del sistema', es
decir, que no corresponden a ning\'un usuario concreto sino que existen por
cuestiones de compatibilidad o para la correcta ejecuci\'on de algunos 
programas. Algunas de estas cuentas no tienen contrase\~na, o tienen una
conocida por todo el mundo, por lo que representan una grave amenaza a la 
seguridad: hemos de deshabilitarlas para evitar que alguien pueda conectar
a nuestro equipo mediante ellas. Algunos ejemplos de este tipo de cuentas
son {\tt guest}, {\tt demo}, {\tt uucp}, {\tt games}, {\tt 4DGifts} o {\tt 
lp}.\\
Para deshabilitar una cuenta, en Unix no tenemos m\'as que insertar un asterisco
en el campo {\it passwd} en la l\'{\i}nea correspondiente del fichero de claves 
(generalmente {\tt /etc/passwd} o\\ {\tt /etc/shadow}). De esta forma, una 
entrada como 
\tt
\begin{quote}
\begin{verbatim}
toni:7atzxSJlPVVaQ:1001:10:Toni Villalon:/export/home/toni:/bin/sh
\end{verbatim}
\end{quote}
\rm
pasar\'{\i}a a convertirse en 
\tt
\begin{quote}
\begin{verbatim}
toni:*7atzxSJlPVVaQ:1001:10:Toni Villalon:/export/home/toni:/bin/sh
\end{verbatim}
\end{quote}
\rm
Aparte de este tipo de cuentas, hemos de tener un especial cuidado con las 
cuentas de usuario que no tienen contrase\~na o que tienen una clave d\'ebil;
para detectar este \'ultimo problema podemos utilizar programas adivinadores 
como {\it Crack}, mientras que para evitarlo podemos utilizar {\it NPasswd} o
{\it Passwd+}, adem\'as de sistemas {\it Shadow Password} para que los usuarios
no puedan leer las claves cifradas. Para detectar cuentas sin contrase\~na 
(aunque tambi\'en {\it Crack} nos
las indicar\'a), podemos utilizar la siguiente orden, obviamente sustituyendo
{\tt /etc/passwd} por el fichero de claves de nuestro sistema:
\tt
\begin{quote}
\begin{verbatim}
anita:~# awk -F: '$2=="" {print $1}' /etc/passwd
\end{verbatim}
\end{quote}
\rm
Por \'ultimo, hay que decir que una correcta pol\'{\i}tica de cuentas pasa por
deshabilitar la entrada de usuarios que no utilicen el sistema en un tiempo
prudencial; por ejemplo, podemos cancelar las cuentas de usuarios que no 
hayan conectado a la m\'aquina en los \'ultimos dos meses, ya que son firmes
candidatas a que un pirata las aproveche para atacarnos; la orden {\tt finger}
nos puede ayudar a detectar este tipo de usuarios.
\item Negaciones de servicio\\
Un tipo de ataque que ni siquiera suele necesitar de un acceso al sistema es el
conocido como la negaci\'on de servicio ({\it Denial of Service}). Consiste 
b\'asicamente en perjudicar total o parcialmente la disponibilidad de un 
recurso, por ejemplo utilizando grandes cantidades de CPU, ocupando toda la 
memoria del sistema o incluso deteniendo una m\'aquina. Obviamente las 
negaciones de servicio m\'as peligrosas son las que detienen el sistema o 
alguno de sus servicios de forma remota:\\
Las paradas de m\'aquina
son, por norma general, fruto de un fallo en la implementaci\'on de red del 
n\'ucleo: por ejemplo, la llegada de un paquete con una cabecera extra\~na, de
una longitud determinada, o con una cierta prioridad, puede llegar a detener la
m\'aquina si ese paquete no se trata correctamente en la implementaci\'on del
sistema de red. La mejor forma de prevenir estos ataques (que no suelen dejar
ning\'un rastro en los ficheros de {\it log}) es actualizar el n\'ucleo de 
nuestro Unix peri\'odicamente, manteniendo siempre la \'ultima versi\'on 
estable.\\
En el caso de la detenci\'on de servicios determinados, habitualmente los 
ofrecidos desde {\tt inetd}, la negaci\'on de servicio suele ser fruto de un 
excesivo n\'umero de peticiones `falsas' al demonio correspondiente; por 
ejemplo, un atacante enmascara su direcci\'on IP para sobrecargar de peticiones
un demonio como {\tt in.telnetd} hasta detenerlo. Para evitar estos ataques 
podemos incrementar el n\'umero de peticiones simult\'aneas que un demonio 
acepta (en la opci\'on {\tt wait} de la l\'{\i}nea correspondiente en el 
fichero {\tt /etc/inetd.conf}), aunque esto tambi\'en implica peligros de 
negaci\'on de servicio (puede aumentar demasiado el tiempo de respuesta del
equipo); una forma mucho m\'as recomendable de actuar no es prevenir estos 
ataques sino minimizar sus efectos: si enviamos una se\~nal {\sc sighup} al
demonio {\tt inetd} \'este relee su configuraci\'on, por lo que el servicio
bloqueado vuelve a funcionar\footnote{Obviamente, las conexiones ya 
establecidas no se pierden.}. Por tanto, es recomendable enviar una de estas
se\~nales de forma autom\'atica cada cierto tiempo; podemos planificar esta
acci\'on para que {\tt cron} la ejecute cada vez que despierte, incluyendo en
nuestro archivo {\it crontab} una l\'{\i}nea como la siguiente:
\tt
\begin{quote}
\begin{verbatim}
* * * * *        /usr/bin/pkill -HUP inetd 2>&1 >/dev/null
\end{verbatim}
\end{quote}
\rm
\end{itemize}
\section{Detecci\'on}
La mayor\'{\i}a de problemas de seguridad en sistemas de I+D implican accesos
no autorizados, bien por usuarios externos a la m\'aquina o bien por usuarios
internos que consiguen un privilegio mayor del que tienen asignado.
>C\'omo detectar estos problemas? Hacer esto puede ser algo muy complicado si 
el atacante es un pirata con cierta experiencia y no hemos tomado algunas 
medidas en nuestro sistema antes de que el ataque se produzca. Aqu\'{\i} se 
presentan unos mecanismos que pueden indicar que alguien ha accedido 
ilegalmente a nuestro equipo.
\begin{itemize}
\item Logs\\
Casi cualquier actividad dentro del sistema es susceptible de ser monitorizada
en mayor o menor medida. Unix ofrece un estupendo sistema de {\it logs} que
guarda informaci\'on en ficheros contenidos generalmente en {\tt /var/adm/}, 
{\tt /var/log/} o {\tt /usr/adm/}; esta informaci\'on var\'{\i}a desde los 
usuarios que han conectado al sistema \'ultimamente hasta los mensajes de error
del n\'ucleo, y puede ser consultada con \'ordenes como {\tt who} o {\tt last},
o con un simple editor de textos.\\
Aunque un atacante que consiga privilegios de {\tt root} en el equipo puede
modificar (<o borrar!) estos archivos\footnote{O cualquier usuario con permiso 
de escritura en ellos; los usuarios ni siquiera han de tener permiso de lectura 
en la 
mayor\'{\i}a de los ficheros de {\it log}.}, los {\it logs} son con frecuencia 
el primer indicador de un acceso no autorizado o de un intento del mismo. 
Dependiendo de nuestra configuraci\'on ({\tt /etc/syslog.conf}), pero 
generalmente en los archivos {\tt messages} o {\tt syslog}, podemos ver mensajes
que pueden indicar un ataque al sistema; a continuaci\'on se presentan algunos
de ellos:
\begin{itemize}
\item \tt
\small{Nov 12 05:35:42 anita in.telnetd[516]: refused connect from bg.microsoft.com}\\
\rm
\normalsize
Este mensaje (conexi\'on rehusada a un servicio) en sistemas con {\it 
TCP-Wrappers} instalado indica que alguien ha intentado conectar a nuestro 
equipo desde una m\'aquina no autorizada a hacerlo.
\item \tt
\small{Nov  7 23:06:22 anita in.telnetd[2557]: connect from localhost}\\
\rm
\normalsize
Alguien ha conectado con \'exito a nuestro equipo desde una determinada 
m\'aquina; no implica que haya accedido con una nombre de usuario y su 
contrase\~na, simplemente que ha tenido posibilidad de hacerlo.
\item \tt
\small{SU 11/17 03:12 - pts/3 toni-root}\\
\rm
\normalsize
El usuario {\tt toni} ha intentado conseguir privilegios de administrador 
mediante la orden {\tt su}; si lo hubiera conseguido, en lugar de un signo 
{\tt `-'} aparecer\'{\i}a un {\tt `+'}. En Solaris, esto se registra en el
fichero {\tt sulog}, aunque en el fichero {\tt messages} se notifica si el 
{\tt su} ha fallado.   
\end{itemize}
\item Procesos\\
Si un atacante ha conseguido acceso a nuestro equipo, y dependiendo de sus
intenciones, es probable que ejecute programas en el sistema que permanecen
en la tabla de procesos durante un largo periodo de tiempo; t\'{\i}picos 
ejemplos son {\it sniffers} (programas para capturar tr\'afico de red) o {\it 
bouncers} (programas para ocultar una direcci\'on en {\sc irc}). Debemos 
desconfiar de procesos que presenten un tiempo de ejecuci\'on elevado, 
especialmente si no es lo habitual en nuestro sistema. Incluso si el nombre
del proceso no es nada extra\~no (obviamente un pirata no va a llamar a su
analizador de tr\'afico {\tt sniffer}, sino que le dar\'a un nombre que no 
levante sospechas, como {\tt xzip} o {\tt ltelnet}) es muy conveniente que
nos preocupemos de comprobar cu\'al es el programa que se est\'a ejecutando.\\
Algo que nos puede ayudar mucho en esta tarea es la herramienta de seguridad
{\tt lsof}, que nos indica los ficheros abiertos por cada proceso de nuestro
sistema, ya que programas como los {\it sniffers} o los {\it crackers} de claves
suelen mantener archivos abiertos para almacenar la informaci\'on generada.
\item Sistemas de ficheros\\
Otro punto que puede denotar actividades sospechosas en la m\'aquina es su
sistema de ficheros:\\ 
Por un lado, hemos de estar atentos al n\'umero de archivos
setuidados en el sistema: es frecuente que un pirata que ha conseguido 
privilegios de {\it root} guarde archivos con este bit activo para volver a
conseguir esos privilegios de una forma m\'as sencilla (por ejemplo, una copia
de un {\it shell} setuidado como {\tt root} dar\'a privilegios de administrador
a cualquiera que lo ejecute).\\
Adem\'as, los intrusos suelen crear directorios `dif\'{\i}ciles' de localizar,
donde poder guardar herramientas de ataque: por ejemplo, si alguien es capaz
de crear el directorio {\tt /dev/.../}, seguramente cuando el administrador
haga un listado de {\tt /dev/} ni se dar\'a cuenta de la existencia de un
directorio con un nombre tan poco com\'un como {\tt `...'}.\\
Otra actividad relacionada con el sistema de ficheros es la sustituci\'on de
ciertos programas que puedan delatar una presencia extra\~na, como {\tt ps},
{\tt who} o {\tt last}, o programas cr\'{\i}ticos como {\tt /bin/login} por
versiones `troyanizadas' que no muestren nada relacionado con el atacante; por
ejemplo, alguien podr\'{\i}a sustituir el programa {\tt /bin/login} por otro
que aparentemente se comporte igual, pero que al recibir un nombre de usuario
concreto otorgue acceso al sistema sin necesidad de clave. Un ejemplo muy 
simple de este tipo de troyanos es el siguiente: alguien mueve el archivo 
{\tt /bin/ps} a {\tt /bin/OLDps} y a continuaci\'on ejecuta
\tt
\begin{quote}
\begin{verbatim}
anita:~# cat >/bin/ps
#!/bin/sh
/bin/OLDps $1|grep -v '^    toni'|grep -v grep|grep -v OLD
^D
anita:~#
\end{verbatim}
\end{quote}
\rm
A partir de ahora, cuando alguien teclee {\tt ps -ef} no ver\'a los procesos
del usuario {\tt toni}. Se puede seguir un mecanismo similar\footnote{Realmente
el mecanismo suele ser m\'as elaborado; aqu\'{\i} se ha utilizado una forma 
muy simple de ocultaci\'on \'unicamente a modo de ejemplo.} con programas como 
{\tt w}, {\tt finger}, {\tt last}, {\tt who} o {\tt ls} para conseguir ocultar
a un usuario conectado, sus procesos, sus ficheros\ldots\\ 
Otro s\'{\i}ntoma que denota la presencia de un problema de seguridad puede
ser la modificaci\'on de ciertos ficheros importantes del sistema; por ejemplo,
un atacante puede modificar {\tt /etc/syslog.conf} para que no se registren
ciertos mensajes en los archivos de {\it log}, o {\tt /etc/exports} para 
exportar directorios de nuestro equipo. El problema de este estilo m\'as 
frecuente es la generaci\'on de nuevas entradas en el fichero de claves con 
{\sc uid} 0 (lo que implica un total privilegio); para detectar este tipo de
entradas, se puede utilizar la siguiente orden: 
\tt
\begin{quote}
\begin{verbatim}
anita:~# awk -F: '$3=="0" {print $1}' /etc/passwd
root
anita:~#
\end{verbatim}
\end{quote}
\rm
Obviamente, si como salida de la orden anterior obtenemos alg\'un otro nombre
de usuario, aparte del {\it root}, ser\'{\i}a conveniente cancelar la cuenta
de ese usuario e investigar por qu\'e aparece con {\sc uid} 0.\\
Detectar este tipo de problemas con el sistema de ficheros de nuestro equipo 
puede ser una tarea complicada; la soluci\'on m\'as r\'apida pasa por instalar
{\it Tripwire}, comentado en este mismo punto.
\item Directorios de usuarios\label{userdir}\\
Dentro del sistema de ficheros existen unos directorios especialmente 
conflictivos: se trata de los {\tt \$HOME} de los usuarios. La conflictividad
de estos directorios radica principalmente en que es la zona m\'as importante 
del sistema de archivos donde los usuarios van a tener permiso de escritura,
por lo que su control (por ejemplo, utilizando {\it Tripwire}) es a priori 
m\'as dif\'{\i}cil que el de directorios cuyo contenido no cambie tan 
frecuentemente. Algunos elementos dentro de estos directorios que pueden 
denotar una intrusi\'on son los siguientes:
\begin{itemize}
\item Hemos de chequear el grupo y propietario de cada archivo para comprobar
que no pertenecen a usuarios privilegiados en lugar de a usuarios normales, o a 
grupos especiales en lugar de a grupos gen\'ericos ({\tt users}, {\tt 
staff}\ldots).  Por ejemplo, si el padre de los directorios de usuario es {\tt 
/export/home/}, podemos buscar dentro de \'el ficheros que pertenezcan al 
administrador con la orden
\tt
\begin{quote}
\begin{verbatim}
anita:~# find /export/home/ -user root -type f -print
\end{verbatim}
\end{quote}
\rm
\item >Hay archivos setuidados o setgidados en los directorios de usuario? No
deber\'{\i}a, por lo que su existencia es algo bastante sospechoso\ldots
\item La existencia de c\'odigo fuente, generalmente C, de {\it exploits}
(programas que aprovechan un fallo de seguridad en el {\it software} para
utilizarlo en beneficio del atacante) puede ser
indicativo de una contrase\~na robada, o de un usuario intentando conseguir
un privilegio mayor en el sistema. >C\'omo saber si un c\'odigo es un {\it
exploit} o una pr\'actica de un alumno? La respuesta es obvia: ley\'endolo.
>Y si se trata de ficheros ejecutables en lugar de c\'odigo fuente? {\tt man
strings}.
\end{itemize}
\item El sistema de red\\
Estar atentos al sistema de red de nuestro equipo tambi\'en nos puede 
proporcionar indicios de accesos no autorizados o de otro tipo de ataques contra
el sistema. Por ejemplo, si utilizamos {\tt netstat} para comprobar las 
conexiones activas, y detectamos una entrada similar a
\tt\small
\begin{quote}
\begin{verbatim}
anita.telnet         luisa.2039           16060      0 10136      0 ESTABLISHED
\end{verbatim}
\end{quote}
\rm\normalsize
esto indica que desde el {\it host} {\it luisa} alguien est\'a conectado a
nuestro sistema mediante {\it telnet}; puede haber accedido o no (si ha tecleado
un nombre de usuario y la contrase\~na correcta), pero el hecho es que la
conexi\'on est\'a establecida.\\
Otro m\'etodo muy seguido por los piratas es asegurar la reentrada al sistema
en caso de ser descubiertos, por ejemplo instalando un programa que espere 
conexiones en un cierto puerto y que proporcione un {\it shell} sin necesidad 
de {\it login} y {\it password} (o con los mismos predeterminados); por 
ejemplo, si el programa espera peticiones
en el puerto 99, el atacante puede acceder al sistema con un simple {\it 
telnet}:
\tt
\begin{quote}
\begin{verbatim}
luisa:~# telnet anita 99
Trying 192.168.0.3...
Connected to anita.
Escape character is '^]'.
Sun Microsystems Inc.   SunOS 5.7       Generic October 1998
anita:~#
\end{verbatim}
\end{quote}
\rm
Podemos detectar los puertos que esperan una conexi\'on en nuestro sistema
tambi\'en con la orden {\tt netstat}:
\tt
\small
\begin{quote}
\begin{verbatim}
anita:~# netstat -P tcp -f inet -a|grep LISTEN
      *.sunrpc             *.*                0      0     0      0 LISTEN
      *.32771              *.*                0      0     0      0 LISTEN
      *.ftp                *.*                0      0     0      0 LISTEN
      *.telnet             *.*                0      0     0      0 LISTEN
      *.finger             *.*                0      0     0      0 LISTEN
      *.dtspc              *.*                0      0     0      0 LISTEN
      *.lockd              *.*                0      0     0      0 LISTEN
      *.smtp               *.*                0      0     0      0 LISTEN
      *.8888               *.*                0      0     0      0 LISTEN
      *.32772              *.*                0      0     0      0 LISTEN
      *.32773              *.*                0      0     0      0 LISTEN
      *.32774              *.*                0      0     0      0 LISTEN
      *.printer            *.*                0      0     0      0 LISTEN
      *.listen             *.*                0      0     0      0 LISTEN
      *.6000               *.*                0      0     0      0 LISTEN
      *.32775              *.*                0      0     0      0 LISTEN
anita:~#
\end{verbatim}
\end{quote}
\rm
\normalsize
\item Tripwire\\
Quiz\'as una de las formas m\'as efectivas de detectar accesos no autorizados 
es mediante el programa {\it Tripwire}. La idea es sencilla: en un sistema 
`limpio' (por ejemplo, reci\'en instalado, antes de ser conectado a red) se 
aplica una funci\'on de resumen ({\it message digest}) sobre los ficheros 
importantes del equipo, (por ejemplo, ficheros en {\tt /etc/}, {\tt /bin/} o 
{\tt /sbin/}). Los resultados de este proceso se almacenan en un medio que a 
partir de ese momento ser\'a de s\'olo lectura, como un disco flexible 
protegido contra escritura o un CD-ROM, y peri\'odicamente volvemos a aplicar 
el resumen sobre los ficheros de nuestro equipo; si se detecta un cambio (por
ejemplo, una variaci\'on en el tama\~no, un cambio de propietario, la
desaparici\'on de un archivo\ldots), {\it Tripwire} nos lo va a indicar. Si no 
lo hemos realizado nosotros, como administradores, es posible (muy posible) que 
ese fichero haya sido modificado en beneficio propio por un intruso.
\end{itemize}
\section{Recuperaci\'on}
>Qu\'e hacer cuando se detecta una intrusi\'on en la m\'aquina? Muchos 
administradores se hacen esta pregunta cuando se dan cuenta de que su seguridad
ha sido quebrada. Por supuesto, en esta situaci\'on se pueden hacer muchas 
cosas, desde ignorar el hecho y dejar que el pirata haga lo que quiera en 
nuestro sistema (obviamente, esto no es recomendable) hasta intentar localizar
al intruso mediante denuncia y orden judicial para tracear la llamada; esto
tampoco es habitual, ya que es muy dif\'{\i}cil demostrar ante un juez que un 
atacante ha violado nuestra seguridad, por lo que s\'olo vamos a perder tiempo
y dinero. Lo habitual en entornos universitarios es intentar detectar si el
atacante pertenece a la comunidad universitaria (en cuyo caso se le puede 
sancionar), restaurar la integridad del equipo de forma que un ataque similar 
no vuelva a tener \'exito, y poner de nuevo la m\'aquina a trabajar (recordemos
que la disponibilidad suele ser lo m\'as importante en organizaciones de I+D). 
Pero, hagamos lo que hagamos, hay que cumplir una norma b\'asica: {\bf no 
ponernos nerviosos}; si no logramos mantener la calma podemos ser incluso m\'as
perjudiciales para el sistema que el propio intruso o podemos poner a \'este
nervioso, lo que puede convertir un simple fisgoneo en una p\'erdida 
irrecuperable de datos.\\
\\Desde el punto de vista de Unix, es posible que nos interese localizar el 
fallo de seguridad a\-pro\-ve\-cha\-do por el pirata para cerrarlo y que el problema
no vuelva a ocurrir; sin entrar en temas complejos como el {\it jailing} o la
simulaci\'on, una de las formas que tenemos para realizar esta tarea es 
mo\-ni\-to\-ri\-zar las actividades del intruso, incluso arriesgando la integridad del
sistema (podemos hacer una copia de seguridad por lo que pueda pasar). Para 
realizar esto, hemos de ser conscientes de que si alguien ha conseguido 
privilegios de administrador en la m\'aquina, puede haber modificado desde los
programas del sistema hasta las librer\'{\i}as din\'amicas, pasando incluso por
el subsistema de {\it accounting} de procesos. Por tanto, hemos de desconfiar
de los resultados que cualquier orden nos proporcione, ya que el intruso puede
haberlos modificado para ocultar sus actividades. Si queremos auditar las
actividades del atacante hemos de utilizar programas `nuevos', a ser posible
compilados est\'aticamente (sin dependencia de librer\'{\i}as din\'amicas): 
podemos utilizar versiones de c\'odigo fuente disponible para adecuarlas a 
nuestro sistema, compilarlas est\'aticamente en un sistema similar al atacado\footnote{El pirata tambi\'en puede haber modificado el compilador.},
y utilizarlas en la m\'aquina donde est\'a el intruso.\\
\\El proceso de modificar librer\'{\i}as din\'amicas habitualmente excede los 
conocimientos del atacante medio de entornos I+D; como adem\'as conseguir
programas est\'aticos para nuestro equipo suele
ser complejo y lento en la mayor\'{\i}a de situaciones, y un objetivo b\'asico
es devolver el sistema a su funcionamiento normal cuanto antes, lo habitual no
es utilizar programas compilados de forma est\'atica sino confiar en que el
intruso no haya modificado librer\'{\i}as din\'amicas y utilizar versiones
`limpias' de programas din\'amicos; por ejemplo, podemos utilizar los programas
originales del sistema operativo que se encuentran en el CD-ROM de instalaci\'on
del mismo.\\
\\Si en lugar de intentar monitorizar las actividades del intruso en el sistema
lo \'unico que queremos es echarlo de nuestra m\'aquina (esto tiene su parte
positiva, pero tambi\'en su parte negativa), hemos de tener siempre presente
que desconocemos lo que ha hecho; la forma m\'as efectiva de tirarlo de nuestro
equipo es desconectando el cable de red (mucho mejor que utilizar {\tt ifconfig}
para detener la tarjeta o {\tt shutdown} para parar el sistema, ya que el
atacante puede haber contaminado estos programas para que realicen una actividad
diferente a la que en teor\'{\i}a est\'an destinados). Pero no nos podemos 
limitar \'unicamente a desconectar el cable de red: el atacante puede tener
procesos activos en el sistema, bombas l\'ogicas, o simplemente tareas 
destructivas planificadas con {\tt at} o {\tt cron}; hemos de chequear todo
el sistema en busca de este tipo de amenazas. Un lugar interesante donde
el intruso nos puede causar un problema grave es en los ficheros de 
inicializaci\'on de la m\'aquina, situados generalmente en {\tt /etc/rc?.d/}
o {\tt /etc/rc.d/}.\\ 
\\Una vez detectado y solucionado el problema de seguridad hemos de restaurar
un estado fiable de la m\'aquina, esto es, un estado similar al que ten\'{\i}a
antes de ser atacada. Aunque en muchos lugares se indica restaurar una copia de
seguridad anterior al ataque, esto presenta graves problemas: realmente no
sabemos con certeza cuando se produjo el ataque al sistema, sino que en todo
caso sabemos cu\'ando se detect\'o el mismo, por lo que corremos el peligro de
utilizar una copia de seguridad que ya est\'e contaminada por el atacante. Es
mucho m\'as seguro reinstalar el sistema completo y actualizarlo para solucionar
los fallos que posibilitaron la entrada del intruso, por ejemplo aplicando 
{\it patches} sobre la versi\'on que hemos instalado.\\
\\Restaurar y actualizar el sistema operativo y sus programas puede ser una
tarea pesada, pero no implica ninguna complicaci\'on: con toda probabilidad 
tenemos a nuestra disposici\'on los CD-ROMs con el {\it software} que hemos
de instalar; los problemas reales 
surgen con los archivos de los usuarios: evidentemente, no podemos decirles que
para evitar posibles conflictos de seguridad se les han borrado sus archivos, 
sino que lo habitual va a ser mantener el estado de sus ficheros justo como
estaban durante el ataque o, en caso de que este haya eliminado o corrompido
informaci\'on, tal y como estaban exactamente antes del ataque. Por tanto, 
especialmente si mantenemos el estado de los ficheros de usuario, hay que
estar atentos a posibles problemas que estos puedan introducir en el sistema,
comentados en el apartado \ref{userdir}.\\
\section{Recomendaciones de seguridad para los usuarios}
Con frecuencia la parte m\'as complicada de una pol\'{\i}tica de seguridad es
concienciar a los usuarios de la necesidad de medidas b\'asicas de prevenci\'on
contra ataques. Demasiados usuarios opinan que las historias de {\it crackers}
que atacan ordenadores s\'olo suceden en las pel\'{\i}culas o en organizaciones
militares de alta seguridad; nada m\'as lejos de la realidad: en cualquier
universidad ocurren a diario incidentes de seguridad, de los que s\'olo una
peque\~na parte se detecta (y muchos menos se solucionan). Ser\'{\i}a pues
muy recomendable para el administrador imprimir una hoja con las medidas de 
seguridad b\'asicas o la pol\'{\i}tica del sistema, y entregar una copia a 
cada usuario al crear su cuenta.
\begin{itemize}
\item Contrase\~nas aceptables\\
Es conveniente que los usuarios elijan claves medianamente resistentes a ataques
de diccionario; una contrase\~na como {\tt patata} o {\tt valencia} es un 
gran agujero de seguridad para la m\'aquina, aunque el usuario que la usa no 
tenga ning\'un privilegio especial. Hemos de ver la seguridad como una cadena
cuya fuerza depende principalmente del eslab\'on m\'as d\'ebil: si falla \'este,
falla toda la cadena. Sin embargo, el problema de estas claves es que pueden
llegar a ser dif\'{\i}ciles de recordar, de forma que mucha gente opta por 
apuntarlas en el monitor de su estaci\'on o en la parte inferior de sus 
teclados; obviamente, esto es casi peor que el problema inicial, ya que como
administradores no podemos controlar estas situaciones la mayor parte de las
veces. Podemos (y ser\'{\i}a lo recomendable) recomendar a los usuarios que
utilicen combinaciones de may\'usculas, min\'usculas, n\'umeros y s\'{\i}mbolos
para generar sus claves, pero de forma que la combinaci\'on les pueda resultar
familiar: por ejemplo, combinar n\'umeros y letras de la matr\'{\i}cula del
coche con algunos s\'{\i}mbolos de separaci\'on; claves de este estilo
podr\'{\i}an ser {\tt V\#GF\&121}, {\tt @3289?DH} o {\tt JKnB0322}. Obviamente
estas claves son m\'as resistentes a un ataque que {\tt beatles}, pero tampoco
son seguras: las acabamos de escribir.
\item Confidencialidad de las claves\\
Hemos de concienciar a nuestros usuarios de que {\bf las contrase\~nas no se
comparten}: no es recomendable `prestar' su clave a otras personas, ajenas o
no al sistema, ni por supuesto utilizar la misma clave para diferentes 
m\'aquinas. Este \'ultimo punto muchas veces se olvida en sistemas de I+D, donde
el usuario se ve obligado a utilizar {\it passwords} para muchas actividades y
tiende invariablemente a usar la misma contrase\~na; incluso se utiliza la 
clave de acceso a un equipo Unix para autenticarse en juegos de red (MUDs o
IRC) o, lo que es igual de grave, para acceder a equipos Windows, de forma que
las vulnerabilidades de seguridad de estos sistemas se trasladan a Unix.
\item Conexiones cifradas\\
Hay que potenciar entre los usuarios el uso de programas como {\tt ssh/scp} o\\ 
{\tt ssl-telnet/ssl-ftp} para conectar al equipo. La parte cliente de estos
programas es muy simple de utilizar, y nos puede ahorrar muchos dolores de 
cabeza como administradores. Incluso existen clientes para Windows y MacOS, por
lo que nadie tiene excusa para no usar sistemas cifrados (se puede conseguir 
que su uso sea completamente {\bf transparente} al usuario); casi la mejor 
forma de que los usuarios los utilicen es dejando de ofrecer ciertos servicios
sin cifra, como {\tt telnet}, {\tt ftp}, {\tt rlogin} o {\tt rsh}.
\item Ejecuci\'on de programas\\
Nunca, bajo ning\'un concepto, instalar o ejecutar {\it software} que no
provenga de fuentes fiables; hay que prestar atenci\'on especial a programas
que nos env\'{\i}en por correo o por {\sc irc}, ya que se puede tratar de 
programas trampa que, desde borrar todos nuestros ficheros, a enviar por correo
una copia del archivo de contrase\~nas, pueden hacer cualquier cosa: imaginemos
que un `amigo' nos env\'{\i}a un juego a trav\'es de cualquier medio -- 
especialmente {\sc irc} -- y nosotros lo ejecutamos; incluso disponer del
c\'odigo fuente no es ninguna garant\'{\i}a (>qu\'e usuario medio lee un 
c\'odigo en C de, quiz\'as, miles de l\'{\i}neas?). Ese programa puede hacer
algo tan simple como {\tt rm -rf \$HOME/*} sin que nosotros nos demos cuenta,
con las consecuencias que esta orden implica.
\item Desconfianza\\
Hemos de desconfiar de cualquier correo electr\'onico, llamada telef\'onica o
mensaje de otro tipo que nos indique realizar una determinada actividad en el
sistema, especialmente cambiar la clave o ejecutar cierta orden; con 
frecuencia, un usuario se convierte en c\'omplice involuntario de un atacante:
imaginemos que recibimos una llamada de alguien que dice ser el administrador
del sistema y que nos recomienda cambiar nuestra clave por otra que \'el nos
facilita, con la excusa de comprobar el funcionamiento del nuevo {\it software} 
de correo. Si hacemos esto, esa persona ya tiene nuestra contrase\~na para 
acceder ilegalmente a la m\'aquina y hacer all\'{\i} lo que quiera; hemos de 
recordar siempre que el administrador no necesita nuestra ayuda para comprobar 
nada, y si necesita cambiar nuestra clave, lo puede hacer \'el mismo.
\item Un \'ultimo consejo\ldots\\
Cualquier actividad sospechosa que detectemos, aunque no nos implique 
directamente a nosotros, ha de ser notificada al administrador o responsable de
seguridad del equipo. Esta notificaci\'on, a ser posible, no se ha de realizar
por correo electr\'onico (un atacante puede eliminar ese {\it mail}), sino en
persona o por tel\'efono.\\
En muchas ocasiones, cuando un usuario nota un comportamiento extra\~no en el
sistema, no notifica nada pensando que el administrador ya se ha enterado del
suceso, o por miedo a quedar en rid\'{\i}culo (quiz\'as que lo que nosotros 
consideramos `extra\~no' resulta ser algo completamente normal); esta 
situaci\'on es err\'onea: si se trata de una falsa alarma, mucho mejor, 
pero\ldots >y si no lo es?
\end{itemize}
\section{Referencias r\'apidas}
\subsection{Prevenci\'on}
\begin{itemize}
\item Cerrar los servicios de {\tt inetd} que no sean estrictamente necesarios.
\item No lanzar demonios en el arranque de m\'aquina que no sean estrictamente
necesarios.
\item Minimizar el n\'umero de ficheros setuidados o setgidados en la m\'aquina.
\item Instalar {\it TCP Wrappers} y utilizar una pol\'{\i}tica restrictiva:
{\tt echo ALL:ALL >/etc/hosts.deny}. 
\item Utilizar {\it TCP Wrappers} para controlar el acceso a nuestro {\tt 
sendmail}, o utilizar un {\it wrapper} propio para este demonio.
\item Sustituir {\tt telnet}, {\tt ftp} o similares por {\tt ssh} y {\tt scp}.
\item No permitir ficheros {\tt \$HOME/.rhosts} en los directorios de usuarios,
y no establecer relaciones de confianza en {\tt /etc/hosts.equiv}.
\item Deshabilitar las cuentas del sistema que no corresponden a usuarios 
reales ({\tt uucp}, {\tt lp}\ldots).
\item Instalar un sistema {\it Shadow Password} para que los usuarios no 
puedan leer las claves cifradas. 
\item Deshabilitar las cuentas de usuarios que no conecten al sistema.
\item Utilizar versiones actualizadas del n\'ucleo del sistema operativo.
\item Evitar sobrecargas de servicio planificando {\tt pkill -HUP inetd} en
nuestro fichero {\it crontab}.
\end{itemize}
\subsection{Detecci\'on}
\begin{itemize}
\item Configurar {\it Tripwire} nada m\'as instalar el sistema y guardar sus
resultados en un medio fiable; cada cierto tiempo, ejecutar {\it Tripwire} para
comparar sus resultados con los iniciales.
\item Chequear peri\'odicamente los {\it logs} en busca de actividades 
sospechosas.
\item Utilizar \'ordenes como {\tt ps}, {\tt netstat} o {\tt last} para 
detectar cualquier evento fuera de lo normal en el sistema, pero no confiar 
ciegamente en los resultados que se nos muestran en pantalla: seamos paranoicos.
\item Comprobar peri\'odicamente la integridad de ficheros importantes de 
nuestro sistema, como {\tt /etc/passwd}, {\tt /etc/exports}, {\tt 
/etc/syslog.conf}, {\tt /etc/aliases} o ficheros de arranque.
\item Comprobar tambi\'en elementos como los permisos o el propietario de los
ficheros que se encuentran en los directorios de usuarios. 
\end{itemize}
\subsection{Recuperaci\'on}
\begin{itemize}
\item Nunca hay que ponerse nervioso: nuestra m\'aquina ni ha sido la primera
ni lamentablemente ser\'a la \'ultima en sufrir un ataque. No es el fin del
mundo.
\item Desconfiar de cualquier programa que se encuentre en el sistema; utilizar
programas del CD-ROM del sistema operativo, o versiones est\'aticas de los
mismos, para tracear las actividades del intruso.
\item Si es posible, reinstalar el sistema operativo completo y aplicarle los
parches de seguridad que el fabricante pone a nuestra disposici\'on\footnote{O 
que se distribuyen por Internet.}; permanecer 
atentos a los directorios de usuarios y a los programas que \'estos contienen.
\item Si pensamos que la integridad del sistema peligra mucho, desconectar 
directamente el cable de red: utilizar {\tt ifconfig down} o detener el sistema
con {\tt shutdown} puede incluso acarrearnos problemas.
\item Obviamente, antes de poner el sistema de nuevo a funcionar en red, estar
completamente seguro que los problemas de seguridad que el atacante aprovech\'o
est\'an solucionados.
\end{itemize}
\subsection{Usuarios}
\begin{itemize}
\item No elegir claves de menos de seis caracteres, y combinar may\'usculas,
min\'usculas, n\'umeros, signos de puntuaci\'on\ldots cualquier cosa que nos 
permita el teclado.
\item No apuntar nuestras claves ni compartirlas con otras personas.
\item No utilizar nuestra contrase\~na de acceso en otros sistemas, 
especialmente juegos en red o equipos Windows.
\item Sustituir {\tt telnet} y {\tt ftp} por {\tt ssh} y {\tt scp} o similares.
\item Nunca ejecutar programas que nos envien por correo o que consigamos a 
partir de fuentes poco fiables (como un `amigo' que nos pasa un programa por
{\sc irc}). Tampoco ejecutar \'ordenes cuyo funcionamiento desconocemos, 
especialmente si alguien desconocido nos indica teclear `algo' para ver el
resultado.
\item Desconfiar de llamadas telef\'onicas o correo electr\'onico que nos incita
a realizar cualquier actividad dentro del sistema, especialmente cambiar nuestra
clave; si estas situaciones se producen, indicarlo inmediatamente al responsable
de seguridad del equipo, mediante tel\'efono o en persona.
\item Ante cualquier actividad sospechosa que se detecte es recomendable 
ponerse en contacto con el responsable de seguridad o el administrador, a ser 
posible por tel\'efono o en persona.
\end{itemize}
%\end{document}
