\chapter{Gesti\'on de la seguridad}
\section{Introducci\'on}
La gesti\'on de la seguridad de una organizaci\'on puede ser -- y en muchos
casos es -- algo infinitamente complejo, no tanto desde un punto de vista
puramente t\'ecnico sino m\'as bien desde un punto de vista organizativo; no 
tenemos m\'as
que pensar en una gran universidad o empresa con un n\'umero elevado de
departamentos o \'areas: si alguien que pertenece a uno de ellos abandona la
organizaci\'on, eliminar su acceso a un cierto sistema no implica ning\'un 
problema t\'ecnico (el administrador s\'olo ha de borrar o bloquear al usuario,
algo inmediato), pero s\'{\i} graves problemas organizativos: para empezar, 
>c\'omo se entera un administrador de sistemas que un cierto usuario, que no
trabaja directamente junto a \'el, abandona la empresa? >qui\'en decide si al
usuario se le elimina directamente o se le permite el acceso a su correo 
durante un mes? >puede el personal del \'area de seguridad decidir bloquear el
acceso a alguien de cierto `rango' en la organizaci\'on, como un directivo o 
un director de departamento, nada m\'as que este abandone la misma? >y si 
resulta que es amigo del director general o el rector, y luego este se enfada? 
Como vemos, desde un punto de vista t\'ecnico no existe ning\'un escollo 
insalvable, pero s\'{\i} que existen desde un punto de vista de la gesti\'on de 
la seguridad\ldots\\
\\Hoy en d\'{\i}a, una entidad que trabaje con cualquier tipo de entorno 
inform\'atico, desde peque\~nas empresas con negocios no relacionados 
directamente con las nuevas tecnolog\'{\i}as hasta grandes {\it telcos} de
\'ambito internacional, est\'a -- o deber\'{\i}a estar -- preocupada por su
seguridad. Y no es para menos: el n\'umero de amenazas a los entornos 
inform\'aticos y de comunicaciones crece casi exponencialmente a\~no tras a\~no,
alcanzando cotas inimaginables hace apenas una d\'ecada. Y con que el futuro de 
la interconexi\'on de sistemas sea tan solo la mitad de prometedor de lo que 
nos tratan de hacer creer, es previsible que la preocupaci\'on por la seguridad
vaya en aumento conforme nuestras vidas est\'en m\'as y m\'as `conectadas' a
Internet.\\
\\Hasta hace poco esta preocupaci\'on de la que estamos hablando se centraba 
sobre todo en los aspectos m\'as t\'ecnicos de la seguridad: alguien 
convenc\'{\i}a a alg\'un responsable t\'ecnico que con la implantaci\'on de un 
cortafuegos corporativo se acabar\'{\i}an todos los problemas de la 
organizaci\'on, y por supuesto se eleg\'{\i}a el m\'as caro aunque despu\'es
nadie supiera implantar en \'el una pol\'{\i}tica correcta; poco despu\'es, y
en vista de que el {\it firewall} no era la panacea, otro comercial avispado
convenc\'{\i}a a la direcci\'on que lo que realmente estaba de moda son los
sistemas de detecci\'on de intrusos, y por supuesto se `dejaba caer' un producto
de este tipo en la red (se `dejaba caer', no se `implantaba'). En la actualidad,
como las siglas est\'an tan de moda, lo que se lleva son las PKIs, que aunque
nadie sepa muy bien como calzarlas en el entorno de operaciones, quedan de
maravilla sobre las {\it slides}\footnote{Las {\it slides} es como muchos 
llaman a las transparencias de toda la vida, pero en formato 
PowerPoint\ldots Y es que ya se sabe: cuantos m\'as t\'erminos anglosajones
podamos soltar en una charla, m\'as parecer\'a que sabemos ;-)} de las 
presentaciones comerciales de turno.\\
\\Por fortuna, las cosas han empezado a cambiar (y digo `por fortuna' a pesar 
de ser una persona m\'as t\'ecnica que organizativa); hoy en d\'{\i}a la
seguridad va m\'as all\'a de lo que pueda ser un cortafuegos, un sistema de 
autenticaci\'on biom\'etrico o una red de sensores de detecci\'on de intrusos:
ya se contemplan aspectos que hasta hace poco se reservaban a entornos 
altamente cerrados, como bancos u organizaciones militares. Y es que nos hemos
empezado a dar cuenta de que tan importante o m\'as como un buen {\it firewall} 
es un plan de continuidad del negocio en caso de cat\'astrofe -- especial y 
desgraciadamente desde el pasado 11 de septiembre --, y que sin una 
pol\'{\i}tica de seguridad correctamente implantada en nuestra organizaci\'on
no sirven de nada los controles de acceso (f\'{\i}sicos y l\'ogicos) a la
misma. Se habla ahora de la {\bf gesti\'on de la seguridad} como algo 
cr\'{\i}tico para cualquier organizaci\'on, igual de importante dentro de la 
misma que los sistemas de calidad o las l\'{\i}neas de producto que 
desarrolla.\\
\\Algo que sin duda ha contribuido a todo esto es la aparici\'on -- m\'as o
menos reciente -- de normativas y est\'andares de seguridad, de \'ambito tanto
nacional como internacional, y sobre todo su aplicaci\'on efectiva; no tenemos
m\'as que mirar la Ley Org\'anica de Protecci\'on de Datos de Car\'acter
Personal en Espa\~na: desde que la Agencia de Protecci\'on de Datos impone
sanciones millonarias a quienes incumplen sus exigencias, todo el mundo se
preocupa de la correcta gesti\'on de su seguridad. Tambi\'en han sido 
importantes la transformaci\'on del British Standard 7799 en una norma ISO 
(17799) a la que referirse a la hora de hablar de la definici\'on de 
pol\'{\i}ticas dentro de una organizaci\'on, y la definici\'on del informe 
UNE 71501 IN como requisito para proteger y gestionar la seguridad de los
sistemas de informaci\'on dentro de las organizaciones.\\
\\Hasta tal punto se ha popularizado el mundo de la seguridad que surgen 
empresas `especializadas' hasta de debajo de las piedras, y por supuesto todas
cuentan con los mejores expertos, consultores e ingenieros de seguridad 
(t\'{\i}tulos que, al menos que yo sepa, no otorga ninguna universidad
espa\~nola); la paranoia se lleva al l\'{\i}mite cuando se ofrecen 
certificaciones comerciales de seguridad del tipo {\it `Certificado X en 
Seguridad'} o {\it `Ingeniero de Seguridad X'}. Por supuesto, aunque en el 
mercado de la seguridad hay excelentes profesionales, la l\'ogica nos debe 
llevar a desconfiar de este tipo de publicidad, pero sin llegar al extremo de 
descuidar nuestra seguridad por no confiar en nadie: como veremos, la seguridad 
gestionada es en muchas ocasiones una excelente soluci\'on.\\
\\En definitiva, en este cap\'{\i}tulo vamos a intentar hablar de aspectos
relacionados con la gesti\'on de la seguridad corporativa -- entendiendo por
`corporativa' la aplicable a una determinada organizaci\'on, bien sea una
empresa bien sea una universidad --. Comentaremos desde aspectos relacionados
con la definici\'on de pol\'{\i}ticas o los an\'alisis de riesgos hasta el
papel del \'area de Seguridad de una organizaci\'on, incluyendo aproximaciones
a la seguridad gestionada. Como siempre, existe numerosa bibliograf\'{\i}a sobre
el tema que es imprescindible consultar si queremos definir y gestionar 
de forma adecuada la seguridad de nuestra organizaci\'on; especialmente 
recomendables son \cite{kn:iso}, \cite{kn:oss} y \cite{kn:nist97}.
\section{Pol\'{\i}ticas de seguridad}
El t\'ermino {\bf pol\'{\i}tica de seguridad} se suele definir como el 
conjunto de requisitos definidos por los responsables directos o indirectos
de un sistema que indica en t\'erminos generales qu\'e est\'a y qu\'e no est\'a
permitido en el \'area de seguridad durante la operaci\'on general de dicho
sistema (\cite{kn:iso88}). Al tratarse de `t\'erminos generales', aplicables
a situaciones o recursos muy diversos, suele ser necesario refinar los 
requisitos de la pol\'{\i}tica para convertirlos en indicaciones precisas
de qu\'e es lo permitido y lo denegado en cierta parte de la operaci\'on del 
sistema, lo que se denomina {\bf pol\'{\i}tica de aplicaci\'on espec\'{\i}fica} 
(\cite{kn:muf93}).\\
\\Una pol\'{\i}tica de seguridad puede ser {\bf prohibitiva}, si todo lo que
no est\'a expresamente permitido est\'a denegado, o {\bf permisiva}, si todo
lo que no est\'a expresamente prohibido est\'a permitido. Evidentemente la
primera aproximaci\'on es mucho mejor que la segunda de cara a mantener la
seguridad de un sistema; en este caso la pol\'{\i}tica contemplar\'{\i}a todas
las actividades que se pueden realizar en los sistemas, y el resto -- las no
contempladas -- ser\'{\i}an consideradas ilegales.\\
\\Cualquier pol\'{\i}tica ha de contemplar seis elementos claves en la seguridad
de un sistema inform\'atico (\cite{kn:par94}):
\begin{itemize}
\item Disponibilidad\\
Es necesario garantizar que los recursos del sistema se encontrar\'an 
disponibles cuando se necesitan, especialmente la informaci\'on cr\'{\i}tica.
\item Utilidad\\
Los recursos del sistema y la informaci\'on manejada en el mismo ha de ser 
\'util para alguna funci\'on.
\item Integridad\\
La informaci\'on del sistema ha de estar disponible tal y como se almacen\'o
por un agente autorizado.
\item Autenticidad\\
El sistema ha de ser capaz de verificar la identidad de sus usuarios, y los 
usuarios la del sistema.
\item Confidencialidad\\
La informaci\'on s\'olo ha de estar disponible para agentes autorizados, 
especialmente su propietario.
\item Posesi\'on\\
Los propietarios de un sistema han de ser capaces de controlarlo en todo 
momento; perder este control en favor de un usuario malicioso compromete la
seguridad del sistema hacia el resto de usuarios.
\end{itemize}
Para cubrir de forma adecuada los seis elementos anteriores, con el objetivo 
permanente de
garantizar la seguridad corporativa, una pol\'{\i}tica se suele dividir en
puntos m\'as concretos a veces llamados {\bf normativas} (aunque las 
definiciones concretas de cada documento que conforma la infraestructura de
nuestra pol\'{\i}tica de seguridad -- pol\'{\i}tica, normativa, est\'andar,
procedimiento operativo\ldots -- es algo en lo que ni los propios expertos se
ponen de acuerdo). El est\'andar ISO 17799 (\cite{kn:iso}) define las siguientes
l\'{\i}neas de actuaci\'on:
\begin{itemize}
\item Seguridad organizacional.\\
Aspectos relativos a la gesti\'on de la seguridad dentro de la organizaci\'on
(cooperaci\'on con elementos externos, {\it outsourcing}, estructura del 
\'area de seguridad\ldots).
\item Clasificaci\'on y control de activos.\\
Inventario de activos y definici\'on de sus mecanismos de control, as\'{\i}
como etiquetado y clasificaci\'on de la informaci\'on corporativa.
\item Seguridad del personal.\\
Formaci\'on en materias de seguridad, clausulas de confidencialidad, reporte
de incidentes, monitorizaci\'on de personal\ldots
\item Seguridad f\'{\i}sica y del entorno.\\
Bajo este punto se engloban aspectos relativos a la seguridad f\'{\i}sica de los
recintos donde se encuentran los diferentes recursos -- incluyendo los humanos
-- de la organizaci\'on y de los sistemas en s\'{\i}, as\'{\i} como la
definici\'on de controles gen\'ericos de seguridad.
\item Gesti\'on de comunicaciones y operaciones.\\
Este es uno de los puntos m\'as interesantes desde un punto de vista 
estrictamente t\'ecnico, ya que engloba aspectos de la seguridad relativos a
la operaci\'on de los sistemas y telecomunicaciones, como los controles de
red, la protecci\'on frente a {\it malware}, la gesti\'on de copias de 
seguridad o el intercambio de {\it software} dentro de la organizaci\'on.
\item Controles de acceso.\\
Definici\'on y gesti\'on de puntos de control de acceso a los recursos 
inform\'aticos de la organizaci\'on: contrase\~nas, seguridad perimetral, 
monitorizaci\'on de accesos\ldots
\item Desarrollo y mantenimiento de sistemas.\\
Seguridad en el desarrollo y las aplicaciones, cifrado de datos, control de
{\it software}\ldots
\item Gesti\'on de continuidad de negocio.\\
Definici\'on de planes de continuidad, an\'alisis de impacto, simulacros de
cat\'astrofes\ldots
\item Requisitos legales.\\
Evidentemente, una pol\'{\i}tica ha de cumplir con la normativa vigente en el
pa\'{\i}s donde se aplica; si una organizaci\'on se extiende a lo largo de
diferentes paises, su pol\'{\i}tica tiene que ser coherente con la normativa
del m\'as restrictivo de ellos. En este apartado de la pol\'{\i}ica se 
establecen las relaciones con cada ley: derechos de propiedad intelectual, 
tratamiento de datos de car\'acter personal, exportaci\'on de cifrado\ldots 
junto a todos los aspectos relacionados con registros de eventos en los 
recursos ({\it logs}) y su mantenimiento.
\end{itemize}
\section{An\'alisis de riesgos}
En un entorno inform\'atico existen una serie de recursos (humanos, t\'ecnicos,
de infraestructura\ldots) que est\'an expuestos a diferentes tipos de riesgos:
los `normales', aquellos comunes a cualquier entorno, y los excepcionales,
originados por situaciones concretas que afectan o pueden afectar a parte de una
organizaci\'on o a toda la misma, como la inestabilidad pol\'{\i}tica en un
pa\'{\i}s o una regi\'on sensible a terremotos (\cite{kn:pla83}). Para tratar
de minimizar los efectos de un problema de seguridad se realiza lo que 
denominamos un {\bf an\'alisis de riesgos}, t\'ermino que hace referencia al 
proceso necesario para responder a tres cuestiones b\'asicas sobre nuestra 
seguridad:
\begin{itemize}
\item >qu\'e queremos proteger?
\item >contra qui\'en o qu\'e lo queremos proteger? 
\item >c\'omo lo queremos proteger?
\end{itemize}
En la pr\'actica existen dos aproximaciones para responder a estas cuestiones,
una cuantitativa y otra cualitativa. La primera de ellas es con diferencia la 
menos usada, ya que en muchos casos implica c\'alculos complejos o datos 
dif\'{\i}ciles de estimar. Se basa en dos par\'ametros fundamentales: la
probabilidad de que un suceso ocurra y una estimaci\'on del coste o las 
p\'erdidas en caso de que as\'{\i} sea; el producto de ambos t\'erminos es lo
que se denomina {\bf coste anual estimado} (EAC, {\it Estimated Annual Cost}),
y aunque te\'oricamente es posible conocer el riesgo de cualquier evento (el 
EAC) y tomar decisiones en funci\'on de estos datos, en la pr\'actica la 
inexactitud en la estimaci\'on o en el c\'alculo de par\'ametros hace 
dif\'{\i}cil y poco realista esta aproximaci\'on.\\
\\El segundo m\'etodo de an\'alisis de riesgos es el cualitativo, de uso muy
difundido en la actualidad especialmente entre las nuevas `consultoras' de
seguridad (aquellas m\'as especializadas en seguridad l\'ogica, cortafuegos,
{\it tests} de penetraci\'on y similares). Es mucho m\'as sencillo e intuitivo
que el anterior, ya que ahora no entran en juego probabilidades exactas sino
simplemente una estimaci\'on de p\'erdidas potenciales. Para ello se 
interrelacionan cuatro elementos principales: las amenazas, por definici\'on 
siempre presentes en cualquier sistema, las vulnerabilidades, que potencian el
efecto de las amenazas, el impacto asociado a una amenaza, que indica los 
da\~nos sobre un activo por la materializaci\'on de dicha amenaza, y los 
controles o salvaguardas, contramedidas para minimizar las vulnerabilidades 
(controles preventivos) o el impacto (controles curativos). Por ejemplo, una 
amenaza ser\'{\i}a un pirata que queramos o no (no depende de nosotros) va a 
tratar de modificar nuestra p\'agina {\it web} principal, el impacto ser\'{\i}a
una medida del da\~no que causar\'{\i}a si lo lograra, una vulnerabilidad
ser\'{\i}a una configuraci\'on incorrecta del servidor que ofrece las p\'aginas,
y un control la reconfiguraci\'on de dicho servidor o el incremento de su
nivel de parcheado. Con estos cuatro elementos podemos obtener un indicador 
cualitativo del nivel de riesgo asociado a un activo determinado dentro de la 
organizaci\'on, visto como la probabilidad de que una amenaza se materialice
sobre un activo y produzca un determinado impacto.\\
\\En Espa\~na es interesante la metodolog\'{\i}a de an\'alisis de riesgos
desarrollada desde el Consejo Superior de Inform\'atica (Ministerio de
Administraciones P\'ublicas) y denominada {\sc magerit} (Metodolog\'{\i}a de
An\'alisis y GEsti\'on de Riesgos de los sistemas de Informaci\'on de las
AdminisTraciones p\'ublicas); se trata de un m\'etodo formal para realizar un
an\'alisis de riesgos y recomendar los controles necesarios para su
minimizaci\'on. {\sc magerit} se basa en una aproximaci\'on cualitativa que
intenta cubrir un amplio espectro de usuarios gen\'ericos gracias a un enfoque
orientado a la adaptaci\'on del mecanismo dentro de diferentes entornos,
generalmente con necesidades de seguridad y nivel de sensibilidad tambi\'en
diferentes. En la p\'agina {\it web} del Consejo Superior de
Inform\'atica\footnote{\tt http://www.map.es/csi/} podemos encontrar
informaci\'on m\'as detallada acerca de esta metodolog\'{\i}a, as\'{\i} como
algunos ejemplos de ejecuci\'on de la misma.\\
\\Tras obtener mediante cualquier mecanismo los indicadores de riesgo en nuestra
organizaci\'on llega la hora de evaluarlos para tomar decisiones organizativas
acerca de la gesti\'on de nuestra seguridad y sus prioridades. Tenemos por una
parte el {\it riesgo calculado}, resultante de nuestro an\'alisis, y este 
riesgo calculado se ha de comparar con un cierto umbral ({\it umbral de 
riesgo}) determinado por la pol\'{\i}tica de seguridad de nuestra 
organizaci\'on; el umbral de riesgo puede ser o bien un n\'umero o bien una
etiqueta de riesgo (por ejemplo, nivel de amenaza alto, impacto alto, 
vulnerabilidad grave, etc.), y cualquier riesgo calculado superior al umbral
ha de implicar una decisi\'on de reducci\'on de riesgo. Si por el contrario el
calculado es menor que el umbral, se habla de {\it riesgo residual}, y el
mismo se considera asumible (no hay porqu\'e tomar medidas para reducirlo). El
concepto de asumible es diferente al de {\it riesgo asumido}, que denota 
aquellos riesgos calculados superiores al umbral pero sobre los que por 
cualquier raz\'on (pol\'{\i}tica, econ\'omica\ldots) se decide no tomar medidas 
de reducci\'on; evidentemente, siempre hemos de huir de esta situaci\'on.\\
\\Una vez conocidos y evaluados de cualquier forma los riesgos a los que nos 
enfrentamos podremos definir las pol\'{\i}ticas e 
implementar las soluciones pr\'acticas -- los mecanismos -- para minimizar sus
efectos. Vamos a intentar de entrar con m\'as detalle en c\'omo dar respuesta a 
cada una de las preguntas que nos hemos planteado al principio de este punto:
\subsection{Identificaci\'on de recursos}
Debemos identificar todos los recursos cuya integridad pueda ser amenazada de
cualquier forma; por ejemplo, \cite{kn:rfc1244} define b\'asicamente los 
siguientes: 
\begin{itemize}
\item {\it Hardware}\\
Procesadores, tarjetas, teclados, terminales, estaciones de trabajo, ordenadores
personales, impresoras, unidades de disco, l\'{\i}neas de comunicaci\'on, 
servidores, {\it routers}\ldots
\item {\it Software}\\
C\'odigos fuente y objeto, utilidades, programas de diagn\'ostico, sistemas 
operativos, programas de comunicaci\'on\ldots
\item Informaci\'on\\
En ejecuci\'on, almacenados en l\'{\i}nea, almacenados fuera de l\'{\i}nea,
en comunicaci\'on, bases de datos\ldots
\item Personas\\
Usuarios, operadores\ldots
\item Accesorios\\
Papel, cintas, t\'oners\ldots
\end{itemize}
Aparte del recurso en s\'{\i} (algo tangible, como un {\it router}) hemos de 
considerar la visi\'on intangible de cada uno de estos recursos (por ejemplo
la capacidad para seguir trabajando sin ese {\it router}). Es dif\'{\i}cil
generar estos aspectos intangibles de los recursos, ya que es algo que va a
depender de cada organizaci\'on, su funcionamiento, sus seguros, sus 
normas\ldots No obstante, siempre hemos de tener en cuenta algunos aspectos
comunes: privacidad de los usuarios, imagen p\'ublica de la organizaci\'on,
reputaci\'on, satisfacci\'on del personal y de los clientes -- en el caso de una
universidad, de los alumnos --, capacidad de procesamiento ante un 
fallo\ldots\\
\\Con los recursos correctamente identificados se ha de generar una lista 
final, que ya incluir\'a {\bf todo} lo que necesitamos proteger en nuestra 
organizaci\'on.
\subsection{Identificaci\'on de amenazas}
Una vez conocemos los recursos que debemos proteger es la hora de identificar
las vulnerabilidades y amenazas que se ciernen contra ellos. Una vulnerabilidad
es cualquier situaci\'on que pueda desembocar en un problema de seguridad, y
una amenaza es la acci\'on espec\'{\i}fica que aprovecha una vulnerabilidad 
para crear un problema de seguridad; entre ambas existe una estrecha relaci\'on:
sin vulnerabilidades no hay amenazas, y sin amenazas no hay vulnerabilidades.\\
\\Se suelen dividir las amenazas que existen sobre los sistemas inform\'aticos
en tres grandes grupos, en funci\'on del \'ambito o la forma en que se pueden
producir:
\begin{itemize}
\item Desastres del entorno.\\
Dentro de este grupo se incluyen todos los posibles problemas relacionados con
la ubicaci\'on del entorno de trabajo inform\'atico o de la propia 
organizaci\'on, as\'{\i} como con las personas que de una u otra forma est\'an
relacionadas con el mismo. Por ejemplo, se han de tener en cuenta desastres
naturales (terremotos, inundaciones\ldots), desastres producidos por elementos
cercanos, como los cortes de fluido el\'ectrico, y peligros relacionados con 
operadores, programadores o usuarios del sistema.
\item Amenazas en el sistema.\\
Bajo esta denominaci\'on se contemplan todas las vulnerabilidades de los 
equipos y su {\it software} que pueden acarrear amenazas a la seguridad, como 
fallos en el sistema operativo, medidas de protecci\'on que \'este ofrece, 
fallos en los programas, copias de seguridad\ldots
\item Amenazas en la red.\\
Cada d\'{\i}a es menos com\'un que una m\'aquina trabaje aislada de todas las
dem\'as; se tiende a comunicar equipos mediante redes locales, intranets o la
propia Internet, y esta interconexi\'on acarrea nuevas -- y peligrosas -- 
amenazas a la seguridad de los equipos, peligros que hasta el momento de la
conexi\'on no se suelen tener en cuenta. Por ejemplo, es necesario analizar
aspectos relativos al cifrado de los datos en tr\'ansito por la red, a proteger
una red local del resto de internet, o a instalar sistemas de autenticaci\'on
de usuarios remotos que necesitan acceder a ciertos recursos internos a la
organizaci\'on (como un investigador que conecta desde su casa a trav\'es de
un m\'odem).
\end{itemize}
Algo importante a la hora de analizar las amenazas a las que se enfrentan
nuestros sistemas es analizar los potenciales tipos de atacantes que pueden
intentar violar nuestra seguridad. Es algo normal que a la hora de hablar de
atacantes todo el mundo piense en {\it crackers}, en piratas inform\'aticos mal
llamados {\it hackers}. No obstante, esto no es m\'as que el fruto de la 
repercusi\'on que en todos los medios tienen estos individuos y sus acciones;
en realidad, la inmensa mayor\'{\i}a de problemas de seguridad vienen dados por
atacantes internos a la organizaci\'on afectada. En organismos de I+D estos
atacantes suelen ser los propios estudiantes (rara vez el personal), as\'{\i}
como piratas externos a la entidad que aprovechan la habitualmente mala 
protecci\'on de los sistemas universitarios para acceder a ellos y conseguir
as\'{\i} cierto {\it status} social dentro de un grupo de piratas. Los
conocimientos de estas personas en materias de sistemas operativos, redes o
seguridad inform\'atica suelen ser muy limitados, y sus actividades no suelen
entra\~nar muchos riesgos a no ser que se utilicen nuestros equipos para atacar
a otras organizaciones, en cuyo caso a los posibles problemas legales hay que
sumar la mala imagen que nuestras organizaciones adquieren.\\
\\No siempre hemos de contemplar a las amenazas como actos intencionados contra
nuestro sistema: muchos de los problemas pueden ser ocasionados por accidentes,
desde un operador que derrama una taza de caf\'e sobre una terminal hasta un
usuario que tropieza con el cable de alimentaci\'on de un servidor y lo 
desconecta de la l\'{\i}nea el\'ectrica, pasando por temas como el borrado 
accidental de datos o los errores de programaci\'on; decir {\it `no lo hice a
prop\'osito'} no ayuda nada en estos casos. Por supuesto, tampoco tenemos que
reducirnos a los accesos no autorizados al sistema: un usuario de nuestras
m\'aquinas puede intentar conseguir privilegios que no le corresponden, una
persona externa a la organizaci\'on puede lanzar un ataque de negaci\'on de
servicio contra la misma sin necesidad de conocer ni siquiera un {\it login}
y una contrase\~na, etc.
\subsection{Medidas de protecci\'on}
Tras identificar todos los recursos que deseamos proteger, as\'{\i} como las
posibles vulnerabilidades y amenazas a que nos exponemos y los potenciales 
atacantes que pueden intentar violar nuestra seguridad, hemos de estudiar c\'omo
proteger nuestros sistemas, sin ofrecer a\'un implementaciones concretas para
protegerlos (esto ya no ser\'{\i}an pol\'{\i}ticas sino mecanismos). Esto 
implica en primer lugar cuantificar los da\~nos que cada posible vulnerabilidad
puede causar teniendo en cuenta las posibilidades de que una amenaza se pueda
convertir en realidad. Este c\'alculo puede realizarse partiendo de hechos
sucedidos con anterioridad en nuestra organizaci\'on, aunque por desgracia en
muchos lugares no se suelen registrar los incidentes acaecidos. En este caso, y
tambi\'en a la hora de evaluar los da\~nos sobre recursos intangibles, existen
diversas aproximaciones como el m\'etodo Delphi, que b\'asicamente consiste en
preguntar a una serie de especialistas de la organizaci\'on sobre el da\~no y 
las p\'erdidas que cierto problema puede causar; no obstante, la experiencia
del administrador en materias de seguridad suele tener aqu\'{\i} la \'ultima
palabra a la hora de evaluar los impactos de cada amenaza.\\
\\La clasificaci\'on de riesgos de cara a estudiar medidas de protecci\'on 
suele realizarse en base al nivel de importancia del da\~no causado y a la
probabilidad aproximada de que ese da\~no se convierta en realidad; se trata
principalmente de no gastar m\'as dinero en una implementaci\'on para proteger
un recurso de lo que vale dicho recurso o de lo que nos costar\'{\i}a 
recuperarnos
de un da\~no en \'el o de su p\'erdida total. Por ejemplo, podemos seguir un
an\'alisis similar en algunos aspectos al problema de la mochila: llamamos
$R_{i}$ al riesgo de perder un recurso {\it i} (a la probabilidad de que se 
produzca un ataque), y le asignamos un valor de 0 a 10 (valores m\'as altos
implican m\'as probabilidad); de la misma forma, definimos tambi\'en de 0 a 10
la importancia de cada recurso, $W_{i}$, siendo 10 la importancia m\'as alta. 
La evaluaci\'on del riesgo es entonces el producto de ambos valores, llamado
peso o riesgo evaluado de un recurso, $WR_{i}$, y medido en dinero perdido por
unidad de tiempo (generalmente, por a\~no):
\begin{center}
$WR_{i} = R_{i}\times W_{i}$
\end{center}
De esta forma podemos utilizar hojas de trabajo en las que, para cada recurso, 
se muestre su nombre y el n\'umero asignado, as\'{\i} como los tres valores 
anteriores. Evidentemente, los recursos que presenten un riesgo evaluado mayor
ser\'an los que m\'as medidas de protecci\'on deben poseer, ya que esto 
significa que es probable que sean atacados, y que adem\'as el ataque puede
causar p\'erdidas importantes. Es especialmente importante un grupo de 
riesgos denominados {\it inaceptables}, aquellos cuyo peso supera un cierto
umbral; se trata de problemas que no nos podemos permitir en nuestros sistemas,
por lo que su prevenci\'on es crucial para que todo funcione correctamente.\\
\\Una vez que conocemos el riesgo evaluado de cada recurso es necesario 
efectuar lo que se llama el an\'alisis de costes y beneficios. B\'asicamente 
consiste en comparar el coste asociado a cada problema (calculado anteriormente,
$WR_{i}$) con el coste de prevenir dicho problema. El c\'alculo de este \'ultimo
no suele ser complejo si conocemos las posibles medidas de prevenci\'on que
tenemos a nuestra disposici\'on: por ejemplo, para saber lo que nos cuesta
prevenir los efectos de un incendio en la sala de operaciones, no tenemos m\'as
que consultar los precios de sistemas de extinci\'on de fuego, o para saber lo
que nos cuesta proteger nuestra red s\'olo hemos de ver los precios de productos
como {\it routers} que bloqueen paquetes o cortafuegos completos. No s\'olo 
hemos de tener en cuenta el coste de cierta protecci\'on, sino tambi\'en lo que
nos puede suponer su implementaci\'on y su mantenimiento; en muchos casos 
existen soluciones gratuitas para prevenir ciertas amenazas, pero estas 
soluciones tienen un coste asociado relativo a la dificultad de hacerlas
funcionar correctamente de una forma cont\'{\i}nua en el tiempo, por ejemplo 
dedicando a un empleado a su implementaci\'on y mantenimiento.\\
\\Cuando ya hemos realizado este an\'alisis no tenemos m\'as que presentar
nuestras cuentas a los responsables de la organizaci\'on (o adecuarlas al
presupuesto que un departamento destina a materias de seguridad), siempre
teniendo en cuenta que el gasto de proteger un recurso ante una amenaza ha de 
ser inferior al gasto que se producir\'{\i}a si la amenaza se convirtiera en
realidad. Hemos de tener siempre presente que los riesgos se pueden minimizar,
pero {\bf nunca} eliminarlos completamente, por lo que ser\'a recomendable
planificar no s\'olo la prevenci\'on ante de un problema sino tambi\'en la
recuperaci\'on si el mismo se produce; se suele hablar de medidas {\bf 
proactivas} (aquellas que se toman para prevenir un problema) y medidas {\bf 
reactivas} (aquellas que se toman cuando el da\~no se produce, para minimizar 
sus efectos).
\section{Estrategias de respuesta}
>Qu\'e hacer cuando nuestra pol\'{\i}tica de seguridad ha sido violada? La 
respuesta a esta pregunta depende completamente del tipo de violaci\'on que se
haya producido, de su gravedad, de qui\'en la haya provocado, de su 
intenci\'on\ldots Si se trata de accidentes o de problemas poco importantes
suele ser suficiente con una reprimenda verbal o una advertencia; si ha sido
un hecho provocado, quiz\'as es conveniente emprender acciones algo m\'as 
convincentes, como la clausura de las cuentas de forma temporal o peque\~nas
sanciones administrativas. En el caso de problemas graves que hayan sido
intencionados interesar\'a emprender acciones m\'as duras, como cargos legales
o sanciones administrativas firmes (por ejemplo, la expulsi\'on de una 
universidad).\\
\\Una gran limitaci\'on que nos va a afectar mucho es la situaci\'on de la 
persona o personas causantes de la violaci\'on con respecto a la organizaci\'on
que la ha sufrido. En estos casos se suele diferenciar entre usuarios internos
o locales, que son aquellos pertenecientes a la propia organizaci\'on, y
externos, los que no est\'an relacionados directamente con la misma; las 
diferencias entre ellos son los l\'{\i}mites de red, los administrativos, los
legales o los pol\'{\i}ticos. Evidentemente es mucho m\'as f\'acil buscar
responsabilidades ante una violaci\'on de la seguridad entre los usuarios
internos, ya sea contra la propia organizaci\'on o contra otra, pero utilizando
los recursos de la nuestra; cuando estos casos se dan en redes de I+D, 
generalmente ni siquiera es necesario llevar el caso ante la justicia, basta
con la aplicaci\'on de ciertas normas sobre el usuario problem\'atico (desde
una sanci\'on hasta la expulsi\'on o despido de la organizaci\'on).\\
\\Existen dos estrategias de respuesta ante un incidente de seguridad 
(\cite{kn:siy95}):
\begin{itemize}
\item Proteger y proceder.
\item Perseguir y procesar.
\end{itemize}
La primera de estas estrategias, proteger y proceder, se suele aplicar cuando
la organizaci\'on es muy vulnerable o el nivel de los atacantes es elevado; la
filosof\'{\i}a es proteger de manera inmediata la red y los sistemas y restaurar
su estado normal, de forma que los usuarios puedan seguir trabajando 
normalmente. Seguramente ser\'a necesario interferir de forma activa las 
acciones del intruso para evitar m\'as accesos, y analizar el da\~no causado. La
principal desventaja de esta estrategia es que el atacante se da cuenta 
r\'apidamente de que ha sido descubierto, y puede emprender acciones para ser
identificado, lo que incluso conduce al borrado de {\it logs} o de sistemas
de ficheros completos; incluso puede cambiar su estrategia de ataque a un 
nuevo m\'etodo, y seguir comprometiendo al sistema. Sin embargo, esta estrategia
tambi\'en presenta una parte positiva: el bajo nivel de conocimientos de los 
atacantes en sistemas habituales hace que en muchas ocasiones se limiten a 
abandonar su ataque y dedicarse a probar suerte con otros sistemas menos 
protegidos en otras organizaciones.\\
\\La segunda estrategia de respuesta, perseguir y procesar, adopta la 
filosof\'{\i}a de permitir al atacante proseguir sus actividades, pero de forma
controlada y observada por los administradores, de la forma m\'as discreta
posible. Con esto, se intentan guardar pruebas para ser utilizadas en la 
segunda parte de la estrategia, la de acusaci\'on y procesamiento del 
atacante (ya sea ante la justicia o ante los responsables de la organizaci\'on,
si se trata de usuarios internos). Evidentemente corremos el peligro de que
el intruso descubra su monitorizaci\'on y destruya completamente el sistema,
as\'{\i} como que nuestros resultados no se tengan en cuenta ante un tribunal
debido a las artima\~nas legales que algunos abogados aprovechan; la parte 
positiva de esta estrategia es, aparte de la recolecci\'on de pruebas, que
permite a los responsables conocer las actividades del atacante, qu\'e
vulnerabilidades de nuestra organizaci\'on ha aprovechado para atacarla, c\'omo
se comporta una vez dentro, etc. De esta forma podemos aprovechar el ataque
para reforzar los puntos d\'ebiles de nuestros sistemas.\\
\\A nadie se le escapan los enormes peligros que entra\~na el permitir a un
atacante proseguir con sus actividades dentro de las m\'aquinas; por muy 
controladas que est\'en, en cualquier momento casi nada puede evitar que la
persona se sienta vigilada, se ponga nerviosa y destruya completamente nuestros
datos. Una forma de monitorizar sus actividades sin comprometer excesivamente
nuestra integridad es mediante un proceso denominado {\it jailing} o 
encarcelamiento: la idea es construir un sistema que simule al real, pero donde
no se encuentren datos importantes, y que permita observar al atacante sin
poner en peligro los sistemas reales. Para ello se utiliza una m\'aquina,
denominada {\bf sistema de sacrificio}, que es donde el atacante realmente
trabaja, y un segundo sistema, denominado {\bf de observaci\'on}, conectado al
anterior y que permite analizar todo lo que esa persona est\'a llevando a cabo.
De esta forma conseguimos que el atacante piense que su intrusi\'on ha tenido
\'exito y continue con ella mientras lo monitorizamos y recopilamos pruebas
para presentar en una posible demanda o acusaci\'on. Si deseamos construir
una c\'arcel es necesario que dispongamos de unos conocimientos medios o 
elevados de programaci\'on de sistemas; utilidades como {\tt chroot()} nos
pueden ser de gran ayuda, as\'{\i} como {\it software} de simulaci\'on como 
{\it Deception Tookit (DTK)}, que simula el \'exito de un ataque ante el pirata 
que lo lanza, pero que realmente nos est\'a informa del intento de violaci\'on 
producido.\\
\\Sin importar la estrategia adoptada ante un ataque, siempre es recomendable
ponerse en contacto con entidades externas a nuestra organizaci\'on, incluyendo
por ejemplo fuerzas de seguridad (en Espa\~na, Guardia Civil o Polic\'{\i}a 
Nacional), gabinetes jur\'{\i}dicos o equipos de expertos en seguridad 
inform\'atica, como el CERT. En el caso de instituciones de I+D, en Espa\~na
existe IrisCERT ({\tt http://www.rediris.es/cert/}), el equipo de respuesta 
ante emergencias de seguridad de RedIRIS, la red universitaria espa\~nola.
\section{\it Outsourcing}
Cada vez es m\'as habitual que las empresas contraten los servicios de 
seguridad de una compa\~n\'{\i}a externa, especializada en la materia, y que
permita olvidarse -- relativamente, como veremos despu\'es -- al personal de 
esa empresa de los aspectos t\'ecnicos y organizativos de la seguridad, para 
poder centrarse as\'{\i} en su l\'{\i}nea de negocio correspondiente; esta
pol\'{\i}tica es lo que se conoce como {\it outsourcing} y se intenta traducir
por `externalizaci\'on', aplicado en nuestro
caso a la seguridad corporativa. A los que somos puramente t\'ecnicos muchas 
veces se nos olvida que la seguridad en s\'{\i} misma no es ning\'un fin, sino
una herramienta al servicio de los negocios, y por tanto nuestros esfuerzos han
de ir orientados a proteger el `patrimonio' (humano, tecnol\'ogico, 
econ\'omico\ldots) de quien contrata nuestros servicios: al director de una
gran firma probablemente le importe muy poco que hayamos implantado en sus
instalaciones el mejor cortafuegos del mercado junto a un fabuloso sistema 
distribuido de detecci\'on de intrusos si despu\'es un atacante puede entrar
con toda facilidad en la sala de m\'aquinas y robar varias cintas de {\it 
backup} con toda la informaci\'on cr\'{\i}tica de esa compa\~n\'{\i}a; y si esto
sucede, simplemente hemos hecho mal nuestro trabajo.\\
\\>Por qu\'e va a querer una empresa determinada que personas ajenas a la misma
gestionen su seguridad? Al fin y al cabo, estamos hablando de la protecci\'on
de muchos activos de la compa\~n\'{\i}a, y encomendar esa tarea tan cr\'{\i}tica
a un tercero, de quien en principio -- ni en final -- no tenemos porqu\'e 
confiar, no parece a primera vista una buena idea\ldots Existen diferentes
motivos para llegar a externalizar nuestra seguridad; por un lado, como hemos
comentado, un {\it outsourcing} permite a la empresa que lo contrata 
despreocuparse relativamente de su seguridad para centrarse en sus l\'{\i}neas
de negocio. Adem\'as, al contratar a personal especializado -- al menos en 
principio -- en la seguridad se consigue -- tambi\'en en principio -- un nivel
mayor de protecci\'on, tanto por el factor humano (el contratado ha de tener
gente con un alto nivel en diferentes materias de seguridad para poder ofrecer
correctamente sus servicios) como t\'ecnico (dispondr\'a tambi\'en de productos
y sistemas m\'as espec\'{\i}ficos, algo de lo que probablemente el contratante 
no puede disponer tan f\'acilmente). Te\'oricamente, estamos reduciendo riesgos
a la vez que reducimos costes, por lo que parece que nos encontramos ante la
panacea de la seguridad.\\
\\Desgraciadamente, el mundo real no es tan bonito como lo se puede escribir 
sobre un papel; el {\it outsourcing} presenta {\it a priori} graves 
inconvenientes, y quiz\'as el m\'as importante sea el que ya hemos adelantado:
dejar toda nuestra seguridad en manos de desconocidos, por muy buenas 
referencias que podamos tener de ellos. Muchas empresas dedicadas a ofrecer
servicios de gesti\'on externa de seguridad est\'an formadas por ex-piratas
(<incluso existen algunas de ellas que se jactan de esto!), lo cual no deja de
ser contradictorio: estamos dejando al cuidado de nuestro reba\~no a lobos, o
cuanto menos ex-lobos, algo que plantea, o debe plantear, ciertas 
cuestiones \'eticas. No voy a expresar de nuevo mi punto de vista (que no deja
de ser una mera opini\'on) acerca de los piratas, porque creo que ya ha quedado 
suficientemente claro en diferentes puntos de este documento, as\'{\i} que cada
cual act\'ue como su conciencia o sus directivos le indiquen. Por supuesto, 
tampoco quiero meter a todo este tipo de compa\~n\'{\i}as en un mismo saco, 
porque por l\'ogica habr\'a de todo, ni entrar ahora a discutir acerca de si
para saber defender un entorno hay que saber atacarlo, porque una cosa es saber
atacar (algo que se puede aprender en sistemas autorizados, o en nuestro
propio laboratorio, sin afectar a ning\'un tercero) y otra defender que s\'olo
un antiguo pirata es capaz de proteger correctamente un sistema.\\
\\Aparte de este `ligero' inconveniente del {\it outsourcing}, tenemos otros
tipos de problemas a tener tambi\'en en cuenta; uno de ellos es justamente el
l\'{\i}mite de uno de los beneficios de esta pol\'{\i}tica: ya que la 
externalizaci\'on permite a una empresa `despreocuparse' de su seguridad, 
podemos encontrar el caso -- nada extra\~no -- de un excesivo 
`despreocupamiento'. Actualmente, el abanico de servicios que ofrece cualquier
consultora de seguridad suele abarcar desde auditor\'{\i}as puntuales hasta
una delegaci\'on total del servicio pasando por todo tipo de soluciones
intermedias, y lo que justifica la elecci\'on de un
modelo u otro es un simple an\'alisis de riesgos: el riesgo de la soluci\'on
externalizada ha de ser menor que el nivel de riesgo existente si se gestiona
la seguridad de forma interna. En cualquier caso, al externalizar se suele
introducir una cierta p\'erdida de control directo sobre algunos recursos de la
compa\~n\'{\i}a, y cuando esa p\'erdida supera un umbral nos encontramos ante 
un grave problema; en {\bf ning\'un} caso es recomendable un desentendimiento
total de los servicios externalizados, y el contacto e intercambio de 
informaci\'on entre las dos organizaciones (la contratante y la contratada) han
de ser cont\'{\i}nuos y fluidos.\\
\\Cuanto m\'as alejada de las nuevas tecnolog\'{\i}as se encuentre la 
l\'{\i}nea de negocio de una determinada empresa, m\'as recomendable suele ser 
para la misma adoptar una soluci\'on de {\it outsourcing} (\cite{kn:vel02});
esto es evidente: una empresa frutera, independientemente de lo grande o 
peque\~na que sea, pero perteneciente a un \'area no relacionada con nuevas
tecnolog\'{\i}as, rara vez va a disponer de los mismos recursos humanos y 
t\'ecnicos para destinar exclusivamente a seguridad que una empresa de 
telecomunicaciones o inform\'atica. Es habitual -- y as\'{\i} debe ser -- que 
el nivel de externalizaci\'on sea mayor conforme la empresa contratante se
aleje del mundo de las nuevas tecnolog\'{\i}as, contemplando un amplio abanico
que abarca desde la gesti\'on de elementos concretos de protecci\'on (como un
{\it firewall} corporativo) o auditor\'{\i}as y {\it tests} de penetraci\'on 
puntuales hasta soluciones de externalizaci\'on total; en cualquier caso, es
necesario insistir de nuevo en el error de `despreocuparse' demasiado de la 
gesti\'on de nuestra seguridad: incluso a esa empresa frutera que acabamos de
comentar le interesar\'a, o al menos as\'{\i} deber\'{\i}a ser, recibir como
poco un informe mensual donde en unas pocas hojas, y sin entrar en aspectos
demasiado t\'ecnicos, se le mantenga al d\'{\i}a de cualquier aspecto relevante
que afecte a su seguridad.\\
\\>Qu\'e areas de nuestra seguridad conviene externalizar? Evidentemente, no
existe una respuesta universal a esta pregunta. Existen \'areas que por su
delicadez o criticidad no conviene casi nunca dejar en manos de terceros, como
es el caso de la realizaci\'on y verificaci\'on de {\it backups}: todos hemos
escuchado historias graciosas -- o terribles, seg\'un en que lado estemos -- 
relacionadas con errores en las copias de seguridad, como ejecutar la 
simulaci\'on de copia en lugar de una copia real para finalizar m\'as 
r\'apidamente el proceso de {\it backup}. No obstante, elementos importantes 
pero no cr\'{\i}ticos {\it a priori}, como los {\it tests} de penetraci\'on, de
visibilidad o las auditor\'{\i}as de vulnerabilidades, que habitualmente se
suelen externalizar, ya que incluso existen empresas de seguridad especializadas
en este tipo de acciones. Otro ejemplo de \'area a externalizar puede ser la
gesti\'on de los cortafuegos corporativos, trabajo que en demasiadas ocasiones
recae sobre el \'area de Seguridad propia y que como veremos en el pr\'oximo
punto no deber\'{\i}a ser as\'{\i}. En definitiva, no podemos dar un listado
donde se indiquen por orden las prioridades de externalizaci\'on, ya que es
algo que depende completamente de cada compa\~n\'{\i}a y entorno; ha de ser el
personal de la propia compa\~n\'{\i}a, asesorado por consultores de seguridad 
y por abogados (recordemos que la LOPD est\'a ah\'{\i}), quien decida qu\'e y
de qu\'e forma gestionar en {\it outsourcing}.
\section{El `\'Area de Seguridad'}
Casi cualquier mediana o peque\~na empresa posee actualmente lo que se viene a
llamar el `\'Area de Seguridad', formada pocas veces a partir de gente que haya
sido incorporada a la plantilla a tal efecto, y muchas a partir del reciclaje
de personal de otras \'areas de la corporaci\'on, t\'{\i}picamente las de 
Sistemas o Comunicaciones. En este punto vamos a hablar brevemente de este
\'area y su posici\'on corporativa, haciendo referencia tanto a sus funciones
te\'oricas como a sus problemas de definici\'on dentro del organigrama de la
organizaci\'on.\\
\\>Cu\'al es la funci\'on de este \'area? Realmente, mientras que todo el 
personal sabe cual es el cometido de la gente de Desarrollo, Sistemas o Bases
de Datos, el del \'area de Seguridad no suele estar definido de una forma 
clara: al tratarse en muchos casos, como acabamos de comentar, de personal 
`reciclado' de otras \'areas, se trabaja mucho en aspectos de seguridad --
para eso se suele crear, evidentemente --, pero tambi\'en se acaba realizando
funciones que corresponden a otras \'areas; esto es 
especialmente preocupante con respecto a Sistemas, ya que en muchas ocasiones
el personal de Seguridad trabaja `demasiado cerca' de esta otra \'area, llegando
a realizar tareas puramente relacionadas con Sistemas, como la gesti\'on de los
cortafuegos (no nos referimos a la definici\'on de pol\'{\i}ticas ni nada 
parecido, sino \'unicamente al manejo del mismo). Y si a esto le a\~nadimos que
a muchos de los que nos dedicamos a este mundo nos gusta tambi\'en todo lo 
relacionado con sistemas -- sobre todo si son Unix :-) --, pues llegamos a una
situaci\'on en la que nadie pone pegas a hacer un trabajo que no le corresponde,
con lo cual se vicia el \'area de Seguridad centr\'andose \'unicamente en
aspectos t\'ecnicos pero descuidando otros que son igual o m\'as importantes.
Por si esto fuera poco, existe una serie de funciones en conflicto a la hora
de gestionar la seguridad corporativa, t\'{\i}picamente la del administrador
de seguridad frente a la del administrador de sistemas, de bases de datos, o
incluso frente al operador de sistemas y los desarrolladores.\\
\\Te\'oricamente, el \'area de Seguridad ha de estar correctamente definida y
ser independiente de 
cualquier otra de la compa\~n\'{\i}a, y por supuesto de la direcci\'on de la
misma: aunque en la pr\'actica sea casi imposible conseguirlo, no podemos 
definir una pol\'{\i}tica de obligado cumplimiento para todos los trabajadores
excepto para nuestros jefes. Evidentemente, ha de contar con el apoyo total de
la direcci\'on de la entidad, que debe estudiar, aprobar y respaldar 
permanentemente, y de forma anticipada, las decisiones de seguridad que el 
\'area decida llevar a cabo (siempre dentro de unos l\'{\i}mites, est\'a
claro\ldots).\\
\\El trabajo del \'area debe ser m\'as normativo que t\'ecnico: no podemos 
dedicar
al personal de la misma a cambiar contrase\~nas de usuarios o a gestionar
(entendido por `manejar') los cortafuegos corporativos, sino que el \'area de
Seguridad debe definir pol\'{\i}ticas e implantar mecanismos que obliguen a su
cumplimiento, o cuanto menos que avisen a quien corresponda en caso de que una
norma no se cumpla. T\'ecnicamente esto no es siempre posible, ya que ni todos
los sistemas ni todas las aplicaciones utilizadas tienen porqu\'e ofrecer 
mecanismos que nos ayuden en nuestra seguridad, pero cuando lo sea es funci\'on 
del \'area bien
su implantaci\'on o bien su auditor\'{\i}a (si es implantado por otro \'area).
Si una determinada aplicaci\'on no soporta las exigencias definidas en la
pol\'{\i}tica de seguridad, pero a\'un as\'{\i} es imprescindible su uso, el
\'area de Seguridad debe recordar que el cumplimiento de la normativa es 
igualmente obligatorio; al oir esto, mucha gente puede poner el grito en el 
cielo: en realidad, si el programa no cumple las especificaciones del \'area de
Seguridad, lo l\'ogico ser\'{\i}a prohibir su uso, pero funcionalmente esto no 
es siempre (realmente, casi nunca) posible: no tenemos m\'as que pensar en una 
aplicaci\'on corporativa que venga gestionando desde hace a\~nos las
incidencias de la organizaci\'on, y que evidentemente la direcci\'on no va a 
sustituir por otra `s\'olo' por que el \'area de Seguridad lo indique. Si 
nuestra pol\'{\i}tica marca que la longitud de clave m\'{\i}nima es
de seis caracteres, pero esta aplicaci\'on -- recordemos, vital para el buen
funcionamiento de la organizaci\'on -- acepta contrase\~nas de cuatro, el 
usuario {\bf no debe} poner estas claves tan cortas por mucho que la 
aplicaci\'on las acepte; si lo hace
est\'a violando la pol\'{\i}tica de seguridad definida, y el hecho de que el 
programa le deje hacerlo no es ninguna excusa. La pol\'{\i}tica es en este 
sentido algo similar al c\'odigo de circulaci\'on: no debemos sobrepasar los 
l\'{\i}mites de velocidad, aunque las caracter\'{\i}ticas mec\'anicas de 
nuestro coche nos permitan hacerlo y aunque no siempre tengamos un policia 
detr\'as que nos est\'e vigilando.\\
\\Aparte de la definici\'on de pol\'{\i}ticas y la implantaci\'on (o al menos la
auditor\'{\i}a) de 
mecanismos, es tarea del \'area de Seguridad la realizaci\'on de an\'alisis de
riesgos; aunque el primero sea con diferencia el m\'as costoso, una vez hecho
este el resto no suele implicar mucha dificultad. Por supuesto, todo esto ha de
ser cont\'{\i}nuo en el tiempo -- para entender porqu\'e, no tenemos m\'as que
fijarnos en lo r\'apido que cambia cualquier aspecto relacionado con las 
nuevas tecnolog\'{\i}as -- y permanente realimentado, de forma que la 
pol\'{\i}tica de seguridad puede modificar el an\'alisis de riesgos y 
viceversa. Asociados a los riesgos se definen planes de contingencia para
recuperar el servicio en caso de que se materialice un problema determinado;
esta documentaci\'on ha de ser perfectamente conocida por todo el personal al
que involucra, y debe contemplar desde los riesgos m\'as bajos hasta los de
nivel m\'as elevado o incluso las cat\'astrofes: >qu\'e pasar\'{\i}a si 
ma\~nana nuestro CPD se incendia o el edificio se derrumba?, >cu\'anto 
tardar\'{\i}amos en recuperar el servicio?, >sabr\'{\i}a cada persona qu\'e 
hacer en este caso?\ldots 
