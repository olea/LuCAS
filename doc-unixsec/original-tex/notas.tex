\chapter{Notas del autor}
El mundo de la seguridad inform\'atica es demasiado amplio y complejo como para
ser tratado exhaustivamente en ning\'un trabajo, mucho menos en uno tan simple
como este; aqu\'{\i} \'unicamente he intentado resumir
una visi\'on global de diferentes aspectos relacionados con la seguridad, 
especialmente con Unix y redes de computadores (estas \'ultimas tan de moda hoy
en d\'{\i}a\ldots Unix por desgracia no tanto). Este trabajo est\'a casi 
completamente extra\'{\i}do de mi proyecto final de carrera, que estudiaba la
seguridad en los sistemas Unix y la red de la Universidad Polit\'ecnica
de Valencia (UPV), de forma que si aparece alguna referencia a `nuestra red' o
`nuestros equipos' -- aunque he intentado eliminar todos los ejemplos y 
comentarios relativos a UPV, por motivos obvios -- ya sabemos de qu\'e se 
trata. A pesar de haberlo revisado bastantes veces (lo bueno de no tener vida 
social es que uno tiene mucho tiempo para leer ;-), evidentemente 
existir\'an errores y faltar\'an datos que podr\'{\i}an haber aparecido, por lo 
que agradecer\'e cualquier
sugerencia o cr\'{\i}tica (constructiva, las destructivas directamente a 
{\tt /dev/null}) que se me quiera hacer. Para ponerse en contacto conmigo
se puede utilizar la direcci\'on de correo electr\'onico que utilizo 
habitualmente: {\tt toni@aiind.upv.es}.\\ 
\\Durante la realizaci\'on de este proyecto ni se han maltratado animales ni
se han utilizado productos Microsoft; personalmente, siempre he considerado 
rid\'{\i}culo
hablar de seguridad en Unix -- incluso de seguridad en general -- y hacerlo
utilizando productos de una compa\~n\'{\i}a que tantas veces ha demostrado su
desinter\'es por la tecnolog\'{\i}a frente a su inter\'es por el {\it 
marketing}. 
El trabajo entero ha sido creado sobre diversos clones de Unix (principalmente
Solaris y Linux, y en menor medida HP-UX, BSD/OS, IRIX, AIX e incluso Minix). 
El texto ha sido escrito \'{\i}ntegramente con {\tt vi} ({\tt vi} es {\bf EL} 
editor, el resto de editores no son {\tt vi} ;-) y compuesto
con \LaTeX, de Leslie Lamport; realmente, algunos fragmentos han sido 
extra\'{\i}dos de documentos que hice hace tiempo con {\tt troff} (s\'{\i}, 
{\tt troff}), de Joe Ossanna y Brian Kernighan, transformados a \LaTeX\ mediante
{\tt tr2tex}, de Kamal Al--Yahya, y retocados con algo de paciencia. Para las 
figuras simples he utilizado el 
lenguaje PIC, tambi\'en de Brian Kernighan, y para las que son m\'as complejas
{\tt xfig}. La captura de alguna pantalla se ha hecho con {\tt xwd} y {\sc
gimp}, y el retoque y transformaci\'on de im\'agenes con este \'ultimo junto a 
{\tt xv} y {\tt xpaint}.\\
\\Quiero agradecer desde aqu\'{\i} la colaboraci\'on desinteresada de algunas
personas que han hecho posible este trabajo (m\'as concretamente, que hicieron
posible mi proyecto final de carrera): Pedro L\'opez (Departamento
de Inform\'atica de Sistemas y Computadores, UPV),
Jon Ander G\'omez (Departamento de Sistemas Inform\'aticos y Computaci\'on,
UPV), Vicent Benet (Centro de C\'alculo, UPV), Jos\'e Manuel Pasamar
(Centro de C\'alculo, UPV) y Albert Ortiz (Universitat Polit\`ecnica de
Catalunya). Y por supuesto a mi director, Ismael Ripoll (Departamento de 
Inform\'atica de Sistemas y Computadores, UPV).\\
\\Tras publicar la versi\'on 1.0 de este trabajo, algunos de los primeros 
comentarios que se me han hecho trataban sobre los posibles problemas legales
derivados de la falta de una licencia para el documento; desconozco hasta qu\'e
punto esos problemas son reales, pero de cualquier forma para tratar de 
evitarlos he decidido adoptar la {\it Open Publication License} como formato de
licencia bajo la que distribuir mi trabajo, al menos de forma temporal. Eso
b\'asicamente implica (en castellano plano) que puedes imprimir el documento, 
leerlo, fotocopiarlo, regalarlo o similares, pero {\bf no} venderlo; este
trabajo es gratuito y pretendo que lo siga siendo. Si alguien lo encuentra
\'util, que me apoye moralmente con un {\it e-mail} :), y si alguien lo 
encuentra {\bf muy} \'util (lo dudo) que destine el dinero que crea que 
pagar\'{\i}a por esto a cosas m\'as \'utiles. >Sab\'{\i}as que cada minuto
mueren de hambre aproximadamente doce ni\~nos en el tercer mundo? En el tiempo
que te puede costar leer estas notas con un m\'{\i}nimo de inter\'es habr\'an
muerto unos veinticinco; mientras que nosotros nos preocupamos intentando 
proteger nuestros sistemas, hay millones de personas que no pueden perder el 
tiempo en esas cosas: est\'an demasiado ocupadas intentando sobrevivir.\\
\\Ah, por \'ultimo, ser\'{\i}a imperdonable no dar las gracias a la gente que ha
le\'{\i}do este trabajo y me ha informado de erratas que hab\'{\i}a en \'el;
he intentado corregir todos los fallos encontrados, pero a\'un habr\'a errores, 
por lo que repito lo que dec\'{\i}a al principio: todos los comentarios 
constructivos son siempre bienvenidos. Debo agradecer especialmente a David 
Cerezo el inter\'es que demostr\'o en las versiones iniciales de este 
documento, as\'{\i} como todas las observaciones que sobre las mismas me hizo 
llegar.
\vspace{1cm}\\
{\bf NOTAS A LA VERSI\'ON 2.0}\\
No hay mucho que a\~nadir a lo dicho hace casi dos a\~nos; y es que las cosas
apenas han cambiado: el panorama en Espa\~na -- en cuanto a seguridad se 
refiere -- sigue siendo desolador, las empresas tecnol\'ogicas caen d\'{\i}a a
d\'{\i}a, la investigaci\'on en materias de seguridad (si exceptuamos la 
Criptograf\'{\i}a) es nula, y poco m\'as. S\'olo dar las gracias una vez m\'as 
a todos los que 
han publicado o se han hecho eco de este documento (Kript\'opolis, HispaLinux,
IrisCERT, Hispasec, Asociaci\'on de Internautas\ldots) y tambi\'en a toda la 
gente que lo ha leido (al menos en parte ;-) y ha perdido unos minutos 
escribi\'endome un {\it e--mail} con alg\'un comentario; realmente es algo que
se agradece, y aunque tarde en responder al correo, siempre trato de 
contestar.\\
\\Como algunos de los comentarios acerca del documento que me han llegado
hablaban del `excesivo' tama\~no del mismo, en esta nueva versi\'on he cambiado
la forma de generar el fichero PDF; he convertido todas las im\'agenes a
formato PNG y despu\'es utilizado {\tt pdflatex} para compilar los ficheros,
habiendo modificado previamente el c\'odigo mediante un sencillo {\it script}. 
Aunque a\'un ocupa bastante, hay que tener en cuenta que estamos hablando de
unas 500 p\'aginas de documento\ldots
\vspace{1cm}\\
{\bf TODO}\\
Igual que hace casi dos a\~nos, sigue en pie la intenci\'on de crear 
cap\'{\i}tulos nuevos (las redes privadas virtuales es mi principal tema 
pendiente) y de comentar la seguridad de mecanismos como DNS, RPC, NIS o 
NFS\ldots espero disponer de algo m\'as de tiempo para poder hacerlo. Quiero 
tambi\'en escribir m\'as acerca de la detecci\'on de intrusos, no s\'e si en 
este documento o en uno aparte, ya que es quiz\'as el tema que m\'as me 
interesa y en lo que m\'as trabajo actualmente. Y finalmente, en mi lista de
cosas para hacer, pone {\bf dormir} (s\'{\i}, lo pone en negrita) como algo que 
tambi\'en queda pendiente :)
\vspace{1cm}\\
{\bf HISTORY}\\
{\bf Versi\'on 1.0} (Julio\'{}00): Documento inicial.\\
{\bf Versi\'on 1.1} (Agosto\'{}00): Peque\~nas correcciones e inclusi\'on de la
{\it Open Publication License}.\\
{\bf Versi\'on 1.2} (Septiembre\'{}00): M\'as correcciones. Ampliaci\'on del 
cap\'{\i}tulo dedicado a servicios de red.\\
{\bf Versi\'on 2.0} (Mayo\'{}02): Cap\'{\i}tulos dedicados a los sistemas de 
detecci\'on de intrusos y a los ataques remotos contra un sistema. Sustituci\'on
del cap\'{\i}tulo referente al n\'ucleo de algunos sistemas Unix por varios
cap\'{\i}tulos que tratan particularidades de diferentes clones con mayor 
detalle. Desglose del cap\'{\i}tulo dedicado a los sistemas cortafuegos en dos,
uno te\'orico y otro con diferentes casos pr\'acticos de estudio. Ampliaci\'on
de los cap\'{\i}tulos dedicados a autenticaci\'on de usuarios (PAM) y a 
criptograf\'{\i}a (funciones resumen). Ampliaci\'on del cap\'{\i}tulo dedicado
a pol\'{\i}ticas y normativa, que ahora pasa a denominarse {\it `Gesti\'on de
la seguridad'}.\\
{\bf Versi\'on 2.1} (Julio\'{}02): Alguna correcci\'on m\'as e inclusi\'on de
la {\it GNU Free Documentation License} (implica que el c\'odigo fuente en 
\TeX\ pasa a ser libre).
