%%%%%%%%%%%%%%%%%%%%%
% Seccion: Examenes %
%%%%%%%%%%%%%%%%%%%%%
\section{Ex�menes}

Esta es una gu�a de preguntas que se pueden formular en un ex�men te�rico. Son preguntas conceptuales que requieren de respuestas concisas.

\begin{enumerate}

\item ?`Donde se puede encontrar ayuda dentro de un sistema GNU/Linux?

\item ?`A qu� se refiere la sigla <<GNU>> en el nombre GNU/Linux?

\item ?`Qu� es una distribuci�n? Nombre algunas.

\item ?`Qu� tipos de permisos se le pueden asignar a los archivos?

\item ?`Cuales son las ventajas y desventajas que existen entre un entorno gr�fico, y un entorno de terminal de texto?

\item Adem�s del precio, ?`que otra gran ventaja tiene el software libre frente al sofware cerrado?

\item ?`C�mo se representan los dispositivos (perif�ricos) en GNU/Linux? Ejemplifique.

\item ?`Donde se guardan las configuraciones del sistema?

\item ?`Qu� tipo de archivos y que importancia tienen los que se encuentran en los directorios \comando{/bin} y \comando{/sbin}?

\item ?`Qu� herramientas usar�a para comprimir y descomprimir archivos en el GNU/Linux?

\item ?`Cuales son las ventajas de tener un administrador de paquetes como el \comando{rpm} para administrar la instalaci�n/desinstalaci�n de software?

\item ?`Con qu� herramientas se pueden cambiar los permisos de archivos?

\item ?`Qu� ventajas presenta la organizaci�n del sistema de archivos de GNU/Linux?

\item ?`A que se refiere la filosof�a RTFM?

\item ?`Cuales son las funciones principales que se pueden realizar con los paquetes RPM?

\item ?`Cu�l es la licencia por la cual se distribuye la mayor�a del software para GNU/Linux incluyendo este mismo curso? ?`Qu� caracter�sticas tiene?

\item ?`Qu� navegadores de p�ginas web se pueden utilizar en GNU/Linux?

\end{enumerate}

