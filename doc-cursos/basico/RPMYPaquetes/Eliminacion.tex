%%%%%%%%%%%%%%%%%%%%%%%%
% Secci�n: Eliminaci�n %
%%%%%%%%%%%%%%%%%%%%%%%%
\section{Eliminaci�n}

Muchas veces los discos r�gidos no son suficientes para cubrir todas
las necesidades. A falta de espacio en disco, se debe eliminar
los paquetes menos usados o los m�s grandes.

La eliminaci�n es tan f�cil como la Instalaci�n. Instalamos el paquete
``xmms'', ahora vamos a proceder a eliminarlo.

Esta vez no hace falta poner el CD-ROM de GNU/Linux.

\begin{enumerate}

 \item Se carga el ``kpackage'' mediante una terminal o consola
       escribiendo \comando{kpackage} (siendo {\tt root}) o yendo al
       men� \menu{K-Utilidades-kpackage}. (fig. \ref{fig:kpackage-Inicial})

 \item Seleccionar el paquete a desinstalar, como por ejemplo
       \rama{RPM-Application-Multimedia}-xmms.
 (fig. \ref{fig:kpackage-Seleccionado-App-MM-Xmms})
 \figura{Selecci�n del paquete a eliminar}{kpackage-Seleccionado-App-MM-Xmms}

 \item Hay que presionar \boton{Desinstalar} en la pantalla
       principal. Saldr� un cuadro de di�logo similar al de la figura
	\ref{fig:kpackage-Desinstalar}.
 \figura{Desinstalar un paquete}{kpackage-Desinstalar}

 \item luego \boton{Desinstalar}

\end{enumerate}

Es un procedimiento muy muy simple... hay que tener cuidado con varios
paquetes:

\begin{itemize}
\item \rama{RPM-Applications}-kpackage
\item \rama{RPM-System Enviroment-Base}-rpm
\end{itemize}

Son los programas usados para instalar/desinstalar programas.  No
hay problemas en desinstalar ninguno de ellos pero al querer volverlos a
instalar... no existe el instalador de paquetes. (Jurar�a que estaba ah�
hace segundos)

Una categor�a muy peligrosa es \rama{RPM-System Enviroment} y todas sus
ramas.  A pesar de que estamos trabajando en memoria, si se corta la
energ�a luego de eliminar esos paquetes, es muy probable que no se
pueda arrancar m�s el sistema. Y existen muchas cosas mas entretenidas
para hacer con GNU/Linux que reinstalar todo de cero.

\newpage
