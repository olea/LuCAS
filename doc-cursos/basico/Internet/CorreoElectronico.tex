%%%%%%%%%%%%%%%%%%%%%%%%%%%%%%%
% Secci�n: Correo electr�nico %
%%%%%%%%%%%%%%%%%%%%%%%%%%%%%%%
\section{Correo electr�nico}

El mayor tr�fico de Internet actual no es el producido por los
programas de Web sino por el Correo Electr�nico.

Para utilizar el correo electr�nico hay que poseer una cuenta POP3
en alg�n servidor. Los datos importantes son:

\begin{itemize}
\item Cuenta
\item Clave
\item Servidor de POP3
\item Servidor de SMTP
\end{itemize}

Un programa bastante bueno de correos es el \comando{kmail} se encuentra
en \rama{K-Internet-Cliente de correo}. La primera vez que es usado
tiene que crear un directorio dentro del HOME para guardar los correos.
(Fig. \ref{fig:kmail-PrimeraVez}).

\figura{Primera vez que se ejecuta \comando{kmail}}{kmail-PrimeraVez}
\figura{Menu Opciones-Identidad en \comando{kmail}}{kmail-OpcionesIdentidad}

Luego aparece el menu de Opciones-Identidad
(Fig. \ref{fig:kmail-OpcionesIdentidad}) donde hay que completar con los
datos que van a aparecer en los emails a enviar. Existe una leng�eta
\emph{Red}, all� hay que poner SMPT como servidor de correos a
enviar. Si bien en Linux viene un servidor de correos llamado
\comando{sendmail} que puede requerir mayor configuraci�n y no esta
dentro de los l�mites del curso b�sico.

Tambi�n hay que a�adir una cuenta POP llendo a \boton{A\~{n}adir...}.
Aqu� hay que completar con datos del servidor, como muestra la figura
\ref{fig:kmail-ConfigurarCuenta}

\figura{Configurar una cuenta POP3 en \comando{kmail}}{kmail-ConfigurarCuenta}

Con estos datos ser�a suficiente como para enviar correos electr�nicos
y poder recibir. Un ejemplo terminado esta en figura \ref{fig:kmail-OpcionesRed}

\figura{Ejemplo de \comando{kmail} configurado}{kmail-OpcionesRed}

Un tema avanzado para este curso pero interesante para conocer y
tenerlo en cuenta es la leng�eta PGP. PGP significa Pretty Good
Privacy (privacidad bastante buena) que es el nombre de un programa de
encriptaci�n de mensajes para comunicaciones seguras. Es necesario
tener instalado el programa \comando{pgp} o bien la version GNU del
que se llama \comando{gpg} (este �ltimo lo recomendamos).

Al enviar el primer email va a pedir que introduzcamos un archivo de
firma. Por ahora no existe ninguno, entonces con el editor de textos
hay que escribir un simple archivo que contenga la firma.  Por
tradici�n se elige el archivo \comando{.signature} porque varios
programas de correos lo utilizan. De esta forma la firma ser� la misma
no importa que cliente de correos estemos usando.

 

