%%%%%%%%%%%%%%%%%%%%%%%%%%%%%%
% Secci�n: Utilidades varias %
%%%%%%%%%%%%%%%%%%%%%%%%%%%%%%
\section{Utilidades varias}

Es verdad que Internet comercialmente existe desde hace bastante poco,
y en muchos paises lleg� tarde, pero la tegnolog�a TCP/IP tiene varios
a�os. A lo largo de todos esos a�os se fueron desarrolando
herramientas para ayudar a mantener o diagn�sticar esas redes.

La intenci�n de este curso no es de cubrir el mantenimiento de redes
TCP/IP, tan s�lo mencionar que herramientas son �tiles para eventules
problemas usando Internet.

Todas las herramientas que utilizamos tienen origenes en consolas.
Por lo tanto se puede usar una terminal. En este curso no vamos a 
mostrar el uso en terminales porque puede llegar a tornarse tedioso
y cr�ptico en en ciertos ejemplos. 

En reemplazo vamos a utilizar una herramienta llamada \comando{knu} que
es un \emph{frontend} de los programas a tratar haciendo el uso
de los mismos m�s amigable.

\subsection{Ping}

?`C�mo sabemos si una computadora esta dentro de la red?

Es una pregunta relativamente simple de contestar sabiendo su \emph{
N�mero IP}. \comando{knu} se encuentra en \rama{K-Internet-Utilidades
de Red}. Se ingresa el n�mero IP (por ejemplo {\tt 127.0.0.1}
\footnote{{\tt 127.0.0.1 es la m�quina local.}}) y a continuaci�n
\boton{!`Adelante!} como muestra la figura
\ref{fig:UtilidadesDeRed-Ping}.

\figura{Ejemplo del uso de \comando{ping} en \comando{knu}}{UtilidadesDeRed-Ping}

Dar� informaci�n de la conexi�n. Normalmete en enlaces entre m�quinas
el dato que m�s importa es {\tt time} que expresa la latencia de red.
Aqu� es irrelevante, puesto que es la m�quina local.  De no andar esta
simple prueba es posible de que la configuraci�n de red no sea
correcta.

Tambi�n se pueden poner nombres de m�quinas, como por ejemplo {\tt
www.google.com} siempre que est� clickeado el rectangulo que dice
\emph{Resolver el nombre}.




