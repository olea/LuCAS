\begin{practica}

\begin{ejercicio}
  \enunciado Con los comandos aprendidos, mostrar de un directorio de
  varios archivos los primeros 10.

%  \solucion
\end{ejercicio}

\begin{ejercicio}
  \enunciado Modificar el ejercicio anterior para mostrar los 10 archivos
  de mayor tama�o, ordenados alfab�ticamente.

%  \solucion
\end{ejercicio}

\begin{ejercicio}
  \enunciado En un directorio con varios archivos, mostrar s�lo los
  que tienen una determinada terminaci�n como por ejemplo
  \archivo{.txt} utilizando \comando{grep} y \comando{find}.

%  \solucion
\end{ejercicio}

\begin{ejercicio}
  \enunciado Utilizando \comando{find} o una composici�n de varios
  comandos por tuber�as, mostrar s�lo los \emph{enlaces simb�licos} 
  existentes, en caso de no poseer, crear varios en varios subdirectorios con
  el comando \comando{ln}

%  \solucion
\end{ejercicio}


\end{practica}