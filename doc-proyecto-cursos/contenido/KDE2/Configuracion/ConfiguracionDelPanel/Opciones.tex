\porcion{Opciones}

\autor{\LDP}
\colaborador{\JAB}
\revisor{}
\traductor{}

%% Revisar
%\figura{Opciones del Panel}{CentroDeControlAplicaciones-Panel-Opciones}

Aqu� se podr�n personalizar las diferentes opciones de comportamiento
del Panel (figura
\ref{fig:CentroDeControlAplicaciones-Panel-Opciones}). En el primer
cuadro \emph{Letreros de ayuda de men�} se encuentra la opci�n
\emph{Mostrar ayudas de men�} que activa o desactiva el uso de unos
peque�os carteles que aparecen como ayuda cuando se deja quieto el
puntero del mouse sobre alg�n sitio del Panel (en ingl�s se los llama
\emph{Tooltips}). Por defecto esta opci�n est� activada, y al estarlo,
se utiliza la barra de deslizamiento para configurar el tiempo de
retardo que tomar� en aparecer cada una de estas ayudas.

En el cuadro \emph{Visuales} se tienen 3 opciones a personalizar, la
primera, \emph{Auto-ocultar panel} hace que el Panel se esconda
autom�ticamente (si est� activada), pudiendo configurar la
velocidad y el retardo de esta acci�n. La opci�n \emph{Auto-ocultar
barra de tareas} es lo mismo que la anterior, pero para la barra de
tareas; y la opci�n \emph{Animar mostrar/esconder} de estar activada
(por defecto lo est�), muestra una animaci�n al esconderse el Panel o
la barra de tareas.

El cuadro inferior --\emph{Otros}-- configura varios aspectos
funcionales del bot�n de men� \boton{K} del Panel.  La primer
opci�n \emph{Entradas del Men� Personal Primero} hace que aparezcan
las configuraciones personales del men� antes que los men�es que
vienen por defecto en el sistema. La segunda opci�n \emph{Carpetas de
Men� Primero} organiza el men� de tal forma que las categor�as
aparezcan arriba de las opciones simples. La tercer opci�n \emph{Reloj
muestra el tiempo en formato AM/PM} sirve para cambiar el modo del
reloj entre AM, PM y 24hs. Por �ltimo, la cuarta opci�n \emph{Reloj
muestra el tiempo en latidos de Internet} cambia el modo del reloj, del
tradicional a un tipo de hora ``universal'' definido por la empresa
SWATCH para el uso del mismo en Internet.

