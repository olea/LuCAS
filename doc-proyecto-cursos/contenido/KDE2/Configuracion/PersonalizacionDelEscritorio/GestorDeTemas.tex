\porcion{Gestor de Temas}

\autor{\LDP}
\colaborador{\JAB}
\revisor{}
\traductor{}

Es posible cambiar a�n m�s el aspecto del entorno, cambiar el color
de los bordes de las ventanas, de la barra de herramientas, cambiar
los botones de maximizado, minimizado, etc.  Esto se hace por medio de
los \emph{Temas de Escritorio}, el KDE trae algunos preinstalados,
pero en Internet hay muchos m�s, se los puede conseguir en
\sitio{http://kde.themes.org}.

Cuando se selecciona el \emph{Gestor de Temas} en el \emph{Centro de Control
de KDE}, aparece una pantalla como la que se puede ver en la figura
\ref{fig:CentroDeControlEscritorio-GestorDeTemas}; simplemente se debe
seleccionar el Tema de Escritorio en la lista de la izquierda y la
visualizaci�n previa de la derecha se ir� actualizando para mostrar al
usuario una im�gen de ejemplo del Tema seleccionado. Si se necesita
agregar alg�n Tema que se haya conseguido por Internet por ejemplo, lo
que se debe hacer es oprimir \boton{Add...} y seleccionar el archivo
del Tema nuevo (�stos archivos son del estilo
\comando{nombre-del-tema.ktheme}), y autom�ticamente se agregar� a la
lista para poder ser seleccionado.

\figura{Gestor de Temas de Escritorio del KDE}{fig:CentroDeControlEscritorio-GestorDeTemas}{width=10cm}{KDE2/Configuracion/PersonalizacionDelEscritorio/CentroDeControlEscritorio-GestorDeTemas.png}

Si por alguna raz�n se quiere volver al Tema original del KDE,
simplemente hay que seleccionar el Tema ``Default'' o presionar
\boton{Predeterminado} y la apariencia del escritorio vuelve a ser
como en el principio.
