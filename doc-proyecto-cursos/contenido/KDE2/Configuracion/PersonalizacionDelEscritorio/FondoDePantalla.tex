\porcion{Fondo de pantalla}

\autor{\LDP}
\colaborador{\JAB}
\revisor{}
\traductor{}

En la categor�a \emph{LookNFeel} del \emph{Centro de control KDE} se
tienen las diferentes opciones para configurar el aspecto y la
funcionalidad del entorno, la opci�n \emph{Fondo} \footnote{Tambi�n se puede acceder haciendo click con el boton derecho sobre el escritorio y seleccionanado la opcion ``Fondo''} se utiliza para configurar el color o imagen
del fondo de los escritorios.

\figura{Configuraci�n del fondo de pantalla}{fig:CentroDeControlEscritorio-Fondo}{width=10cm}{KDE2/Configuracion/PersonalizacionDelEscritorio/CentroDeControlEscritorio-Fondo}

Como se ve en la figura \ref{fig:CentroDeControlEscritorio-Fondo}, se
puede configurar cada escritorio por separado, o unificar la
configuraci�n para todos seleccionado la opci�n \emph{Fondo com�n}. En
la pagina \emph{Fondo} se selecciona el tipo de fondo a utilizar,
estos pueden ser:
\begin{itemize}
	\item{Un color �nico de fondo.}
	\item{Varios tipos de gradientes entre dos colores.}
	\item{Un programa de fondo.}
	\item{Una p�gina web como fondo de pantalla.}
\end{itemize}
En la p�gina \emph{Papel Tapiz} configuramos si queremos poner como
fondo de pantalla un archivo de imagen y de que manera mostrarlo,
adem�s permite seleccionar m�ltiples papeles tapices que se van
cambiando cada un intervalo de tiempo determinado,para realizar esto
seleccionar la opci�n ``M�ltiples Papeles Tapiz'' y presionar
\boton{Configuraci�n m�ltiple...}, para agregar las im�genes a mostrar
y configurar el intervalo de tiempo.  Por ultimo se le puede dar un
toque mas ``art�stico'' al fondo de nuestro escritorio, para esto
seleccionamos la p�gina \emph{Avanzado} donde podemos mezclar el color
de fondo seleccionado en la primer p�gina con el papel tapiz elegido,
existen distintos tipos de mezclas con diferentes grados de intensidad.
Con \boton{Aplicar} guardamos los cambios realizados.





