\porcion{Estilo del entorno}

\autor{\LDP}
\colaborador{\JAB}
\revisor{}
\traductor{}


Otro aspecto en la personalizaci�n del entorno, es el estilo de
dibujado de los \emph{Objetos visuales}. KDE permite elegir entre
distintos estilos, por ejemplo: el estilo t�pico de Windows95 ( ver
\ref{fig:CentroDeControlEscritorio-Estilo-W95} ), el estilo
predeterminado de KDE2 y el estilo del Sistema 001 ( ver
\ref{fig:CentroDeControlEscritorio-Estilo-System001}. Podemos cambiar
y previsualizar los estilos seleccion�ndolos de la lista de
estilos. Puede suceder que el cambio de estilo tambi�n modifique el
cambio de colores del entorno.

\figura{Estilo de objetos visuales como Windows95}
{fig:CentroDeControlEscritorio-Estilo-W95} 
{width=10cm}{KDE2/Configuracion/PersonalizacionDelEscritorio/CentroDeControlEscritorio-Estilo-W95}
\figura{Estilo de objetos visuales como System 001}
{fig:CentroDeControlEscritorio-Estilo-System001}
{width=10cm}
{KDE2/Configuracion/PersonalizacionDelEscritorio/CentroDeControlEscritorio-Estilo-System001}



La opci�n que dice \emph{Barra de Men� arriba de la pantalla al estilo
de MacOS} simular�a de alguna manera el estilo de los men�s como se
utilizan en las computadoras Macintosh, es decir, en vez de que cada
ventana tenga su propia barra de men�s, existe una barra de men�s
general para todas las aplicaciones arriba de la pantalla, a medida
que se va intercambiando de ventana en ventana (de aplicaci�n en
aplicaci�n), esta barra de men�s va cambiando.


La segunda  opci�n, que dice \emph{Aplicar fuentes y colores a
aplicaciones no-KDE} sirve para darle a las aplicaciones que no son
espec�ficas de KDE (el Netscape por ejemplo), un ``look'' parecido a
las dem�s aplicaciones que si lo son, asign�ndole el color de ventanas
y botones bastante similares.

La tercera opci�n, \emph{Usar anti-alias para fuentes e iconos},
permite utilizar en KDE2 las t�cnicas de dibujado de los servidores
gr�ficos que mejora la calidad de textos y gr�ficos, disminuyendo el
efecto serrucho de sus bordes.

Debajo de �sto, hay un cuadro llamado \emph{Estilo y opciones de las
barras de herramientas} que se utiliza para configurar las barras
de herramientas, las opciones son:

\figura{Barras de herramientas con solo iconos}
{fig:CentroDeControlEscritorio-Estilo-SoloIconos}
{width=10cm}
{KDE2/Configuracion/PersonalizacionDelEscritorio/CentroDeControlEscritorio-Estilo-SoloIconos}

\figura{Barras de herramientas con solo texto}
{fig:CentroDeControlEscritorio-Estilo-SoloTexto}
{width=10cm}
{KDE2/Configuracion/PersonalizacionDelEscritorio/CentroDeControlEscritorio-Estilo-SoloTexto}

\figura{Barras de herramientas con iconos junto a texto}
{fig:CentroDeControlEscritorio-Estilo-TextoJuntoIconos}
{width=10cm}
{KDE2/Configuracion/PersonalizacionDelEscritorio/CentroDeControlEscritorio-Estilo-TextoJuntoIconos}

\figura{Barras de herramientas con texto sobre iconos}
{fig:CentroDeControlEscritorio-Estilo-TextoBajoIconos}
{width=10cm}
{KDE2/Configuracion/PersonalizacionDelEscritorio/CentroDeControlEscritorio-Estilo-TextoBajoIconos}


