\porcion{Personalizaci�n del comportamiento de las ventanas}

\autor{\JAB}
\colaborador{\LDP}
\revisor{}
\traductor{}

\figura{Secci�n de configuraci�n del escritorio}{fig:PersonalizacionDeLasVentanas}
{width=10cm}
{KDE2/Configuracion/PersonalizacionDeLasVentanas}


Otro aspecto importante en la personalizaci�n del escritorio en KDE es
c�mo configurar el modo de comportamiento de las ventanas,esto se
realiza en la secci�n \emph{LookNFeel}...\emph{Comportamiento de las ventanas} del
\emph{Centro de Centro KDE} (figura \ref{fig:PersonalizacionDeLasVentanas}), lo primero que podemos
determinar es la pol�tica de focalizaci�n\footnote{Una ventana tiene
el foco cuando se puede interactuar con ella por teclado y mouse, es decir cuando est� en
primer plano} de las ventanas, estas pueden ser:

\begin{description}

\item[Pulsar para focalizar]: Solamente clickear sobre la
      ventana para activarla.  
\item[Foco sigue al rat�n]: Simplemente
      ubicar el puntero del mouse sobre la ventana. Si el puntero
      luego se mueve a otra �rea del escritorio donde no hay ventanas,
      la ultima retiene el foco.  
\item[Foco bajo el rat�n]: La
      ventana bajo el foco posee el foco, sin embargo si el puntero del
      mouse esta sobre un �rea del escritorio donde no hay ventanas,
      ninguna ventana se activara.
\item[Foco estrictamente bajo el rat�n]: Similar a <<Foco bajo el rat�n>> pero mas estricto en su interpretaci�n.

\end{description}

	
Una vez determinada la pol�tica de focalizaci�n, hay dos opciones que
determinar, las cuales solo tienen efecto con las ultimas tres
pol�ticas. Marcando <<Subir autom�ticamente>>, KDE lleva una ventana
al frente si el mouse esta sobre la ventana por un determinado periodo
de tiempo ( esta opci�n es �til para la pol�tica: <<Foco sigue al
rat�n>> ), poniendo en manifiesto la capacidad de configuraci�n de
KDE, tambi�n es posible configurar el retardo a utilizar. Un periodo
entre 0 y 300 ms es correcto para un manejo tranquilo de las
ventanas. Si no se utiliza la opci�n de <<Subir autom�ticamente>>
asegurarse de que este seleccionada la opci�n de <<Pulsar para
subir>>.

La siguiente pagina \emph{Acciones} posee una amplia variedad de
configuraciones para la personalizaci�n de las acciones a que debe
tomar KDE en el momento que el usuario interact�a con una o varias
ventanas. Primeramente seleccionamos que acci�n tomar al hacer doble
click sobre la barra de titulo\footnote{Secci�n superior de la ventana
donde se encuentra el titulo o nombre de esta.}, esta puede ser, en
primer lugar, maximizar la pantalla (si no lo esta) y restaurarla (si
estaba maximizada), la otra opci�n es recoger la ventana, lo cual
significa <<arrollarla como una persiana>> quedando solo la barra de
titulo. Las dem�s opciones nos permiten personalizar que acciones
tomar con cada bot�n del mouse al presionar en un determinado lugar de
una ventana seg�n el estado que posea en ese momento.  Algunas de las
acciones son:
\begin{description}

\item[Subir]: Lleva la ventana al frente de todas las dem�s.	
\item[Bajar]: Lleva la ventana atr�s de todas las dem�s.
\item[Mover]: Mover la ventana.
\item[Redimensionar]: Cambiar el tama�o de la ventana.
\item[mas..]

\end{description}

Para probar los cambios, como siempre en el \emph{Centro de Control
KDE} presionar \boton{Aplicar}, en caso de querer volver al origen de
todo, presionamos \boton{Predeterminado}.


Por ultimo, en la pagina \emph{Avanzado} encontramos configuraciones
sobre el dibujado y animaciones de las ventanas. Algunas de estas son:

\begin{description}

\item[Mostrar contenido en ventanas en movimiento] Esta opci�n, 
cuando est� activada, hace que al mover una ventana, se vea todo el
contenido de la ventana en movimiento, si se posee una m�quina no muy
potente, es aconsejable desactivar esta opci�n, as� al mover una
ventana solo veremos el recuadro en movimiento.

\item[Mostrar contenido en ventanas al redimensionar] Es la misma
 funcionalidad que la opci�n anterior, pero cuando se redimensiona una
 ventana.  

\item[Animar la minimizaci�n y restauraci�n] Permite
 visualizar como se minimiza o restaura una ventana, pudiendo
 determinar la velocidad. 

\item[Permitir mover y modificar el tama�o de las ventanas maximizadas].

\end{description}
