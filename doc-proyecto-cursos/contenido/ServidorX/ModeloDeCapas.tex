%%%%%%%%%%%%%%%%%%%%%%%%%%%%%%%%%%%%%%%%%%%%%%
% Porci�n: Modelo de capas del protocolo X11 %
%%%%%%%%%%%%%%%%%%%%%%%%%%%%%%%%%%%%%%%%%%%%%%

% Nota: Pensar un mejor t�tulo
\porcion{Modelo de capas del protocolo X11}

\autor{\LDP}
\colaborador{}
\revisor{}
\traductor{}

El hecho de que el servidor X haya sido dise�ado para ser portable a
diferentes arquitecturas hizo que el funcionamiento de �ste con las
aplicaciones gr�ficas funcionen en modo de capas. La divisi�n de
tareas en distintas capas es algo que se ve en muchos aspectos del
funcionamiento de GNU/Linux dada la facilidad de mantenimiento de los
programas que se logra.

\figura{Modelo de capas con el entorno
KDE}{fig:ModeloDeCapasConKDE}{width=6cm}{ServidorX/ModeloDeCapasConKDE}

En la figura \ref{fig:ModeloDeCapasConKDE} vemos que como capa inferior, se
tiene al hardware, el cual interact�a con su capa inmediata superior,
que en este caso es el servidor X, es decir que X se encarga de lidiar
con las particularidades de cada tarjeta de v�deo, manejar las
funciones de red, y el dibujado b�sico en pantalla\footnote{Cuando se
habla de dibujado b�sico, se est� hablando de pintado de puntos en
pantalla.}. Estas funciones de dibujado, X se las provee a su capa
superior la cual depender� del entorno gr�fico que se est�
manejando. En el caso de KDE por ejemplo, la capa superior a X es la
biblioteca llamada Qt, �sta provee las funciones de armado de
ventanas, movimiento de las mismas, y dibujado de botones, flechas,
punteros, etc. las cuales son aprovechadas por la capa correspondiente
a las \emph{kde-libs} es decir, a las bibliotecas de KDE dedicadas a
dar una colecci�n de funciones de construcci�n de cajas de di�logos
est�ndar, men�es, etc. a las aplicaciones KDE y al entorno KDE
propiamente dicho.

En el caso del entorno gr�fico GNOME\footnote{Se dan estos dos
ejemplos ya que son los entornos gr�ficos mas populares en estos
d�as.} el esquema de capas es muy similar al anterior, como se ve en
la figura \ref{fig:ModeloDeCapasConGNOME}

\figura{Modelo de capas con el entorno
GNOME}{fig:ModeloDeCapasConGNOME}{width=6cm}{ServidorX/ModeloDeCapasConGNOME}

Lo �nico que cambia son las capas por encima del servidor X, y es por
eso que es posible tener en un mismo equipo varios entornos gr�ficos y
usar el que mas se ajuste a las necesidades del usuario, a diferencia
de otros sistemas operativos que s�lo poseen una opci�n.

Lo interesante del modelo de capas, es que �stas son intercambiables:
si por ejemplo se cambia la tarjeta de v�deo (la capa mas inferior),
no es necesario tener que cambiar el resto del conjunto de programas,
s�lo basta con que la capa superior (el servidor X en este caso) pueda
comunicarse con la nueva capa de hardware para que todo funcione
correctamente. Otra ventaja es que si el desarrollo de las diferentes
capas se realiza por diferentes grupos de personas, cuando se
actualiza una de las capas el conjunto en su totalidad se actualiza
autom�ticamente, si por ejemplo el equipo GNOME lanza su nueva versi�n
de la biblioteca GTK+, solamente har� falta instalar esta nueva
versi�n y el resto de los programas que conforman las otras capas no
deber�an tocarse, quedando todo funcionando
correctamente\footnote{Esto es lo que normalmente ocurre, pero a veces
puede tomar un poco m�s de trabajo dependiendo del nivel de
actualizaci�n.}.



