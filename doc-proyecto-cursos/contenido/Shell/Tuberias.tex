\porcion{Tuber�as (pipes)}
\autor{\NC}
\colaborador{}
\revisor{}
\traductor{}
\label{subsection:tuberias}

Podr�amos representar cada programa como una <<caja negra>> que
tiene una entrada y una salida que se pueden unir entre ellos.

%Debido a que la entrada por defecto es el teclado y la 
%salida por defecto es terminal, graficaremos cuando sea necesario
%ambos.

El ejemplo que utilizamos se encuentra esquematizado en la figura
\ref{fig:redireccion-sort} siendo la entrada est�ndar
el teclado y la salida est�ndar <<el terminal>> o por simplicidad la
pantalla.

\figura{Esquema de entrada y salida est�ndar del ejemplo}{fig:redireccion-sort}%
{width=6cm}%
{Shell/RedireccionIO-Diagrama-Sort.png}

Vamos a suponer un caso ficticio donde necesitamos todas las
definiciones de cada palabra en un texto. Primero las ordenamos
alfab�ticamente, luego utilizamos un comando ficticio llamado
\comando{diccionario} que toma palabras de la entrada est�ndar y
las reescribe junto a su significado en la salida est�ndar.

Su esquema se ve en la figura \ref{fig:redireccion-dict}.
En este caso nombramos por separado las entradas y salidas est�ndares de los
dos programas, pero la <<uni�n>> entre ambos programas se puede considerar
como un s�lo <<tubo>>. 

\figura{Esquema de entrada y salida est�ndar del ejemplo 2}{fig:redireccion-dict}%
{width=12cm}%
{Shell/RedireccionIO-Diagrama-Diccionario}

En ese tubo, el flujo est� en un estado intermedio, donde est� ordenado,
pero no tiene las definiciones de diccionario.

En la l�nea de comandos esto se escribe de la siguiente manera:

\begin{vscreen}
$ sort | dicccionario
\end{vscreen}
%$

Donde el caracter \comando{|} representa la conexi�n entre la
salida est�ndar de un programa y la entrada est�ndar de otro.

Con este fuerte y simple concepto se pueden concatenar gran cantidad
de programas como si fuera una l�nea de producci�n en serie para
generar resultados complejos.

Para mejorar nuestro ejemplo sacaremos las palabras repetidas, antes
de mostrarlas con definiciones. Suponiendo que exista un programa
llamado \comando{sacar-repetidas}, la l�nea de comando ser�a:

\begin{vscreen}
$ sort | sacar-repetidas | diccionario
\end{vscreen}
%$

Simple, utilizando herramientas sencillas logramos algo un poco m�s
complicado. El inconveniente que tenemos en este ejemplo es que hay
que escribir aquello a procesar. Normalmente queremos utilizar archivos
como entrada de nuestros datos. Es necesario un comando que env�e a
salida est�ndar un archivo, as� se procesa como la entrada est�ndar
del \comando{sort} y contin�a el proceso normalmente. Este comando es
\comando{cat}. La sintaxis es simple {\tt cat
  \emph{nombre-de-archivo}}.

Quedando nuestro ejemplo:

\begin{vscreen}
$ cat archivo.txt | sort | sacar-repetidas | diccionario
\end{vscreen}
%$

... crea un glosario de las palabras que se encuentren en \archivo{archivo.txt}

La combinaci�n de comandos es incalculable y brinda posibilidades
enormes. Veamos algunos ejemplos.

\begin{ejemplo}

En el caso que se quieran buscar procesos con el string \emph{http}:

\begin{vscreen}
$ ps ax | grep http
3343 ?        S      0:00 httpd -DPERLPROXIED -DHAV
3344 ?        S      0:00 httpd -DPERLPROXIED -DHAV
3975 ?        S      0:00 httpd -DPERLPROXIED -DHAV
12342 pts/6    S      0:00 grep http
\end{vscreen}
%$

Si queremos eliminar la ultima linea podemos volver a usar
\comando{grep} con la opcion \comando{-v}

\begin{vscreen}
$ ps ax | grep http | grep -v grep
3343 ?        S      0:00 httpd -DPERLPROXIED -DHAV
3344 ?        S      0:00 httpd -DPERLPROXIED -DHAV
3975 ?        S      0:00 httpd -DPERLPROXIED -DHAV
\end{vscreen}
%$

Se pueden filtrar las l�neas que contengan la palabra
\emph{linux} del archivo \archivo{arch1.txt} y luego
mostrarlas en un paginador como \comando{less}.

\begin{vscreen}
$ cat arch1.txt | grep linux | less
\end{vscreen}
%$

Podemos enviar los resultados  por correo a un amigo,

\begin{vscreen}
$ cat arch1.txt | grep linux | mail amigo@email.com
\end{vscreen}
%$


\end{ejemplo}
