\porcion{Smarthost con \comando{sendmail}}
\autor{\LDP}
\revisor{\LLC}
\traductor{}
\colaborador{}

Cuando \comando{sendmail} recibe un mensaje de correo que no
corresponde a una cuenta local del sistema, debe intentar entregar el
mensaje al servidor que corresponda para que llegue
correctamente. Algunas veces esto no es posible: por ejemplo un
servidor que est� aislado de Internet que se utilice como servidor
interno de correo de una oficina. Dicho servidor recibe un correo
desde uno de los usuarios de la oficina, dirigido a una cuenta
externa, entonces \comando{sendmail} deduce que no debe entregarlo
localmente; el siguiente paso a seguir es encontrar qu� servidor en
Internet maneja el dominio al cual pertenece la cuenta destino, pero
en este caso como el servidor se encuentra aislado de Internet. Esto
ser�a un problema.

Gracias a la funcionalidad de \emph{mail relaying} de
\comando{sendmail}, se puede configurar el servidor de correo para que
entregue todos los mensajes no locales a un servidor conocido. Este
servidor \emph{pasarela} estar�a conectado a Internet y por lo tanto
podr�a hacer llegar los mensajes dirigidos hacia fuera de la oficina
correctamente.

Para activar esta funcionalidad, se debe agregar la l�nea:

\begin{vscreen}
define(`SMART_HOST', `nombre.de.host')
\end{vscreen}

en el archivo \archivo{sendmail.mc} de configuraci�n de macros m4, o
agregar la l�nea:

\begin{vscreen}
DSnombre.de.host
\end{vscreen}

en el archivo de configuraci�n
\archivo{/etc/mail/sendmail.cf}\footnote{Se debe tener cuidado pues
siempre existe una l�nea con el comando \comando{DS} en el archivo
\archivo{sendmail.cf}, aunque no posea un \emph{SmartHost} definido.}.


