\porcion{Administraci�n de Cola de Mensajes}
\autor{\NC}
\colaborador{}
\revisor{\LLC}
\traductor{}

El servidor \comando{sendmail} posee una cola de mensajes a enviar
guardada en el directorio \archivo{/var/spool/mqueue} que contiene
archivos donde guarda los correos a enviar e informaci�n adicional
sobre el email.

Si bien se puede entrar al directorio para ver los mensajes a enviar, 
existe una utilidad que facilita el acceso: \comando{mailq}

\begin{vscreen}
[usuario@maquina usuario]$ mailq
Mail queue is empty
\end{vscreen}
%$

En el ejemplo no existe ning�n email a despachar. En el caso de que 
exista efectivamente un email, mostrar� algo similar a:

\begin{vscreen}
[root@maquina usuario]# mailq
MTA Queue status...
                /var/spool/mqueue (1 request)
-----Q-ID----- --Size-- -----Q-Time----- ------------Sender/Recipient-----------
g3M1J2S1028121      312 Sun Apr 21 22:20 <remitente@dominio.remitente.org.ar>
                 (host map: lookup (dominio.remoto.org.ar): deferred)
                                         <admin@dominio.remoto.org.ar>
\end{vscreen}

Para borrar un mensaje que est� en la cola de mails, se 
utiliza el identificador. En este caso {\tt g3M1J2S1028121},
y se busca en el directorio \archivo{/var/spool/mqueue} 
los archivos que comiencen con {\tt df} y {\tt qf} y seguido el
identificador.

\begin{vscreen}
[root@maquina usuario]# cd /var/spool/mqueue
[root@maquina mqueue]# ls *g3M1J2S1028121
dfg3M1J2S1028121  qfg3M1J2S1028121
\end{vscreen}

y luego borrar los archivos con \comando{rm *g3M1J2S1028121} como \usuario{root}.
