\porcion{Uso diario de GnuPG}
\autor{\LDP}
\colaborador{\SGG}
\revisor{\LLC}
\traductor{}

Una vez configurado el programa de cifrado, lo primero que se debe
hacer es \emph{generar un certificado de revocaci�n}, el cual servir�
para poder anular el par de claves en caso de olvido de la contrase�a,
o robo de la clave privada. Esto se hace de la siguiente forma:

\begin{vscreen}
usuario@maquina:~$ gpg --output cert-revocacion.asc --gen-revoke Usuario

sec  1024D/62B70584 2001-04-22   Usuario (233-8847) <usuario@maquina.dominio.com>

Create a revocation certificate for this key? y
Please select the reason for the revocation:
  1 = Key has been compromised
  2 = Key is superseded
  3 = Key is no longer used
  0 = Cancel
(Probably you want to select 1 here)
Your decision?
\end{vscreen}
%$

El argumento que se le debe pasar al par�metro \comando{--gen-revoke}
es el nombre que previamente se ingres� cuando se configur� GPG. El
programa pide la raz�n por la cual se genera el certificado de
revocaci�n y un comentario, luego GPG genera el archivo
\archivo{cert-revocacion.asc} (que es un archivo de texto) con el
certificado incluido. Este archivo deber�a guardarse en alg�n medio de
almacenamiento y esconderse donde nadie tenga acceso. Una de las
opciones v�lidas puede ser imprimir el archivo, guardar la hoja en
alg�n lugar seguro y luego borrar el archivo.

Por cada persona a la que se quiera enviar mensajes cifrados, se debe
tener almacenada su clave p�blica en el \emph{anillo de llaves} de
GPG. Para ver la lista de claves p�blicas que se tiene en cualquier
instante, se ejecutar� el siguiente comando:

\begin{vscreen}
usuario@maquina:~$ gpg --list-keys
/home/usuario/.gnupg/pubring.gpg
--------------------------------
pub  1024D/62B70584 2001-04-22 Usuario (233-8847) <usuario@maquina.dominio.com>
sub  2048g/7459EB6A 2001-04-22
\end{vscreen}
%$

Inicialmente se tendr� �nicamente la propia clave p�blica en el
anillo. Cuando se reciba alguna clave p�blica de otra persona, se la
debe agregar al anillo de la siguiente manera:

\begin{vscreen}
usuario@maquina:~$ gpg --import <nombre_archivo>
\end{vscreen}
%$

Donde \comando{nombre\_archivo} corresponde al archivo donde se
tengan la o las claves p�blicas a ingresar.

De manera similar, se necesitar� exportar la propia clave p�blica a un
archivo de texto para poder enviarla a aquellas personas que quieran
enviar mensajes cifrados. Esto se hace como sigue:

\begin{vscreen}
usuario@maquina:~$ gpg --armor --output <archivo_exportado> --export <nombre>
\end{vscreen}
%$

Donde \comando{archivo\_exportado} ser� el archivo donde se
almacenar� la clave p�blica en formato de texto, y \comando{nombre}
deber� corresponder con el nombre o la direcci�n de correo electr�nico
propia que se haya ingresado en los datos del programa. De hecho, este
comando sirve para exportar cualquier clave p�blica, la propia o la de
cualquiera que est� en el anillo, proveyendo el correspondiente nombre
o direcci�n electr�nica.

La clave p�blica exportada tendr� aproximadamente el siguiente
formato\footnote{De hecho esta clave que se incluye es la clave
  p�blica de uno de los autores de este manual, Lucas Di Pentima.}:

\begin{vscreen}
-----BEGIN PGP PUBLIC KEY BLOCK-----
Version: GnuPG v1.0.4 (GNU/Linux)
Comment: For info see http://www.gnupg.org

mQGiBDq6OwIRBADXaEdL7bTUR8HtgfNYz+Bzm5oEGM/vm3tUB1yDgMGZAkugCZt4
b41HX2LXCqmfMLQclS+0B6gYSfYGNG1v/VdGFdXEkrWzVV3QcTEFVDC79qac06eD
zerdIgjgVaPKjwIY8b9i1I2OisjY4Ey5gw2w1CoepCj9DMZZZ7d0tLJhhwCg5OB4
UXQtitHd8L6ASkc9Eyjg6x0EAKg61fRranXcCWhaVDoPvKvjThCTf53wVaw9eSVy
CJwy4f18cMRC+MYT5J9wni4dC2I1YkLGuNrUgb0SVVS2TZkcaI/4LZvIzxSupMww
yMedstQfYe1kzjY/ODnE3OYXyW6k5eEfBopNGO/J300/YDZ/OCXj/zq/TOb9Ztls
SEGSA/9XeoEUqe6lolXMJMAK1eO0TkgFm1B5R+mjnkXrh8z/Ofgw+HuD0Pr9imQi
YMk9ymT+swhC08hv5lDPe0iOHc5biyvKh6t654vIiJF8sds8hkiwW5RYmiVeB3hG
R4dBfVuOEGQXYNabwGSiqECJpXeCl2q7NVvijIUJFqsLTPcBd7Q7THVjYXMgRGkg
UGVudGltYSAoVGVsOiA1NCAzNDIgNDU5MzEyMikgPGx1Y2FzQGx1bml4LmNvbS5h
cj6IVwQTEQIAFwUCOro7AgULBwoDBAMVAwIDFgIBAheAAAoJEI+YP89qpU/Jo18A
oJQ0OyVfioYgCxxbU6f1buN6uKsfAJ4nVyK+YW+RnofcbHJga/SsxAJ2drkBDQQ6
ujsGEAQA8g+dqkNhhywhkj3d8dCqVicq8JjEFQyWIQkP1/Qg0CshljSWjeX1D/bn
fEOkxqt/oy/+ClqMMKABO764NEcu3B7zz16OK/uLuvddY3vfZiA82XJdxu5wXqk4
dKA+iorx5xtE2eNxVAFydXnR7KkiN4HOoVDNxTNGKA22wra3ND8AAwUD/iSd3NoP
zPdhF7/laBefp7vdDo7LRn3iLe7m1NbvXvtYMtNvtWP9LIjq7q1iHqsZw+5Xymkl
LbhaMhfUIoZhqaLr1L1IajuLZA8wUmYeHK/ZswLoEK0bJBYfVxS8gbpJG89PiQXK
PFPwWSHmesR13+nKUtoAsEsOaxRM/f4PPPlziEYEGBECAAYFAjq6OwYACgkQj5g/
z2qlT8ka+ACg24eZNxUhYJ+FF6P7Cd3CPZ/dlhsAnj8NrXcAEN6BbOdTWaS6FDmt
a3FW
=SP6W
-----END PGP PUBLIC KEY BLOCK-----
\end{vscreen}
