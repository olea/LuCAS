\porcion{Conclusi�n}
\autor{}
\colaborador{\SGG}
\revisor{\LLC}
\traductor{}

Este curso aborda el tema del cifrado como una herramienta de
comunicaci�n segura con otras personas a trav�s del uso de un servicio
no seguro, como lo es el correo electr�nico. Esto no quiere decir que
GPG s�lo sirva para cifrar y descifrar mensajes de texto. Al
contrario, GPG sirve para cifrar cualquier tipo de archivo, no
necesariamente que se deba enviar a otra persona, por ejemplo si el
alumno tiene un conjunto de documentos que quiere almacenar en un
CD-ROM como copia de seguridad, pero que no quiere que ninguna persona
que tenga acceso a ese CD-ROM pueda usar esos documentos, puede
cifrarlos con su propia clave p�blica y almacenar en CD-ROM los
archivos cifrados resultantes, teniendo la seguridad de que nadie podr�
descifrarlos al no tener la clave privada correspondiente.

A continuaci�n se describir�n los pasos necesarios para integrar GPG
con el cliente de correo \comando{pine} para poder usar el cifrado en
los mensajes de correo electr�nico de manera f�cil y c�moda.
