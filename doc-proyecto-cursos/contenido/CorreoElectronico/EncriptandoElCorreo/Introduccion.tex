\porcion{La seguridad del correo electr�nico}
\autor{}
\colaborador{\SGG}
\revisor{\LLC}
\traductor{}

\label{sec:gpg}
Generalmente se cree que el correo electr�nico es totalmente privado.
Que nadie, excepto el receptor del mensaje, puede leer el mensaje que
se env�a. Esto dista de ser verdad. De hecho, es mucho m�s f�cil espiar
el correo electr�nico que el correo postal, dado que a fin de
cuentas, el correo electr�nico termina siendo un archivo en alg�n o
algunos equipos en la red, a medida que viaja a su destino final.

Con lo dicho anteriormente, no se debe pensar que cada mensaje que se
env�a es le�do por alguna persona distinta a la que en realidad se le
envi� el mensaje, pero es muy recomendable tener siempre en mente que
un mensaje que viaja por la red puede ser revisado \footnote{Esto
  obviamente es ilegal, es una violaci�n a la privacidad y muchos
  pa�ses ya tienen leyes sobre este tema.}. Tarde o temprano, se
necesitar� enviar informaci�n sensible por correo electr�nico.
Entonces habr� que utilizar alg�n m�todo para \emph{cifrar} la
informaci�n de manera tal que s�lo el receptor del mensaje pueda
descifrarla correctamente.

Esta secci�n se encarga de introducir al alumno al tema del cifrado de
datos utilizando una herramienta libre: el \emph{GnuPG (GNU Privacy
Guard)}.
