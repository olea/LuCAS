\porcion{Leyendo y enviando correo}
\autor{}
\colaborador{\SGG}
\revisor{\LLC}
\traductor{}

Entrando a una carpeta, estaremos en el �ndice de mensajes de la
misma, vi�ndose algo parecido a la figura
\ref{fig:ClientesDeCorreo-Pine-IndiceDeMensajes}. Los mensajes que se
reciban nuevos tendr�n una <<N>> a la izquierda.

%\figura{�ndice de mensajes del \comando{pine}}{ClientesDeCorreo-Pine-IndiceDeMensajes}

Para ver el mensaje seleccionado, naturalmente se pulsa \boton{Enter},
como en la figura \ref{fig:ClientesDeCorreo-Pine-VistaDeEmail}. Para
responder el mail seleccionado, se utiliza la tecla \boton{R}. El
programa preguntar� si se quiere citar el mensaje anterior, esto es
bueno hacerlo, siempre y cuando las citas se mantengan en un l�mite
adecuado y no se abuse de las mismas.

%\figura{Leyendo mensajes en \comando{pine}}{ClientesDeCorreo-Pine-VistaDeEmail}

Si se quiere escribir un mensaje desde cero, la tecla \boton{C} en las
diferentes secciones del programa tiene la misma funcionalidad,
\emph{componer mensajes}. Al activar esta funci�n, el programa cargar�
el editor de mensajes, con el encabezado del mensaje a enviar arriba,
y el lugar para el cuerpo por debajo, como se puede observar en la
figura \ref{fig:ClientesDeCorreo-Pine-ComposicionDeMensaje}.

%\figura{Pantalla de composici�n de mensajes del \comando{pine}}{ClientesDeCorreo-Pine-ComposicionDeMensaje}

Se supone que el alumno sabe c�mo enviar correo electr�nico. Este
curso no se trata de ello, as� que no se explicar�n las funciones de
cada campo del encabezado del mensaje a componer, lo que s� vale la
pena aclarar, es el uso de la libreta de direcciones (addressbook) y
los archivos inclu�dos (file attach) dentro de los mensajes. Cuando el
cursor se encuentra en el campo \comando{To:}, pulsando \boton{Ctrl-T}
se puede seleccionar la direcci�n de destino que tengamos almacenada
en la libreta de direcciones.

Cuando se necesite enviar uno o varios archivos por correo
electr�nico, se debe posicionar el cursor en el campo
\comando{Attachmnt:} y pulsando \boton{Ctrl-T}, se carga un navegador
de disco como el que se ve en la figura
\ref{fig:ClientesDeCorreo-Pine-NavegadorDeArchivos}.

%\figura{Seleccionando un archivo a incluir en el mensaje}{ClientesDeCorreo-Pine-NavegadorDeArchivos}

Una caracter�stica del \comando{pine} que no se ha visto, es la del
\emph{addressbook}. Con esta libreta de direcciones se puede mantener
toda la lista de contactos de forma f�cil. Estando en la pantalla
principal, se puede acceder pulsando la tecla \boton{A}, y de esta
forma se obtiene la lista de contactos ingresados. Primeramente se
tendr� la lista vac�a. Cuando se necesite agregar un nuevo contacto,
se utiliza (como se puede ver en el men�) la tecla \boton{@},
apareciendo de esta manera una pantalla como la que se ve en la figura
\ref{fig:ClientesDeCorreo-Pine-AgregarContacto}.

%\figura{Agregando un contacto en la libreta de direcciones del \comando{pine}}{ClientesDeCorreo-Pine-AgregarContacto}

En el campo \comando{Nickname:} se debe ingresar el alias de la
persona que luego se utilizar� en el campo \comando{To:} cuando se
compone un mensaje. De esta manera, no har� falta escribir todo el
nombre de la persona.
