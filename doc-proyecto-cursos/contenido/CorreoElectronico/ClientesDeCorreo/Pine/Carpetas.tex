\porcion{Carpetas}
\autor{}
\colaborador{\SGG}
\revisor{\LLC}
\traductor{}

El uso de carpetas se hace evidentemente necesario cuando uno comienza
a recibir grandes cantidades de correo. Hemos hablado ya del
\emph{correo SPAM}, y por m�s que tengamos cuidado en no dejar en
muchos lugares nuestra direcci�n electr�nica, tarde o temprano caer�
en manos de un empresario sin escr�pulos que la utilizar� para enviar
su molesta publicidad. Es una actitud un tanto pesimista, pero es lo
que la experiencia a trav�s de los a�os ha demostrado en m�s de una
ocasi�n.

Tarde o temprano recibiremos cientos de mensajes diarios, y gran parte
de ellos ser�n de SPAM, ?`c�mo evitar desesperarse con tanta cantidad
de mensajes, si los interesantes s�lo son algunos pocos?. La respuesta
est� en el filtrado de mensajes, y la divisi�n de los mismos en
carpetas; el filtrado es algo que se explicar� m�s adelante. Por ahora
nos abocaremos a la tarea de crear carpetas.

%\figura{Lista de carpetas del \comando{pine}}{ClientesDeCorreo-Pine-ListaDeCarpetas}

En la figura \ref{fig:ClientesDeCorreo-Pine-ListaDeCarpetas} se puede
ver la lista de carpetas por defecto que existe cuando el
\comando{pine} es reci�n iniciado. Estas carpetas en realidad son
archivos que se encuentran almacenados en el directorio personal de
cada usuario, dentro del subdirectorio \comando{mail/}\footnote{A
  excepci�n de la carpeta \comando{INBOX}, que es una referencia al
  buz�n de entrada del servidor de correos. Generalmente este archivo
  tiene el mismo nombre de cada usuario, y se encuentra en el
  directorio \comando{/var/spool/mail/}}.

De las funciones del men�, se tienen 3 que sirven para administrar las
carpetas: \comando{Add}, \comando{Delete} y \comando{Rename}. Est� de
m�s explicar para que sirven.

Con las flechas del teclado se puede ir seleccionando una u otra
carpeta, y pulsando \boton{Enter} entraremos al �ndice de mensajes de
la carpeta seleccionada.

Vale la pena nombrar la funcionalidad de las tres carpetas por defecto
que el \comando{pine} crea al inicio:
\begin{description}
\item[INBOX] En esta carpeta se ir�n almacenando los mensajes que
  lleguen. El alumno ver� m�s adelante que mediante el uso de
  filtrado se podr� seleccionar la carpeta donde se guardan mensajes
  espc�ficos.
\item[sent-mail] Esta carpeta almacena los mensajes enviados por el
  usuario. Es �til a la hora de necesitar reenviar alg�n mensaje que
  por una u otra raz�n no ha llegado a su destino.
\item[saved-messages] Como ya se ver�, los mensajes pueden ser movidos
  de carpeta en carpeta. La carpeta por defecto a donde son movidos
  los mensajes es �sta. Generalmente la carpeta \comando{INBOX}
  contiene aquellos mensajes nuevos que van llegando, el usuario
  entonces los va leyendo, respondiendo y si cree conveniente, los
  guarda para un futuro, de lo contrario, los elimina. La carpeta
  \comando{saved-messages} es para esto.
\end{description}
