\porcion{Introducci�n}
\autor{\LDP}
\colaborador{}
\revisor{\LLC}
\traductor{}

?`Para qu� sirven los permisos de usuario? Bueno, es una pregunta
bastante obvia teniendo en cuenta que GNU/Linux es un sistema
operativo multiusuario. Cuando muchas personas utilizan un mismo
equipo, debe haber un mecanismo que sirva para diferenciar los
archivos de un usuario de los dem�s archivos.

Un concepto no del todo correcto es pensar que los usuarios se
utilizan exclusivamente por personas. Los procesos\footnote{Recordemos
  que un proceso es un programa en ejecuci�n.} que se ejecutan en un
sistema GNU/Linux tienen tambi�n un usuario <<due�o>>, que coincide
generalmente con el usuario que ejecut� dicho programa. Adem�s,
los \emph{demonios}\footnote{Procesos que dan servicios, como por
  ejemplo el servidor de p�ginas web.} tienen su propio usuario por
cuestiones de seguridad.

En esta secci�n se ver� el tema de los permisos de usuario desde el
punto de vista de un usuario com�n, y no de un administrador. M�s
adelante se ver�n los detalles de mayor complejidad que generalmente incumben
al administrador.
