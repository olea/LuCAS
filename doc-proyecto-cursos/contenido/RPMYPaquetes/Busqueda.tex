%%%%%%%%%%%%%%%%%%%%%
% Secci�n: B�squeda %
%%%%%%%%%%%%%%%%%%%%%
\porcion{Buscando paquetes}
\autor{\NC}
\colaborador{}
\revisor{}
\traductor{}


Se pueden buscar en un buscador com�n. Como puede ser
\sitio{www.google.com} o \sitio{www.altavista.com}. Si bien no es la
forma m�s �ptima, muchas veces es un buen comienzo.

Otra alternativa es ir a sitios especializados en el tema. Existen
varios \emph{rpmfinders}\footnote{Buscadores de paquetes RPM.}
para buscar infinidad de paquetes. Al principio suele deconcertar la
cantidad que existen. No s�lo para computadoras tipo PC sino para
diferentes arquitecturas.\footnote{Recordemos que Linux es un sistema
operativo multiplataforma.}

Un ejemplo de \emph{rpmfinder} es \sitio{http://www.rpmfind.net/linux/RPM}

Siempre nos debemos asegurar de encontrar los paquetes que digan {\tt
i386} para 386, 486 y Pentium I, {\tt i586} para Pentium MMX y K6 II/III
o {\tt i686} para Pentium II/III y similares.

%%%%%%%%%%%%
%%%% describir como copiar a la PC
Para bajar el paquete que hayamos seleccionado bastar� con presionar
el enlace correspondiente y se nos mostrar� el di�logo Guardar Como
de nuestro navegador. A la hora de elegir la ruta d�nde almacenar
nuestro paquete hemos de tener cuidado de no cambiar el nombre del
paquete puesto que, como hemos visto, este nos da informaci�n, sobre
versiones y tipos de arquitecturas.


