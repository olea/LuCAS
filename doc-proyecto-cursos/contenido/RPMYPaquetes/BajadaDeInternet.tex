%%%%%%%%%%%%%%%%%%%%%%%%%%%
% Secci�n: Bajada de Inet %
%%%%%%%%%%%%%%%%%%%%%%%%%%%
\porcion{Consiguiendo paquetes nuevos}
\autor{\NC}
\colaborador{}
\revisor{}
\traductor{}

Es muy probable que los programas especializados no se encuentren en el
CD-ROM de GNU/Linux, al igual que las versiones actualizadas que se
deber�n copiar de otros lugares. Un buen lugar es Internet puesto
que normalmente las �ltimas versiones aparecen all�.

Los paquetes deben ser elaborados a partir de una cantidad de archivos
a instalar junto con datos de configuraci�n. Muchos programadores no
fabrican su Software en paquetes RPM. Por lo tanto tiene que haber
alguna persona dedicada a ``empaquetar'' esos archivos.

Esto quiere decir que, si bien existen versiones nuevas de algunos
programas, no necesariamente est�n en paquetes RPM.




