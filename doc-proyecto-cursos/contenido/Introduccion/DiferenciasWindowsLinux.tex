\porcion{Diferencias entre Windows y Linux}
\autor{}
\colaborador{}
\revisor{\LLC}
\traductor{}

GNU/Linux, a diferencia de Windows, es multitarea real y multiusuario y
posee un esquema de seguridad basado en usuarios y permisos de
lectura, escritura y ejecuci�n establecidos a los archivos y
directorios. Esto significa que cada usuario es propietario de sus
archivos y otro usuario no puede acceder a estos archivos. Esta
propiedad no permite el contagio de virus entre archivos de diferentes
usuarios.

Una diferencia, quiz�s la m�s importante de todas, con respecto
a cualquier sistema operativo comercial, es el hecho de que es
\emph{software libre}, ?`qu� quiere decir esto? que junto con el
sistema, se puede obtener el c�digo fuente de cualquier parte del
mismo y modificarlo a gusto. �sto da varias ventajas, por ejemplo:

\begin{enumerate}
\item La seguridad de saber \emph{qu� hace} un programa tan solo viendo
el c�digo fuente, o en su defecto, tener la seguridad que al estar el
c�digo disponible, nadie va a agregar <<caracter�sticas ocultas>> en los
programas que distribuye.
\item La libertad que provee la licencia GPL permite a cualquier
programador modificar y mejorar cualquier parte del sistema. Esto da
como resultado que la calidad del software incluido en GNU/Linux sea
muy buena.
\item El hecho de que el sistema sea mantenido por una gran comunidad de
programadores y usuarios alrededor del mundo, provee una gran
velocidad de respuesta ante errores de programas que se van
descubriendo, que ninguna compa��a comercial de software puede
igualar.
\end{enumerate}

Adem�s de las ventajas anteriormente enumeradas, GNU/Linux es ideal
para su utilizaci�n en un ambiente de trabajo. Al menos dos razones justifican
esto:

\begin{enumerate}
\item Al ser software libre, no existe el costo de las licencias,
y una copia del sistema GNU/Linux puede instalarse en tantas
computadoras como se necesite.
\item Existen utilidades para el trabajo en oficina que son compatibles
con las herramientas de la serie MS-Office.
\end{enumerate}
