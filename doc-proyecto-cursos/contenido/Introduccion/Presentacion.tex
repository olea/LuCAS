%%% Parte de la Guia Completa a Linux ver 1.2.2
%%% de Jaime E. Gomez <jgomez@uniandes.edu.co>
%%% http://linuxcol.uniandes.edu.co/infolinux/docs/guia_linux/guia_linux.html
\porcion{Presentaci�n}
\autor{\jeg}
\colaborador{\NC}
\revisor{\LLC}
\traductor{}

Se pretende que aquellas personas que sigan los pasos de esta gu�a
sean capaces de instalar GNU/Linux en su propio computador y saber c�mo
manejar este sistema operativo. Se espera cumplir los objetivos y que
los lectores sean capaces de perdonar los errores que
seguramente existir�n. 

El temario est� dividido en cinco cap�tulos: el primero se dedica a 
la presentaci�n de GNU/Linux, historia y caracter�sticas t�cnicas,
el segundo muestra la instalaci�n de GNU/Linux; el tercero ense�a al 
usuario a utilizar las interfaces gr�ficas; el cuarto cap�tulo ilustra
la forma de realizar las configuraciones b�sicas para tener un 
sistema funcional y el quinto y �ltimo lo instruye en las �rdenes 
b�sicas y el funcionamiento general del sistema.

Esta gu�a se basa originalmente en el libro {\bf Curso Linux} del
Club de Inform�tica {\bf Disk�bolo} (http://diskobolo.mat.ucm.es) de
la Universidad Complutense de Madrid, escrito por Francisco Javier
Ahijado Mart�n-Navarro ({\it iCesofT}) y
 David Flores Santacruz ({\it Castor})
en el a�o 1997. En el a�o 1998 es actualizado, y 
ampliado por integrantes de el grupo de Usuarios de Linux en Colombia
{\bf LinuxCOL} (http://www.linuxcol.org) en la Universidad de Los
Andes en Santaf�
de Bogot�, Colombia, particularmente su director: Jaime Enrique G�mez
Hern�ndez ({\it Kasandra}). 

Hoy en d�a, {\bf Gu�a Completa a Linux} comparte a�n algunos p�rrafos
originales de {\bf Curso Linux}; pero ya lo supera ampliamente, no s�lo 
en tama�o sino en su contenido. Esta maneja informaci�n actualizada a 
la fecha e incluye, desde su versi�n 1.0, versiones para seis diferentes
distribuciones incluidas: {\bf Red Hat}, {\bf Mandrake}, {\bf Conectiva}, 
{\bf SuSE}, {\bf Slackware} y {\bf Debian}.

%\input{../linux/estilo}

