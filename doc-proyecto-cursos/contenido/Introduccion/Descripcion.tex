%%% Parte de la Guia Completa a Linux ver 1.2.2
%%% de Jaime E. Gomez <jgomez@uniandes.edu.co>
%%% http://linuxcol.uniandes.edu.co/infolinux/docs/guia_linux/guia_linux.html
\porcion{?'Qu� es GNU/Linux?}
\autor{\jeg}
\colaborador{\NC}
\revisor{\LLC}
\traductor{}

GNU/Linux es un sistema operativo desarrollado inicialmente para la
arquitectura de procesadores 386 de {\bf Intel}. Actualmente es 
el sistema
operativo que m�s plataformas soporta, incluyendo procesadores
de diversos proveedores y arquitecturas como 
Alpha (AXP) de {\bf DEC}/{\bf Compaq},
RS6000 de {\bf IBM}, 
M68K y PowerPC de {\bf Motorola} usados por {\bf Apple} e {\bf IBM}, 
IA64 o Itanium de {\bf Intel}, 
Sparc y Ultra Sparc de {\bf Sun}, 
procesadores MIPS usados por {\bf Silicon Graphics}, {\bf IBM}, {\bf DEC}
y muchos otros. 
Sin embargo, la versi�n m�s utilizada sigue siendo
sobre la arquitectura i386 y sus equivalentes, incluyendo la familia 
Pentium (Cl�sico, Pro, II , III y IV) de {\bf Intel}, la familia K6 (I, II y III),
y la familia Athlon (Duron y Athlon) de {\bf AMD}, los 686 y M6 de {\bf Cyrix},
los Winchip de {\bf IDT} y otros compatibles.

Es de esperarse que este panorama cambie a corto plazo con la 
introducci�n en el mercado de computadores de escritorio de los 
procesadores de 64 bits: la reducci�n de precio de los Alpha y su 
licenciamiento a {\bf Samsumg}; el Itanium de {\bf Intel} ya 
empieza a aparecer en servidores y en el 2002, {\bf AMD} lanzar� los
propios (SledgeHammer/ClawHammer) del que el dise�o ya se encuentra
listo y es publicitado en la red con el nombre de x86-64. 
Por esa raz�n, esta gu�a est� primordialmente orientada hacia
la instalaci�n en 
un PC, aunque no deja de ser �til como referencia para cualquier otra 
plataforma: una vez
instalado, GNU/Linux se utiliza y administra de la misma manera en cualquiera de 
ellas.

Podr�amos definir GNU/Linux como un sistema operativo basado en la filosof�a
de dise�o de UNIX y por esto muestra una cantidad de caracter�sticas
como multiusuario, multitarea, memoria protegida, de consumo de recursos
bajo demanda, etc. Como puede verse, no es nada obsoleto como algunas 
personas han tratado de afirmar: un auto �ltimo modelo es tecnolog�a 
de punta a�n cuando se base en los mismos principios alguna 
vez descritos y puestos en pr�ctica por Daimler en 1898. 
Este se ha actualizado, mejorado, perfeccionado al nivel que todos
conocemos hoy en d�a.

Al definirlo multitarea y multiusuario implica que puede haber 
varios usuarios utilizando un computador al mismo tiempo, y varios
procesos ejecut�ndose a la vez. Siendo rigurosos, a menos que se tenga un
computador con m�s de un procesador no es posible ejecutar varios procesos
al mismo tiempo. Sin embargo el cambio entre un proceso y otro es tan r�pido que
dan la impresi�n de estar ejecut�ndose a la vez.

Como se ha mencionado antes, GNU/Linux es un sistema operativo que se
desarroll� para 386. Por lo tanto los requerimientos m�nimos son un
procesador 386 o superior. GNU/Linux necesita al menos 4 Mbytes de memoria
para poder ejecutarse, sin embargo con 4 Mbytes solo servir� para realizar
tareas simples como enrutador o servidor de impresi�n sin interfaz
gr�fica de usuario. Es necesario tener 8 Mbytes o m�s para disponer de
una interfaz gr�fica. El espacio en disco duro requerido depende de lo
que se quiera hacer y los programas que se deseen instalar. Una
instalaci�n m�nima del sistema pueden ser s�lo 40 Mbytes, pero se
recomiendan al menos unos 200 Mbytes de disco para una instalaci�n 
peque�a.
Considerando las distribuciones esto puede cambiar: por ejemplo 
con {\bf Slackware} es posible tener instalaciones m�nimas de 40 Mbytes, 
mientras que {\bf SuSE} pide 80 Mb y con {\bf RedHat} o {\bf Mandrake}
es necesario 
alrededor de 300 Mb para el contenido m�nimo. En el otro extremo 
pueden usarse hasta 2 Gbytes de RAM en el kernel 2.2 y hasta 64 Gb 
en el 2.4, mientras que {\bf Mandrake} en su versi�n libre puede 
llegar a ocupar 2.3 Gb en disco y los paquetes disponibles para 
{\bf Debian} podr�an superar f�cilmente los 10 Gb.

