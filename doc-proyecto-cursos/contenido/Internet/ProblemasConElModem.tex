\porcion{Problemas con el modem}
\autor{\NC}
\colaborador{}
\revisor{\LLC}
\traductor{}

Antes de empezar a ilusionarse con conectarse a Internet, hay que
saber si el modem es o no un \emph{Winmodem}. Esta l�nea especial de
modems (o no tan modems) utilizan \emph{drivers} propietarios que la
mayor�a s�lo funcionan en Windows. En caso de una futura compra de
modem, lo primero que hay que fijarse es que no sea uno de estos.

Sabiendo que no es un \emph{Winmodem} el modem instalado, hay muchos
problemas que pueden tener los modems. Desde los m�s simples, como ser
que la l�nea de tel�fono est� desconectada, hasta los m�s complicados
y misteriosos que cuesta bastante encontrar sus causas. En el archivo
\comando{/usr/doc/HOWTO/Modem-HOWTO} hay mucha referencia sobre modems
y en especial un cap�tulo dedicado a problemas (Troubleshooting).

Para quienes no leen ingl�s, existe una traducci�n de ese archivo,
junto con varios HOWTOs (traducidos al castellano como COMOs),
llamado \comando{Modem-COMO} y puede estar en el directorio
\comando{/usr/doc/HOWTO/translations/es}.

