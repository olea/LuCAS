\porcion{Problemas con el pppd}
\autor{\NC}
\colaborador{}
\revisor{\LLC}
\traductor{}

Despu�s de pulsar \boton{Conectar}, el programa de \emph{Conexi�n a Internet},
llamado \comando{kppp}, transfiere los datos ingresados al
\comando{pppd} para que �ste realice la conexi�n en s�.

El \comando{pppd} es llamado tambi�n \emph{demonio pppd} y en el caso 
de que no se pueda conectar, puede surgir con varios cuadros de di�logo.

Uno muy com�n y con poca referencia es el que dice: ``El demonio pppd
muri� inesperadamente''. Esta frase suena absurda en especial a
usuarios reci�n iniciados en Linux.

Normalmente cuando el proveedor de Internet corta la conexi�n sin
motivo, aparece el demonio muerto. Esto puede ser por diversos
motivos:

\begin{itemize} 
 \item El provedor da ocupado y el modem no detecta que es se�al de
ocupado. (muchas compan�as tienden a poner grabaciones).

 \item Al ingresar una clave incorrecta, el provedor no da explicaci�n
y corta la comunicaci�n. 

 \item La cuenta puede estar siendo usada y el provedor s�lo permite 
una conexi�n por cuenta.
\end{itemize}

