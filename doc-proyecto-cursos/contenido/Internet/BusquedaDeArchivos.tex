%%%%%%%%%%%%%%%%%%%%%%%%%%%%%%%%%
% Secci�n: B�squeda de archivos %
%%%%%%%%%%%%%%%%%%%%%%%%%%%%%%%%%
\porcion{B�squeda de archivos}
\autor{\NC}
\colaborador{}
\revisor{\LLC}
\traductor{}

Es muy com�n querer buscar archivos en Internet, especialmente
software. La comunidad GNU/Linux se dedica a diario a mejorar y mejorar
los programas y documentos. Por lo tanto, es muy com�n tener una
distribuci�n vieja pero estar al d�a con los programas m�s usados.

La extensi�n original en el mundo de Unix para los paquetes fue
\comando{.tar.gz} o \comando{.tgz} que era simplemente un �rbol de
directorio con archivos.  Hoy, gracias a varios programadores, existen
paquetes que fueron pensados no s�lo para instalar sino para
desinstalar, mantenerlos e instalarlos por partes: los llamamos
RPM\footnote{RPM es un tipo especial de paquetes, hay varios tipos
como por ejemplo DEB.} en nuestro curso.
