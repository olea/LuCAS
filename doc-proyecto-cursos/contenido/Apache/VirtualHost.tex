\porcion{La directiva VirtualHost}
\autor{\NC}
\colaborador{}
\revisor{\LLC}

Para atender varios dominios en un s�lo servidor Apache se debe usar
la directiva de bloque {\tt \verb+<+VirtualHost\verb+>+}. Como su
nombre indica, realiza el trabajo de host virtual. Podemos
configurar las opciones comunes en el bloque principal y las opciones
espec�ficas a los hosts en los bloques correspondientes.

\begin{ejemplo}

\begin{vscreen}
User apache
Group apache

<VirtualHost www.dom1.com.ar>
ServerName www.dom1.com.ar
ServerAdmin webmaster@un-lugar.org.ar
DocumentRoot /var/www/sitio1/
</VirtualHost>

<VirtualHost www.dom2.com.ar>
ServerName www.dom2.com.ar
ServerAdmin webmaster@un-lugar.org.ar
DocumentRoot /var/www/sitio2/
</VirtualHost>
\end{vscreen}
\end{ejemplo}

El nombre de Host es necesario para diferenciar la petici�n del
usuario. Tambi�n se puede utilizar una direcci�n IP si existen m�ltiples
interfaces.

Cuando se ejecute Apache, dependiendo la petici�n del cliente, se
utilizar�n distintos \emph{Document Root}\footnote{se podr�a traducir como:
\emph{Directorio Ra�z de Documentos}}. Pero como configuraci�n en com�n, 
la ejecuci�n se har� con el usuario {\tt apache} en ambos casos.


Ahora, podemos combinar la directiva de bloque {\tt
\verb+<+Directory\verb+>+} para dar opciones particulares a cada
directorio de los hosts virtuales, pues sus
\emph{Document Root}s son distintos. 
