\porcion{C�mo generar un m�dulo}
\autor{\LDP}
\colaborador{\NC}
\revisor{}
\traductor{}

Cuando compilamos el n�cleo debemos especificar cuales controladores se
compilaran como m�dulos. Esto es muy sencillo, s�lo hay que poner la
letra {\tt M} en el men�. 

%% revisar
%\figura{Seleccionando opciones como m�dulos}{CompilandoNucleo-Modulos}

Siempre que terminamos de configurar la opciones del n�cleo hay que
ejecutar \comando{make dep}.

Ejecutando \comando{make modules} se compilar�n todos los m�dulos que
sean necesarios. Esto puede tardar desde unos pocos segundos hasta una
hora, dependiendo del hardware, la configuraci�n (cuantos m�dulos se
eligieron) y la versi�n del n�cleo.

Lo �nico que falta es copiar los m�dulos reci�n compilados al lugar
indicado (el directorio \archivo{/lib/modules/(versi�n del N�cleo)/}).
Esto se puede hacer \emph{manualmente} con \comando{cp} o tipear
\comando{make modules\_install}.

