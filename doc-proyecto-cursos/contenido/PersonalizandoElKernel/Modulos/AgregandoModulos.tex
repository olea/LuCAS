\porcion{Agregando m�dulos}
\autor{\LDP}
\colaborador{\NC}
\revisor{}
\traductor{}

La forma de agregar un m�dulo es relativamente simple. El comando
es \comando{insmod} y su sintaxis es:

\begin{vscreen}
insmod modulo [parametros]
\end{vscreen}

Siendo {\tt modulo} el nombre del m�dulo y {\tt parametros} los
parametros de ese modulo, que configuran al dispositivo
que controla. La documentaci�n de los par�metros se encuentran en
\archivo{/usr/src/linux/Documentation}.

El gran inconveniente de \comando{insmod} es que no controla las
dependencias necesarias, s�lo intenta cargar el m�dulo, si la
operaci�n no tiene �xito, finaliza su ejecuci�n.

Debido a que es casi imposible tener en mente todo el �rbol de
dependencias, existe una utilidad que realiza comprobaciones. Esta
utilidad es \comando{modprobe}. \comando{modprobe} utiliza
\comando{insmod} en el orden correcto y su sintaxis es:

\begin{vscreen}
modprobe modulo [parametros]
\end{vscreen}