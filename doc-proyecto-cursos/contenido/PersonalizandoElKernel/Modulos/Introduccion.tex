\porcion{Introduccion a los m�dulos}
\autor{\LDP}
\colaborador{\NC}
\revisor{}
\traductor{}

Una mejora extraordinaria al n�cleo fue la \emph{modularizaci�n} del
mismo. En un principio el n�cleo era \emph{monol�tico}, es decir, un
gran archivo que conten�a todos los controladores para los
dispositivos.

Un n�cleo monol�tico es m�s eficiente que uno modularizado, en parte
porque toda referencia se conoce en tiempo de compilaci�n y por otro
lado el sistema entero est� en memoria siempre. Como desventaja tiene
su gran tama�o, poca flexibilidad de incorporar nuevos controladores y
no acepta cambios en el c�digo existente.

Los m�dulos como contrapartida, se pueden cargar y descargar de
memoria en cualquier momento. Dando la libertad de poder utilizar s�lo
lo necesario. Y si estamos programando un controlador para cualquier
perif�rico, compilamos el controlador, lo cargamos a memoria, lo
probamos y luego se puede sacar de memoria, recompilar y seguir
probando. Todo esto sin rearrancar el sistema, ni cerrar los programas
que estamos usando.