\porcion{Instalando LILO}
\autor{\LDP}
\colaborador{\NC}
\revisor{}
\traductor{}

Una vez que est� correctamente configurado (mediante el archivo
\archivo{/etc/lilo.conf}), es necesario escribir el sector de arranque
del dispositivo\footnote{Utilizamos la palabra \emph{dispositivo} en
  vez de \emph{disco} porque puede ser que estemos configurando otra
  alternativa de arranque.}.  

Un error com�n es pensar que s�lo editando el archivo se guarda la
configuraci�n, hay que recordar que es un archivo m�s, incluso se
puede utilizar otro archivo.

Para grabar hay que ejecutar:
\begin{vscreen}
root@maquina:/root# lilo
Adding linux *
Adding windows
root@maquina:/root#
\end{vscreen}

Y listo. Si es que no surgi� ning�n problema. 

El asterisco ({\tt *}) indica que  n�cleo  se cargar� por
defecto (\emph{default}). En nuestro caso es la entrada que contiene
{\tt label=linux}.

La tecla \boton{TAB} muestra todas las posibilidades de nucleos a
cargar. Obviamente muestra el contenido de {\tt label}.
