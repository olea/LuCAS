\porcion{LILO}
\autor{\LDP}
\colaborador{\NC}
\revisor{}
\traductor{}
\label{subseccion:LILO}

Al hablar del n�cleo es inevitable hablar del cargador de Linux
o bien conocido como LILO\footnote{LInux LOader: Cargador de Linux} que
es el encargado de cargar en memoria el n�cleo y largarlo a correr.

El n�cleo es un archivo mas. Normalmente se encuentra en el disco
r�gido\footnote{Es probable que en otros cursos veamos como arrancar
  una m�quina sin disco r�gido a trav�s de la red}. Similar a un
archivo ejecutable, alg�n proceso debe ser el encargado de cargarlo y
luego ejecutarlo. Como todav�a no se encuentra nada en memoria, la
BIOS ejecuta c�digo de un lugar especial en el disco, llamado
\emph{boot sector}, que contendr� a LILO.

Una de las grandes funciones de LILO es la selecci�n de n�cleo a usar.
Normalmente al compilar diferentes n�cleos hay que elegir, por
ejemplo, entre alguno que tenga soporte para
\emph{clusters}\footnote{Cluster es una forma de utilizar varias
  computadoras para que todas calculen al mismo tiempo como si fuera
  una sola} 
o para emular \emph{SCSI} con un dispositivo \emph{IDE}\footnote{Es muy
  com�n cuando se desea utilizar una grabadora de CD-R}.

Tambi�n se pueden tener n�cleos de otros sistemas operativos (en el
caso de alg�n problema serio neurol�gico) como OS/2, toda la gama de
Windows, otros UNIX, etc.

El n�cleo de Linux acepta par�metros para personalizarlo o en el caso
de que no pueda auto detectar ciertos dispositivos o recursos. Estos
par�metros deben darse antes de que se cargue el n�cleo en si. Un ejemplo
ser�a:

\begin{vscreen}
LILO: linux mem=256M
\end{vscreen}

en este caso se saltea la auto detecci�n de cantidad de memoria
realizada por Linux y se presume que existen 256 MB de memoria.

Una lista m�s detallada de estos par�metros se encuentra en 
\archivo{/usr/src/linux/Documentation}. En este directorio est� 
toda la documentaci�n de los desarrolladores de n�cleo, separada
por m�dulo.


