%%% Parte de la Guia Completa a Linux ver 1.2.2
%%% de Jaime E. Gomez <jgomez@uniandes.edu.co>
%%% http://linuxcol.uniandes.edu.co/infolinux/docs/guia_linux/guia_linux.html
\porcion{Iniciando la instalaci�n}
\autor{\jeg}
\colaborador{\NC}
\colaborador{\SGG}
\revisor{\LLC}
\traductor{}

Para poder instalar GNU/Linux se ha de iniciar una versi�n especial del
sistema operativo preparada para realizar todo el proceso.
Para hacer esto existen como m�nimo tres opciones disponibles:

\begin{enumerate}

\item{{\bf Inicio desde el CD-ROM}: Si el computador es relativamente
nuevo seguramente puede iniciar el sistema operativo directamente desde 
el CD-ROM con tan solo dejar el disco en la unidad lectora y 
reiniciando el computador. Es necesario cambiar en el BIOS del 
computador la secuencia de arranque para iniciar desde
el CD-ROM.
En el momento de inicio del computador presione la tecla [{\sf Del}]
o la que indique el sistema para iniciar el {\it setup}:  puede
ser [{\sf F1}] � [{\sf F2}] seg�n el fabricante. 
Una vez dentro, escoja, {\sf BIOS FEATURES SETUP} y cambie la 
secuencia de inicio {\sf Boot Sequence} a algo como \verb+CDROM,C,A+.
Esta secuencia puede cambiar, en algunos {\it BIOS} se tiene
un men� propio para la secuencia de arranque 
(figura~\ref{fig_setup_cdrom}). 
Si no se quiere estar cambiando los par�metros
no hay que preocuparse, simplemente se act�a como si el computador no
tuviese esta caracter�stica.

\figura
{Selecci�n CD-ROM como primer dispositivo de inicio}
{fig_setup_cdrom}
{width=11cm}{images_gen/setup_cdrom.png}

}


\item{{\bf Desde DOS/Windows}: Si se utiliza DOS/Windows, al introducir
el CD-ROM se podr� observar la pantalla de arranque que permite realizar
disquetes de arranque y seguir varios tutoriales 
(ver figura~\ref{fig_win_autorun}).

\figura
{Men� de inicio desde Windows}
{fig_win_autorun}
{width=6cm}{images/win_autorun.png}

}

\item{{\bf Disquetes de Arranque}: Si a�n as� falla y no se consigue 
iniciar la instalaci�n queda la posibilidad de arrancar desde disquete.
Primero es necesario escoger una imagen de disquete indicada
para las condiciones de instalaci�n. En el directorio \verb+images+ 
se encuentra una serie de im�genes para diferentes medios de 
instalaci�n, tanto local como remota:

\begin{itemize}
\item{\verb+hd.img+: Instalaci�n est�ndar un disco duro.}
\item{\verb+cdrom.img+: Instalaci�n est�ndar desde CD-ROM.}
\item{\verb+network.img+: Instalaci�n por red usando los protocolos FTP, NFS o HTTP}
\item{\verb+pcmcia.img+: Instalaci�n en port�tiles con tarjetas PCMCIA. 
			La fuente puede ser CD-ROM o disco duro.}
\end{itemize}

Para hacer el disco de arranque se introduce un disquete de 3.5 pulgadas
sin errores en la unidad.
Si est� usando MS-DOS/Windows, en el directorio \verb+dosutils+ 
del CD-ROM, se encuentra el programa \verb+rawrite.exe+ que se utiliza 
de la siguiente forma:

\begin{vscreen}
 C:\> d:
 D:\> cd \dosutils
 D:\dosutils> rawrite
 Enter disk image source file name: ..\images\laimgn.img
 Enter target diskette drive: a:
 Please insert a formatted diskette into drive A: and
 press --ENTER-- : [Enter]
 D:\dosutils>
\end{vscreen}

o en una sola l�nea de comandos 

\begin{vscreen}
 D:\DOSUTILS>rawrite -f ..\images\laimgn.img -d a:
\end{vscreen}

Es recomendable realizar esto desde el modo DOS, ya que desde una
consola MS-DOS de Windows pueden presentarse problemas.
Si no desea salirse a DOS es recomendable usar \verb+rawritewin.exe+
el cual realiza la misma tarea desde una ventana de Windows (ver 
figura~\ref{fig_rawritewin}). 

\figura
{Copiando disco de arranque desde Windows}
{fig_rawritewin}
{width=6cm}{images_gen/rawritewin.png}

Si se encuentra en linux basta con la siguiente l�nea de comandos para
transferir la imagen al disquete.

\begin{vscreen}
# dd if=laimgn.img of=/dev/fd0 
\end{vscreen}

}

\end{enumerate}


\figura
{Flujo de la instalaci�n}
{fig_flujo_inst}
{width=7cm}{images/flujo_install.png}

