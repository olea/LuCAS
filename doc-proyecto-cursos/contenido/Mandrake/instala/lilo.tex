%%% Parte de la Guia Completa a Linux ver 1.2.2
%%% de Jaime E. Gomez <jgomez@uniandes.edu.co>
%%% http://linuxcol.uniandes.edu.co/infolinux/docs/guia_linux/guia_linux.html
\porcion{Instalando LILO}\label{sec_inst_lilo}
\autor{\jeg}
\colaborador{\NC}
\revisor{\LLC}
\traductor{}

Como se mencion� en la secci�n anterior, LILO ({\it LInux LOader})
es un peque�o programa que se instala usualmente en el MBR y que
permite seleccionar qu� sistema operativo va a arrancar. �ste
es necesario a�n cuando GNU/Linux sea el �nico sistema en el computador.

A continuaci�n se pregunta por opciones de su instalaci�n 
(figura~\ref{fig_instala_LILO}), en su orden: 

El tipo de sistema a instalar:

\begin{itemize}
\item{{\sf Grub}: Opci�n alterna a LILO ofrecida por \distribucion  }
\item{{\sf LILO con men� gr�fico}: La nueva versi�n bastante llamativa}
\item{{\sf LILO con men� texto}: Es posible que el men� gr�fico no funcione}
\end{itemize}

Como dispositivo de arranque es recomendado instalarlo en 
el MBR y para esto se selecciona {\sf /dev/hda}. 
No seleccione {\sf /dev/hda1} porque
seguramente destruir� el sistema de archivos de Windows/DOS. 
Por defecto LILO configura la {\sf Demora antes
de arrancar la imagen por omisi�n} en 5 segundos y 
estos pueden ser cambiados a su gusto. 

\figura
{Instalaci�n del LILO}
{fig_instala_LILO}
{width=11cm}{Mandrake/configura/install/pan_lilo.png}

Si se presiona [{\sf Avanzada}] se obtiene una extensi�n de
opciones, que incluye el uso de {\sf lba}, el cual es 
recomendado para discos duros nuevos, pero algunos BIOS
no lo soportan. El modo {\sf compacto} se ignora ya que s�lo es
necesario en casos muy particulares. El modo de v�deo Normal;
si se quiere borra el /tmp al inicio, el cual es recomendado,
y finalmente la memoria RAM del sistema.
Este �ltimo valor puede ser peligroso en algunas tarjetas 
madres que usan parte de la memoria RAM para el v�deo. Entonces
una m�quina con 64 Mbytes realmente tiene 56 porque ha usado
8 Mbytes para el v�deo. Si escoge un valor equivocado por 
exceso, su Linux no arrancar�.

Sea cuidadoso al escoger un modo de v�deo de {\it frame buffer}, i.e.
diferente a Normal, puede ser que su tarjeta de v�deo no lo soporte. 
El LILO instala por defecto una opci�n de v�deo est�ndar por seguridad. 
En caso de tener problema con su elecci�n, use {\it linux-nonfb} 
para entrar sin problema. 

Se presiona [{\sf Aceptar}] y la pantalla muestra las opciones de los
diferentes modos de arranque de LILO, que incluye los diferentes
sistemas operativos presentes en el disco.
Si tiene m�s usuarios en su m�quina es posible que desee que
este arranque Windows por defecto y
s�lo cuando se escoja en el boot, arranque GNU/Linux. 
Para realizar esto se se�ala {\sf dos} y se presiona [{\sf Modificar}].
En la pantalla se activa el bot�n de Sistema
por defecto {\sf Por omisi�n}. Se presiona [{\sf Aceptar}].
Una vez se est� satisfecho con la configuraci�n de LILO, se 
presiona [{\sf Hecho}].

