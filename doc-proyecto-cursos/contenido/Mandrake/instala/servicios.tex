%%% Parte de la Guia Completa a Linux ver 1.2.2
%%% de Jaime E. Gomez <jgomez@uniandes.edu.co>
%%% http://linuxcol.uniandes.edu.co/infolinux/docs/guia_linux/guia_linux.html
\porcion{Servicios en el Arranque}
\autor{\jeg}
\colaborador{\NC}
\revisor{\LLC}
\traductor{}

Todo GNU/Linux al iniciar lanza varios programas que proveen servicios
al sistema y a usuarios, conocidos como {\it daemons} o servidores
silenciosos, y mal traducidos se conocen como {\it demonios}. Estos 
programas de servicios son muy
livianos, pero es in�til e inseguro tenerlos corriendo si no se van 
a usar. En la siguiente pantalla se escogen cu�les van a ser
lanzados al inicio del sistema. Al colocar el rat�n sobre el nombre,
se abre una ventana de ayuda que describe qu� hace cada uno. Se van
a mencionar s�lo algunos de los m�s importantes a tener en cuenta.

\begin{itemize}
\item{{\sf autofs}: Controla el montaje autom�tico de dispositivos 
			extraibles como el CD-ROM y el disquete.}
\item{{\sf crond}: Ejecuta programas con una frecuencia o fechas programadas.}
\item{{\sf cupsd}: El programa que maneja las tareas de impresi�n \verb+CUPS+.}
\item{{\sf drakfont}: Mantiene actualizadas las fuentes para X-Window.}
\item{{\sf httpd}: Apache, servidor de p�ginas WWW y programas CGI.}
\item{{\sf kudzu}: Detecci�n y configuraci�n autom�tica de hardware.}
\item{{\sf linuxconf}: Realiza tareas de configuraci�n pendientes. }
\item{{\sf lpd}: El programa que maneja las tareas de impresi�n \verb+lpr+.}
\item{{\sf named}: El programa de servidor de nombres (DNS).}
\item{{\sf network}: Activa y desactiva las interfaces de red.}
\item{{\sf pcmcia}: Mantiene los dispositivos PCMCIA en los port�tiles.}
\item{{\sf postfix}: Agente de transporte de correo o MTA ({\it Mail Transport Agent}).}
\item{{\sf postgresql}: Servicio para la base de datos PostgreSQL.}
\item{{\sf proftp}: El servidor de FTP preferido por \distribucion .}
\item{{\sf smb}: Servicio de conexi�n a MS-Windows (Samba).}
\item{{\sf squid}: El muy conocido proxy-cache.}
\item{{\sf sound}: Activa y desactiva dispositivos de sonido.}
\item{{\sf sshd}: Acepta conexiones usando {\it Secure Shell}.}
\item{{\sf syslog}: Sistema para mantener bit�coras.}
\item{{\sf usb}: Activa y desactiva dispositivos usb.}
\item{{\sf xfs}: Servidor de fuentes para X-Window.}
\item{{\sf xinet}: Activa otros demonios y servicios como rsh, rlogin, etc.}
\end{itemize}

