\porcion{Zona Horaria}
\autor{\jeg}
\colaborador{\NC}
\revisor{\LLC}
\traductor{}

En este momento, el proceso de instalaci�n muestra 
una pantalla con el resumen de la configuraci�n hecha
hasta ahora y algunos valores tomados por defecto. 
Para configurar cualquiera de ellos basta con 
presionar con el rat�n el bot�n asociado.

El resumen presenta los siguientes valores: 

\begin{itemize}
\item{{\bf Rat�n}: Serie 2 botones gen�rico}
\item{{\bf Teclado}: Espa�ol }
\item{{\bf Zona Horaria}: Europa/Madrid}
\item{{\bf Impresora}: Sin Impresora}
\end{itemize}


El primero a cambiar es la zona horaria. En la caja de 
di�logo que se muestra est�
una lista jer�rquica en forma de �rbol de Continente/Ciudad 
que rige las diferentes zonas
horarias en el mundo. Si no se encuentra la ciudad, se escoge
una que tenga la misma hora que su ubicaci�n, por ejemplo
{\sf Am�rica/Bogot�} para toda Colombia
(figura~\ref{fig_zona_horaria}).

Esta selecci�n es muy importante para las correcciones horarias de
los pa�ses con estaciones. A la pregunta {\sf El reloj de su
computador usa GMT?} se contestar� negativamente.

\figura
{Escogencia de la zona horaria}
{fig_zona_horaria}
{width=11cm}{Mandrake/configura/install/pan_zona_horaria.png}

