%%% Parte de la Guia Completa a Linux ver 1.2.2
%%% de Jaime E. Gomez <jgomez@uniandes.edu.co>
%%% http://linuxcol.uniandes.edu.co/infolinux/docs/guia_linux/guia_linux.html
\porcion{Configuraci�n de hardware}\label{sec_conf_hardw}
\autor{\jeg}
\colaborador{\NC}
\revisor{\LLC}
\traductor{}

La inclusi�n de nuevo hardware o de hardware que no fue
apropiadamente instalado es una de las labores que eran
bastante temidas hace alg�n tiempo. \distribucion provee varias
herramientas que permiten la detecci�n y configuraci�n
de estos equipos.

\iconowindows

La herramienta m�s poderosa que posee \distribucion para la
configuraci�n de {\it hardware} es \verb+HardDrake+
el cual integra tres niveles de programas:


\figura
{{\tt HardDrake}: Herramienta de administraci�n de dispositivos}
{fig_harddrake}
{width=9cm}{Mandrake/configura/harddrake.png}


\begin{itemize}
\item{{\sf Librer�a Detect}: Es la m�quina de auto-detecci�n reuniendo
                        varias herramientas como isapnptools y SuperProbe}
\item{{\sf Harddrake}: La interfaz gr�fica que combina la detecci�n con 
                        la configuraci�n}
\item{{\sf Harddrake Wizard}: Es una herramienta de configuraci�n gen�rica
                        y es el reemplazo de \verb+Soundrake+ y \verb+Etherdrake+}
\end{itemize}


Para iniciar la herramienta, en una consola se escribe 
\verb+harddrake+ o desde el bot�n de configuraci�n de hardware 
en \verb+DrakConf+.
Esto puede tomar algo de tiempo mientras realiza su tarea.
Una vez se activa, muestra un administrador gr�fico 
de dispositivos (figura ~\ref{fig_harddrake}).

Se se�ala el dispositivo que se desea configurar y se 
ejecuta la herramienta particular para �l, por ejemplo
la herramienta de configuraci�n de sonido i.e. \verb+sound-wizard+
(figura ~\ref{fig_harddrake_sound}) que aparece cuando
se presiona [{\sf Corra herramienta de configuraci�n  }] 
({\it Run configuration Tool}).

\figura
{{\tt HardDrake}: Sound-wizard}
{fig_harddrake_sound}
{width=9cm}{Mandrake/configura/harddrake-soundwizard.png}

En esta ventana se pueden cambiar los datos de interrupciones (IRQ) o
dem�s, cuesti�n que no es muy buena idea ya que �stos han sido pre-seleccionados
como consecuencia de la detecci�n. Cuando se presiona [{\sf OK}] se 
ejecuta una prueba de tres sonidos: 8 bits, 16 bits y MIDI. Si se oyeron
correctamente, se presiona nuevamente [{\sf OK}] para confirmar
la configuraci�n. M�s adelante se presentan otras formas de configurar
el sonido.





