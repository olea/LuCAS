%%% Parte de la Guia Completa a Linux ver 1.2.2
%%% de Jaime E. Gomez <jgomez@uniandes.edu.co>
%%% http://linuxcol.uniandes.edu.co/infolinux/docs/guia_linux/guia_linux.html
\porcion{Configuraci�n de LILO}\label{sec_conf_lilo}
\autor{\jeg}
\colaborador{\NC}
\revisor{\LLC}
\traductor{}

Muchas veces el ordenador es compartido por varias personas
que no quieren saber nada de LILO ni de Linux y lo �nico que
quieren es que se inicie su sistema operativo sin problemas.

\iconoconsola 
Para ello se puede configurar LILO para que inicie un sistema
operativo por defecto. Esto se hace entrando en el sistema
como {\it root} y escribiendo la orden: \verb+lilo -D windows -d 50+

Con lo que se configura {\it windows} como sistema operativo por
defecto y que cargue �ste a los 5 segundos de no pulsar
ninguna tecla. Suponiendo claro, que se haya definido as� el
nombre de la partici�n\footnote{Algunas distribuciones le
asignan el nombre {\it DOS} a esta partici�n}.

Este proceso puede hacerse tambi�n editando el archivo de
configuraci�n de LILO \verb+/etc/lilo.conf+ que para este
momento debe lucir como:

{\footnotesize
\begin{vscreen}
boot=/dev/hda
map=/boot/map
install=/boot/boot.b
vga=0x0f06
default=linux
keytable=/boot/es.klt
lba32
prompt
timeout=30
message=/boot/message
#
image=/boot/vmlinuz
	label=linux
	root=/dev/hda3
	vga=788
#
image=/boot/vmlinuz
	label=linux-nonfb
	root=/dev/hda3
#
image=/boot/vmlinuz
	label=failsafe
	root=/dev/hda3
#
other=/dev/hda1
	label=windows
	table=/dev/hda
#
other=/dev/fd0
	label=floppy
	unsafe
#
\end{vscreen}
}

Donde se puede observar que LILO se instala en el MBR del disco
primario {\it boot=/dev/hda}; 
el disco se trata como lba32 (m�s de 1024 cilindros);
el teclado es Espa�ol (es.klt); 
va a preguntar por el sistema de arranque {\it prompt};
el tiempo de espera {\it timeout} est� en 3 segundos (en d�cimas de segundo) y 
el sistema por omisi�n ({\it default}) es el sistema {\it linux}.

Despu�s del comentario {\it\#} se
identifican varios p�rrafos correspondientes a los sistemas
instalados y modos de arranque, identificados por un {\it label}:
\begin{itemize}
\item {\sf linux}:Linux instalado en la partici�n {\it /dev/hda3} usando
	un modo de VGA de {\it Frame Buffer: 788} para inicio gr�fico.
\item {\sf windows}:MS-Windows instalado en la partici�n {\it /dev/hda1}
\item {\sf linux-nonfb}:Mismo linux sin inicio gr�fico.
\item {\sf failsafe}:Mismo linux pero para emergencias. 
\item {\sf floppy}:Arranque de un disquete de sistema.
\end{itemize}

Una vez salvado es necesario ejecutar \verb+# /sbin/lilo+ para
activar los cambios.
Sea muy cuidadoso con los cambios y valores asignados en este
archivo, ya que pueden terminar da�ando la entrada a su instalaci�n
y se haga necesario el uso de su disquete de rescate.

\iconowindows
El sitio natural para esta tarea es el centro de control o 
\verb+DrakConf+, en el cual se escoge
[{\sf Boot}] ${\rightarrow}$ [{\sf Boot Config}] lo que 
lanza la herramienta \verb+drakboot+ (figura~\ref{fig_drakboot}). 
Lo primero que se observa en la parte superior de la ventana 
es la configuraci�n de LILO/Grub Mode en la cual se presiona 
el bot�n de [{\sf Configure}] y se lanza una nueva ventana
que es la misma que se utiliz� en la instalaci�n y se
puede seguir paso a paso en la secci�n~\ref{sec_inst_lilo}.

\figura
   {{\tt Drakboot}}
   {fig_drakboot}
   {width=9cm}{Mandrake/drakboot.png}


\clearpage
\iconomundo 
Para configurar LILO tambi�n se puede usar \verb+Linuxconf+. 
Para este caso se usar� la interfaz de red (www): en un 
navegador se escribe la direcci�n del servidor al puerto 98 con
http://my\_servidor:98 (figura ~\ref{fig_linuxconf_bienvenida}) .

\figura
{Bienvenida a {\tt Linuxconf}}
{fig_linuxconf_bienvenida}
{width=10cm}{Linuxconf/linuxconf_bienvenida.png}
                         
Es conveniente resaltar que para entrar a \verb+linuxconf+ por 
red, es necesario usar el password de {\it root} y esto es
un posible agujero de seguridad. No lo haga a menos que est� 
seguro de no ser ``escuchado'' por la red. 

Una vez se ha entrado, se va a {\sf Modo de Arranque} y se tiene
la configuraci�n de LILO. Se divide en tres grupos 

\begin{itemize}
\renewcommand{\labelitemi}{$\star$}
\item{Configurar
	\begin{itemize}
	\item{Valores por defecto de LILO}
	\item{Configuraciones de LILO}
	\item{Configuraciones de otros OS bajo LILO}
	\end{itemize}
}
\item{Cambiar
	\begin{itemize}
	\item{Configuraciones de arranque por defecto}
	\end{itemize}
}
\item{Agregar
	\begin{itemize}
	\item{Un kernel nuevo}
	\item{Un kernel que Ud. compil�}
	\item{Modo de arranque por defecto}
	\end{itemize}
}
\end{itemize}

En cada una de ellas se pueden ver los valores actuales de 
LILO. Pero para la labor a realizar s�lo interesa la
{\sf configuraci�n de arranque por defecto} en la cual
se puede seleccionar el cambio a DOS/Windows 
(figura ~\ref{fig_linuxconf_arranque_DOS}). 

\figura
{{\tt Linuxconf}:Configuraci�n de OS de arranque}
{fig_linuxconf_arranque_DOS}
{width=10cm}{Linuxconf/linuxconf_arranque_dos.png}

Se presiona aceptar y se confirma la activaci�n de
la nueva configuraci�n. Al reiniciar el computador,
�ste debe arrancar DOS/Windows por defecto.


