%%% Parte de la Guia Completa a Linux ver 1.2.2
%%% de Jaime E. Gomez <jgomez@uniandes.edu.co>
%%% http://linuxcol.uniandes.edu.co/infolinux/docs/guia_linux/guia_linux.html
\porcion{Printerdrake}

\autor{\jeg}
\colaborador{\NC}
\revisor{\LLC}
\traductor{}

Aun cuando se ha configurado una impresora durante la
instalaci�n con \verb+DrakX+, no se est� exento de
afinar su configuraci�n,
a�adir una nueva o simplemente cambiar de impresora.

\iconoconsola
Como se menciona al inicio del presente cap�tulo,
el centro de control de \distribucion, \verb+DrakConf+ 
dispone de un conjunto de botones para diferentes 
tareas de administraci�n, entre ellas incluye un bot�n 
de configuraci�n de Impresoras presionando
[{\sf Sistema}]${\rightarrow}$[{\sf Printer}] el cual llama a la 
misma herramienta utilizada en el momento de la instalaci�n:
 \verb+printerdrake+. Por esto la Secci�n
~\ref{sec_impresora} puede ser seguida paso a paso para instalar
las impresoras. Se escoge su versi�n en texto para las
ilustraciones.

Al iniciar \verb+printerdrake+ muestra las colas instaladas
(figura ~\ref{fig_prdrk_colas}) y un bot�n de adici�n [{\sf Add}]. 
Si no se ha instalado una impresora previamente el sistema inicia
por defecto a�adiendo una cola. 

\figura
{{\tt Printerdrake}: Colas de impresi�n}
{fig_prdrk_colas}
{width=6cm}{Mandrake/configura/impresora/colas.png}

Se presenta ahora la posibilidad de escoger la conexi�n a la
impresora: remotas usando LPR (Unix) o SMB (MS-Windows),
o local por puerto paralelo. Se escoger� una impresora local.

\figura
{{\tt Printerdrake}: Tipos de conexi�n de impresoras}
{fig_prdrk_where}
{width=9cm}{Mandrake/configura/impresora/where.png}


\verb+Printerdrake+ tratar� de detectar el dispositivo donde se 
encuentra conectada la impresora, que por defecto es el primer 
puerto paralelo conocido como {\it /dev/lp0}.

Aparecer� entonces la ventana de descripci�n con toda
la informaci�n de la impresora: el Nombre: lp por 
defecto\footnote{ Herencia
del formato de LPD}, la descripci�n y la ubicaci�n. 

\figura
{{\tt Printerdrake}: Descripci�n de impresora}
{fig_prdrk_descr}
{width=9cm}{Mandrake/configura/impresora/descrip.png}


La instalaci�n presentar� una lista de impresoras
(figura ~\ref{fig_prdrk_modelo}). Se selecciona
la correspondiente o la que m�s se parezca al modelo.
Una vez seleccionada se pregunta si desea hacer una
prueba, la cual se recomienda.

\figura
{{\tt Printerdrake}: Modelos de impresoras}
{fig_prdrk_modelo}
{width=9cm}{Mandrake/configura/impresora/modelo.png}
                         

Una vez configurada la impresora se presentar� nuevamente
la pantalla de colas y puede presionar hecho [{\bf Done}].
configuraci�n pulsa en [{\sf OK}].



