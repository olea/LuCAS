%%% Parte de la Guia Completa a Linux ver 1.2.2
%%% de Jaime E. Gomez <jgomez@uniandes.edu.co>
%%% http://linuxcol.uniandes.edu.co/infolinux/docs/guia_linux/guia_linux.html
\porcion{Configuraci�n y Control de acceso}
\autor{\jeg}
\colaborador{\NC}
\revisor{\LLC}
\traductor{}

\distribucion  provee desde la versi�n 7.0 un centro de control
para la configuraci�n y administraci�n gr�fica del sistema:
\verb+DrakConf+ y ahora cambiando su nombre a 
\verb+Mandrake Control Center+. Esta consiste en un conjunto de botones
distribuidos en un �rbol jer�rquico que activan diferentes programas 
de prop�sito espec�fico. Estos lanzan peque�os programas hechos en 
Perl o Python, como la herramienta de detecci�n de hardware 
(\verb+HardDrake+), y m�dulos independientes de una versi�n
propia de {\verb+Linuxconf+ \footnote{Linuxconf es la
meta-herramienta de configuraci�n de Linux que se describir�
m�s adelante.}} (1.16) (figura ~\ref{fig_DrakConf}).

\figura
{Linux Mandrake {\tt DrakConf}}
{fig_DrakConf}
{width=10cm}{Mandrake/DrakConf.png}

Lo m�s interesante de \verb+DrakConf+ es que usa los mismos programas
que la instalaci�n \verb+DrakX+ para tareas de configuraci�n, e
inclusive el 90\% de ellos funcionan tambi�n en consola de texto aparte
de X-window. Estos programas son, en orden alfab�tico:

\begin{itemize}
\item{{\verb+diskdrake+}: Fdisk gr�fico}
\item{{\verb+drakboot+}: Modos de inicio}
\item{{\verb+drakfloppy+}: Creaci�n disquete de arranque}
\item{{\verb+drakfont+}: Administrador de fuentes}
\item{{\verb+drakgw+}: Compartir conexi�n a Internet}
\item{{\verb+draknet+}: Redes}
\item{{\verb+drakxservices+}: Servicios de arranque}
\item{{\verb+keyboardrake+}: Teclado}
\item{{\verb+harddrake+}: {\it Hardware}}
\item{{\verb+menudrake+}: Men�s del sistema}
\item{{\verb+mousedrake+}: Rat�n}
\item{{\verb+modemconf+}: M�dem}
\item{{\verb+packdrake+}: Creaci�n de paquetes rpm}
\item{{\verb+printerdrake+}: Impresoras}
\item{{\verb+rpmdrake+}: Paquetes rpm}
\item{{\verb+tinyfirewall+}: {\it Wizard} para cortafuegos}
\item{{\verb+userdrake+}: Usuarios}
\item{{\verb+XFdrake+}: X-window}
\end{itemize}


\distribucion a�n incluye algunas herramientas muy �tiles de la casa
matriz {\bf Red Hat} como \verb+sndconfig+, \verb+kudzu+, \verb+netconfig+
y otros.
