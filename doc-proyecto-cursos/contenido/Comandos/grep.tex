\porcion{El comando \comando{grep}}
\autor{\LDP}
\colaborador{\NC}
\revisor{\LLC}
\traductor{}

Escribir en salida est�ndar aquellas l�neas que concuerden con un
patr�n. Su sintaxis es como sigue:

\begin{vscreen}
grep [opciones] PATR�N [ARCHIVO...]
grep [opciones] [-e PATR�N | -f ARCHIVO] [ARCHIVO...]
\end{vscreen}

Este comando realiza una b�squeda en los \comando{ARCHIVOs} (o en la
entrada est�ndar, si no se especifica ninguno) para encontrar l�neas
que concuerden con \comando{PATR�N}. Por defecto \comando{grep}
imprime en pantalla dichas l�neas. Sus opciones m�s interesantes son:

\begin{description}
\item[-c] Modifica la salida normal del programa, en lugar de imprimir
  por salida est�ndar las l�neas coincidentes, imprime la cantidad de
  l�neas que coincidieron en cada archivo.
\item[-e PATR�N] Usar \comando{PATR�N} como el patr�n de b�squeda, muy
  �til para proteger aquellos patrones de b�squeda que comienzan con
  el signo <<->>.
\item[-f ARCHIVO] Obtiene los patrones del archivo \comando{ARCHIVO}.
\item[-H] Imprimir el nombre del archivo con cada coincidencia.
\item[-r] Buscar recursivamente dentro de todos los subdirectorios del
  directorio actual.
\end{description}

El patr�n de b�squeda normalmente es una palabra o una parte de una
palabra. Tambi�n se pueden utilizar \emph{expresiones regulares}, para
realizar b�squedas m�s flexibles.


\begin{ejemplo}
Si se quisiera buscar la ocurrencia de todas las palabras que
comiencen con <<a>> min�scula, la ejecuci�n del comando ser�a algo
as�:

\begin{vscreen}
$ grep 'a*' archivo
\end{vscreen}
%$ <- Esto es para que el XEmacs no me joda la colorificaci�n por culpa
% del s�mbolo $ de mas arriba. (Lucas - 7/4/2001)

Tambi�n se pueden aprovechar las tuber�as para realizar filtros, el ejemplo
anterior es equivalente a:

\begin{vscreen}
$ cat archivo | grep 'a*'
\end{vscreen}
%$

\end{ejemplo}
