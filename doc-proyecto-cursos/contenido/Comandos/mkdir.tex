\porcion{El comando \comando{mkdir}}
\autor{\LDP}
\colaborador{\NC}
\revisor{\LLC}
\traductor{}

Este comando es bastante simple; su finalidad es la creaci�n de
directorios, y su sintaxis es as�:

\begin{vscreen}
mkdir [opciones] directorio...
\end{vscreen}

Sus opciones son las que siguen:

\begin{description}
\item[-m modo] Establece los permisos de los directorios creados.
\item[-p] Crea los directorios padre que falten para cada argumento
  \comando{directorio}.
\end{description}

\begin{ejemplo}

Para crear un directorio en el directorio actual:

\begin{vscreen}
$ mkdir directorio
\end{vscreen}
%$

Tambi�n se pueden crear directorios en otros subdirectorios
existentes:

\begin{vscreen}
$ mkdir /dir_existente/directorio
\end{vscreen}
%$


\end{ejemplo}
