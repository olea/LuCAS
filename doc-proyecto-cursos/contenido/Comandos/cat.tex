\porcion{El comando \comando{cat}}
\autor{\LDP}
\colaborador{}
\revisor{\LLC}
\traductor{}

Se utiliza para concatenar archivos y mostrarlos por la salida
est\'andar (normalmente la pantalla). Su sintaxis es muy simple:

\begin{vscreen}
cat [opci\'on] [archivo]...
\end{vscreen}

Donde \comando{archivo} puede ser uno o m\'as archivos. Si no se
especifica este segundo par\'ametro, \comando{cat} tomar\'a la entrada
de la entrada est\'andar (normalmente el teclado).

Sus opciones m\'as comunes son:

\begin{description}
\item[-n] Numera todas las l\'ineas de salida.
\item[-b] Numera aquellas l\'ineas de salida que no est\'en en blanco.
\end{description}

Este comando es muy utilizado en conjunci\'on con otros comandos, por
lo tanto se deja el ejemplo para la secci\'on de tuber\'ias.


