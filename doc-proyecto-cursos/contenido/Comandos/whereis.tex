\porcion{El comando \comando{whereis}}
\autor{\LDP}
\colaborador{}
\revisor{\LLC}
\traductor{}

Este comando se utiliza para localizar el archivo binario, el c�digo
fuente y la p�gina de manual de un determinado comando. Su sintaxis es
como sigue:

\begin{vscreen}
whereis [opciones] archivo...
\end{vscreen}

La lista de opciones m�s utilizadas es:

\begin{description}
\item[-b] Buscar solamente el archivo binario.
\item[-m] Buscar solamente la p�gina manual.
\item[-s] Buscar solamente el c�digo fuente.
\end{description}

Como ejemplos, se ve lo siguiente:

\begin{ejemplo}

\begin{vscreen}
usuario@maquina:~/$ whereis -m whereis
whereis: /usr/share/man/man1/whereis.1.gz

usuario@maquina:~/$ whereis passwd
passwd: /usr/bin/passwd /etc/passwd /usr/man/passwd.5.gz
\end{vscreen}

En el primer case se ha pedido la p�gina de manual del mismo
comando \comando{whereis}\footnote{El lector ya est� advertido que al
  autor le gustan las \emph{recursividades}, por favor perd�nenlo.},
mientras que en el segundo se han pedido todos los archivos que tengan
que ver con el comando \comando{passwd}.

\end{ejemplo}

