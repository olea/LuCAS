\porcion{El comando \comando{df}}
\autor{\LDP}
\colaborador{}
\revisor{\LLC}
\traductor{}

Provee informaci�n sobre la utilizaci�n del espacio en disco en los
diferentes sistemas de archivos montados en el sistema. Para un
sistema GNU/Linux, quedarse sin espacio libre es algo bastante grave,
ya que muchos \emph{demonios} y programas en general utilizan el
directorio \archivo{/tmp} para guardar informaci�n mientras se
ejecutan. La sintaxis de \comando{df} es la siguiente:

\begin{vscreen}
df [opciones] [sistema-de-archivo...]
\end{vscreen}

Si no se provee del argumento \comando{sistema-de-archivo},
\comando{df} informar� acerca de todos los sistemas de archivos
montados y en funcionamiento. Las opciones de \comando{df} m�s
relevantes son:

\begin{description}
\item[-h] Imprimir los tama�os de forma m�s legible para humanos.
\item[-i] Informar sobre la utilizaci�n de los
  nodos-�\footnote{Abreviaci�n de \emph{nodos �ndice}, en ingl�s
    i-nodes}. Los nodos-� son estructuras internas del sistema de
  archivos. Cuando �ste se queda sin nodos-� libres, por m�s que haya
  espacio libre en disco, no se podr�n crear nuevos archivos hasta que
  se liberen nodos-�, generalmente esto no pasa a menos que se haya generado
  una enorme cantidad de archivos muy peque�os.
\item[-k] Mostrar los tama�os en bloques de 1024 bytes.
\item[-m] Mostrar los tama�os en bloques de mega-bytes.
\end{description}

Un ejemplo de ejecuci�n del \comando{df} es:

\begin{vscreen}
usuario@maquina:~/$ df
Filesystem           1k-blocks      Used Available Use% Mount 
/dev/hda2              2949060   2102856    696400  75% /
/dev/hda1                23302      2593     19506  12% /boot
/dev/hda4             10144728   5506796   4637932  54% /home
/dev/hdb2              3678764   3175268    503496  86% /u
\end{vscreen}
%$
