\porcion{El comando \comando{wc}}
\autor{\LDP}
\colaborador{}
\revisor{\LLC}
\traductor{}

El nombre del comando \comando{wc} proviene de \emph{word count}, y
como es de suponer, sirve para contar palabras. Pero no s�lo palabras
como se ver\'a a continuaci�n. Su sintaxis es como sigue:

\begin{vscreen}
wc [opci�n...] [archivo...]
\end{vscreen}

Si se omite el argumento \comando{archivo}, \comando{wc} tomar� los
datos (naturalmente) de la entrada est�ndar.

La lista de opciones m�s importantes es la siguiente:

\begin{description}
\item[-c] Contar bytes.
\item[-l] Contar l�neas.
\item[-w] Contar palabras.
\end{description}

\begin{ejemplo}
Como ejemplo, se pueden contar las l�neas del archivo
\comando{/etc/passwd} y de esta manera se sabr� r�pidamente cu�ntos
usuarios tiene definidos el sistema:

\begin{vscreen}
usuario@maquina:~/$ wc -l /etc/passwd
     32 /etc/passwd
\end{vscreen}
%$
\end{ejemplo}

Se pueden combinar varios argumentos a la vez.
