\porcion{El comando \comando{date}}
\autor{\LDP}
\colaborador{}
\revisor{\LLC}
\traductor{}

Este comando tiene dos funciones: una es la de mostrar en pantalla la
fecha del sistema (en varios formatos, como veremos a continuaci�n),
la otra es la funci�n de configurar la hora del sistema, pero para que
esta funcionalidad se cumpla, se debe ejecutar el comando desde una
sesi�n de \emph{root}. La sintaxis de este comando es:

\begin{vscreen}
date [opci�n...] [+FORMAT]
date [opci�n] [MMDDhhmm[[CC]AA][.ss]]
\end{vscreen}

\comando{FORMAT} controla el formato con que se mostrar� la fecha,
alguna de las opciones de este argumento son:

\begin{description}
\item[\%a] D�a de la semana abreviado.
\item[\%A] D�a de la semana completo.
\item[\%b] Nombre del mes abreviado.
\item[\%B] Nombre del mes completo.
\item[\%d] D�a del mes.
\item[\%m] N�mero de mes.
\item[\%H] Hora, en formato 24h.
\item[\%M] Minuto.
\item[\%S] Segundos.
\end{description}

Existen much�simas m�s opciones de formato que alentamos al lector a
verlas en la p�gina de manual del comando \comando{date}.

\begin{ejemplo}

\begin{vscreen}
usuario@maquina:~/$ date
Sun Apr  8 15:09:32 ART 2001
usuario@maquina:~/$ date +"%A %d %B"
Sunday 08 April
\end{vscreen}

\end{ejemplo}
