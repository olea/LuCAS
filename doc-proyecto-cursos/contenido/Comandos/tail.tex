\porcion{El comando \comando{tail}}
\autor{\LDP}
\colaborador{\NC}
\revisor{\LLC}
\traductor{}

El comando \comando{tail} es al \comando{head} como el \comando{less} es al
\comando{more}\footnote{Perdonen al autor, aunque a veces es
  interesante mezclar conceptos matem�ticos con inform�ticos, �ste no
  es el caso.}. El comando \comando{tail} escribe a la salida est�ndar
la �ltima parte de un archivo. Su sintaxis es:

\begin{vscreen}
tail [opci�n...] [archivo...]
\end{vscreen}

Al igual que \comando{head}, si no se le proporciona un argumento
\comando{archivo}, este comando tomar� su entrada desde la entrada
est�ndar. Alguna de sus opciones son las siguientes:

\begin{description}
\item[-c N] Escribe los �ltimos \comando{N} bytes.
\item[-n N] Escribe las �ltimas \comando{N} l�neas.
\item[-f] Escribir la �ltima parte del archivo a medida que va
  creciendo. Esta opci�n es muy �til para monitorear archivos de
  registro que van creciendo con el tiempo.
\end{description}

\begin{ejemplo}

Un uso muy com�n de \comando{tail} es utilizarlo para inspeccionar logs
(o bit�coras) del sistema.

\begin{vscreen}
$ tail -n 10 /var/log/messages
\end{vscreen}
%$
mostrar� las ultimas 10 l�neas del log \archivo{messages}.

En el caso que se quiera tener un seguimiento de un log en especial
se puede utilizar la opci�n \comando{-f}

\begin{vscreen}
$ tail -n 10 -f /var/log/messages
\end{vscreen}
%$
mostrar� las ultimas 10 l�neas del log \archivo{messages}. Y luego
quedar� a la espera de nuevas l�neas por aparecer en el final del
archivo.

\end{ejemplo}
