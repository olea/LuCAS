\porcion{El comando \comando{rm}}
\autor{\LDP}
\colaborador{\NC}
\revisor{\LLC}
\traductor{}

He aqu� un comando peligroso, \comando{rm} se utiliza para borrar
archivos o directorios, su sintaxis es:

\begin{vscreen}
rm [opciones] archivo...
\end{vscreen}

Se debe \emph{siempre} pensar dos veces lo que se est� haciendo antes
de ejecutar este comando. Quiz�s esto parezca una advertencia para
tontos, pero m�s a�n cuando se est� administrando un equipo que da
servicios a varios usuarios, un <<teclazo>> en falso, y f�cilmente se
pierden datos importantes. Sus opciones m�s utilizadas son:

\begin{description}
\item[-f] No imprimir mensajes de error, ni pedir al usuario
 una confirmaci�n por cada archivo que se vaya a borrar.
\item[-r] Borrar los contenidos de directorios recursivamente.
\item[-v] Muestra el nombre de cada archivo eliminado.
\end{description}

el argumento \comando{archivo} puede ser tanto un nombre de archivo,
como una expresi�n regular.

\begin{ejemplo}

Para borrar un archivo en el directorio actual llamado
\archivo{arch1.txt}.

\begin{vscreen}
$ rm arch1.txt 
\end{vscreen}
%$
pedir� confirmaci�n. Para no tener que confirmar se utiliza la opcion
\comando{-f}

\begin{vscreen}
$ rm -f arch1.txt
\end{vscreen}
%$

El comando \comando{rm} no permite borrar directorios directamente. Pero existen
opciones que pueden ayudar para eliminarlo sin confirmaci�n. Cuidado con el uso
de estas opciones.

\begin{vscreen}
$ rm -rf directorio
\end{vscreen}
%$

Eliminar� completamente el directorio sin importar su contenido.

\end{ejemplo}
