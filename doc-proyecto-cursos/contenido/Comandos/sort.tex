\porcion{El comando \comando{sort}}
\autor{\LDP}
\colaborador{}
\revisor{\LLC}
\traductor{}

Este comando se utiliza para ordenar l�neas de texto a partir de
varios criterios, su sintaxis es similar a la de todos los comandos:

\begin{vscreen}
sort [opci�n...] [archivo...]
\end{vscreen}

Si no se le provee al menos un argumento \comando{archivo}, este
comando tomar� su entrada de la entrada est�ndar, ya veremos esto en
la secci�n \ref{sec:redireccion}.

El criterio de orden que utiliza \comando{sort} por defecto es
alfab�tico, esto se debe tener en cuenta siempre que se necesite
ordenar listas de n�meros, si no se le especifica a \comando{sort} que
debe ordenar num�ricamente, tomar� a los n�meros como una lista de
caracteres y el ordenamiento no ser\'a el esperado.

A continuaci\'on se listan las opciones m\'as com\'unmente usadas:

\begin{description}
\item[-n] Utilizar ordenamiento num\'erico.
\item[-r] Ordenar en forma inversa.
\item[-f] Tratar igualmente a las may\'usculas y min\'usculas.
\item[-d] Utilizar el m\'etodo de ordenamiento de diccionario (s\'olo
  toma en cuenta espacios en blanco y caracteres alfanum\'ericos).
\end{description}

\begin{ejemplo}

Dami\'an necesita limpiar su directorio personal de archivos
innecesarios y para ello obtuvo una lista similar a esta:

\begin{vscreen}
384746	MP3
2613	Mail
82716	Internet
8534	Fotos
132	zaxxon
5921	proyectos
\end{vscreen}

La cual tiene almacenada en un archivo
\archivo{/home/damian/lista.txt}. Para poder ver f\'acilmente cu\'al
directorio ocupa m\'as espacio y as\'i revisarlo y limpiarlo, Dami\'an
ejecutar\'a el siguiente comando:

\begin{vscreen}
$ sort -nr lista.txt
\end{vscreen}

Obteniendo de esta manera la lista ordenada num\'ericamente y en forma
descendiente:

\begin{vscreen}
384746  MP3
82716   Internet
8534    Fotos
5921    proyectos
2613    Mail
132     zaxxon
\end{vscreen}

Notar que si no se utiliza la opci\'on \comando{-n}, la lista se
ordenar\'a en forma incorrecta:

\begin{vscreen}
8534    Fotos
82716   Internet
5921    proyectos
384746  MP3
2613    Mail
132     zaxxon
\end{vscreen}

\end{ejemplo}


