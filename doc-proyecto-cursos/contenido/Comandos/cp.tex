\porcion{El comando \comando{cp}}
\autor{\LDP}
\colaborador{}
\revisor{\LLC}
\traductor{}

Se utiliza para copiar archivos, su sintaxis es la siguiente:

\begin{vscreen}
cp [opciones] archivo-origen camino-destino
cp [opciones] archivos-origen... directorio-destino
\end{vscreen}

Entre las opciones m�s relevantes, se tienen:

\begin{description}
\item[-f] Borrar los archivos de destino ya existentes.
\item[-d] Copia los enlaces simb\'olicos tal cual son, en lugar de 
  copiar los archivos a los que apuntan.
\item[-p] Preservar los permisos, el usuario y el grupo del archivo a
  copiar.
\item[-R] Copia directorios recursivamente.
\item[-a] Equivalente a utilizar las opciones \comando{-dpR}.
\item[-u] No copia un archivo si en el destino ya
  existe tal archivo, y \'este tiene igual tiempo de modificaci�n o m�s
  reciente.
\item[-v] Da informaci�n en pantalla sobre los archivos que se van
  copiando.
\end{description}

\begin{ejemplo}

Sup\'ongase que el sistema donde el usuario \emph{juancito} trabaja
normalmente tiene un directorio \comando{/usr/local/respaldos},
especialmente destinado para almacenar copias de respaldo de los datos
de sus usuarios, y \emph{juancito} necesita hacer su copia de respaldo
semanal\footnote{\emph{juancito} es un usuario sabio que hace copias
de respaldo, `?usted las hace?}, entonces el comando que ejecutar\'a
es el siguiente:

\begin{vscreen}
$ cp -dpR /home/juancito /usr/local/respaldos
\end{vscreen}

Que es equivalente a ejecutar:

\begin{vscreen}
$ cp -a /home/juancito /usr/local/respaldos
\end{vscreen}

Con esto, \emph{juancito} copia todos sus archivos con los permisos y 
atributos exactamente igual a como est\'an en los archivos originales, 
y s\'olo debe especificar su directorio personal, ya que la opci\'on 
\comando{-R} se encarga de incluir todos los archivos que se encuentran
dentro del mismo.

\end{ejemplo}

\begin{ejemplo}

Pedro se encuentra trabajando en el laboratorio de inform\'atica de su
escuela donde en el servidor de archivos existe un directorio
\comando{/usr/local/tp} que contiene los directorios de todos los
grupos de trabajo de la escuela. Pedro pertenece al grupo 15 y
necesita actualizar su grupo de copias del trabajo pr\'actico 2,
entonces usar\'a el comando \comando{cp} de la siguiente manera:

\begin{vscreen}
$ cp -uv /usr/local/tp/grupo15/tp2/* /home/pedro/TPs/2
\end{vscreen}

Con la opci\'on \comando{-v}, Pedro puede ver cuales archivos se han
actualizado, y con la opci\'on \comando{-u} s\'olo copia aquellos
archivos mas recientes que los que \'el ya posee.

\end{ejemplo}
