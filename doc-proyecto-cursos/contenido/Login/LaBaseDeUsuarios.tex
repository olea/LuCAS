\porcion{La base de datos de los usuarios}
\autor{\LDP}
\colaborador{Pedro Pablo Fabrega}
\revisor{\LLC}
\traductor{}

Se ha visto que para iniciar una sesi�n de trabajo en un sistema GNU/Linux
se debe suministrar al sistema una pareja de nombre de
usuario/clave. Estos datos se almacenan en un archivo llamado
\archivo{/etc/passwd}. Este archivo contiene una l�nea por cada
usuario del sistema. Cada l�nea consta de una serie de campos
separados por dos puntos (:). Estos campos son, en el orden que
aparecen:

\begin{enumerate}
\item \textbf{Nombre de usuario}. Es el nombre con el que nos
  presentamos al sistema, con el que tenemos que responder a
  \comando{Login:} y por el que nos identifica el sistema.
\item \textbf{Clave cifrada}. El siguiente campo es la clave de acceso
  al sistema. Esta clave no se guarda como se introduce, sino que se
  almacena transformada mediante el algoritmo \textbf{DES} para que
  nadie pueda averiguarla.
\item \textbf{UID}\label{UID}. Identificador de usuario. Es el n�mero
  de usuario que tiene cada cuenta abierta en el sistema. El sistema
  trabaja de forma interna con el UID, mientras que nosotros
  trabajamos con el nombre de usuario. Ambos son equivalentes.
\item \textbf{GID}\label{GID}. Identificador de grupo. Es el n�mero de
  grupo principal al que pertenece el usuario.
\item \textbf{Nombre completo de usuario}. Este es un campo meramente
  informativo, en el que se suele poner el nombre completo del
  usuario.
\item \textbf{Directorio personal}. Este campo indica el directorio
  personal de un usuario, en el cual el usuario puede guardar su
  informaci�n.
\item \textbf{Int�rprete de comandos}. El �ltimo campo indica un
  programa que se ejecutar� cuando el usuario inicie una sesi�n de
  trabajo. Normalmente este campo es un int�rprete de comandos
  (<<shell>> en ingl�s) que proporciona una l�nea de �rdenes para que
  el usuario trabaje. Ejemplo:
\end{enumerate}

\begin{vscreen}
usuario:xKxd6YkHSs:505:705:Usuario:/home/usuario:/bin/bash  
   ^          ^    ^   ^     ^           ^          ^  
   |          |    |   |     |           |          | 
   |          |    |   |     |           |  I. de comandos 
   |          |    |   |     |      directorio personal 
   |          |    |   |  Nombre completo del usuario
   |          |    |  N�mero de grupo (GID)    
   |          |   N�mero de usuario (UID)  
   |       Clave cifrada 
 Nombre de usuario
\end{vscreen}


