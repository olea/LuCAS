\porcion{Iniciando una sesi�n}
\autor{\LDP}
\colaborador{Pedro Pablo Fabrega}
\revisor{\LLC}
\traductor{}

Existen diferentes m�todos para poder conectar los terminales al
sistema:

\begin{itemize}
\item En primer lugar podemos conectarnos a un GNU/Linux a trav�s del
   \emph{puerto serie (RS232)}, con una terminal no inteligente o
  bien con otro equipo y un emulador de terminales. En ambos casos
  existe un programa que atiende las solicitudes de conexi�n a trav�s
  del puerto serie. Cuando hay una solicitud de conexi�n, este
  programa la atiende solicitando al usuario que se identifique ante
  el sistema. Cuando termina la conexi�n, este programa se reactiva
  para seguir atendiendo nuevas solicitudes.
\item Mediante \emph{tarjeta de red}. En este caso, tenemos un
  programa que escucha las solicitudes de conexi�n a trav�s de la
  tarjeta de red. Cuando llega una solicitud, este programa se desdobla
  de forma que una parte atiende la conexi�n y otra contin�a
  atendiendo nuevas conexiones. As�, podemos tener m�s de una conexi�n
  a trav�s de la tarjeta de red. Algunos servicios que proveen esta
  funcionalidad son el \comando{telnet} (sin encriptaci�n de datos) y
  el \comando{ssh} (Secure Shell, con encriptaci�n de datos). Esto se
  ver� m�s adelante.
\item La \emph{consola}. Evidentemente, en un sistema GNU/Linux
  tambi�n podemos trabajar desde el teclado y monitor que est�n
  conectados directamente al sistema. Normalmente en la mayor�a de las
  distribuciones, en la consola hay hasta 6 terminales virtuales,
  accediendo a cada una de ellas con \boton{Alt-F1} a \boton{Alt-F6}.
\end{itemize}

Una vez que se ha conseguido conectar a un sistema GNU/Linux tenemos
que iniciar una sesi�n de trabajo. GNU/Linux es un sistema
multiusuario, y esto exige que el usuario se presente al sistema y que
�ste lo acepte como usuario reconocido. As�, cada vez que iniciamos
una sesi�n GNU/Linux nos responde con

\begin{vscreen}
Login:  
\end{vscreen}

a lo que se debe responder con el nombre de usuario. Acto seguido,
GNU/Linux solicita una clave para poder comprobar que el usuario es
quien dice ser:

\begin{vscreen}
Password:
\end{vscreen}

En este caso se teclea la clave de acceso. Por motivos de seguridad
esta clave no aparecer� en la pantalla. Si la pareja nombre de
usuario/clave es correcta, el sistema inicia un int�rprete de comandos
con el que se puede trabajar.

Habitualmente ser� el s�mbolo \comando{\$}, aunque puede ser tambi�n
el s�mbolo \comando{\%} (si usamos un shell C). Cuando es el
administrador (root) quien est� trabajando en el sistema, el indicador
que aparece es \comando{\#}.

