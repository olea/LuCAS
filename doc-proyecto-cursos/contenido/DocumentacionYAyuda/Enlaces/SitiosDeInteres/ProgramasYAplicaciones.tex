\porcion{Programas y aplicaciones}
\autor{\LDP}
\colaborador{\SGG}
\revisor{Leonardo Mart�nez}
\traductor{}

%+++++++++++++++++++++++  Programas y aplicaciones ++++++++++++++++++++++

Hay multitud de programas y aplicaciones para GNU/Linux, la mejor forma de
encontrarlos es visitando las siguientes p�ginas:

\begin{description}

   \item[DistroWatch.com]
   	\sitio{http://www.distrowatch.com/}, Comparativas de las distribuciones
	                                     GNU/Linux m�s importantes. Un buen
					     punto de partida para conocer las
					     distintas distribuciones existentes

   \item[Freshmeat]
	\sitio{http://freshmeat.net}, Este sitio est� dedicado a la recolecci�n de
	                              Software Libre de todo tipo, dividido por
				      categor�as

   \item[ibiblio]
   	\sitio{http://metalab.unc.edu}, Repositorio de software y distribuciones
	                                de GNU/Linux

   \item[Kernel.org]
   	\sitio{http://www.kernel.org}, Sitio principal para obtener el c�digo
	                               fuente de Linux

   \item[LinuxISO]
   	\sitio{http://www.linuxiso.org}, P�gina donde podr�s encontrar las imagenes
	                                 ISO de algunas distribuciones de GNU/Linux,
					 listas para ser grabadas

   \item[RmpFind]
   	\sitio{http://rpmfind.net}, Buscador de aplicaciones, la caracter�stica
	                            de este es que todos los programas est�n
				    empaquetados en RPM

   \item[SourceForge]
	\sitio{http://sourceforge.net/}, Servicios gratuitos para los desarrolladores
	                                 de Software Libre, en el que se podr�n
					 encontrar aplicaciones de c�digo abierto
					 de todo tipo

\end{description}
