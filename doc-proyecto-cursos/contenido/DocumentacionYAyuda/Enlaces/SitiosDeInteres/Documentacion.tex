\porcion{Documentaci�n}
\autor{\LDP}
\colaborador{\SGG}
\revisor{Leonardo Mart�nez}
\traductor{}

%+++++++++++++++++++++++++ Documentaci�n ++++++++++++++++++++++++++++++++++++

La siguiente lista presenta las p�ginas m�s importantes de documentaci�n
disponibles en Internet:

\begin{description}

   \item[INSFLUG]
	\sitio{http://www.insflug.org/}, p�gina donde se coordina la traducci�n
	                                 ``oficial'' de documentos breves, como
					 los COMOs y PUFs o Preguntas de Uso
					 Frecuente\footnote{FAQs en Ingl�s}

   \item[NuLies]
   	\sitio{http://nulies.hispalinux.es/}, proyecto dedicado a la traducci�n al
	                                      castellano de aquellas partes del n�cleo
					      Linux cuya traducci�n sea �til para la
					      inmensa mayoria de hispanoparlantes

   \item[PAMELi]
   	\sitio{http://ditec.um.es/~piernas/manpages-es/}, los esfuerzos de la
	                                      traducci�n al castellano de las
					      p�ginas del manual se encuentran en
					      esta localizaci�n

   \item[Pedro Reina]
 	\sitio{http://pedroreina.org}, P�gina personal de Pedro Reina, profesor de
	                               Ense�anza Secundaria. En esta p�gina est�
				       disponible el curso de inform�tica realizado
				       por Pedro, basado en herramientas de
				       libre distribuci�n

   \item[Proyecto LuCAS]
   	\sitio{http://lucas.hispalinux.es/}, como se puede leer en la p�gina del
	                                     Proyecto, \emph{LuCAS} es la mayor
					     biblioteca en espa�ol dedicada a
					     GNU/LiNUX de todo el planeta''

   \item[LDP]
   	\sitio{www.linuxdoc.org}, en \emph{El Proyecto de Documentaci�n de Linux}\footnote{en
	                          Ingl�s: The Linux Documentation Project, de ah�
				  el acr�nimo LDP}, podremos encontrar documentaci�n
				  en Ingl�s libre y de calidad para el sistema
				  operativo GNU/Linux

\end{description}