% Porci�n basada en la "Gu�a de instalaci�n oficial de Red Hat Linux"
% traducida al castellano. Esta gu�a est� disponible en:
% http://lucas.hispalinux.es/htmls/manuales.html

\porcion{P�ginas \comando{Info}}
\autor{Gu�a de instalaci�n oficial de Red Hat Linux}
\colaborador{\SGG}
\revisor{\LLC}
\traductor{}

Mientras \comando{man} es el formato de documentaci�n m�s extendido,
\comando{info} es mucho m�s potente. Proporciona enlaces de hipertexto
para hacer mucho m�s sencilla la lectura de documentos grandes y muchas
ayudas para quien escribe la documentaci�n.

Para leer la documentaci�n de \comando{info}, use el programa \comando{info}
sin ning�n argumento. Le presentar� una lista de documentaci�n disponible.
Si no puede encontrar algo, es probable que sea debido a que no tiene
instalado el paquete que incluye dicha documentaci�n.

El sistema \comando{info} es un sistema basado en hipertexto. El texto
que aparezca resaltado es un enlace que apunta a m�s informaci�n. Las teclas
m�s importantes cuando maneja \comando{info} son:

\begin{description}
   \item [\boton{Tab}]
      para mover el cursor a un enlace
   \item[\boton{Intro}]
      para seguir un enlace
   \item[\boton{p}]
      le devolver� a la p�gina previa
   \item[\boton{n}]
      le enviar� a la p�gina siguiente
   \item[\boton{u}]
      subir� un nivel de la documentaci�n
   \item[\boton{q}]
      para abandonar info
\end{description}

La mejor manera de usar \comando{info} es leer la documentaci�n
\comando{info} acerca de �l. Si lee la primera p�gina que \comando{info}
le presente, ser� capaz de comenzar a usarlo.
