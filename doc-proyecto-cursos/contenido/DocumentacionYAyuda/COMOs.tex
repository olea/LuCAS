\porcion{COMOs}
\autor{\LDP}
\colaborador{\SGG}
\revisor{Leonardo Mart�nez}
\traductor{}

Los \emph{COMOs}\footnote{En Ingl�s se les llama HOWTOs} son documentos en
forma de ``recetas'' sobre c�mo realizar diferentes tareas
espec�ficas en el sistema. Estos documentos se encuentran en el
directorio principal de documentaci�n de GNU/Linux:
\comando{/usr/share/doc/}, en el subdirectorio \comando{HOWTO}\footnote{
Las traducciones al espa�ol de muchos de estos \emph{COMOs}, se encuentran en
el subdirectorio \comando{HOWTO/es/}, en caso de estar instalados}.
La documentaci�n est� disponible en texto plano y formato HTML, gracias
a esto podemos emplear cualquier paginador (\comando{less} por ejemplo)
o navegador web para visualizarla.

% (Comentado hasta saber como se tratar�n las referencias cruzadas)
% Si los \emph{COMOs} no se encuentran en el sistema, se pueden
% instalar de la forma indicada en el cap�tulo \ref{cha:rpm}.

% Secci�n basada en la "Gu�a de instalaci�n oficial de Red Hat Linux"
% traducida al castellano. Esta gu�a est� disponible en:
% http://lucas.hispalinux.es/htmls/manuales.html

Tambi�n puede encontrar ficheros cuyo nombre termina en .gz. Dichos
ficheros han sido comprimidos con \comando{gzip} para ahorrar espacio,
luego debe descomprimirlos antes de leerlos. Una forma de leer los
\emph{COMOs}, sin llenar su disco con versiones descomprimidas de
los mismos, es utilizar \comando{zless}:

La orden \comando{zless} utiliza las mismas teclas que \comando{less},
de modo que pueda moverse con facilidad a trav�s del documento.
