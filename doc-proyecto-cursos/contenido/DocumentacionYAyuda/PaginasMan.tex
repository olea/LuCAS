\porcion{P�ginas del Manual}
\autor{\LDP}
\colaborador{\SGG}
\revisor{Leonardo Mart�nez}
\traductor{}

Se les llama \emph{P�ginas del Manual} a la documentaci�n en
l�nea del sistema. Existe una p�gina del manual por casi cada comando
de GNU/Linux. La forma de consultar esta documentaci�n (en un
terminal de texto) es mediante la orden \comando{man} seguida del
comando sobre el cual se quiere obtener la informaci�n. Por ejemplo, si
necesitamos ver las opciones del comando \comando{mount} (para montar
sistemas de archivos), se deber�a ejecutar \comando{man mount}.

% Secci�n basada en la "Gu�a de instalaci�n oficial de Red Hat Linux"
% traducida al castellano. Esta gu�a est� disponible en:
% http://lucas.hispalinux.es/htmls/manuales.html

La p�gina del manual se muestra mediante el programa \comando{less},
(el cual facilita la lectura del documento pantalla a pantalla), por lo
que todas las opciones de \comando{less} funcionar�n cuando lea una p�gina
del manual. Las teclas m�s importantes de \comando{less} son:

\begin{description}
   \item [\boton{q}]
      para salir
   \item[\boton{Intro}]
      para avanzar l�nea a l�nea
   \item[\boton{Espacio}]
      para avanzar p�gina a p�gina
   \item[\boton{b}]
      para retroceder una p�gina
   \item[\boton{/}]
      seguido de un texto y de la tecla \comando{Intro} para buscar dicho
      texto
   \item[\boton{n}]
      para encontrar la siguiente coincidencia de la b�squeda actual
\end{description}
