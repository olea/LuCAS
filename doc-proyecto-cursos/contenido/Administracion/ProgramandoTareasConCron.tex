\porcion{Programaci�n de tareas con \comando{cron}}
\autor{\LDP}
\colaborador{}
\revisor{\LLC}
\traductor{}

En muchas ocasiones se hace necesario alg\'un modo de automatizar
tareas administrativas, con el fin de aliviar de trabajo al operador
del sistema. Estas tareas var\'ian en importancia y fecuencia,
teniendo como ejemplos las siguientes:

\begin{itemize}
\item Chequeo del estado de las interfaces de red.
\item Env\'io de boletines informativos a los usuarios.
\item Limpieza de archivos temporales.
\item Auditor\'ias autom\'aticas de seguridad.
\item Realizaci\'on de copias de seguridad.
\end{itemize}

Existe un servicio en GNU/Linux destinado a atender estas
necesidades, llamado \emph{crond}, que se incluye en pr\'acticamente
todas las distribuciones. 

Cada usuario tiene su propio perfil de tareas autom\'aticas, adem\'as
de la lista de tareas del sistema. El comando para manejar este
servicio es el \comando{crontab} y su sintaxis es la siguiente:

\begin{vscreen}
crontab [-u usuario] OPCI\'ON
\end{vscreen}

Se le puede especificar el modificador \comando{-u usuario} para
cambiar la lista de tareas de un usuario en especial, pero esto
requiere tener permisos de administrador. Las opciones posibles para
el comando \comando{crontab} son:

\begin{description}
\item[-l] Listar las tareas planificadas.
\item[-r] Eliminar la lista de tareas planificadas.
\item[-e] Editar la lista de tareas planificadas.
\end{description}

El formato del archivo de \comando{crontab} para describir la lista de
tareas planificadas se describe a continuaci\'on:

\begin{vscreen}
c1 c2 c3 c4 c5 <comando de tarea a planificar>
\end{vscreen}

\comando{c1} a \comando{c5} son campos que definen en que momento y
con que frecuencia ejecutar la tarea espec\'ifica:

\begin{description}
\item[c1] Minuto en que se ejecutar\'a la tarea (de 0 a 59).
\item[c2] Hora en que se ejecutar\'a la tarea (de 0 a 23).
\item[c3] D\'ia del mes en que se ejecutar\'a la tarea (de 1 a 31).
\item[c4] Mes en que se ejecutar\'a la tarea (de 1 a 12).
\item[c5] D\'ia de la semana en que se ejecutar\'a la tarea (de 0 a 7).
\end{description}

Cada campo adem\'as puede especificar rangos, por ejemplo si en el
campo \comando{c2} se usa un valor \comando{5-9}, la tarea planificada
se ejecutar\'a cada hora desde las 5 hasta las 9.

Cada campo tambi\'en puede especificar listas de valores, separados
por comas. As\'i si en una tarea se le configura el campo \comando{c4}
el valor \comando{1,7,10} la tarea se ejecutar\'a en Enero, Julio y
Octubre.

En conjunci\'on con los rangos de valores, se pueden especificar
intervalos, por ejemplo si necesitamos que una tarea chequee el nivel
de uso de un servidor cada 2 horas dentro del horario pico, digamos de
8 de la ma\~{n}ana hasta las 8 de la noche, el campo \comando{c2} se
puede escribir con el valor \comando{8-20/2}.

Como \'ultimo detalle a nombrar, cada campo puede tener como valor un
asterisco (*), que significa que la tarea se ejecutar\'a en todos los
valores de dicho campo, es decir que si por ejemplo se necesita
ejecutar un proceso que env\'ie un bolet\'in informativo a los
usuarios todos los d\'ias a las 7 de la ma\~{n}ana, la entrada
siguiente ser\'a necesaria en \comando{cron}:

\begin{vscreen}
0 7 * * * enviar-boletin-informativo.sh
\end{vscreen}

A continuaci\'on se enumerar\'an algunos ejemplos con referencia a los
casos de uso presentados al comienzo del tema.

\begin{ejemplo}
El servidor posee una tarjeta de red no totalmente soportada, y por lo
tanto, se desactiva constantemente. Para solucionar este problema (sin
tener que instalar una nueva tarjeta en el servidor) se puede
planificar un proceso de chequeo del estado de la interfaz de red, que
en caso de no estar en funcionamiento, la active. Para esto se
programar\'a el \comando{cron} para que ejecute dicho proceso cada
media hora:

\begin{vscreen}
*/30 * * * * chequear-tarjeta-de-red.sh
\end{vscreen}
\end{ejemplo}

\begin{ejemplo}
Se necesita enviar un bolet\'in informativo a los usuarios de un sitio
web, todos los lunes de cada semana, entre los meses de Marzo y
Noviembre. Se programar\'a \comando{cron} para que los boletines sean
enviados a las 7 de la ma\~{n}ana:

\begin{vscreen}
0 7 * 3-11 1 enviar-boletin-informativo.sh
\end{vscreen}
\end{ejemplo}

\begin{ejemplo}
Se necesita automatizar la tarea de copias de respaldo en un servidor,
para esto es necesario programar la copia de respaldo incremental
(diaria) todos los d\'ias excepto los domingos a las 4:00, cuando el
servidor no tiene tanta carga, y los domingos se realizar\'a la copia
de respaldo completa, tambi\'en a las 4:00 porque se supone que el
servidor est\'a menos cargado los d\'ias domingo.

Para realizar esto se necesitar\'an 2 entradas en el
\comando{crontab}, como las que se muestran a continuaci\'on:

\begin{vscreen}
0 4 * * 1-6   respaldo-incremental.sh
0 4 * * 7     respaldo-total.sh
\end{vscreen}

\end{ejemplo}

A nivel sistema, las distribuciones modernas tienen configurado el
\comando{cron} de tal manera que se puedan planificar ciertas tareas
tan s\'olo copiando los \emph{scripts} necesarios a directorios
espec\'ificos:

\begin{vscreen}
/etc/cron.daily/		(diariamente)
/etc/cron.hourly/		(cada hora)
/etc/cron.monthly/		(cada mes)
/etc/cron.weekly/		(semanalmente)
\end{vscreen}

\'Estos \emph{scripts} se ejecutar\'an con nivel de
administrador, es por eso que se debe tener especial cuidado con lo
que se coloca en esos directorios.

