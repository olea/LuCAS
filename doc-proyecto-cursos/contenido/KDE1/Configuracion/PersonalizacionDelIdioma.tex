\porcion{Personalizaci�n del idioma}

\autor{\LDP}
\colaborador{\JAB}
\revisor{}
\traductor{}

Una de las primeras cosas que quiz�s uno quiera hacer para sentirse
m�s c�modo en el escritorio, es configurarlo para que use el mismo
idioma que uno habla. 

Generalmente el KDE est� configurado por defecto para usar el idioma
Ingl�s, para cambiarlo, se debe cargar el \emph{KDE Control Center},
desplegar la categor�a \emph{Desktop}, y seleccionar la opci�n
\emph{Language}, como se ve en la figura \ref{fig:CambioDeIdioma}

\figura{Como primer paso: Cambiar el idioma al Espa�ol}{CambioDeIdioma}

Se tienen tres listas desplegables, en la primera seleccionar el
idioma ``Spanish'', y oprimir \boton{OK}; el sistema avisar� que se
necesita reiniciar el escritorio (solamente el escritorio) para que
los cambios hagan efecto, por lo tanto, el segundo paso ser�
seleccionar la opci�n \boton{Logout} del bot�n
\boton{K}\footnote{Al mejor estilo Windows95, el KDE tiene el
bot�n de men� a la izquierda en la barra de botones, cuyo dibujo es el
logo mismo de KDE}.

El cambio de lenguaje en el entorno KDE provee un reemplazo casi
completo de idioma en el escritorio as� como tambi�n en las
aplicaciones KDE. De existir alguna aplicaci�n KDE instalada sin
soporte para el idioma seleccionado, el sistema asignar� a esa
aplicaci�n el segundo o tercer idioma configurado en el cuadro que se
muestra en la figura \ref{fig:CambioDeIdioma}, por defecto se ha
dejado que las aplicaciones sin soporte de Espa�ol usen el Ingl�s para
desplegar sus mensajes.

Al volver a ingresar al entorno se podr�n notar las diferencias, de
ahora en adelante, se tratar�n las opciones en Espa�ol.