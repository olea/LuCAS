\porcion{Personalizaci�n del escritorio}
\label{sec:escritorio}

\autor{\LDP}
\colaborador{\JAB}
\revisor{}
\traductor{}

Antes de seguir adelante, es conveniente aclarar el concepto de
<<m�ltiples escritorios>>, esta es una caracter�stica que proveen la
mayor�a de los entornos de escritorio, en una misma sesi�n de X tener
varios escritorios, esto posibilita trabajar con m�s comodidad. KDE
por defecto tiene configurados 4 escritorios, en la barra de
herramientas se pueden observar los botones llamados \boton{Uno}
\boton{Dos} \boton{Tres} y \boton{Cuatro}, los que se utilizan para
acceder a dichos escritorios. Tambi�n existe una forma de cambiar de
escritorios de una manera m�s directa: usando la combinaci�n de
teclas \comando{Ctrl-F1} para el primero, \comando{Ctrl-F2} para el
segundo, etc. Estos escritorios m�ltiples le proveen al usuario m�s
espacio para colocar sus aplicaciones.

