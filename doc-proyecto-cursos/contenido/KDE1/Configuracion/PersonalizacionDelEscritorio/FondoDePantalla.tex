\porcion{Fondo de pantalla}

\autor{\LDP}
\colaborador{\JAB}
\revisor{}
\traductor{}

En la categor�a \emph{Escritorio} del \emph{Centro de control KDE} se
tienen las diferentes opciones para configurar el aspecto y la
funcionalidad del entorno, la primera opci�n es \emph{Fondo}, y se
utiliza para configurar el color o im�gen del fondo de los
escritorios.

\figura{Configuraci�n del fondo de pantalla}{CentroDeControlEscritorio-Fondo}

Como se ve en la figura \ref{fig:CentroDeControlEscritorio-Fondo}, se
puede configurar cada escritorio por separado, o unificar la
configuraci�n para todos seleccionado la opci�n \emph{Fondo
com�n}. En el sector \emph{Colores} se puede seleccionar un color de
fondo o utilizando la opci�n \emph{Dos colores}, se logra un
``degrad�'' entre ambos. 

En la secci�n \emph{Tapiz} se encuentra una lista de las im�genes
que pueden utilizarse para el fondo. En la lista desplegable de
\emph{Disposici�n} existen varias opciones para la forma de mostrar
en pantalla dichos tapices.

Para que el fondo de pantalla cambie autom�ticamente, puede
activarse activar la opci�n \emph{Al azar}.

Si se activa la opci�n \emph{Fijar en el panel}, en la barra de
herramientas del KDE aparecer� un �cono especial para poder cambiar la
configuraci�n del fondo directamente sin necesidad de entrar en el
Centro de Control KDE.
