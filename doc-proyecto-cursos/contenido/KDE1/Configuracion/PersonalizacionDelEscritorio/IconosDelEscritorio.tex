\porcion{�conos de Escritorio}

\autor{\LDP}
\colaborador{\JAB}
\revisor{}
\traductor{}

Los �conos en el escritorio tienen ciertas propiedades de
visualizaci�n y localizaci�n que se pueden modificar en esta secci�n.

\figura{Configuraci�n de iconos del escritorio}{CentroDeControlEscritorio-IconosDelEscritorio}

En la figura \ref{fig:CentroDeControlEscritorio-IconosDelEscritorio}
se ve c�mo se pueden modificar las distancias horizontal y vertical de
la ``Red de fondo'' (una grilla imaginaria que se utiliza para
posicionar los �conos en el escritorio). Adem�s de esto, activando la
opci�n \emph{Texto Transparente para los �conos del Escritorio}, se le
puede dar una mejor apariencia al texto que poseen los iconos por
debajo.

Las opciones \emph{Color del frente del �cono} y \emph{Color del fondo
del �cono} corresponden a la configuraci�n de los colores de las
letras y del fondo de ellas respectivamente. Por �ltimo, si se activa
la opci�n \emph{Mostrar Ficheros Ocultos en el Escritorio} podr�n
visualizarse aquellos archivos ``invisibles''\footnote{Se denominan
``invisibles'' a aquellos archivos cuyo nombre comienzan con un punto}
que se encuentren en el escritorio; normalmente �sta opci�n no tiene
mucha utilidad.