\porcion{Colores}

\autor{\LDP}
\colaborador{\JAB}
\revisor{}
\traductor{}

Los colores de los diferentes componentes de las ventanas pueden
cambiarse de forma tal de darle al escritorio la apariencia que a uno
m�s le agrade. Esto se puede hacer en la secci�n de \emph{Colores} que
puede observarse en la figura
\ref{fig:CentroDeControlEscritorio-Colores}. Este cuadro se divide en
tres partes m�s una de visualizaci�n previa.

\figura{Configuraci�n de colores en KDE}{CentroDeControlEscritorio-Colores}

Cada configuraci�n de colores se puede guardar como un esquema
individual, en el cuadro \emph{Esquema de colores} se pueden ver
varios esquemas ya definidos, simulando los colores de varios
entornos. Pueden definirse nuevos esquemas presionando \boton{A�adir}
y definiendo un nombre para ese nuevo esquema, una vez hecho esto, en
el cuadro \emph{Color del widget\footnote{Widget se le llama a cada
componente de una ventana, un bot�n, un recuadro, la barra de t�tulo,
etc.}} se elige cada componente que se quiera asignar un color
espec�fico y luego se presiona el bot�n de color seleccionando as� el
color a utilizar. En la visualizaci�n previa se puede ir controlando los
cambios de colores a medida que se realizan.

El cuadro \emph{Contraste} posee una barra de nivel que establece el
contraste entre los colores seleccionados.
