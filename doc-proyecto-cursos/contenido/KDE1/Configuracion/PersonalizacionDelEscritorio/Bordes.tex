\porcion{Bordes}

\autor{\LDP}
\colaborador{\JAB}
\revisor{}
\traductor{}

Los bordes activos se utilizan para <<moverse>> de un escritorio a
otro a trav�s del uso del mouse. Cuando est�n activados, si se
lleva el puntero hacia el borde derecho de la pantalla y se lo deja un
instante, el KDE activa el escritorio consecutivo. Lo mismo pasa si se
lleva el mouse hasta la izquierda de la pantalla.

Seleccionando la opci�n \emph{Activar bordes activos} como se ve en
la figura \ref{fig:CentroDeControlEscritorio-Bordes}, se puede
establecer el tiempo de retardo que requieren los bordes activos para
funcionar.

\figura{Configuraci�n de bordes}{CentroDeControlEscritorio-Bordes}

Otras opciones en la secci�n inferior del cuadro son los
\emph{Bordes M�gicos}, que se utilizan para <<magnetizar>> los
bordes de las ventanas entre s�.
